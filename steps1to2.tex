We will first imitate the first two steps of Yazdi's construction, constructing two infinite families of non-orientable surfaces.

\textbf{Step 1:} We will define two families of surfaces $P_{n,k}$ and $Q_{n,k}$, first defining $P_{n,k}$ in the following way. Let $T_1$ be a non-orientable surface of genus 12 with 3 boundary components $c,d$ and $e$. We give $c, d$ and $e$ a consistent choice of orientation (\textcolor{red}{Is this the right thing to do here?}). Let $p$ (respectively $q$) be a puncture (respectively a marked point) on the boundary component $e$ of $T$. Let $r$ and $s$ be the two oriented arcs connecting $p$ and $q$ in $\partial T$. See Figure X for a picture of $T$. Let $T_{i,j}$ be copies of the surface $T$, where $i,j \in \mathbb{Z}$. We will use similar notations to refer to the boundary components of $T_{i,j}$. Define the infinite surface $N_{\inf}$ as the quotient $$N_{\inf} \coloneqq \left( \bigcup T_{i,j} \right)/\sim,$$ where $i,j \in \mathbb{Z}$