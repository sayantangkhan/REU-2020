\section{Universal Bounds}

In this section we will prove the second part of Theorem \ref{thm:stretch}, an extension of Yazdi's upper bound that is independent of the number of punctures. As Yazdi states , the trade off here is that we must assume the genus is sufficiently large when compared to the number of punctures.

\textcolor{red}{List of Results in this section and what they depend on:} \caleb[margin]{I'm trying to figure out which proofs we need to elaborate on and which ones we can completely skip}
\begin{enumerate}
    \item Lemma 4.1: Just number theory
    \item Lemma 4.3: Just number theory
    \item Lemma 4.5: Depends just on the combinatorics of the curves, and our curves follow the exact same combinatorics
    \item The graph $\overline{\Gamma}$ that Yazdi defines in this section should be the exact same for our example since the ways our curves intersect outside the building blocks is the same 
    \item Lemma 4.7: Depends on the graph $\overline{\Gamma}$
    \item Lemma 4.8: This might be something we have to prove. It seems to only depend on the combinatorics of the curves but it might take some more work.
    \item Lemma 4.9: Pretty direct from previous lemmas
    \item Lemma 4.10: Easy
    \item Lemma 4.11:
\end{enumerate}

\subsection{Preliminary Results}

Before going into our second construction, we need to state some of the number theoretic results, notations and definitions by Yazdi that we will be using. 

The Jacobsthal function $j(n)$ is defined as, for a given $n \in \mathbb{N}$, what is the smallest integer $j(n)$ such that any sequence of $j(n)$ consecutive integers contains an element relatively prime to $n$? An important corollary of work done by Iwaniec \cite{iwaniec1978problem} on this function is that:
\begin{corollary}[Iwaniec, Yazdi (?)]
There exists a number $K \geq 2$ such that for every $n \geq 2$, each of the intervals
\begin{align*}
    [\log^2(n),K\log^2(n)], [K\log^2(n),2K\log^2(n)],[2K\log^2(n),3K\log^2(n)],\dots
\end{align*}
contains a number that is relatively prime to $n$.
\end{corollary}

This result is important in defining the following set.Let $\mathcal{S}_n$ be the following set
\begin{align*}
    \mathcal{S}_n \coloneqq \{a \in \mathbb{N} \vert a \geq \log^2(n), \gcd(n,a) = 1\}.
\end{align*}

Then Yazdi shows that this is not a `sparse' subset of the natural numbers.

\begin{lemma}[Yazdi]
The ratio of any two consecutive members of $\mathcal{S}_n$ is bounded above by $2K$, where $K$ is the constant in corollary above. IN particular, $K$ is independent of $n$. Moreover, if $a_1$ is the smallest element of $\mathcal{S}_n$ then $a_1 \leq K\log^2(n)$.
\end{lemma}

In order to define the new maps that we will use for the universal bound, we will introduce notation used by Yazdi in his construction.

Given $n \in \mathbb{N}$, we define $\overline{n} \in \mathbb{Z}$ as follows

\begin{gather*}
    n \equiv 1 (\mod 2) \implies 
\end{gather*}
