\section{Thurston norm for non-orientable $3$-manifolds}
\label{sec:thur-norm-non-orientable}
In his original manuscript defining what is now called the Thurston norm, Thurston defined a norm on $H_2(M;\RR)$ where $M$ is an orientable 3-manifold \cite{thurston1986norm}.  He wrote ``Most of this paper works also for non-orientable manifolds but for simplicity we only deal with the orientable case.''
In this section, we recall the Thurston norm for orientable manifolds in Section \ref{sec:backgr-thurst-norm} and then adapt it to the non-orientable setting.
In Section \ref{sec:thurst-norm-cohom} we describe the challenge of defining the Thurston norm on $H_2(M;\RR)$ if $M$ is non-orientable and present the solution of defining the Thurston norm instead on $H^1(M;\RR)$.
However, Poincar\'e duality does not hold for non-orientable manifolds.
We therefore define a condition -- {\it relative orientability} on a pair consisting of a manifold and an embedded surface.
A surface that is relatively orientable in a non-orientable 3-manifold $M$ will have a corresponding cohomology class in $H^1(M;\ZZ)$, giving a version of Poincar\'e duality for non-orientable 3-manifolds as stated in Theorem~\ref{thm:Poincare-duality}.
Finally in Section \ref{sec:oriented-sums}, we define the oriented sum for relatively oriented embedded surfaces in non-orientable manifolds.

\subsection*{Notation}

We set up the following notational convention we adhere to in Sections \ref{sec:thur-norm-non-orientable} and \ref{sec:mapping-classes-with}.
\begin{itemize}
    \item $M$ will refer to an compact $3$-manifold. Unless otherwise specified, we will not assume anything about the orientability of $M$.
    \item $N$ will refer to a non-orientable compact $3$-manifold. $\wt{N}$ will refer to the orientation double cover of $N$.
    \item A subscript $\varphi$ on $N$ or $M$ will indicate that the $3$-manifold is the mapping torus of the homeomorphism $\varphi$.
    \item $\no$ will refer to a embedded non-orientable surface in $N$. The pre-image of $\no$ in $\wt{N}$ will be referred to as $\wt{\no}$.
    \item If $\alpha$ is a function or $1$-form on $N$, then $\wt{\alpha}$ will refer to the pullback to $\wt{N}$.
\end{itemize}

\subsection{Thurston norm and mapping tori}
\label{sec:backgr-thurst-norm}

Let $M$ be an compact orientable closed $3$-manifold.
%The Thurston norm is a semi-norm on its second homology group $H_2(M; \RR)$; to define the norm, we need to make some preliminary remarks and define a function.
Let $S$ be a connected surface embedded in $M$  The complexity of $S$ is $\chi_-(S) = \max\left\{-\chi(S),0\right\}$.
If the surface $S$ has multiple components $S_1, \ldots, S_m$ then $\chi_-(S) = \displaystyle\sum_{i=1}^m\chi_-(S_i)$.
The elements in $H_2(M ; \ZZ)$ can be represented by embedded surfaces inside of $M$ \cite[Lemma 1]{thurston1986norm}.


\p{Thurston norm} Let $a$ homology class in $H_2(M; \ZZ)$.  Define the integer valued norm $x:H_2(M;\ZZ)\rightarrow \ZZ$ as the following:
\begin{align*}
  x(a) = \min\{\chi_-(S) \mid [S] = a \text{ and $S$ is compact, properly embedded and oriented}\}.
\end{align*}

We then linearly extend $x$ to $H_2(M;\mathbb{Q})$.
There is then a unique continuous extension of $x$ to $\RR$: an $\RR$-valued function on $H_2(M; \RR)$ called the \emph{Thurston norm}.
The unit ball for the Thurston norm is a convex polyhedron.%, and thus it makes sense to talk about the \emph{faces} of the unit ball.

%In this paper, we'll focus on $3$-manifolds arising from mapping tori constructed from surfaces and their homeomorphisms.
\p{Mapping tori} Let $S$ be a surface, and let $\varphi: S \to S$ be a homeomorphism.  A {\it mapping torus} of $S$ by $\varphi$ is a $3$-manifold $M_\varphi$ given by the identification:
\begin{align*}
  M_\varphi \coloneqq \frac{S \times [0,1]}{(x,1) \sim (\varphi(x), 0)}.
\end{align*}

A mapping torus is a surface bundle over $S^1$, or \emph{fibrations over $S^1$}, denoted $S\rightarrow M\rightarrow S^1$.
A fibration defines a flow on $M$, called the \emph{suspension flow}, where for any $x_0\in S$ and $t_0\in S^1$ the pair $(x_0,t_0)$ is sent to $(x_0,t_0+t)$.
The fiber of a fibration is the preimage of any point $\theta \in S^1$ under the projection map from $M_{\varphi} \to S^1$.
The fiber as a subset of $M$ is only well defined up to isotopy, since we don't specify the choice of $\theta$, but the homology class of the fiber in $H_2(M_{\varphi}; \RR)$ is well defined.

The theory of Thurston norm becomes useful for answering the following question: Given a compact oriented $3$-manifold $M$, in how many ways can $M$ be expressed as a mapping torus of a surface and a homeomorphism?
This question was answered by the following remarkable theorem of Thurston \cite{thurston1986norm}.
We use the restatement from \cite{yazdi2018pseudo}.
\begin{thm}[Thurston]
  \label{thm:Thur1}
  Let $M$ be an orientable 3-manifold.  Let $\mathcal{F}$ be the set of homology classes in $H_2(M; \RR)$ that are representable by fibers of fibrations of $M$ over the circle.
\begin{enumerate}[(i)]
\item Elements of $\mathcal{F}$ are in one-to-one correspondence with (non-zero) lattice points inside some union of cones over open faces of the unit ball in the Thurston norm.
\item If a surface $F$ is transverse to the suspension flow associated to some fibration of
  $M \xrightarrow[]{} S^1$ then $[F]$ lies in the closure of the corresponding cone in $H_2(M;\RR)$.
\end{enumerate}
\end{thm}
The class $[F]$ referred to in the theorem has orientation such that the positive flow direction is pointing outwards relative to the surface.
The open faces whose cones contain the fibers of fibrations are what are referred to as \emph{fibered faces}.
The takeaway from this theorem is that if we want to determine if a surface $S$ embedded in a $3$-manifold is homologous to the fiber of some fibration, it suffices to check whether the homology class $[S]$ lies in one of the open cones corresponding to the fibrations.

The goal for the rest of this section is to prove a version of Theorem \ref{thm:Thur1} for compact non-orientable $3$-manifolds.
Most of the work in the proof will involve reducing the version for non-orientable $3$-manifolds to the orientable version by passing to the double cover.

\subsection{Thurston norm on cohomology of non-orientable mapping tori}
\label{sec:thurst-norm-cohom}

%\subsubsection*{The problem with homology in non-orientable $3$-manifolds}
A na\"ive first attempt at defining the Thurston norm would be to define it on the second homology group, like in the orientable case.
However, if the norm is defined in that fashion, the non-orientable version of Theorem \ref{thm:Thur1} will not be true.
Consider a compact non-orientable surface $\no$, a homeomorphism $\varphi: \no \to \no$, and the associated mapping torus $N_{\varphi}$.
Clearly, $N_{\varphi}$ fibers over $S^1$, and $\no$ is the fiber of this fibration.  However, the homology class associated to $\no$ is the zero homology class, since the top-dimensional homology of non-orientable compact surfaces is $0$-dimensional.

Our workaround for this problem will be to deal with the first cohomology $H^1(N_{\varphi})$ rather than the second homology $H_2(N_{\varphi})$.
By Poincar\'e duality they are the same for orientable 3-manifolds, but that is not true for non-orientable $3$-manifolds.

\paragraph{Poincar\'e Duality}
To see why Poincar\'e duality for non-orientable 3-manifolds fails, we first consider the orientable case.
We look at a construction of the isomorphism between $H^1(M; \ZZ)$ to $H_2(M; \ZZ)$ for an orientable manifold $M$.  To do so, we set up a correspondence between elements of $H^1(M; \ZZ)$ and homotopy classes of maps from $M$ to $S^1$ as follows.
Given a cohomology class $[\alpha]$ in $H^1(M; \ZZ)$, choose a representative $1$-form $\alpha$, and a basepoint $y_0$ in $M$.
The associated map $f_{\alpha}$ is given by the following formula.
\begin{align}\label{form:map}
  f_{\alpha}(y) \coloneqq  \int_{y_0}^y \alpha \mod \ZZ.
\end{align}
Changing the basepoint or the representative $1$-form gives a different map to $S^1$ that is homotopic to the original map (see Section 5.1 of \cite{calegari2007foliations} for the details).
One can recover the $1$-form $\alpha$ from the map $f_{\alpha}$ by pulling back the canonical length form $d\theta$ on $S^1$ along $f_{\alpha}$.

Let $\theta \in S^1$ be a regular value and let $S = f_{\alpha}^{-1}(\theta)$ be a surface.
To construct the associated homology class, we choose an orientation on $S$ by declaring that the outwards pointing normal vectors on $S$ are assigned a positive value by the form $\alpha$.
Then $S$ inherits an orientation from the orientation on $M$, and we have defined a
fundamental class $[S]$.
We claim that $[S]$ is the Poincar\'e dual to $\alpha$.
\begin{lem}
  Let $\theta$ and $\theta'$ be two regular values of the function $f_{\alpha}$ and let $S=f_\alpha^{-1}(\theta)$ and $S'=f_\alpha^{-1}(\theta')$.
  Then for any closed $2$-form $\omega$ on $M$, the following identity holds:
  \begin{align*}
    \int_{S} \omega = \int_{S'} \omega.
  \end{align*}
  Furthermore, the following identity also holds:
  \begin{align*}
    \int_S \omega = \int_M \alpha \wedge \omega.
  \end{align*}
  In particular, the homology class of $S$ is Poincar\'e dual to $\alpha$.
\end{lem}
\begin{proof}
  The first part of the lemma follows from the fact that $S$ and $S'$ are homologous, i.e. $f^{-1}_{\alpha}([\theta, \theta'])$ is a singular $3$-chain that has $S$ and $S'$ as boundaries.
  From Stokes' theorem, we get the following:
  \begin{align*}
    \int_{S - S'} \omega &= \int_{f_{\alpha}^{-1}([\theta, \theta'])} d\omega \\
                         &= 0.
  \end{align*}

  To prove the second claim, observe that we can break up the second integral as a product integral:
  \begin{align*}
    \int_M \alpha \wedge \omega &= \int_{S^1} \left(   \int_{f_{\alpha}^{-1}(\theta)} \omega \right) d\theta.
  \end{align*}
  The above equation is true because $\alpha$ is the pullback of $d\theta$ along the map $f_{\alpha}$.
  Observe that the inner integral only makes sense when $\theta$ is a regular value, but by Sard's theorem, almost every $\theta \in [0,1]$ is a regular value, so the right hand side is well-defined.
  By the first claim, the inner integral is a constant function, as we vary over the $\theta$ which are regular values of $f_{\alpha}$.
  Then the integral of $d\theta$ over $S^1$ is $1$, giving us the identity we want:
  \begin{align*}
    \int_M \alpha \wedge \omega = \int_S \omega.
  \end{align*}
\end{proof}
What we have here is an explicit formula for the Poincar\'e duality map from $H^1(M; \RR)$ to $H_2(M; \RR)$.
For orientable $3$-manifolds, this is an isomorphism, and more specifically the following theorem is true.
\begin{thm}[Poincar\'e duality for orientable $3$-manifolds]
  \label{thm:orientable-poincare-duality}
  Let $M$ be an orientable $3$-manifold, and let $S$ be an oriented embedded surface. Then there exists a $1$-form
  $\alpha$ and a regular value $\theta \in S^1$ such that $S$ and $f_{\alpha}^{-1}(\theta)$ are homologous surfaces.
\end{thm}

Note that the map from the space of $1$-forms to homology classes of an embedded surface still makes sense for a non-orientable $3$-manifold $N$.
However in that case the map from $H^1(N; \ZZ)$ to $H_2(N; \ZZ)$ has a nontrivial kernel, as we saw in the example where $N_{\varphi}$ was the mapping torus of a non-orientable surface, and $\no$ was the fiber.

\paragraph{Defining the Thurston norm on cohomology}
Let $N$ be a non-orientable $3$-manifold, and $\wt{N}$ its orientation double cover.
We will denote by $\iota$ the orientation reversing deck transformation
of $\wt{N}$, and the covering map $\wt{N} \to N$ by $p$.
If $N=N_\varphi$ is the mapping torus of the non-orientable surface $\no$ and a self-homeomorphism $\varphi: \no \to \no$, then $\wt{N}$ is the mapping torus of $(\os, \wt{\varphi})$, where $\os$ is the orientable double cover of $\no$, and $\wt{\varphi}$ is the orientation preserving lift of $\varphi$.

Since we have already concluded that the first cohomology is the right space on which to define the Thurston norm, we need to relate $H^1(N; \RR)$ and $H^1(\wt{N}; \RR)$.
We do so by pulling back $H^1(N;\RR)$ to $H^1(\wt{N};\RR)$ via $p$.

\begin{lem}
  \label{lem:injective}
  The pullback $p^{\ast}:H^1(N;\RR)\rightarrow H^1(\wt{N};\RR)$ maps $H^1(N; \RR)$ bijectively to the $\iota^{\ast}$-invariant subspace of   $H^1(\wt{N}; \RR)$.
\end{lem}
\begin{proof}
  For any $1$-form $\alpha$ on $N$, $p^{\ast}(\alpha)$ will be $\iota^{\ast}$-invariant.
  To check that $p^\ast$ is injective, consider a $1$-form $\alpha$ on $N$ such that $p^{\ast}\alpha$ is exact.
  Then there exists a smooth function $g:\wt{N} \to \RR$ such that the following relation holds:
    \begin{align*}
        dg = p^{\ast} \alpha.
    \end{align*}
    But since $p^{\ast}\alpha$ is $\iota^{\ast}$-invariant, we must have $dg = \iota^{\ast} dg$.
    Because $\iota^\ast$ commutes with the exterior derivative, we have $dg = d(\iota^{\ast}g)$.
    That means $g$ and $\iota^{\ast}g$ differ by a constant, but that constant must be $0$ since $\iota^2$ is the identity map.
    Thus $g$ is $\iota$-equivariant and therefore descends to a function on $N$ and $\alpha$ must be exact, which proves injectivity of $p^{\ast}$.

    To show surjectivity, let $[\alpha]$ be an element in $H^1(\wt{N}; \RR)$ that is $\iota^{\ast}$-invariant and let $\alpha$ be a representative.
    Since the cohomology class $[\alpha]$ is $\iota$-invariant, $\alpha$ and $\iota^{\ast}\alpha$ must differ by an exact form.
    \begin{align*}
        \alpha - \iota^{\ast}(\alpha) = dg
    \end{align*}
    Applying $\iota^\ast$ to both sides of the equality, we have $\iota^{\ast}dg = -dg$.
    This means that the $1$-form $\beta = \alpha - \frac{dg}{2}$ is an $\iota^{\ast}$-invariant representative of the cohomology class $[\alpha]$.
    Since $\beta$ is $\iota^{\ast}$-invariant, we can push it forward by the projection map $p$, defining the value of its pushforward locally.
    This gives a well defined $1$-form on $N$, whose pullback is $\beta$, and proves surjectivity of $p^{\ast}$.
\end{proof}
\begin{rem}
  Note that if we change the coefficients in the statement of this lemma from $\RR$ to $\ZZ$, the proof of injectivity follows through, but the proof of surjectivity does not.
\end{rem}

Lemma \ref{lem:injective} tells us that $H^1(N; \RR)$ is a subspace of $H^1(\wt{N}; \RR)$, so we define the Thurston norm on $H^1(N; \RR)$ by restricting the Thurston norm on the orientable 3-manifold $\widetilde{N}$ to the subspace $p^*(H^1(N;\RR))$ of $H^1(\widetilde{N};\RR)$.

\paragraph{Thurston norm for non-orientable $3$-manifolds}
  Let $N$ be a non-orientable 3-manifold and $\wt{N}$ its orientation double cover, $\wt{x}$ be the Thurston norm on $H^1(\wt{N};\RR)$ defined in Section \ref{sec:backgr-thurst-norm}, and let $\alpha \in H^1(N;\RR)$.
  The \emph{Thurston norm on $H^1(N; \RR)$}, is the norm $x: H^1(N;\RR) \rightarrow \RR$ defined:
  \begin{align*}
    x(\alpha) \coloneqq \wt{x}(p^{\ast}\alpha).
  \end{align*}

We now extend properties of the Thurston norm for orientable manifolds to the Thurston norm on $H^1(N;\RR)$.
\begin{thm}
  The unit ball with respect to the dual Thurston norm on $\left( H^1(N; \RR) \right)^{\ast}$ is a polyhedron in $(H^1(N,\RR))^\ast$ whose vertices are lattice points $\{\pm \beta_1, \ldots \pm \beta_k\}$.
  The unit ball $B_1$ with respect to Thurston norm is a polyhedron given by the following inequalities.
  \begin{align*}
    B_1 = \left\{ a\in H^1(N,\RR) \mid \left| \beta_i(a) \right| \leq 1 \text{ for $1\leq i \leq k$} \right\}
  \end{align*}
\end{thm}

\begin{proof}
  The proof is identical to the original proof of Thurston \cite[Theorem 2]{thurston1986norm}.
  Because the norm of an element of $H^1(N; \ZZ)$ is the Thurston norm of the corresponding element in $H^1(\wt{N}; \ZZ)$, the norm of any element in $H^1(N; \ZZ)$ is always an integer.
  The rest of the proof works exactly like it does in the orientable case.
\end{proof}

Note that defining the Thurston norm on $H^1(N; \RR)$ rather than $H_2(N; \RR)$ is not quite satisfactory.
In particular, fibers of fibrations are embedded surfaces in $N$, and in the orientable case, the norm on the second homology relates these fibers to the Thurston norm.
In Section \ref{sec:weak-inverse-poinc}, we develop a version of Poincar\'e duality for non-orientable 3-manifolds that will let us translate a family of embedded surfaces into $1$-forms, allowing us to talk about their Thurston norm.

\subsection{Weak inverse to the Poincar\'e duality map}
\label{sec:weak-inverse-poinc}

Let $M$ be a 3-manifold.  Regardless of whether $M$ is orientable or not, we can construct a dual map $f_\alpha$, using equation \eqref{form:map}.
The preimage of a regular value $\theta \in S^1$, denoted by $f_{\alpha}^{-1}(\theta)$, will be an embedded surface in $M$.
When $M$ is orientable, Poincar\'e duality determines a cohomology class in $H^1(M; \ZZ)$ corresponding any embedded surface $S$.
In that case, let $\alpha$ be a $1$-form representative $\alpha$ of a cohomology class and let $\theta$ of $f_{\alpha}$.  The surfaces $f_{\alpha}^{-1}(\theta)$ and $S$ are homologous.
In this section, we state a weaker version of Poincar\'e duality for non-orientable surfaces in Theorem \ref{thm:Poincare-duality}, that lets us associate $1$-forms to a certain class of embedded surfaces in non-orientable $3$-manifolds.
The condition we need to impose upon the embedded surfaces is \emph{relative orientability}.

\p{Relative orientability}
  Let $M$ be a $3$-manifold, and $S$ an embedded surface in $M$.
  The surface $S$ is said to be \emph{relatively oriented with respect to $M$} if there is a nowhere vanishing vector field on $S$ that is transverse to the tangent plane of $S$.
  Two such vector fields are said to induce the same orientation if locally they induce the same orientation after choosing a local frame for $S$.
  A surface $S$ is \emph{relatively oriented} if both $S$ and the choice of positive normal vector field are specified.

Note that relative orientability is a strictly weaker notion than orientability.
If $S$ and $M$ are orientable, then $S$ is relatively orientable with respect to $M$.
But even if $M$ is non-orientable, a non-orientable embedded surface $S$ may be relatively orientable with respect to $M$.
For instance, let $S$ be the fiber of a non-orientable mapping torus $N_{\varphi}$.
The preimage under the projection map to $S^1$ of a non-vanishing vector field is a non-vanishing vector field on $M$ that is always transverse to the fiber.

On the other hand, for orientable $M$ and $S$, if $S$ is relatively oriented with respect to $S$, then a choice of orientation on $S$ determines an orientation on $M$ and vice versa.

\p{A surface that is not relatively orientable in a 3-manifold}
  Let $S$ be the standard torus $\RR^2/\ZZ^2$, and let $\varphi$ map $(x,y)$ to $(-x, y)$.
  Then $\varphi$ is an orientation-reversing homeomorphism.
  Therefore the mapping torus $M_\varphi$ is non-orientable.
  Consider a vertical line $\gamma$ in $S$ preserved by $\varphi$, i.e. the line
  $x = 0$.
  The image of $\gamma$ in $S$ under the suspension flow in $M$ is a subsurface of $M$, which we'll call $S'$.
  The normal direction to $S'$ when restricted to $S$ is $\frac{\partial}{\partial x}$.
  Because the suspension flow reverses the direction of $\gamma$, the normal vector field cannot be continuously extended to all of $M$.
  This means that the surface $S'$ is not relatively orientable in $M$ (despite being orientable itself.)

However, if both $M$ and an embedded surface are non-orientable, the surface will be relatively orientable.
\begin{prop}
  \label{prop:relative-orientability}
  Let $N$ be a non-orientable $3$-manifold, and let $\no$ be an embedded connected non-orientable surface in $N$.
  Then $\no$ is relatively orientable with respect to $N$.
\end{prop}
\begin{proof}
  Let $\wt{N}$ be the orientation double cover of $N$, and $\wt{\no}$ be the preimage of $\no$ under the double cover.
  The restriction of the orientation reversing deck transformation $\iota:\wt{N}\rightarrow\wt{N}$ to $\wt{\no}$ is an orientation reversing homeomorphism of $\wt{\no}$.
  Let $(v_1, v_2)$ be positively oriented local frame for the tangent space to $\wt{\no}$.
  Let $n$ be an outward pointing transverse vector to $\wt{\no}$ so the local frame $(v_1, v_2, n)$ is positively oriented.
  Since $\iota$ reverses the orientation of both $\wt{\no}$ and $\wt{N}$, $(\iota(v_1), \iota(v_2))$ and $(\iota(v_1), \iota(v_2), \iota(n))$ are both negatively oriented.
  Then $\iota(n)$ is outward pointing.
  Therefore the outward pointing transverse direction on $\wt{\no}$ descends to an outward pointing transverse direction on $\no$, and $\no$ is relatively orientable in $N$.
\end{proof}

We care about relatively orientable surfaces because these surfaces can be mapped to cohomology classes.
\begin{thm}[Poincar\'e duality for non-orientable $3$-manifolds]
  \label{thm:Poincare-duality}
  Let $N$ be a non-orientable $3$-manifold, and let $\no$ be a relatively oriented embedded surface.  Let $\wt{N}$ be the orientation double cover with covering map $p:\wt{N}\to N$.
  Then there exists a cohomology class $[\alpha]$ in $H^1(N; \ZZ)$, such that for any $1$-form representative $\alpha$, and any regular value $\theta$ of $f_{\alpha}$, the preimages $p^{-1}(\no)$ and $p^{-1}(f_{\alpha}^{-1}(\theta))$ are homologous in $\wt{N}$.
  Furthermore, $\alpha$ assigns positive values to the positively oriented transverse vector field on $\no$.
\end{thm}
\begin{rem}
  The only statement in the above theorem that does not follow from the orientable version of Poincar\'e duality is the fact that the dual to the preimage of $\no$ in the double cover $\wt{N}$ is a pullback of an element of $H^1(N; \ZZ)$. Also, the conclusion of the theorem is not strong enough for our application: we will later need to show that $\no$ and $f_{\alpha}^{-1}(q)$ are homeomorphic surfaces, but to do so, we will need to impose stronger conditions on $\no$ and $N$, and use a result of Thurston.
\end{rem}

%The idea of the proof of this theorem is fairly straightforward.
%Starting with the embedded surface $S$ in $M$, we look at the pre-image $\wt{S}$ in the orientation double cover $\wt{M}$.
%We show that the Poincar\'e dual to $\wt{S}$ is $\iota^{\ast}$-invariant.
\begin{lem}
  \label{lem:PD1}
  Let $N$ be a non-orientable 3-manifold and let $\wt{N}$ and the map $p:\wt{N}\to N$ be its orientation double cover.  Let $\iota$ be the generator of the deck group of $\wt{N}$
  Let $\no$ be a relatively oriented embedded surface in $N$, and let $\wt{\no}$ be its preimage under $p$ in $\wt{N}$.
  Then the Poincar\'e dual to $[\wt{\no}]$ is $\iota^{\ast}$-invariant.
\end{lem}
\begin{proof}
  If $\no$ is relatively oriented with respect to $N$, then the relative orientation lifts to a relative orientation of $\wt{\no}$ with respect to $\wt{N}$.
  Since $\wt{N}$ is oriented, the relative orientation of $\wt{\no}$ defines an orientation of $\wt{\no}$,\becca[inline]{Need to say this outside the proof somewhere. SK: Mentioned in the paragraph defining relative orientability.} and thus the homology class $[\wt{\no}]$ is well defined.

  We show first that $\iota$ reverses the orientation on $\wt{\no}$.
  Let $(v_1, v_2, v_3)$ be a local frame for some point in $\wt{\no}$ such that
  $v_3$ is the outward pointing transverse vector field.
  Because $\no$ is relatively oriented in $N$, the outward pointing transverse vector field on $\wt{\no}$ must descend to an outward pointing transverse vector field on $\no$.\becca[inline]{Didn't this used to have more justification? SK: I realized that the justification boils down to saying that the lift of a non-vanishing vector field is non-vanishing, so I removed it. I can add it back if necessary.}
  This means $\iota$ takes outward pointing vector fields to outward pointing vector fields, and $\iota(v_3)$ must be outward pointing.
  Since $\iota$ reverses the orientation on $\wt{N}$ but preserves the direction of $\iota(v_3)$, $\iota$ must reversing the orientation on the pair $(v_1, v_2)$.
  In particular, that means $\iota$ reverses the orientation on $\wt{\no}$.

  This means $[\wt{\no}]$ is in the $-1$-eigenspace of the $\iota_{\ast}$ action on $H_2(\wt{N}; \RR)$.
  Let the cohomology class $[\wt{\alpha}]$ be the the Poincar\'e dual to $[\wt{\no}]$.
  Then there exists representative 1-form $\wt{\alpha}$ that is $\iota^{\ast}$-invariant.
  This follows from the following chain of equalities which hold for all closed $2$-forms $\omega$.
  We use the fact that $\iota^2= \mathrm{id}$ in the first and third equalities:
  \begin{align*}
    \int_{\iota_{\ast}\wt{\no}} \omega &= \int_{\wt{\no}} \iota^{\ast}\omega &&\text{(By a change of variables)} \\
                                     &= \int_{\wt{N}} \wt{\alpha} \wedge \iota^{\ast} \omega &&\text{(Poincar\'e duality)} \\
                                     &=\int_{\wt{N}} \iota^{\ast} \left( \iota^{\ast}\wt{\alpha} \wedge \omega \right) \\
    &= \int_{\wt{N}} - \left( \iota^{\ast} \wt{\alpha} \wedge \omega \right) &&\text{($\iota$ is orientation reversing)}
  \end{align*}
  On the other hand, the following equalities follow from the fact that $\iota_{\ast}[\wt{\no}] = -[\wt{\no}]$.
  \begin{align*}
    \int_{\iota_{\ast}\wt{\no}} \omega &= - \int_{\wt{\no}} \omega \\
                              &= - \int_{\wt{N}} \wt{\alpha} \wedge \omega
  \end{align*}
  Because $$\int_{\wt{N}}\wt{\alpha}\wedge\omega=\int_{\wt{N}}\iota^\ast\wt{\alpha}\wedge\omega$$ for all $\omega$, it follows that $\wt{\alpha}$ is
  $\iota^{\ast}$-invariant.
\end{proof}

Let us continue denoting the Poincar\'e dual to $[\wt{\no}]$ by $[\wt{\alpha}]$.  The class $[\wt{\alpha}]$ is an $\iota^{\ast}$-invariant element of $H^1(\wt{N}; \ZZ)$, but it is not clear that $[\wt{\alpha}]$ is the pullback of an element of $H^1(N; \ZZ)$ under $p$.
In the next lemma, we show that is indeed the case, i.e. $[\wt{\alpha}]$ is the pullback of an element in $H^1(N; \ZZ)$.
\begin{lem}
  \label{lem:PD2}
  % Let $N$ be a non-orientable 3-manifold and let $\wt{N}$ and the map $p:\wt{N}\to N$ be its orientation double cover.
  There exists an $[\alpha] \in H^1(N; \ZZ)$ such that $\wt{\alpha} = p^{\ast} \alpha$.
\end{lem}
\begin{proof}
  To show that $[\wt{\alpha}]$ is the pullback of an element in $H^1(N; \ZZ)$, it will suffice to show that the integral of $\wt{\alpha}$ along any closed curve or arc in $\wt{N}$ that projects down to a closed curve in $N$ is an integer.
  Since $\wt{\alpha} \in H^1(\wt{N}; \ZZ)$, the integral along any closed curve will be an integer, it will suffice to check along arcs that project down to closed curves in $N$.
  Let $\delta$ be such an arc, starting at a point $x_0 \in \wt{N}$.
  Since its image in $N$ must close up, $\delta$ must necessarily end at $\iota(x_0)$.
  The next goal is to show that
    $\displaystyle\int_{\delta} \wt{\alpha}$ is in $\ZZ.$
  Consider the closed curve $\wt{\gamma}$, obtained by taking the union of $\delta$ and $\iota(\delta)$.
  Because $\wt{\alpha}$ is the $\iota^{\ast}$-invariant, we have that $ \displaystyle\int_{\delta} \wt{\alpha} = \displaystyle\int_{\iota(\delta)} \wt{\alpha}.$  Therefore $$\int_{\wt{\gamma}}\wt{\alpha}=2\int_{\delta}\wt{\alpha}.$$ It will suffice to show that $\displaystyle\int_{\wt{\gamma}}\wt{\alpha}$ is an even integer.
  %\begin{align*}
   % \int_{\wt{\gamma}} \wt{\alpha} \in 2\ZZ.
  %\end{align*}
  Without loss of generality, we can assume that the curve $\wt{\gamma}$ intersects the surface $\wt{\no}$ transversally at all points.
  Since $\wt{\alpha}$ is a representative of the Poincar\'e dual to $[\wt{\no}]$, the integral of $\wt{\alpha}$ along $\wt{\gamma}$ is the signed intersection number of $\wt{\gamma}$ with $\wt{\no}$.
  The intersection number must be even, for if $\wt{\gamma}$ and $\wt{\no}$ intersect at a point $y$, then they also intersect at $\iota(y)$, by $\iota$-invariance of $\wt{\no}$ and $\wt{\gamma}$. This proves the lemma.
\end{proof}
We now have everything we need to finish proving Theorem \ref{thm:Poincare-duality}.
\begin{proof}[Proof of Theorem \ref{thm:Poincare-duality}]
  %Starting with a relatively oriented surface $\no$ in $N$, we look at its pre-image $\wt{\no}$ in $\wt{N}$ under the orientation double cover.
  Let $\wt{\no}=p^{-1}(\no).$
  The relative orientation of the $\wt{\no}$ gives us the homology class $[\wt{\no}]$.  Let the $1$-form $\wt{\alpha}$ be the Poincar\'e dual to $[\wt{\no}]$.
  By Lemmas \ref{lem:PD1} and \ref{lem:PD2}, we have that $[\wt{\alpha}]$ is the pullback of an element in $H^1(N; \ZZ)$, which is the Poincar\'e dual to the relatively oriented surface $\no$.
  It follows from the orientable version of Poincar\'e duality that $\wt{\no}$ and $p^{-1}(f_{\alpha}^{-1}(\theta))$ are homologous surfaces in $\wt{N}$.
\end{proof}

We will now strengthen the conclusion of Theorem \ref{thm:Poincare-duality} to get a more concrete relationship between $\no$ and $f_{\alpha}^{-1}(\theta)$.
To do so, we will need to define \emph{incompressible surfaces}, and state a result of Thurston.

\p{Incompressible surfaces}
  Let $S$ be a surface with positive genus embedded in a $3$-manifold $M$.
  The surface $S$ is said to be \emph{incompressible} if there does not exist an embedded disc $D$ in $M$ such that $D \cap S = \partial D$, and $D$ intersects $S$ transversally.

\begin{thm}[Theorem 4 of \cite{thurston1986norm}]
  \label{thm:Thur2}
Let $M$ be an oriented 3-manifold that fibers over $S^1$.  Let $S$ be an incompressible surface embedded in $M$.  If $S$ is homologous to a fiber, then $S$ is isotopic to the fiber.
\end{thm}

We can now state the strengthened version of Poincar\'e duality for non-orientable $3$-manifolds.
\begin{thm}
  \label{thm:strong-duality}
  Let $N$ be a compact non-orientable $3$-manifold, and let $\no$ be a relatively oriented incompressible surface.
  Let $[\alpha] \in H^1(N; \ZZ)$ be the dual to $\no$, as given by Theorem \ref{thm:Poincare-duality}.
  If $[\alpha]$ has a $1$-form representative $\alpha$ that vanishes nowhere on $N$, then $\no$ is homeomorphic to $f_{\alpha}^{-1}(\theta)$ for all $\theta \in \no^1$.
\end{thm}

Before we prove this theorem, we need the following lemma.
\begin{lem}
  \label{lem:lift-of-incompressible}
  If $\no$ is a relatively oriented incompressible surface in a non-orientable $3$-manifold $N$, then the lift $\wt{\no}$ of $\no$ in $\wt{N}$, the orientation double cover of $N$, is also incompressible.
\end{lem}
\begin{proof}
  The fact that $\no$ is incompressible in $N$ implies that the inclusion induced map on fundamental groups is injective.
  \begin{align*}
    \pi_1(\no) \hookrightarrow \pi_1(N)
  \end{align*}
  Since $\pi_1(\wt{\no})$ and $\pi_1(\wt{N})$ are subgroups of $\pi_1(\no)$ and $\pi_1(N)$ respectively, the map $\pi_1(\wt{\no}) \to \pi_1(\wt{N})$ must also be injective.
  Furthermore, $\wt{\no}$ is two-sided in $\wt{N}$: in such cases, the induced map on fundamental groups being injective is equivalent to incompressibility.
  This proves the lemma.
\end{proof}

\begin{proof}[Proof of Theorem \ref{thm:strong-duality}]
  We have that the representative $1$-form $\alpha$ vanishes nowhere.
  That means the map $f_{\alpha}: N \to \no^1$ is a fibration, since it has full rank everywhere.
  Consider now the lift $\wt{\no}$ of $\no$ and $f_{\alpha} \circ p$ of $f_{\alpha}$ to the orientation double cover $\wt{N}$.
  By Lemma \ref{lem:lift-of-incompressible}, $\wt{\no}$ is incompressible, and the lift of the fibration is still a fibration.
  By Theorem \ref{thm:Poincare-duality}, we must have that $\wt{\no}$ and $p^{-1}(f_{\alpha}^{-1}(\theta))$ are homologous.
  Theorem \ref{thm:Thur2} then tells us $\wt{\no}$ must be isotopic to the fiber of $p \circ f_{\alpha}$.
  But $\wt{\no}$ and $p^{-1}(f_{\alpha}^{-1})(\theta)$ are two sheeted covers of the surfaces $\no$ and $f_{\alpha}^{-1}(\theta)$ respectively.
  If $\wt{\no}$ and $p^{-1}(f_{\alpha}^{-1})(\theta)$ are homeomorphic, then $\no$ and $f_{\alpha}^{-1}(\theta)$ must also be homeomorphic.
  This proves the result.
\end{proof}

\begin{rem}
  Note that the above proof doesn't tell us if $\no$ and $f_{\alpha}^{-1}(\theta)$ are isotopic.
  To have that, we would require the isotopy on the orientation double cover to be $\iota^{\ast}$-invariant, which we have not done because the current implication is strong enough for our application.
\end{rem}

If we associate to each fibration of $N$ the $1$-form obtained by pulling back the canonical form $d\theta$ on $\no^1$, we are able state a non-orientable version of Theorem \ref{thm:Thur1}.
\begin{thm}
  \label{thm:classifying-fibrations}
  Let $N$ be a compact non-orientable $3$-manifold, and let $\mathcal{F}$ be the elements of $H^1(N; \ZZ)$ corresponding to fibrations of $N$ over $\no^1$.
  \begin{enumerate}[(i)]
  \item Elements of $\mathcal{F}$ are in a one-to-one correspondence with (non-zero) lattice points (i.e. points of $H^1(N; \ZZ)$) inside some union of cones over open faces of the unit ball in the Thurston norm.
  \item If a relatively oriented surface $\no$ is transverse to the suspension flow associated to some fibration $f: N \to \no^1$, then the Poincar\'e dual $[\alpha]$ to $\no$ lies in the  closure of the cone in $H^1(N; \RR)$ containing the $1$-form corresponding to $f$.
  \end{enumerate}
\end{thm}
\begin{proof}
  The union of cones in $H^1(N; \RR)$ is determined by intersecting the pullback of $H^1(N; \RR)$ to the cohomology of the orientable double cover with the union of cones in $H^1(\wt{N}; \RR)$.
  Clearly, every lattice point in this union of cones corresponds to a fibration, since the pullback to $H^1(\wt{N}; \ZZ)$ corresponds to a fibration.
  Conversely, every fibration of $N$ must lie in one of these cones, since the composition with the covering map $p$ gives a fibration of $\wt{N}$ over $\no^1$.

  If the surface $\no$ is transverse to the suspension flow of a fibration $f$, then $\wt{\no}$ is transverse to $p \circ f$, and thus the Poincar\'e dual $\wt{\alpha}$ to $\wt{\no}$ lies in the closure of the cone in $H^1(\wt{N}; \RR)$.
  But since the dual $\alpha$ to $\no$ pulls back to $\wt{\alpha}$, this means $\alpha$ lies in the closure of the cone in $H^1(N; \RR)$.
\end{proof}

\subsection{Oriented sums}
\label{sec:oriented-sums}

The next step in studying embedded non-orientable surfaces will be to describe \emph{oriented sums}.
The oriented sum of two surfaces embedded in a manifold $M$ is additive in both the Euler characteristic and $H^1(M;\RR)$.
This operation is well-known in the case of orientable $3$-manifolds (along with orientable embedded surfaces), but we will sketch out the relevant details.
We then extend the construction relatively orientable embedded surfaces.

\paragraph{Oriented sum for oriented manifolds}

Let $S$ and $S'$ be oriented embedded surfaces in an oriented manifold $M$.
Assume that $S$ and $S'$ intersect transversally.
Thus, $S \cap S'$ is a disjoint union of copies of $S^1$.
For each component of $S\cap S'$, take a tubular neighborhood that has cross section as in Figure \ref{fig:cross-section}.

%\autoref{fig:cross-section}.
\begin{figure}
  \centering
  \incfig[0.2]{cross-section}
  \caption{Cross section of intersection of $S$ and $S'$.}
  \label{fig:cross-section}
\end{figure}

We then perform a surgery on the leaves of $S$ and $S'$ so that the outward pointing normal vector fields match as in Figure \ref{fig:surgery}.
\begin{figure}
  \centering
  \incfig[0.3]{surgery}
  \caption{On the left, the normal vectors on $S$ and $S'$ are consistent. On the right, they are not.}
  \label{fig:surgery}
\end{figure}

By performing this surgery at all the intersections, we get a new submanifold $S''$ (which may have multiple components).
This new submanifold $S''$ is the oriented sum of $S$ and $S'$.
The operation of taking oriented sums is additive on Euler characteristic, as well as the homology classes (and thus the cohomology classes of their Poincar\'e duals).
\begin{align*}
  \chi(S'') &= \chi(S) + \chi(S') \\
  [S''] &= [S] + [S'] \\
\end{align*}

\paragraph{Oriented sum for non-orientable manifolds}

Let $N$ be a non-orientable manifold and let $\no$ and $\no'$ be embedded surfaces in $N$ that are relatively oriented.
We define the oriented sum on $\no$ and $\no'$ as follows.
Let $p:\wt{N}\rightarrow N$ be the orientation double cover and let $\iota$ be the orientation reversing deck transformation of $\wt{N}$.
As above, let $\wt{\no}=p^{-1}(\no)$ and $\wt{\no}'=p^{-1}(\no')$, which are embedded oriented surfaces in $\wt{N}$.
The oriented sum of $\no$ and $\no'$ is the image under $p$ of the oriented sum of $\wt{\no}$ and $\wt{\no}'$ (as defined above for oriented surfaces in oriented manifolds).
We need to show that this operation is well-defined.

As in the proof of Lemma \ref{lem:PD1}, $\iota$ preserves the relative orientation, and thus leaves the outward normal vector fields on $\wt{\no}$ and $\wt{\no}'$ invariant.
Therefore a leaf $\ell$ of $\wt{\no}$ is surgered with a leaf of $\ell'$ of $\wt{\no}'$ if and only if $\iota(\ell)$ and $\iota(\ell')$ are surgered.
Therefore surgery factors through $p$ and the oriented double sum is well-defined.

\begin{example}
  Let $\gamma$ be a component of $\no\cap \no'$ and $\wt{\gamma}_1$ and $\wt{\gamma}_2$ be the path lifts of $\gamma$.
  Consider \autoref{fig:consistency}, which shows the outward point normal vectors to $\wt{\no}$ and $\wt{\no'}$, which determine which leaves are glued together along $\wt{\gamma}_1$ and $\wt{\gamma}_2$.

\begin{figure}
  \centering
  \incfig[0.4]{consistency}
  \caption{Neighborhoods of $\wt{\gamma}_1$ and $\wt{\gamma}_2$, with the outward pointing normal vector field.}
  \label{fig:consistency}
\end{figure}

The normal vector field tells us that the left $\wt{\no}$ leaf gets glued to the bottom $\wt{\no'}$ leaf near $\wt{\gamma}_1$ and $\wt{\gamma}_2$.
Since $\iota(\wt{\gamma}_1)=\wt{\gamma_2})$, the outward pointing normal vector fields point the same (relative) directions.
\end{example}

\paragraph{Additivity}

By the consistency of the oriented sum in $N$ and $\wt{N}$, it easily follows that the oriented sum is additive in Euler characteristic, as well as in terms of Poincar\'e dual, since the Poincar\'e dual was also defined by passing to the orientation double cover.
