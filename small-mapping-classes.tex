\section{Mapping classes with small stretch factors}
\label{sec:mapping-classes-with}

In this section, we provide a strategy to compute pseudo-Anosov homeomorphisms with small stretch factors.
%The key tools from \autoref{sec:thur-norm-non-orientable} are Theorems \ref{thm:strong-duality} and \ref{thm:classifying-fibrations}, and the operation of oriented sums for non-orientable surfaces.

\subsection{Mapping class groups of non-orientable surfaces}
\label{sec:backgr-mapp-class}
Let $\no$ be a non-orientable surface and let $\wt{\no}$ and the covering map $p:\wt{\no}\rightarrow \no$ be its orientation double covering space.
Every homeomorphism $\varphi: \no \to \no$, has a unique orientation-preserving lift $\wt{\varphi}: \wt{\no} \to \wt{\no}$.


A consequence is that lifting homeomorphisms induces a monomorphism between homeomorphisms of $\no$ and orientation-preserving homeomorphisms of $\wt{\no}$.
Every homotopy of $\no$ lifts to a homotopy of $\wt{\no}$.
%Further, if $f,g:\no\to\no$ are homeomorphisms such that their orientation preserving lifts are homotopic, then $f$ and $g$ are homotopic.
%\becca[inline]{Do we need this (the previous) sentence?  It does not follow naturally from the sentence before it and actually requires proof/citation.  But it is only required if we want the inclusion to be injective.}
%\sayantan[inline]{We don't really need this to be injective, but the fact is probably folklore: we could prove it if we needed to, but it would be more elementary covering space stuff, and wouldn't add anything useful.}
%\becca[inline]{That's what I thought (we could also just cite it, there's a paper that proves it somewhere out there).  But I think we should just remove that sentence.}
Therefore there is an inclusion from the mapping class group of $\no$ to the (orientation-preserving) mapping class group of $\wt{\no}$.
This inclusion also respects the Nielsen-Thurston classification of mapping classes, both qualitatively, and quantitatively, as the following proposition shows.
\begin{prop}
  \label{prop:2}
  Let $\varphi:\no\rightarrow\no$ be a homeomorphism and let $\wt{\varphi}:\wt{\no}\rightarrow\wt{\no}$ be the orientation-preserving lift of $\varphi$.  Then:
  \begin{enumerate}[(i)]
  \item $\varphi$ is periodic if and only if $\wt{\varphi}$ is periodic,
  \item $\varphi$ is reducible if and only if $\wt{\varphi}$ is reducible, and
  \item $\varphi$ is pseudo-Anosov if and only if $\wt{\varphi}$ is pseudo-Anosov.  Moreover if $\varphi$ has stretch factor $\lambda$, then $\wt{\varphi}$ also has stretch factor $\lambda$.
  \end{enumerate}
\end{prop}
\begin{proof}
 % It's easy to see that if $\varphi$ is periodic, so it $\wt{\varphi}$, and the other way round.
  %If $\varphi$ is reducible, that means it leaves some multicurve $\gamma$ in $\no$ invariant, which means $\wt{\varphi}$ leaves the pre-image of $\gamma$ invariant as well.
  %Conversely, if $\wt{\varphi}$ leaves some multicurve $\wt\gamma$ invariant, so does $\iota \circ \wt{\varphi}$, since they commute.\becca[inline]{Need more here.  I'll think about it (See Aramayona--Leininger--Souto ``Injections on mapping class groups": there exist (branched) covering spaces where pA lift to reducible.  This is not such a case, but why?)}
  %That means the union of $\wt\gamma$ and $\iota(\wt\gamma)$, where $\iota$ is the orientation reversing deck transformation, is also a multi-curve and thus descends to a multi-curve on $\no$ that is left invariant by $\varphi$.\becca[inline]{The image of the multicurve may be non-simple or trivial}
  %If $\varphi$ is neither periodic nor reducible, it must be pseudo-Anosov.  By exclusion, $\wt{\varphi}$ must also be pseudo-Anosov.
  The fact that the map from $\Mod(\no)$ to $\Mod(\wt{\no})$ is type-preserving follows from Aramayona--Leininger--Souto \cite[Lemma 10]{aramayona2009injections} (while the statement of the Lemma is for orientable surfaces, the argument, which we will skip, is identical for non-orientable surfaces).

  Suppose now that $\varphi:\no\rightarrow\no$ is a psuedo-Anosov homeomorphism with stretch factor $\lambda$ and stable and unstable foliations $\mathcal{F}_s$ and $\mathcal{F}_u$ respectively.
  Let $\wt{\mathcal{F}_s}$ and $\wt{\mathcal{F}_u}$ denote the lifts of the stable and unstable foliations to the orientation double cover.
  We need to show that the following identities hold for all simple closed curves $\gamma$ in $\wt{\no}$ (see \cite[Expos\'e 5]{FLP} for the definition of intersection number with measured foliations; the fact that these identities suffice follows from \cite[Lemma 9.15]{FLP}):
%  \sayantan[inline]{Provide citations for stretch factor claim and also intersection number with foliations definition.}
  \begin{align}
      \label{eq:unstable-foliation}
      i(\gamma, \wt{\varphi}(\wt{\mathcal{F}_u})) &= \lambda \cdot i(\gamma, \wt{\mathcal{F}_u}) \\
      \label{eq:stable-foliation}
      i(\gamma, \wt{\varphi}(\wt{\mathcal{F}_s})) &= \frac{1}{\lambda} \cdot i(\gamma, \wt{\mathcal{F}_s}).
  \end{align}

  To see that these identities hold, we partition $\gamma$ into short arcs $\{\gamma_i\}$ such that the restriction of the covering map $p$ to a neighborhood of each arc is a homeomorphism.
  %The local homeomorphism lets us compute the intersection number for each arc $\gamma_i$ by instead computing the intersection number on the surface $\no$:
  Then we have:
  \begin{align}
  \label{eq:push1}
    i(\gamma_i, \wt{\mathcal{F}_u}) &= i(p(\gamma_i), \mathcal{F}_u) \\
  \label{eq:push2}
    i(\gamma_i, \wt{\varphi}(\wt{\mathcal{F}_u})) &= i(p(\gamma_i), \varphi(\mathcal{F}_u)).
  \end{align}
  Since we know that $\mathcal{F}_u$ is the unstable foliation for $\varphi$ with stretch factor $\lambda$, we can compute the ratio of the right hand side of \eqref{eq:push1} and \eqref{eq:push2}:
  \begin{align}
      \label{eq:ratio}
      i(p(\gamma_i), \varphi(\mathcal{F}_u)) = \lambda \cdot i(p(\gamma_i), \mathcal{F}_u).
  \end{align}
  Combining \eqref{eq:push1}, \eqref{eq:push2}, and \eqref{eq:ratio}, and summing over all $\gamma_i$ gives us that \eqref{eq:unstable-foliation} holds. A similar argument also verifiproves that \eqref{eq:stable-foliation} holds.
\end{proof}



\subsection{Constructing pseudo-Anosov maps on nearby surfaces using oriented sums}
\label{sec:constr-psuedo-anos}
%\becca[inline]{I rewrote the section.  The old version is in comments.}
The goal of this section is to prove that the stretch factor of any pseudo-Anosov homeomorphism provides an asymptotic upper bound for the minimum stretch factor.  We do this in Proposition \ref{prop:asymptotic}.


\begin{prop}\label{prop:asymptotic}
Let $\no_g$ be a non-orientable surface of genus $g$ and let $\varphi:\no_g\rightarrow \no_g$ be a pseudo-Anosov homeomorphism with stretch factor $\lambda$.  Let $N_\varphi$ be the mapping torus of $\no_g$ by $\varphi$.  Let $\no_{g'}$ be an incompressible surface embedded in $N_\varphi$ that is transverse to the supension flow associated to $\varphi$.  Then for all $k\in\ZZ^+$, there is a pseudo-Anosov homeomorphism of the oriented sum $\no_{g}+k\no_{g'}$ with stretch factor at most $\lambda$.
\end{prop}

Our strategy for proving Proposition \ref{prop:asymptotic} is to find fiber bundles of $N_{\varphi}$ over $S^1$ that have fiber $\no_g+k\no_{g'}$.  We then apply a special case of Thurston's hyperbolization theorem, which says that the mapping torus of an orientable surface $S$ by a homeomorphism $\varphi$ is hyperbolic if and only if $\varphi$ is pseudo-Anosov \cite[Theorem 0.1]{thurston_hyp}.  In particular, Thurston's theorem implies that if $M=M_\varphi$ fibers over $S^1$ in two ways, either both mondromies are pseudo-Anosov or neither monodromy is pseudo-Anosov.  Finally, we adapt theorems of Fried and Matsumoto (Theorem \ref{thm:fm}) and Agol--Leininger--Margalit (Theorem \ref{thm:alm}) to work for mapping tori with non-orientable fibers.


\medskip
We will repeatedly use the following two facts for orientable surfaces and 3-manifolds:
\begin{enumerate}
 % \label{thm:norm-minimizing}
 \item An orientable surface $S$ minimizes the Thurston norm in its homology class if and only if $S$ is incompressible.
\item  If an orientable 3-manifold $M$ fibers over $S^1$, then the fiber is incompressible.
\end{enumerate}


%As a consequence, if we can decompose the mapping torus of a pseudo-Anosov map as the mapping torus of some other homeomorphism, then the other homeomorphism must also be pseudo-Anosov.


%Let $\no$ be a non-orientable surface, and $\varphi$ be a pseudo-Anosov homeomorphism.
%Let $N_\varphi$ be the associated mapping torus and $f:N\rightarrow S^1$ the associated fibration.
%Suppose we can construct another relatively oriented surface $\no'$ inside $N$ such that $\no'$ is transverse to the suspension flow direction associated to $f$. Let $\alpha$ be the Poincar\'e dual of $\no$ and $\alpha'$ be the Poincar\'e dual of $\no'$.  By Theorem \ref{thm:classifying-fibrations} $\alpha$ lies in a cone $\mathcal{C}_\varphi \subset H^1(N_\varphi; \ZZ)$ associated to $\varphi$ with other $1$-forms also coming from fibrations.
%Furthermore, $\alpha'$ lies in the closure of $\mathcal{C}_\varphi$.
%All positive integer linear combinations of $\alpha$ and $\alpha'$ are lattice elements of $\mathcal{C}_\varphi$.
%Each linear combination of $\alpha$ and $\alpha'$ is Poincar\'e dual to an oriented sum of $\no$ and $\no'$ in $N$.  Under reasonably mild conditions on $\no$ and $\no'$, we can show that the fiber of $f$ is homeomorphic to the oriented sum $\no + \no'$ of $\no$ and $\no'$.

\begin{prop}
  \label{thm:oriented-sum}
  Let $\no'$ be an incompressible surface embedded in $N$ that is transverse to the suspension flow associated to $\varphi$.
  Let $\alpha$ be the Poincar\'e dual of $\no$ and $\alpha'$ the Poincar\'e dual of $\no'$.
  If the oriented sum of $\no$ and $\no'$ is connected, then $\no + \no'$ is homeomorphic to the fiber of the fibration given by $\alpha + \alpha'$.
\end{prop}
\begin{proof}
  Let $p:\wt{N}\rightarrow N$ be the orientation double cover of $N$.
  The surface $\no$ is incompressible because it is a fiber of $f$; therefore $p^{-1}(\no)$ is also incompressible.  Then the Thurston norm of $\no$ of $\alpha$ is $2\chi_-(\no)$.  Likewise, the Thurston norm of $\alpha'$ is $2\chi_-(\no')$.

By \autoref{thm:classifying-fibrations} (ii), $\alpha'$ lies in the same cone in $H^1(N;\ZZ)$ as $\alpha$.  The Thurston norm $x$ on $H^1(N;\ZZ)$ is linear function on that cone.
 Since the Thurston norm is also linear on oriented sums of $\no$ and $\no'$, we have:
  \begin{align*}
    x(\alpha + \alpha') &= x(\alpha) + x(\alpha') \\
                        &= 2\chi_-(\no) + 2\chi_-(\no') \\
                        &= 2\chi_-(\no + \no').
  \end{align*}
  %The last equality follows from the linearity of the oriented sum.
  Because $2\chi_-(\no+\no')$ achieves the Thurston norm of $\alpha+\alpha'$, the preimage $p^{-1}(\no+\no')$ achieves the Thurston norm of the pullback of $\alpha+\alpha'$ under $p$.  Therefore $p^{-1}(\no+\no')$ is incompressible.  Then $\no+\no'$ is also incompressible.


  By Theorem \ref{thm:classifying-fibrations} (i), we have that $\alpha + \alpha'$ corresponds to some other fibration $f'':N\rightarrow S^1$.
  By Theorem \ref{thm:strong-duality}, the fiber of $f''$ must be homeomorphic to $\no + \no'$.  %Since the Poincar\'e dual to $\no + \no'$ is $\alpha + \alpha'$, and $\no + \no'$ is incompressible.
\end{proof}

In the proof of Proposition \ref{prop:asymptotic}, we will use Proposition \ref{thm:oriented-sum} along with a theorem of Thurston to obtain a pseudo-Anosov homemorphism $\varphi_k$ of the surface of genus of genus $g+kg'$.  We the use Theorems \ref{thm:fm} and \ref{thm:alm} to obtain a upper bound on the stretch factor of $\varphi_k$.

%The next step is to show that the stretch factor of the pseudo-Anosov on the new surfaces is less than the stretch factor of the original surface.  We use the following two theorems that hold for orientable 3-manifolds.

\begin{thm}[Fried \cite{fried1982flow,fried1983transitive}, Matsumoto\cite{matsumoto1987topological}]
  \label{thm:fm}
  Let $M$ be an orientable hyperbolic $3$-manifold and let $\mathcal{K}$ be the union of cones in $H^1(M; \RR)$ whose lattice points correspond to fibrations over $S^1$.
  There exists a strictly convex function $h: \mathcal{K} \to \RR$ satisfying the following properties:
  \begin{enumerate}[(i)]
  \item For all $c > 0$ and $u \in \mathcal{K}$, $h(cu) =  \frac{1}{c}h(u)$,
  \item For every primitive lattice point $u \in \mathcal{K}$, $h(u) = \log(\lambda)$, where $\lambda$ is the
    stretch factor of the pseudo-Anosov map associated to this lattice point, and
  \item $h(u)$ goes to $\infty$ as $u$ approaches $\partial \mathcal{K}$.
  \end{enumerate}
\end{thm}

\begin{thm}[Agol-Leininger-Margalit]
  \label{thm:alm}
  Let $\mathcal{K}$ be a fibered cone for a mapping torus $M$ and let $\overline{\mathcal{K}}$ be its closure
  in $H^1(M;\RR)$. If $u \in \mathcal{K}$ and $v \in \overline{\mathcal{K}}$, then $h(u+v) < h(u)$.
\end{thm}

\begin{proof}[Proof of Proposition \ref{prop:asymptotic}]
The oriented sum $\mathcal{S}=\no_g+k\no_{g'}$ constructed in Proposition \ref{thm:oriented-sum} is a surface of genus $g+kg'$, and is $\mathcal{S}$ is homeomorphic to a fiber of $N_\varphi$ given by $\alpha+k\alpha'$.  Let $\varphi_{k}:\mathcal{S}\rightarrow\mathcal{S}$ be the monodromy of $N_\varphi$ over $\mathcal{S}$.  By Thurston's theorem, $\varphi_k$ is pseudo-Anosov.  We claim that $\varphi_k$ has stretch factor at most $\lambda$.


Let $p:\wt{N}\rightarrow N_\varphi$ be the orientation double cover of $N_\varphi$. Let $h\mid_{N}$ be the restriction of $h$ to the pullback  $p^\ast(H^1(N_\varphi; \RR))$ in $H^1(\wt{N}; \RR)$.
The restriction $h\mid_N$ satisfies all the properties of Theorems \ref{thm:fm} and \ref{thm:alm}.

 Let $\wt{\varphi}$ be the orientation preserving lift of $\varphi$ to $p^{-1}(\no)$.  Since $\wt{\alpha}$ is the pullback of $\alpha$, the $\wt{\varphi}$ is the pseudo-Anosov homeomorphism associated to $\wt{\alpha}$.  By Proposition \ref{prop:2}, the stretch factor of $\wt{\varphi}$ is $\lambda$.

Let $\mathcal{K}$ be the cone in $H^1(N_\varphi;\RR)$ containing $\alpha$.  Since $\no_{g'}$ is transverse to the suspension flow in the direction of $\varphi$, we have that $\alpha'$ is in the closure of $\mathcal{K}$ in $H^1(N;\RR)$.  Let $\wt{\alpha}$ be the pullback of $\alpha$ under $p$ and let $\wt{\alpha}'$ be the pullback of $\alpha'$ under $p$.  Then $h\mid_N(\wt{\alpha}+\wt{\alpha}')<h\mid_N(\wt{\alpha})$.  By Theorem \ref{thm:fm}, $h(\wt{\alpha})$ is equal to the stretch factor of the pseudo-Anosov homeomorphism associated to $\wt{\alpha}$.  Therefore we have $h\mid_N(\wt{\alpha}+\wt{\alpha}')<\log(\lambda)$. It follows that the stretch factor of $\varphi_k$ is less than $\lambda$.
\end{proof}

%
%In this section, we answer the following question:
%\begin{question}
%  If $\no_g$ has a pseudo-Anosov map with stretch factor $\lambda$, does there exist a pseudo-Anosov map on $\no_{g'}$ with stretch factor at most $\lambda$, for all $g' > g$?
%\end{question}
%\becca[inline]{This might just be a note to self: I don't understand what the difference between $\lambda$ and $\lambda'$ being controlled means.}
%\sayantan[inline]{\sout{This refers to getting an upper bound on $|\ln \lambda - \ln \lambda^{\prime}|$ as a function of $g$ and $g'$. I've changed the question to the more precise version.} You were right in pointing out that the theorem at the end of this section doesn't answer the earlier question; rather it answers the following question, which is weaker, but good enough for our purposes.}
%\becca[inline]{Thanks Sayantan.  Now that I understand, I think I don't like that we ask it as a question without providing the answer right away.  It's probably clearer just to say exactly what we are going to do in the section.}
%To do so, we first prove Theorem \ref{thm:oriented-sum}, which allows us to construct different surfaces and pseudo-Anosov maps on them which have the same mapping torus.
%
%We will repeatedly use the following two facts:
%\begin{enumerate}
% % \label{thm:norm-minimizing}
% \item A surface $S$ minimizes the Thurston norm in its homology class if and only if it is incompressible.
%\item  If $M$ fibers over $S^1$, then the fiber is incompressible.
%\end{enumerate}
%\becca[inline]{Is the point that these facts are only known for orientable surfaces and we have to finagle a little to make them work for non-orientable in Theorem 3.2?}
%\sayantan[inline]{Yes, that's right.}
%
%We also have the following theorem of Thurston relating hyperbolicity of mapping tori and pseudo-Anosov maps.
%\begin{thm}[Thurston's Hyperbolization Theorem]
 % If $M$ is the mapping torus of a surface $S$ and a homeomorphism $\varphi$, then $M$ is hyperbolic if and only if $\varphi$ is pseudo-Anosov.
%\end{thm}
%
%%As a consequence, if we can decompose the mapping torus of a pseudo-Anosov map as the mapping torus of some other homeomorphism, then the other homeomorphism must also be pseudo-Anosov.
%In particular, if a manifold is a mapping torus of two different homeomorphisms $\varphi$ and $\varphi'$, then $\varphi$ is pseudo-Anosov if and only if $\varphi'$ is pseudo-Anosov.
%
%Next, we characterize the possible fibers of a mapping torus.
%
%%Let $\no$ be a non-orientable surface, and $\varphi$ be a pseudo-Anosov homeomorphism.
%%Let $N_\varphi$ be the associated mapping torus and $f:N\rightarrow S^1$ the associated fibration.
%%Suppose we can construct another relatively oriented surface $\no'$ inside $N$ such that $\no'$ is transverse to the suspension flow direction associated to $f$. Let $\alpha$ be the Poincar\'e dual of $\no$ and $\alpha'$ be the Poincar\'e dual of $\no'$.  By Theorem \ref{thm:classifying-fibrations} $\alpha$ lies in a cone $\mathcal{C}_\varphi \subset H^1(N_\varphi; \ZZ)$ associated to $\varphi$ with other $1$-forms also coming from fibrations.
%%Furthermore, $\alpha'$ lies in the closure of $\mathcal{C}_\varphi$.
%%All positive integer linear combinations of $\alpha$ and $\alpha'$ are lattice elements of $\mathcal{C}_\varphi$.
%%Each linear combination of $\alpha$ and $\alpha'$ is Poincar\'e dual to an oriented sum of $\no$ and $\no'$ in $N$.  Under reasonably mild conditions on $\no$ and $\no'$, we can show that the fiber of $f$ is homeomorphic to the oriented sum $\no + \no'$ of $\no$ and $\no'$.
%
%\begin{thm}
 % \label{thm:oriented-sum}
 % Let $\no$ be a non-orientable surface and $\varphi$ a homeomorphism of $\no$.
 % Let $N_\varphi$ be the associated mapping torus and $f:N_\varphi\rightarrow S^1$ the fibration.
 % Let $\no'$ be an incompressible surface embedded in $N$ that is transverse to the suspension flow direction associated to $f$.
 % Let $\alpha$ be the Poincar\'e dual of $\no$ and $\alpha'$ the Poincar\'e dual of $\no'$.
 % If the oriented sum of $\no$ and $\no'$ is connected, then $\no + \no'$ is homeomorphic to the fiber of the fibration given by $\alpha + \alpha'$.
%\end{thm}
%\begin{proof}
 % Let $p:\wt{N}\rightarrow N$ be the orientation double cover of $N$.
 % The surface $\no$ is incompressible because it is a fiber of $f$; therefore its pre-image under $p$ is also incompressible.  Therefore the Thurston norm of $\no$ of $\alpha$ is $2\chi_-(\no)$.  Likewise, the Thurston norm of $\alpha'$ is $2\chi_-(\no')$.
%
% Both $\alpha$ and $\alpha'$ lie in a cone over a fibered face in $H^1*N;\ZZ)$.  Therefore the Thurston norm $x$ on $H^1(N;\ZZ)$ is linear function on that cone.
% Since the Thurston norm is linear on oriented sums of $\no$ and $\no'$, we have:
 % \begin{align*}
%    x(\alpha + \alpha') &= x(\alpha) + x(\alpha') \\
   %                     &= 2\chi_-(\no) + 2\chi_-(\no') \\
  %                      &= 2\chi_-(\no + \no').
 % \end{align*}
  %%The last equality follows from the linearity of the oriented sum.
 % Because $2\chi_(\no+\no')$ achieves the Thurston norm of $\alpha+\alpha'$, the preimage $p^{-1}(\no+\no')$ achieves the Thurston norm of the pullback of $\alpha+\alpha'$ under $p$.  Therefore $p^{-1}(\no+\no')$ is incompressible.  Then $\no+\no'$ is also incompressible.
 % \becca[inline]{The previous version of what was above was incomplete.  Perhaps this version is too complete (I think it's correct though).}

 % By Theorem \ref{thm:classifying-fibrations}, we have that $\alpha + \alpha'$ corresponds to some other fibration $f'':N\rightarrow S^1$.
 % By Theorem \ref{thm:strong-duality}, the fiber of $f''$ must be homeomorphic to $\no + \no'$.  %Since the Poincar\'e dual to $\no + \no'$ is $\alpha + \alpha'$, and $\no + \no'$ is incompressible.
%\end{proof}

%Theorem \ref{thm:oriented-sum} is the first step in answering the question posed at the beginning of this subsection.
%If $\no'$ in Theorem \ref{thm:oriented-sum} has Euler characteristic $-1$, then then the oriented sum $\no+k\no'$ has genus equal to the genus of $\no$ plus $k$.
%\becca[inline]{If $\no'$ has Euler characteristic -1, can't we realize all genera greater than 2?}
%The next step is to show that the stretch factor of the pseudo-Anosov on the new surfaces is lesser than the stretch factor of the original surface.

%That is done using the following two theorems, due to Fried--Matsumoto \cite{fried1982flow}, \cite{fried1983transitive}, and \cite{matsumoto1987topological} and Agol--Leininger--Margalit  \cite{agol6983pseudo} in the orientable case.

%\begin{thm}[Fried-Matsumoto]
 % \label{thm:fm}
 % Let $M$ be a hyperbolic $3$-manifold and let $\mathcal{K}$ be the union of cones in $H^1(M; \RR)$ whose lattice points correspond to fibrations over $S^1$.
 % There exists a strictly convex function $h: \mathcal{K} \to \RR$ satisfying the following properties.
 % \begin{enumerate}[(i)]
 % \item For all $c > 0$ and $u \in \mathcal{K}$, $h(cu) =  \frac{1}{c}h(u)$.
 % \item For every primitive lattice point $u \in \mathcal{K}$, $h(u) = \log(\lambda)$, where $\lambda$ is the
 %   stretch factor of the pseudo-Anosov map associated to this lattice point.
  %\item $h(u)$ goes to $\infty$ as $u$ approaches $\partial \mathcal{K}$.
 % \end{enumerate}
%\end{thm}
%
%\begin{thm}[Agol-Leininger-Margalit]
 % \label{thm:alm}
 % Let $\mathcal{K}$ be a fibered cone for a mapping torus $M$ and let $\overline{\mathcal{K}}$ be its closure in $H^1(M;\RR)$. If $u \in \mathcal{K}$ and $v \in \overline{\mathcal{K}}$, then $h(u+v) < h(u)$.
%\end{thm}
%
%\begin{proof}[Proof of Theorems \ref{thm:fm} and \ref{thm:alm} for non-orientable $M$]
%Consider the pullback of $H^1(M; \RR)$ to $H^1(\wt{M}; \RR)$, where $\wt{M}$ is the orientation double cover, and restrict the function $h$ to the image of the pullback.
%It is easy to verify that the restriction satisfies all the listed properties as well.
%\end{proof}
%
%
%This answers the question we posed at the beginning of this subsection, and we use the results of this section in Section \ref{sec:application} to construct pseudo-Anosov maps with small stretch factors.
