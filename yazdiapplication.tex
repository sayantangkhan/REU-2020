\section{Minimal stretch factors for non-orientable surfaces with marked points}
\label{sec:application}

In this section we will use Theorems \ref{thm:classifying-fibrations} and \ref{thm:oriented-sum} to adapt the methods of \cite{yazdi2018pseudo} to non-orientable surfaces. In particular, we prove the following theorem on the asymptotic behavior of the minimal stretch factor of non-orientable surfaces:
\begin{manualtheorem}
  {\ref{thm:stretch1}}
  For any fixed $n \in \mathbb{N}$, there are positive constants $B'_1 = B'_1(n)$ and $B'_2 = B'_2(n)$ such that
  for any $g \geq 2$, the stretch factor satisfies the following inequalities.
  \begin{align*}
    \frac{B'_1}{g} \leq l'_{g,n} \leq \frac{B'_2}{g}
  \end{align*}
\end{manualtheorem}


%Recall that associated to every pseudo-Anosov homeomorphism $f$ there is a number $\lambda(f)$, the
%\textit{dilatation} or \textit{stretch factor}, the amount that the stable and unstable foliations of the
%pseudo-Anosov change by. Given a surface $S$, it is natural to ask what we can say about the set of all
%possible stretch factors, i.e.
%\begin{align*}
%    \left\{\log(\lambda(f)) \mid f \in \text{Mod}(S) \text{ is pseudo-Anosov}\right\}
%\end{align*}

%We call this set the \textit{spectrum} of $S$. A first step at understanding this set of all stretch factors associated to a surface is considering the following quantity
%\begin{align*}
%    l_{g,n} =\min\{\log(\lambda(f)) \mid f \in \text{Mod}(\mathcal{S}_{g,n}) \text{ is pseudo-Anosov}\}.
%\end{align*}

%The study of this minimal stretch factor $l_{g,n}$ was initiated by Penner in his work
%\cite{penner1991bounds}. In this paper Penner studied the asymptotic behavior of minimal stretch factors of orientable surfaces without punctures, i.e. the behavior of $l_{g,0}$. He showed that there exist positive constants
%$A_1$ and $A_2$ such that the following inequalities held for any $g \geq 2$.
%\begin{align*}
%    \frac{A_1}{g} \leq l_{g,0} \leq \frac{A_2}{g}
%\end{align*}
%This showed that asymptotically $l_{g,0}$ behaves like $\frac{1}{g}$ for $g \geq 2$.
%It turns out something similar is true when one starts adding punctures. Yazdi showed
%that for any fixed $n$, $l_{g,n}$ behaves like $\frac{1}{g}$. More precisely, he proved the following theorem.
%\begin{thm}[Theorem 1.2 of \cite{yazdi2018pseudo}]
%\label{thm:yazdi1}
%For any fixed $n \in \mathbb{N}$, there are positive constants $B_1 = B_1(n)$ and $B_2 = B_2(n)$ such that the following inequalities hold for
%any $g \geq 2$.
%\begin{align*}
%    \frac{B_1}{g} \leq l_{g,n} \leq \frac{B_2}{g}.
%\end{align*}
%\end{thm}

%Yazdi's result is one of many recent results in studying the asymptotics of $l_{g,n}$ for different subsets of
%the $(g,n)$ plane. See the introductions of \cite{yazdi2018pseudo} and \cite{tsai2009asymptotic} for more
%examples of results of this form. 

%Yazdi proves an additional result along these lines for a large subset of
%the $(g,n)$ plane, one containing balls of arbitrary large radii.

%\begin{thm}[Yazdi]
%    \label{thm:yazdi2}
%    There exists positive constants $A$, $B$ and $C$ such that for any $n \geq 1$ and $g \geq Cn\log^2(n)$ such that
%    the following inequalities hold.
%    \begin{align*}
%        \frac{B}{g} \leq l_{g,n} \leq \frac{A}{g}
%    \end{align*}

%\end{thm}

%A key tool in Yazdi's proof of this theorem is the fibered face theory of Thurston. With the non-orientable analog of
%Thurston's fibered face theory we developed in the previous section, it's possible to prove an analogous
%theorem for non-orientable punctured surfaces.  Let $\mathcal{N}_{g,n}$ be the genus $g$ non-orientable
%surface with $n$ punctures and let $l_{g,n}'$ be the minimal stretch factor of $\no_{g,n}$.
%\begin{align*}
%  l'_{g,n} = \min\left\{\log(\lambda(f)) \mid f \in \text{Mod}(\mathcal{N}_{g,n})\ \text{is pseudo-Anosov}\right\}
%\end{align*}
%Then we have the following result, analogous to Yazdi's results.
%\begin{thm}
%  \label{thm:stretch1}
%  For any fixed $n \in \mathbb{N}$, there are positive constants $B'_1 = B'_1(n)$ and $B'_2 = B'_2(n)$ such that
%  for any $g \geq 2$, the stretch factor satisfies the following inequalities.
%  \begin{align*}
%    \frac{B'_1}{g} \leq l'_{g,n} \leq \frac{B'_2}{g}
%  \end{align*}
%\end{thm}

Observe that the lower bound for the non-orientable case follows easily from the lower bound for the orientable case.
Let $\varphi$ be a pseudo-Anosov map with the minimal stretch factor on $\no_{g,n}$. Then, by Proposition \ref{prop:2},
$\varphi$ lifts to a map $\wt{\varphi}:\os_{g-1, 2n}\to\os_{g-1,2n}$ (possibly after squaring). Furthermore, $\wt{\varphi}$ has the same
stretch factor as $\varphi$. The former is bounded below by $\frac{B_1}{g}$, and thus the stretch factor of $\varphi$ is bounded
below as well. The more challenging part of the proof is showing the upper bound holds. This will be done by explicitly
constructing pseudo-Anosov maps with small stretch factors, adapting Yazdi's techniques to the non-orientable setting.

We will closely follow Yazdi's construction, which proceeds in five steps.  In steps 1 and 2, a family of small dilatation psuedo-Anosov
maps is constructed on $\os_{g_i,n}$, where $\{g_i\}$ is a sequence of genera going off to infinity, but not
containing every element of $\mathbb{N}$, i.e. there are plenty of gaps. Steps 3 through 5 deal with constructing small dilatation pseudo-Anosov maps on the missing surfaces to fill in the gaps. This is where
Thurston's fibered face theory enters the picture. In this section, we will adapt the steps to work for
non-orientable surfaces.

\paragraph{Step 1: Constructing the surfaces}

The first step in the construction is defining a family of surfaces that exhibit a specific rotational
symmetry. Using this symmetry, if one shows that a power of some homeomorphism is pseudo-Ansov, then so is
the original homeomorphism 
\cite{penner1991bounds}.

%We will try to follow Yazdi's notation as close as we can, in order to make it clear to the reader
%how our construction replicates his.

We begin by defining a family of surfaces $P_{n,k}$. Let $S$ be an orientable surface of genus 5 with 3
boundary components $c,d$ and $e$. Choose an orientation for $S$ and let $c,d$ and $e$ inherit the induced orientations. We obtain a non-orientable surface $T$ from $S$ as follows.  Add two cross caps to $S$ (but retain the orientation of the boundary components of $S$). Remove a point $p$ from the boundary component $e$ and let $q$ be a marked point in $e$.  Let  $r$ and
$s$ be the components of $e\setminus{p,q}$. The resulting surface $T$ is given in Figure \ref{fig:buildingblock}.

\begin{figure}[]
    \centering
    \incfig[0.4]{YazdiTypeSurface}
    \caption{The surface $T$, which will be the building block of the construction.}
    \label{fig:buildingblock}
\end{figure}

Let $T_{i,j}$ be copies of the surface $T$, where $i,j \in \mathbb{Z}$. Let $c_{i,j}, d_{i,j}$ and $e_{i,j}$ be the (oriented) boundary components of $T_{i,j}$ and let $r_{i,j}$ and $s_{i,j}$ be the copies of the arcs $r$ and $s$ in $T_{i,j}$. Define a connected infinite surface $T_\infty$ as the following quotient:
\begin{align*}
  T_\infty \coloneqq \left. \left( \bigcup T_{i,j} \right)\right/\sim
\end{align*}
for all $i$ and $j$ are integers. The gluing $\sim$ is given by the following two families of (orientation-reversing)
identifications:
\begin{align}
\label{identification}
  c_{i,j} &\sim d_{i+1,j} \\
  r_{i,j} &\sim s_{i,j+1}.
\end{align}
Furthermore, the boundary components are glued by an orientation-reversing homeomorphism.


We have two
natural shift maps $\overline{\rho_1},\overline{\rho_2}: T_\infty \to T_\infty$ that act in the
following manner:
\begin{align*}
  \overline{\rho_1}: T_{i,j} &\mapsto T_{i+1, j} \\
  \overline{\rho_2}: T_{i,j} &\mapsto T_{i, j+1}.
\end{align*}

Note that these maps commute. Define the surface $P_{n,k}$ as the quotient of the surface $T_\infty$ by the
covering action of the group generated by $(\overline{\rho_1})^n$ and $(\overline{\rho_2})^k$. Therefore,
$\overline{\rho_1}$ and $\overline{\rho_2}$ induce maps on the surface $P_{n,k}$, which we denote by $\rho_1$
and $\rho_2$.

A natural question at this point is why we chose the surface $T$ for our building block? It has two advantages:
\begin{itemize}
\item The combinatorics of the curves in Figure \ref{fig:curves} make the associated matrix from the Penner construction satisfy
  the conditions of Lemma \ref{lem:spectral}. This is used to prove our family of pseudo-Anosov maps have
  stretch factors bounded above by the quantity we desire.
\item Having a curve $\gamma$ such that it and its image under a given pseudo-Anosov we construct form the boundary of an embedded
  $\RR P^2$ with two boundary components in the mapping torus, which will come into play when extending our family of surfaces in Step 3.
\end{itemize}

\begin{lem}
\label{lem:genera}
Let
\begin{align*}
    g_{n,k} &= (14k - 2)n + 2
\end{align*} for $n \geq 1$ and $k \geq 3$.
    The genus of $P_{n,k}$ is $g_{n,k}$.
\end{lem}
\begin{proof}
%Let $U$ be the quotient of $P_{n,k}$ by the subgroup of $\Mod(P_{n,k})$ generated by $\overline{\rho}_1$.  Therefore $P_{n,k}$ is a $n$-fold cover of $U$.  Because $\overline{\rho}_1$ and $\overline{\rho}_2$ commute, the quotient of $T_\infty$ by $\langle \overline{\rho}_1,\overline{\rho}_2^k$ induces a covering space of $T_\infty$ over $U$.  Let $\pi$ be the covering map $T_\infty\rightarrow U$.  Then $\pi$ is also the composition of the covering map of $T_\infty\to P_{n,k}$ and the covering space of $P_{n,k}\to T_\infty$. Moreover, because $\overline{\rho}_1$ is a deck transformation of $\pi$, each $T_{i,j}$ is a fundamental domain of $\pi$.  The map $\pi$ identifies the boundary components $c_{i,j}$ and $d_{i,j}$ and the arcs $r_{i,j}$ and $s_{i,j}$.  Therefore $\chi(U)=\chi(T_{i,j})=2-% The deck group of the covering space of $T_\infty$ over $U$ is 
  Let $U \subset P_{n,k}$ be the subsurface
  \begin{align*}
    U = \left. \left( \bigcup_{i =0}^{k-1} T_{0,i} \right)\right/\sim
  \end{align*}
  where $\sim$ is given by (\ref{identification}).\becca{Is this right?}
  Then $U$ is a compact, non-orientable surface of genus $12k$ with $2k$ boundary components, and forms a fundamental domain for the covering action of $\overline{\rho_1}$ on $T_\infty$. We can compute the Euler characteristic of $U$ in order to determine the Euler characteristic of $P_{n,k}$:
  \begin{align*}
    \chi(U) &= 2 - 12k - 2k \\
            &= 2 - 14k.
  \end{align*}
  Thus
  \begin{align*}
    \chi(P_{n,k}) &= n \cdot \chi(U)\\
                  &= -n(14k - 2),
  \end{align*}
  since $P_{n,k}$ is formed by gluing $n$ copies of $U$ together along circle boundary components. By the
  relation between genus and Euler characteristic, we have the claimed formula for the genus of $P_{n,k}:$
  \begin{align*}
    g_{n,k} = n(14k-2) + 2
  \end{align*}
\end{proof}

\p{Step 2: Constructing the maps}

We now construct homeomorphisms $f_{n,k}: P_{n,k} \to P_{n,k}$ that are defined as a composition of specific Dehn twists
followed by a finite order mapping class. The key insight is that a power of this map will be a composition of
Dehn twists that satisfy the criteria to be a Penner construction and are therefore themselves pseudo-Anosov. This is how we take
advantage of the rotational symmetry of the $P_{n,k}$.

\begin{figure}[h]
    \centering
    \incfig[0.65]{CurvesOnSurface}
    \caption{The curves $\alpha_i$,  $\beta_j$, and $\gamma$ in $T_{0,0}$}
    \label{fig:curves}
\end{figure}

\begin{figure}[h]
    \centering
    \incfig[0.4]{ExtraCurves}
    \caption{The parts of curves $\beta_2$ and $\beta_7$ on $T_{0,1}$ and $T_{1,0}$}
    \label{fig:extracurves}
\end{figure}

Recall that for non-orientable surfaces, we did not initially have a well-defined notion of a positive or
negative Dehn twist. As we saw in Section \ref{sec:backgr-mapp-class}, in order to use the Penner
construction to construct pseudo-Anosov mapping classes, we need to ensure that the curves about which we twist are marked inconsistently. Construct the multi-curves $\{\alpha_1,\cdots,\alpha_8\}$ in $T_{0,0}$ as shown in Figure \ref{fig:curves}.\becca[inline]{I don't understand the curves $\beta_2,$ $\beta_3$ and $\beta_8$.  How do they close?  Actually, what's the difference between $\beta_2$ and $\beta_3$?}
\caleb[inline]{First, you are absolutely correct about $\beta_2$ and $\beta_3$, I believe that's a mislabeling on my part. I'll go and change the figure so it is properly labeled. The way they close is shown in Figure 6 below.}

Note that our
labeling of the curves in Figure \ref{fig:curves} already gives us an inconsistent marking. For any alpha curve $\alpha_i$, we let the
marking $\phi_{\alpha_i}$ be orientation preserving and for beta curves $\beta_j$ let $\phi_{\beta_j}$ be
orientation reversing. Since alpha curves only intersect beta curves (and vice versa), we have an inconsistent
marking at each point of intersection.

Let $\mathcal{B}$ be the union of all $\beta$ curves except $\beta_1$ in $T_{0,0} \cup T_{0,1} \cup T_{1,0}$
(see Figure \ref{fig:extracurves} above). Let $\rho_1(\mathcal{B})$ be the image of $\mathcal{B}$ under $\rho_1$. Define $\phi_b$
as the composition of Dehn twists along all the curves in the set
$\overline{\mathcal{B}} \coloneqq \mathcal{B} \cup \rho_1(\mathcal{B}) \cup \dots \cup
\rho_1^{n-1}(\mathcal{B})$. Since the curves in $\overline{\mathcal{B}}$ are disjoint, Dehn twists along them
commute and therefore it is not necessary to specify the order in which we compose these Dehn twists in
$\phi_b$. Let $\mathcal{R}$ be the union of all $\alpha$ curves except $\alpha_1$ in $T_{0,0}$. Define
$\overline{\mathcal{R}}$ and $\phi_r$ in the same manner as above.

Let $\alpha_1,\beta_1 \subset T_{0,0}$ be the curves in Figure \ref{fig:curves}. Let $\phi$ be the composition
of Dehn twists along all the curves $\alpha_1, \rho_1(\alpha_1), \dots, \rho_1^{n-1}(\alpha_1)$ followed by
Dehn twists along all the curves $\beta_1,\rho_1(\beta_1),\dots,\rho_1^{n-1}(\beta_1)$. Define the map $f_{n,k}$
in the following manner.
\begin{align*}
    f_{n,k} &\coloneqq \rho_2 \circ \phi \circ \phi_b \circ \phi_r
\end{align*}
It follows from the Penner construction that $(f_{n,k})^k$ is pseudo-Anosov. Hence $f_{n,k}$ itself is
pseudo-Anosov and an invariant train track $\tau_{n,k}$ for $f_{n,k}$ can be obtained from Penner's
construction that we described in Section \ref{sec:backgr-mapp-class}.

\p{Step 3: The Mapping Torus}

We have now constructed an infinite family of non-orientable surfaces and pseudo-Anosov maps, but this is not
enough. By Lemma \ref{lem:genera}, the family does not contain a surface of every genus. In fact, the family does not include surfaces of infinitely many genera. We will use our extension of the Thurston's
fibered face theory to fill in the gaps.  For each $n\in\mathbb{N}$ we will find a fibration of a mapping torus of $f_{n,k}$ that has a fiber that is homeomorphic to $\no_3$, the hyperbolic non-orientable surface of lowest possible genus.\becca{Is this right?}

Let $M_{n,k}$ be the mapping torus of $f_{n,k}$. Likewise, let $\mathcal{K}_{n,k}$ denote the fibered cone of
$H^1(M_{n,k};\mathbb{R})$ corresponding to the map $f_{n,k}$. %We will show that $M_{n,k}$ contains a closed, relatively orientable, incompressible surface homeomorphic to $\mathcal{N}_3$ that is transverse to the suspension flow direction. This will allow us to apply Theorem \ref{thm:oriented-sum} to construct new fibrations of $M_{n,k}$.

\begin{figure}[h]
    \centering
    \incfig[0.4]{GammaCurves}
    \caption{The curves $\gamma$ and $\widehat{\gamma}$ bound an a non-orientable surface of genus 1.}
    \label{fig:gammacurves}
\end{figure}

\begin{lem}
\label{lem:genus3}
Let $M_{n,k}$ be the mapping torus of $f_{n,k}$. Let $\mathcal{K}_{n,k}$ denote the fibered cone of
$H^1(M_{n,k};\mathbb{R})$ corresponding to the map $f_{n,k}$. 
There is a relatively orientable incompressible surface $F_{n,k}$ in $M_{n,k}$ that is homeomorphic to $\mathcal{N}_3$.
Moreover $F_{n,k}$ is transverse to the suspension flow direction given by $f_{n,k}$ and the Poincar\'e dual of $F_{n,k}$ is in
the closure $\overline{\mathcal{K}_{n,k}}$.
\end{lem}
\begin{proof}
  Let $\gamma \subset T_{0,0}$ be the curve shown in Figure \ref{fig:gammacurves}. Note that $\gamma$ and $\phi(\gamma)$ bound a non-orientable surface
  $\hat{F}$ of genus 1 with boundary. For convenience, we will denote $\phi(\gamma)$ by $\widehat{\gamma}$. We are going to follow the image of $\gamma$
  under iterations of the pseudo-Anosov map $f_{n,k}$.  Doing so will allow us to attach annuli to the
  boundary of $\widehat{F}$ to get a closed $\mathcal{N}_3$. Using the facts that $f_{n,k}=\rho_2\circ\phi\circ\phi_b\circ\phi_r$ and both $\phi_r$ and $\phi_b$ act trivially on $\gamma$, we have the following:
  \begin{align*}
    f_{n,k}(\gamma) &= \rho_2 \circ \phi \circ \phi_b \circ \phi_r(\gamma) \\
                    &= \rho_2 \circ \phi(\gamma) \\
                    &= \rho_2(\widehat{\gamma}) \\
    f^2_{n,k}(\gamma) &= \rho_2^2(\widehat{\gamma}) \\
                      &\vdots \\
    f^k_{n,k}(\gamma) &= \rho_2^k(\widehat{\gamma})\\
                      &= \widehat{\gamma}.
  \end{align*}
  That is, for all $1\leq i\leq k$, the curve $f_{n,k}^i(\gamma)$ is the copy of $\widehat{\gamma}$ in the copy of $T_{0,0}$ under rotation by $\rho_2^i$.  
  For $1\leq i\leq k$, let $T_i$ be an annulus in $M_{n,k}$ that connects $f_{n,k}^{i-1}(\gamma)$ to $f_{n,k}^i(\gamma)$ obtained by following the suspension
  flow of $f_{n,k}$ around $M_{n,k}$. We can now construct our embedded surface $F_{n,k}$ by taking the union of
  $T_1,T_2,\dots,T_k$ and $\hat{F}$. Since we are adding an orientable genus to a non-orientable surface of
  genus 1, we see $F_{n,k}$ is homeomorphic to $\mathcal{N}_3$.

  The resulting surface is an embedded non-orientable surface in a non-orientable $3$-manifold, so we have
  relative orientability by Proposition \ref{prop:relative-orientability}.

  The proof that $F_{n,k}$ can be isotoped to be transverse to the suspension flow is the same as the
  proof in \cite{yazdi2018pseudo}, which in turn follows the proof in \cite{leininger2013number}. Let $N(\gamma)$ be a tubular neighbrhood of $\gamma$ in $\hat{F}$, and $\eta: \hat{F} \xrightarrow{} [0,1]$ be a smooth function supported on $N(\gamma)$ with $\eta^{-1}(1) = \gamma$ and such that the derivative of $\eta$ on $\gamma$ vanishes. Denote the suspension flow of the map $f_{n,k}$ by $\phi_t: M_{n,k} \xrightarrow{} M_{n,k}$, where $t \in \mathbb{R}$, and define the map $g: \hat{F} \xrightarrow{} M_{n,k}$ as $g(x) = \phi_{k\cdot \eta(x)}(x)$. Then $g$ restricted to the interior of $\hat{F}$ is an embedding, and satisfies $g(\gamma) = \hat{\gamma}$. The image of $g: \hat{F} \xrightarrow{} M_{n,k}$ is an embedded non-orientable surface of genus three, that is isotopic to the natural embedding of $F_{n,k}$ in $M_{n,k}$, and is tranverse to the suspension flow.  
  %The proof goes through even in this setting essentially because of the local nature of the proof. 
  Therefore, its Poincar\'e dual is in $\overline{\mathcal{K}_{n,k}}$ by Theorem \ref{thm:classifying-fibrations}.
  \caleb[inline]{Do we need to say more here? SK: I think if the Yazdi paper elides the proof of transversality citing the local nature, we should be able to as well. The safest option would be to see what exactly Yazdi quotes before referring to LM13, and make sure we do the same.}
  \becca[inline]{I think we should basically quote Yazdi and/or Leininger--Margalit and attribute it as going through Yazdi and originally due to Leininger--Margalit.}
  \caleb[inline]{So I just copied the exact wording from Yazdi, changing some of the notation to match ours. Not sure if that's too direct even though we cited it.}
   
  Finally, $F_{n,k}$ is incompressible in $M_{n,k}$ because $M_{n,k}$ is hyperbolic, and $F_{n,k}$ is genus $3$, the
  lowest possible genus for a hyperbolic non-orientable surface.
\end{proof}

\p{Step 4: Bounding the Stretch Factor}

Yazdi then finds an upper bound for the log of the stretch factors of the pseudo-Anosov homeomorphisms he constructed. Similarly, we want to find an upper bound for the homeomorphisms we constructed in Step 2.  In order to to do this we use \textit{train tracks}, embedded graphs in our surface with that property that for every vertex $v$ of valence three or greater, all edges adjacent to $v$ have the same tangent vector at $v$. All pseudo-Anosov homeomorphisms come equipped with an \textit{invariant train track}, a train track whose image under the map is homotopic to itself. These invariant train tracks have an associated matrix whose Perron-Frobenius eigenvalue is the stretch factor of our pseudo-Anosov.

%recall in Section \ref{sec:backgr-mapp-class} we saw that pseudo-Anosov maps give rise to matrices whose Perron-Frobenius eigenvalue isthe corresponding stretch factor. So a way to find an upper bound of the stretch factor of the maps we have constructed is to bound the spectral radius of the associated matrices. The following lemma by 

Yazdi uses the Lemma \ref{lem:spectral} to bound the spectral radius of the associated matrices.

\begin{lem}[Lemma 2.3 of \cite{yazdi2018pseudo}]
\label{lem:spectral}
Let $A$ be a non-negative integral matrix, $\Gamma$ be the adjacency graph of $A$, and $V(\Gamma)$ the set of
vertices of $\Gamma$. For each $v \in V(\Gamma)$, define $v^+$ to be the set of vertices $u\in V(\Gamma)$ such that there
is an oriented edge from $v$ to $u$. Let $D$ and $k$ be fixed natural numbers. Assume the following conditions
hold for $\Gamma$:
\begin{enumerate}[(i)]
\item For each $v \in V(\Gamma)$ we have $\deg_{\text{out}}(v) \leq D$,
\item There is a partition $V(\Gamma) = V_1 \cup \dots \cup V_\ell$ such that for each $v \in V_i$ we have
  $v^+ \subset V_{i+1}$, for any $1 \leq i \leq \ell$ except possibly when $i = 1$ or 3 (indices are mod $\ell$),
\item For each $v \in V_1$, we have $v^+ \subset V_2 \cup V_3$,
\item For each $v \in V_3$ we have $v^+ \subset V_3 \cup V_4$, and for $u \in v^+ \cap V_3$ we have
  $u^+ \subset V_4$, and 
\item For all $3 < j \leq k$ and each $v \in V_j$, the set $v^+$ consists of a single element.
\end{enumerate}

Then the spectral radius of $A^{\ell-1}$ is at most $4D^4$.

\end{lem}
With this result in hand, we can now show that the stretch factors for our main family of examples are all
bounded above in the way we hope.

\begin{lem}\label{lem:upperbound}
  Let $\lambda_{n,k}$ be the stretch factor of $f_{n,k}$. Then there exists a universal positive constant $C'$ such that for every $n \geq 1$ and $k \geq 3$, we have the following upper bound on $\log(\lambda_{n,k})$.
  \begin{align*}
   \log(\lambda_{n,k}) \leq C'\frac{n}{g_{n,k}}
  \end{align*}
\end{lem}

\begin{proof}
 We deliberately constructed our examples so our curves are in the same general form as the ones in
 \cite{yazdi2018pseudo} such that all intersections between the curves happen inside the building block $T$, except for the intersections between building blocks given by the beta curves $\beta_3$ and $\beta_8$. %Keeping with this form makes our set of curves also satisfy Lemma \ref{lem:spectral}, but we will still explicitly prove this.

  We define the following multi-curves:
\begin{align*}
  \mathcal{A} &\coloneqq \mathcal{B} \cup \mathcal{R} \cup \{\alpha_1,\beta_1\} =\bigcup_{i=1}^8(\alpha_i\cup\beta_i)\\
  \smallskip
  \overline{\mathcal{A}} &\coloneqq \mathcal{A} \cup \rho_1(\mathcal{A}) \cup \dots \cup \rho_1^{n-1}(\mathcal{A}) \\
  \widehat{\mathcal{A}} &\coloneqq \overline{\mathcal{A}} \cup \rho_2\left(\overline{\mathcal{A}}\right) \cup \dots \cup \rho_2^{k-1}\left(\overline{\mathcal{A}}\right).
\end{align*}
Thus, $\widehat{\mathcal{A}}$ is union of the curves about which we Dehn twist to obtain $f_{n,k}$.

Because $f_{n,k}$ is constructed via the Penner construction, it is pseudo-Anosov and therefore has a corresponding invariant train track $\tau$.  For each $\gamma\subset\widehat{\mathcal{A}}$ there is an associated transverse measure $\mu_\gamma$ on $\tau$ that assigns 1 to all edges in $\gamma$ and 0 to everything else. Let $V_\tau$ be the cone of transverse measures on $\tau$.  Let $H$ be the subspace of $V_\tau$ spanned by the transverse measures $$\{\mu_\gamma | \gamma\subset\widehat{\mathcal{A}}\}.$$
The measures $\mu_\gamma$ form a basis for $H$. \caleb[]{Is it necessary to describe in our specific case why they form a basis?} The subspace $H$ is invariant under the action of $\varphi$ on $V_\tau$, thus $\varphi$ has a linear action on $H$. Let $M$
be the matrix representing this linear action on $H$.  Let $\Gamma$ be the adjacency graph for $M$. Work of Penner \cite{penner1988construction} tells us that the stretch factor of $f_{n,k}$, $\lambda$, is the Perron-Frobenius eigenvalue of $M$.
%Let $A$ be the matrix that $f_{n,k}$ on the subspace of the cone of transverse measures that is spanned by the measures assigning $1$ to single curves in $\hat{\mathcal{A}}$ and 0 to everything else. Let $A$ be said matrix and $\Gamma$ the adjacency graph of $A$. 

To bound the spectral radius of $M$, we need to show that $\Gamma$ satisfies
the criteria of Lemma \ref{lem:spectral}.

%We can now check the conditions of Lemma \ref{lem:spectral}, based on the combinatorics of the curves on our surface:

\begin{enumerate}[(i)]
\item There exists a constant $D'$, independent of $n$ and
  $k$, such that for every (connected) curve $\gamma \in \widehat{\mathcal{A}}$, the geometric
  intersection number between $\gamma$ and $\overline{\mathcal{A}}$ is at most $D'$. %Recall from Section \ref{sec:backgr-mapp-class} that we refer to the linear action of $f_{n,k}$ on the subspace of the cone of transverse measures on our invariant train track corresponding to connected curves in $\hat{\mathcal{A}}$ as $M$.
  Recall that $f_{n,k}=\rho_2\circ\phi\circ\phi_b\circ\phi_r$.  Let $M_1,M_2,M_3$ and $M_4$ be the matrices describing the linear action of $\phi_r,\phi_b,\phi$ and $\rho_2$ on $H$, respectively. The matrix
  $M$ can then be written as a product:
  \begin{align*}
    A = M_4M_3M_2M_1.
  \end{align*}
  For a (connected)
  curve $\delta \in \hat{\mathcal{A}}$, the $L^1$-norm of $A(\mu_\delta)$ is bounded above by the geometric intersection of
  $f_{n,k}(\delta)$ with the curves in $\overline{\mathcal{A}}$, thus each of $M_1$, $M_2$ and $M_3$ will change the norm by
  a factor of at most $(1 + D')$. Since $\rho_2$ will not change intersection numbers, $M_4$ will preserve the
  $L^1$-norm. If we let $D = (1 + D')^3$, then the outward degree of each vertex in $\Gamma$ is at most $D$.
\item  Next we partition the vertices of $\Gamma$.  Recall
$$\widehat{\mathcal{A}} = \bigcup_{i=1}^k \rho_2^{i-2}(\overline{\mathcal{A}}).$$ Then define $V_i$ for
$1 \leq i \leq k$ as the vertices of $\Gamma$ corresponding to elements in
$\rho_2^{i-2}(\overline{\mathcal{A}})$.  Suppose that $v \in V_i$, $i \neq 1,3$, is a vertex
  that corresponds to $\mu_\gamma$ for a curve $\gamma \in \hat{\mathcal{A}}$. Then $\gamma$ must be a curve
  in $\rho_2^{i-2}(\overline{\mathcal{A}})$, for $i \neq 1,3$.  The action of $\phi\circ\phi_b \circ \phi_r$ on $\widehat{\mathcal{A}}$ will preserve the set $\rho_2^{i-2}(\overline{\mathcal{A}})$ for each $i$.  %Therefore the action of $\phi\circ\phi_b\circ\phi_r$ on $H$ will send $\mu_\gamma$ to a sum of $\mu_\eta$ where $\eta$ corresponds to elements of $V_i$. 
  Then $\rho_2$ will rotate the curve $\phi\circ \phi_b\circ\phi_r(\gamma)$ to $\rho_2^{i-1}(\overline{\mathcal{A}})$. That is: $f_{n,k}=\rho_2\circ\phi\circ\phi_b\circ\phi_r$ maps $\mu_\gamma\in H$ to $$\sum_{\zeta\in \mathcal{Z}}\mu_\zeta$$ where $\mathcal{Z}$ is a subset of $\rho_2^{i-1}(\overline{\mathcal{A}})$.  Therefore $f_{n,k}$ maps $v$ to a subset of $V_{i+1}$.
\item We need to see which vertices in $v \in V_1$ have $v^+ \not\subset V_2$. These are the vertices of $V_1$ corresponding to curves that $\phi\circ\phi_b\circ\phi_r$ maps to curves that do not correspond to vertices in $V_1$.  Recall that because $\rho_1$ and $\rho_2$ commute, each vertex of $v\in V_1$ corresponds to a curve in: 
    $$\rho_2^{-1}(\overline{\mathcal{A}})=\rho_2^{-1}(\mathcal{A})\cup\rho_1(\rho_2^{-1}(\mathcal{A}))\cup\cdots\cup\rho_1^{n-1}(\rho_2^{-1}(\mathcal{A})).$$  The elements of $v^+$ that are not in $V_2$ correspond to the images of curves in $\rho_2^{-1}(\overline{\mathcal{A}})$ under $f_{n,k}$ that are not in $\overline{\mathcal{A}}$.  
As in Yazdi, the only curves in $\rho_2^{-1}(\overline{\mathcal{A}})$ that intersect curves in $\overline{\mathcal{A}}$ are those in the set: 
\begin{align*}
    \mathcal{X} = \{ \rho_1^i(\rho_2^{-1}(\beta_7))\,\mid\,0\leq i\leq n-1\}.
  \end{align*}
  Therefore $\phi\circ\phi_b\circ\phi_r$ maps curves in $\mathcal{X}$ to curves in $\rho_2^{-1}(\overline{\mathcal{A}})\cup\overline{\mathcal{A}}$.  Then $f_{n,k}=\rho_2\circ\phi\circ\phi_b\circ\phi_r$ maps curves in $\mathcal{X}$ to curves in $\overline{\mathcal{A}}\cup\rho_2(\overline{\mathcal{A}}).$

  Let $X=\{\mu_\eta|\eta\in\mathcal{X}\}$
%  \becca[inline]{This used to say $\rho_1(\rho_2^{-1}(\beta_8)$.  But I couldn't find $\beta_8$ and $\beta_7$ was the one that agreed with Yazdi.  Also the $i$ was missing.  But Yazdi doesn't have a $-1$ as an exponent for $\rho_2$.}
%  \caleb[inline]{I just checked and he does have a -1 as an exponent for this part, but not for the part below. This is because vertices in $V_1$ correspond to curves in $\rho_2^{-1}(\overline{\mathcal{A}})$. At one point I had just convinced myself this was true and didn't write it down, I need to sit here for a few minutes to see if I can remember.}
%  \caleb[inline]{Okay yeah I got it. So the reason is just that under the action of $f_{n,k}$ the curves specified above will get moved to $\rho_1^i(\beta_7)$ and all of these curves will intersect $\rho_1^j(\rho_2(\alpha_8))$ when $i = j$, but these curves are in $V_3$.}
%  \becca[inline]{But don't we have to watch $\beta_2$?  $\beta_7$ actually seems to end up in $V_2$ (it seems to be disjoint from all twists applied since $\rho_2^{-1}(\beta_7)$ lives in $T_{0,k-1}\cup T_{1,k-1}$ and all twists are in $T_{i,0}$)}
%  \caleb[inline]{Oh but actually, since we are applying all of $f_{n,k}$, won't $\rho_1^i(\rho_2^{-1}(\beta_7))$ first get twisted around curves in $V_2$ by $\phi$ and then $\rho_2$ will send that image to something that lies in $V_2$ and $V_3$? We don't have to worry about $\beta_2$ because $\rho_1^i(\rho_2^{-1}(\beta_2))$ only intersects curves also in $V_1$. }

  Therefore the vertices of $v\in V_1$ corresponding to curves in $X$ will have
  $v^+ \subset V_2 \cup V_3.$  Moreover, $f_{n,k}$ maps the curves $\rho_2^{-1}(\overline{\mathcal{A}})\setminus X$ to curves in $\overline{\mathcal{A}}$.  Thus for any vertex $v\in V_1$ that does not correspond to an element of $X$, the set $v^+$ is contained in $V_2$.
\item  Similarly, we need to see which vertices in $v \in V_3$ have $v^+ \not\subset V_4$.  The vertices of $V_3$ correspond to elements of the form:
$$\rho_2(\overline{\mathcal{A}})=\rho_2(\mathcal{A})\cup\rho_1(\rho_2(\mathcal{A}))\cup\cdots\cup\rho_1^{n-1}(\rho_2(\mathcal{A})).$$
 The set of curves 
 $$\mathcal{Y}=\{\rho_1^i(\rho_2(\alpha_8))\,\mid\,0\leq i\leq n-1\}$$ are precisely the curves corresponding to vertices of $V_3$ that intersect with the curves in $\overline{\mathcal{A}}$, namely the curves
  $\rho_1^i(\beta_8)$.

Therefore elements $v \in V_3$ such that $v^+ \not\subset V_4$ are those that correspond to the elements of
  the following set:
  \begin{align*}
    Y = \{\mu_\eta \mid \exists \,\eta\in\mathcal{Y}\}.
  \end{align*}
  Moreover, for any element $v \in V_3$ corresponding to $Y$ and any
  $u \in v^+ \cap V_3$, the vertex $u$ does not correspond to an element of $Y$ and hence $u^+ \subset V_4$.
\item All the curves corresponding to an element of $V_j$, $3 < j \leq k$ are disjoint from all the curves in
  $\overline{\mathcal{A}}$. Thus, $f_{n,k}$ just acts by rotation.
\end{enumerate}

Set $\lambda = \lambda_{n,k}$.  By Lemma \ref{lem:spectral}, we have:
\begin{gather*}
    \lambda^{k-1} = \rho(M)^{k-1} = \rho(M^{k-1}) \leq 4D^4. 
\end{gather*}
Then the logarithm of $\lambda$ satisfies:
$$\log(\lambda^{k-1})=(k-1)\cdot \log(\lambda) \leq \log(4D^4).$$
Then for $k\geq 2$
    $$\frac{k}{2}\log(\lambda) \leq (k-1)\log(\lambda) \leq \log(4D^4).$$
On the other hand, we know $g_{n,k} = (14k - 2)n + 2 \leq 14kn$. Therefore
\begin{align*}
    \log(\lambda) \leq 2\log(4D^4)\cdot\frac{1}{k} \leq 2\log(4D^4)\cdot \frac{14n}{g_{n,k}}.
\end{align*}
Let $C' \coloneqq 28\log(4D^4)$ to complete the result.
\end{proof}

\p{Step 5: Filling in the Gaps}
Recall that the family of surfaces $P_{n,k}$ that we have constructed have genera in the set $\{(14k-2)n+2\}$.
We now want to construct pseudo-Anosov maps with small
stretch factors on surfaces of the genera not in the set $\{(14k-2)n+2\}$. To do this we use the mapping torus $M_{n,k}= (P_{n,k},f_{n,k})$. Recall from Lemma \ref{lem:genus3} that there exists a relatively incompressible surface $F_{n,k}$ in $M_{n,k}$ that is homeomorphic to $\no_3$.  Let $P_{n,k}^r$ be the oriented sum of $P_{n,k}$ and
$r$ times $F_{n,k}$, as defined in Theorem $\ref{thm:oriented-sum}$.  The surfaces $P_{n,k}^r$ will be surfaces of the remaining genera.

\begin{lem}
  The surfaces $P^r_{n,k}$ have genus $g^r_{n,k} = g_{n,k} + r$. In particular, as $r$ varies between
  $0$ and $14n$, the genera of $P^r_{n,k}$ cover the range between $g_{n,k}$ and $g_{n,k+1}$. Moreover,
  $P^r_{n,k}$ is isotopic to a fiber of a fibration of $M_{n,k}$ with pseudo-Anosov monodromy that fixes $2n$
  of the singularities of its invariant foliation.
\end{lem}

\begin{proof}
  The Euler characteristic of an oriented sum is the sum of the Euler characteristics of the summands:
  \begin{align*}
    \chi(P^r_{n,k}) &= \chi(P_{n,k}) + r\cdot\chi(F_{n,k}) \\
                    &= (-2g_{n,k} + 2)-2r \\
                    &= -2(g_{n,k} + r) + 2.p
  \end{align*}
  \becca[inline]{Do we need to work with the unpunctured surface to get this identity?  We can puncture later...}
  \caleb[inline]{Yes that is an assumption we should clarify earlier. The $P^r_{n,k}$ are not supposed to be punctured yet. We puncture them at the bottom of this page.}
  This proves the identity for the genus of $P^r_{n,k}.$

  By Lemma \ref{lem:genus3} we know that $F_{n,k}$ is incompressible and transverse to the suspension flow of given by $f_{n,k}$.  Therefore Theorem \ref{thm:oriented-sum} gives us that $P^r_{n,k}$ is isotopic to a fiber of a fibration of
  $M_{n,k}$. Let $f^r_{n,k}$ be the first return map of the new fibration of $M_{n,k}$ over $P^r_{n,k}$.  Since
  $f_{n,k}$ is a pseudo-Anosov monodromy of $M_{n,k}$, we have that $M_{n,k}$ is hyperbolic.  Therefore all monodromies of $M_{n,k}$ are pseudo-Anosov and in particular $f^r_{n,k}$ is a pseudo-Anosov map.

  As in the proof of Lemma 3.5 of \cite{yazdi2018pseudo}, the singularities of the stable foliation of
  $f_{n,k}$ that are fixed are the $2n$ intersection points of the axis of $\rho_1$ with
  $P_{n,k}$. By Lemma \ref{lem:genus3}, the surface $F_{n,k}$ can be isotoped to be transverse to
  the suspension flow and disjoint from the orbit of the $2n$ singularities of $f_{n,k}$.  Hence the monodromy
  $f^r_{n,k}$ still fixes the corresponding $2n$ singularities on $P^r_{n,k}$.
\end{proof}

We can now prove the non-orientable version of Lemma 3.6 of \cite{yazdi2018pseudo}.
\begin{lem}
\label{lem:bound}
Let $\lambda_{n,k}^r$ be the stretch factor of $f_{n,k}^r$. Then there exists a constant $C > 0$ such that for every $n \geq 1$, $k \geq 3$, and $0 \leq r \leq 14n$ we have the following upper bound on $\log(\lambda_{n,k}^r)$.
\begin{align*}
  \log(\lambda^r_{n,k}) \leq C\frac{n}{g^r_{n,k}}
\end{align*}
\end{lem}
\begin{proof}
  Let $\mathcal{K} = \mathcal{K}_{n,k}$ be the fibered cone corresponding to $f_{n,k}$ and $h: \mathcal{K} \xrightarrow[]{} \mathbb{R}$
  the function described in Theorem \ref{thm:fm}. Note that $g_{n,k}\geq 42$, therefore we have the following bounds on $g_{n,k}^r$:
  \begin{align*}
    g^r_{n,k} &= g_{n,k} + r \\
              &\leq g_{n,k} + 14n \\
              &< 2g_{n,k}.
  \end{align*}
  Let $\omega$ be the Poincar\'e dual of $P^r_{n,k}$ and $\alpha$ the Poincar\'e dual of $P_{n,k}$.  Then the following string of inequalities holds, the first inequality is by the convexity of $h$, the second inequality is the bound in Lemma \ref{lem:upperbound}, and the third inequality is from the bound on $g^r_{n,k}$ above:
  \begin{align*}
    h([\omega]) &< h([\alpha]) \\
                &\leq C'\frac{n}{g_{n,k}} \\
                &\leq 2C'\frac{n}{g^r_{n,k}}. 
  \end{align*}
\end{proof}

So  our surfaces $P^r_{n,k}$ are isotopic to fibers of fibrations of $M_{n,k}$ with
pseudo-Anosov monodromies with bounded stretch factors.

Recall that all of the $P^r_{n,k}$ will have $2n$ singularities, and we can puncture $n$ of them to view $f^r_{n,k}$ as a map on a non-orientable surface of genus $g^r_{n,k}$ with $n$ punctures. Also note from above we
know $g^r_{n,k}$ covers all natural numbers between $g_{n,k}$ and $g_{n,k+1}$, thus this set of genera
for all $r$ covers all natural numbers larger than $g_{n,3} = 40n + 2$. 

We can now give a proof of Theorem \ref{thm:stretch1}.

%\begin{manualtheorem}{\ref{thm:stretch1}}
%For any fixed $n \in \mathbb{N}$, there are positive constants $B'_1 = B'_1(n)$ and $B'_2 = B'_2(n)$ such that for any $g \geq 2$, the stretch factor satisfies the following inequalities.
 % \begin{align*}
  %  \frac{B'_1}{g} \leq l'_{g,n} \leq \frac{B'_2}{g}
  %\end{align*}
%\end{manualtheorem}
\begin{proof}[Proof of Theorem \ref{thm:stretch1}]
As above, the lower bound follows easily from the lower bound in the orientable setting.  % as demonstrated in the discussion following the initial statement of Theorem \ref{thm:stretch1}. For ease of reading, we replicate that here.  
  Let $f$ be the pseudo-Anosov map with the minimal stretch factor on $\no_{g,n}$. Then, by Proposition \ref{prop:2},
  this map lifts to a map $\wt{f}$ on $\os_{g-1, 2n}$ (possibly after squaring). Furthermore, $\wt{f}$ has the
  same stretch factor as $f$. The former is bounded below by $\frac{B}{g}$, and thus the stretch factor of $f$
  is bounded below as well.
  
  To find the upper bound, let $C'=\frac{C}{2}$ be the value given in Lemma \ref{lem:bound}. Let $B'_2(n)$ be the
  following quantity:
  \begin{align*}
    B'_2(n) = \max\{2C'n, l'_{1,n}, 2l'_{2,n}, \dots, (40n + 1)l'_{40n+1,n}\}
  \end{align*}
  By Lemma \ref{lem:bound}, $B'_2(n)$ is an upper bound for $g\cdot l'_{g,n}$.
\end{proof}
