\section{Background}

The following lemma from Yazdi will be crucial to bounding the stretch factors on the pseudo-Anosov maps we construct.

\begin{lem}[Yazdi]
Let $A$ be a non-negative integral matrix, $\Gamma$ be the adjacency graph of $A$, and $V(\Gamma)$ the set of vertices of $\Gamma$. For each $v \in V(\Gamma)$, define $v^+$ to be the set of vertices $u$ such that there is an oriented edge from $v$ to $u$. Let $D$ and $k$ be fixed natural numbers. Assume the following conditions hold for $\Gamma$: \begin{enumerate}
    \item For each $v \in V(\Gamma)$ we have $\deg_{\text{out}}(v) \leq D$.
    \item There is a partition $V(\Gamma) = V_1 \cup \dots \cup V_k$ such that for each $v \in V_i$ we have $v^+ \subset V_{i+1}$, for any $1 \leq i \leq k$ except possibly when $i = 1$ or 3 (indices are mod $k$).
    \item For each $v \in V_1$, we have $v^+ \subset V_2 \cup V_3$.
    \item For each $v \in V_3$ we have $v^+ \subset V_3 \cup V_4$, and for $u \in V^+ \cap V_3$ we have $u^+ \subset V_4$.
    \item For all $3 < j \leq k$ and each $v \in V_j$, the set $v^+$ consists of a single element.
\end{enumerate}
\end{lem}

In order to construct the pseudo-Anosov maps that we will later use in this paper, we need to use the Penner construction for non-orientable surfaces. To see a more detailed description of this, see \cite{Strenner_2017}. The typical Penner construction takes a pair of filling multicurves $A = \{a_1,\dots,a_n\}$ and $B = \{b_1,\dots,b_m\}$ and states that a composition of Dehn twists $T_{a_i}$ and $T_{b_i}^{-1}$ that uses all curves in the multicurves at least once will be pseudo-Anosov. The issue for non-orientable surfaces is that in defining Dehn twists, we don't have a well-defined notion of a left or right Dehn twist (what direction we are choosing to be the Dehn twist and which is its inverse). The way we get around this for non-orientable surfaces is by having a collection of filling two-sided curves that are \textit{marked inconsistently}. Each two-sided curve on a non-orientable surface has a neighborhood homeomorphic to an anulus. We can choose orientations on all these annuli and if at the intersections of our curves these orientations disagree, then we say our curves are marked inconsistently. 

\textbf{Necessary Theorems for Last Step of Construction:}

\textcolor{red}{The question is, is it easier to use the original theorems and discuss our extended versions or to state our extended versions and use those when proving Yazdi for non-orientable}

For the theorems below, let $M$ be an orientable closed, fibered 3-manifold 
