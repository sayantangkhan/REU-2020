\section{Thurston norm for non-orientable 3-manifolds}
\label{sec:thur-norm-non-orientable}
Thurston defined a norm on $H_2(M;\RR)$ where $M$ is an orientable 3-manifold \cite{thurston1986norm}, and this norm is now called the Thurston norm.  In his manuscript, Thurston wrote ``Most of this paper works also for non-orientable manifolds but for simplicity we only deal with the orientable case.''  However, the details are not explained in Thurston's work or in subsequent literature.  Therefore the goal of this section is to write the details of the Thurston norm for non-orientable 3-manifolds.
We recall the Thurston norm for orientable manifolds in Section \ref{sec:backgr-thurst-norm}.
In Section \ref{sec:thurst-norm-cohom} we describe the challenge of defining the Thurston norm on $H_2(M;\RR)$ if $M$ is non-orientable and present the solution of defining the Thurston norm instead on $H^1(M;\RR)$.
However, Poincar\'e duality does not hold for non-orientable manifolds.
We therefore define a condition -- {\it relative orientability} -- on a pair consisting of a manifold and an embedded surface.
A surface that is relatively orientable in a non-orientable 3-manifold $M$ will have a corresponding cohomology class in $H^1(M;\ZZ)$, giving a version of Poincar\'e duality for non-orientable 3-manifolds as stated in Theorem~\ref{thm:strong-duality}.
Finally in Section \ref{sec:oriented-sums}, we define the oriented sum for relatively oriented embedded surfaces in non-orientable manifolds.

\subsection{Thurston norm and mapping tori}
\label{sec:backgr-thurst-norm}
In this section we recall the Thurston norm for orientable surfaces and how it detects when a 3-manifold fibers over a circle.

\paragraph{Mapping tori} Let $S$ be a surface, and let $\varphi: S \to S$ be a homeomorphism.  A {\it mapping torus} of $S$ by $\varphi$ is the $3$-manifold $M_\varphi$ given by the identification:
\begin{align*}
  M_\varphi \coloneqq \frac{S \times [0,1]}{(x,1) \sim (\varphi(x), 0)}.
\end{align*}

A mapping torus is \emph{fibration over $S^1$}, denoted $S\rightarrow M_\varphi\rightarrow S^1$.
A fibration defines a flow on $M$, called the \emph{suspension flow}, where for any $x_0\in S$ and $t_0\in S^1$ the pair $(x_0,t_0)$ is sent to $(x_0,t_0+t)$.
The fiber of a fibration is the preimage of any point $\theta \in S^1$ under the projection map from $M_{\varphi} \to S^1$.
If we do not specify $\theta$, the fiber as a subset of $M_\varphi$ is only well defined up to isotopy.   The homology class of the fiber in $H_2(M_{\varphi}; \RR)$ is well-defined.

A natural inverse question is to determine when a 3-manifold fibers over a circle, and the possible fibers.  To this end, Thurston established a correspondence between second homology of 3-manifolds and surfaces embedded in 3-manifolds.

\paragraph{Complexity of an embedded surface} Let $M$ be a compact orientable closed $3$-manifold.
Let $S$ be a connected surface embedded in $M$.  The complexity of $S$ is $\chi_-(S) = \max\left\{-\chi(S),0\right\}$.
If the surface $S$ has multiple components $S_1, \ldots, S_m$ then $\chi_-(S) = \displaystyle\sum_{i=1}^m\chi_-(S_i)$.
The elements in $H_2(M ; \ZZ)$ can be represented by embedded surfaces inside of $M$ \cite[Lemma 1]{thurston1986norm}.


\paragraph{Thurston norm} Let $a$ be a homology class in $H_2(M; \ZZ)$.  Define the integer valued norm $x:H_2(M;\ZZ)\rightarrow \ZZ$ as the following:
\begin{align*}
  x(a) = \min\{\chi_-(S) \mid [S] = a \text{ and $S$ is compact, properly embedded and oriented}\}.
\end{align*}

We then linearly extend $x$ to $H_2(M;\mathbb{Q})$.  The \emph{Thurston norm} is the unique continuous $\RR$ valued function that is an extension of $x$ to $H_2(M;\RR)$.
The unit ball for the Thurston norm is a convex polyhedron in $H_2(M;\RR)$.

The following remarkable theorem of Thurston \cite{thurston1986norm} determines all possible fibrations of an oriented 3-manifold over the circle.
We use the restatement of Yazdi \cite{yazdi2018pseudo}.
\begin{thm}[Thurston]
  \label{thm:Thur1}
  Let $M$ be an orientable 3-manifold.  Let $\mathcal{F}$ be the set of homology classes in $H_2(M; \RR)$ that are representable by fibers of fibrations of $M$ over the circle.
\begin{enumerate}[(i)]
\item Elements of $\mathcal{F}$ are in one-to-one correspondence with (non-zero) lattice points inside some union of cones over open faces of the unit ball in the Thurston norm.
\item If a surface $F$ is transverse to the suspension flow associated to some fibration of
  $M \xrightarrow[]{} S^1$ then $[F]$ lies in the closure of the corresponding cone in $H_2(M;\RR)$.
\end{enumerate}
\end{thm}
The class $[F]$ has orientation such that the positive flow direction is pointing outwards relative to the surface.
An open face of the unit ball is said to be a \emph{fibered face} if the cone over the face contains the fibers of a fibration.

The goal for the rest of this section is to prove a version of Theorem \ref{thm:Thur1} for compact non-orientable $3$-manifolds.
Most of the work in the proof will involve reducing the version for non-orientable $3$-manifolds to the orientable version by passing to the double cover.

\subsection{Thurston norm on cohomology of non-orientable mapping tori}
\label{sec:thurst-norm-cohom}

Let $\no$ be a compact non-orientable surface.
A na\"ive first attempt at defining the Thurston norm would be to define it on the $H_2(\no;\RR)$, like in the orientable case.
However, if the norm is defined on $H_2(\no;\RR)$, the non-orientable version of Theorem \ref{thm:Thur1} will not be true.
Let $\varphi: \no \to \no$ be a homeomorphism and let $N_{\varphi}$ be associated mapping torus.
Clearly, $N_{\varphi}$ fibers over $S^1$, and $\no$ is the fiber of this fibration.  However, the homology class associated to $\no$ is the zero homology class, since the top-dimensional homology of non-orientable compact surfaces is $0$-dimensional.

Our workaround for this problem will be to define a norm on the first cohomology $H^1(N_{\varphi})$ rather than the second homology $H_2(N_{\varphi})$.
By Poincar\'e duality they are isomorphic for orientable 3-manifolds, but that is not true for non-orientable $3$-manifolds.

\paragraph{Poincar\'e Duality}
To see why Poincar\'e duality fails for non-orientable 3-manifolds, we will work through the construction of the isomorphism between first cohomology and second homology for orientable 3-manifolds.  Let $M$ be a 3-manifold.
To define the Poincar\'e dual of $H^1(M;\ZZ)$, we first define a homotopy class of maps $M\rightarrow S^1$.  Then we construct an element of $H_2(M;\ZZ)$.
Let $\alpha$ be a 1-form on $M$ and $[\alpha]$ its class in $H^1(M; \ZZ)$.  Fix a basepoint $y_0\in M$.
The associated map $f_{\alpha}:M\rightarrow S^1$ is given by:
\begin{align}\label{form:map}
  f_{\alpha}(y) \coloneqq  \int_{y_0}^y \alpha \mod \ZZ.
\end{align}
The choice of basepoint does not affect the homotopy class of $f_\alpha$
(see Section 5.1 of \cite{calegari2007foliations} for the details).

Now let $\theta \in S^1$ be a regular value so that $S = f^{-1}_{\alpha}(\theta)$ is a surface.
To construct the associated element of $H_2(M;\ZZ)$, we choose an orientation on $S$ by assigning positive values of $\alpha$ to the outward pointing normal vectors on $S$.
Then $S$ inherits an orientation from the orientation on $M$, and we have defined a
fundamental class $[S]\in H_2(M;\ZZ)$.
We claim that $[S]$ is the Poincar\'e dual to $\alpha$.
\begin{lem}
  \label{lem:poincare-duality}
  Let $\theta$ and $\theta'$ be two regular values of the function $f_{\alpha}$ and let $S=f_\alpha^{-1}(\theta)$ and $S'=f_\alpha^{-1}(\theta')$.
  Then for any closed $2$-form $\omega$ on $M$, the following identities hold:
  \begin{enumerate}[(i)]
  \item $\displaystyle
    \int_{S} \omega = \displaystyle\int_{S'} \omega$ and
 \item $\displaystyle
    \int_S \omega = \displaystyle\int_M \alpha \wedge \omega.$
  \end{enumerate}
  In particular, the homology class of $S$ is Poincar\'e dual to $\alpha$.
\end{lem}
\begin{proof}
  To see (i), observe that $S$ and $S'$ are homologous, i.e. $f^{-1}_{\alpha}([\theta, \theta'])$ is a singular $3$-chain that has $S$ and $S'$ as boundaries.
  From Stokes' theorem, we have the following:
  \begin{align*}
    \int_{S - S'} \omega &= \int_{f_{\alpha}^{-1}([\theta, \theta'])} d\omega
                         = 0.
  \end{align*}

  To prove (ii), observe that because $\alpha$ is the pullback of $d\xi$ along the map $f_{\alpha}$ we can write the right hand side:
  \begin{align*}
    \int_M \alpha \wedge \omega &= \int_{S^1} \left(   \int_{f_{\alpha}^{-1}(\xi)} \omega \right) d\xi.
  \end{align*}
  By Sard's theorem, almost every $\xi \in [0,1]$ is a regular value.  Therefore the right hand side is well-defined.
  By (i), the inner integral is a constant function, as we vary over the $\xi$ which are regular values of $f_{\alpha}$.
  Then the integral of $d\xi$ over $S^1$ is $1$, giving us the identity we want:
  \begin{align*}
    \int_M \alpha \wedge \omega = \int_S \omega.
  \end{align*}
\end{proof}
What we have here is an explicit formula for the Poincar\'e duality map from $H^1(M; \RR)$ to $H_2(M; \RR)$.
For orientable $3$-manifolds, this is an isomorphism.%, and more specifically the following theorem is true.
\begin{thm}[Poincar\'e duality for orientable $3$-manifolds]
  \label{thm:orientable-poincare-duality}
  Let $M$ be an orientable $3$-manifold, and let $S$ be an orientable embedded surface. Then there exists a $1$-form
  $\alpha$ and a regular value $\theta \in S^1$ such that $S$ and $f_{\alpha}^{-1}(\theta)$ are homologous surfaces.
\end{thm}

Let $N$ be a non-orientable 3-manifold.  The map above from $H^1(N;\RR)$ to $H_2(N;\RR)$ is still well-defined.
However the map from $H^1(N; \ZZ)$ to $H_2(N; \ZZ)$ has a nontrivial kernel.  For example, when $N_{\varphi}$ is the mapping torus of a non-orientable surface $\no$, as above, the fiber is trivial in $H_2(N;\ZZ)$.

\paragraph{Non-orientable manifolds}
Let $N$ be a non-orientable $3$-manifold.  Let $\wt{N}$ and the covering map $p:\wt{N}\rightarrow N$ be the orientation double covering space of $N$.
Let $\iota$ be the orientation reversing deck transformation
of $\wt{N}$.
Let $N=N_\varphi$ be the mapping torus of the non-orientable surface $\no$ by a homeomorphism $\varphi: \no \to \no$. Then $\wt{N}$ is the mapping torus of $(\os, \wt{\varphi})$, where $\os$ is the orientation double cover of $\no$, and $\wt{\varphi}$ is the orientation preserving lift of $\varphi$.

\paragraph{Defining the Thurston norm on cohomology} In order to define the Thurston norm on $H^1(N;\ZZ)$, we first need to relate $H^1(N; \RR)$ and $H^1(\wt{N}; \RR)$.
We do so by pulling back $H^1(N;\RR)$ to $H^1(\wt{N};\RR)$ via $p$. We also state the following lemma without proof (the proof is elementary).

\begin{lem}
  \label{lem:injective}
  The pullback $p^{\ast}:H^1(N;\RR)\rightarrow H^1(\wt{N};\RR)$ maps $H^1(N; \RR)$ bijectively to the $\iota^{\ast}$-invariant subspace of   $H^1(\wt{N}; \RR)$.
\end{lem}

Next we use Lemma \ref{lem:injective} to define the Thurston norm on $H^1(N; \RR)$.

\paragraph{Thurston norm for non-orientable $3$-manifolds}
  Let $\alpha \in H^1(N;\RR)$ and let $\wt{x}$ be the Thurston norm on $H^1(\wt{N};\RR)\cong H_2(\wt{N};\RR)$.
  The \emph{Thurston norm on $H^1(N; \RR)$}, is the norm $x: H^1(N;\RR) \rightarrow \RR$ defined:
  \begin{align*}
    x(\alpha) \coloneqq \wt{x}(p^{\ast}\alpha).
  \end{align*}

Note that defining the Thurston norm on $H^1(N; \RR)$ rather than $H_2(N; \RR)$ is not quite satisfactory.
In particular, fibers of fibrations are embedded surfaces in $N$.  In the orientable case, the embedded surfaces define the Thurston norm.
In Section \ref{sec:weak-inverse-poinc}, we develop a (weak) version of Poincar\'e duality for non-orientable 3-manifolds.

\subsection{Weak inverse to the Poincar\'e duality map}
\label{sec:weak-inverse-poinc}

We state and prove a weak version of Poincar\'e duality for {\it relative oriented} (non-orientable) surfaces embedded in 3-manifolds as Theorem \ref{thm:strong-duality}.

\paragraph{Relative oriented surfaces}
  Let $M$ be a $3$-manifold, and $S$ an embedded surface in $M$.
  The surface $S$ is said to be \emph{relatively oriented with respect to $M$} if there is a nowhere vanishing vector field on $S$ that is transverse to the tangent plane of $S$.
  Two such vector fields are said to induce the same orientation if they induce the same local orientation after choosing a local frame for $S$.
  A surface $S$ is \emph{relatively oriented} in $M$ if both $S$ and the choice of positive normal vector field are specified.

If $S$ and $M$ are orientable, then $S$ is relatively oriented with respect to $M$.
But even if $M$ is non-orientable, a non-orientable embedded surface $S$ may be relatively oriented in $M$.  In particular, we have the following Lemma.
\begin{lem}
  \label{lem:fibers-relatively-oriented}
  Let $\no$ be the fiber of a fibration $f: N \to S^1$.
  Then $\no$ is relatively oriented in $N$.
\end{lem}
\begin{proof}
Consider a non-zero tangent vector $v$ pointing in the positive direction at a point $\theta \in S^1$.
One can pull back the tangent vector $v$ to a nowhere vanishing vector field over $f^{-1}(\theta) = \no$ because $f$ is a fibration, i.e. a submersion.
The pulled back vector field defines a relative orientation for $\no$ in $N$.
\end{proof}

\paragraph{Orientable manifolds} Now let $M$ be an orientable 3-manifold, and let $S$ be an orientable embedded surface.  If $S$ is relatively oriented with respect to $M$, then a choice of orientation on $S$ determines an orientation on $M$ and vice versa.
We also need to define the notion of \emph{incompressible surfaces} to state our version of Poincar\'e duality.

\paragraph{Incompressible surfaces}
  Let $S$ be a surface with positive genus embedded in a $3$-manifold $M$.
  The surface $S$ is said to be \emph{incompressible} if there does not exist an embedded disc $D$ in $M$ such that $D$ intersects $S$ transversely and $D \cap S = \partial D$.
  The following result of Thurston demonstrates the link between incompressible surfaces and fibers of fibrations.

\begin{thm}[Theorem 4 of \cite{thurston1986norm}]
  \label{thm:Thur2}
Let $M$ be an oriented 3-manifold that fibers over $S^1$.  Let $S$ be an incompressible surface embedded in $M$.  If $S$ is homologous to a fiber, then $S$ is isotopic to the fiber.
\end{thm}


%Relatively orientable incompressible surfaces are those for which our version of Poincar\'e duality holds.

In the remainder of the section, we will be working with a non-orientable 3-manifold $N$ and an embedded non-orientable surface $\no$.   Let $\wt{N}$ and the covering map $p:\wt{N}\rightarrow N$ be the orientation double covering space of $N$.  Let $\wt{\no}$ be the preimage of $\no$ under $p$.  Let $\iota:\wt{N}\rightarrow\wt{N}$ be the orientation-reversing deck transformation of $p$.  We will initiate $N$ and $\no$ in each result below, but we surpress the initiation of the orientation double cover.

\begin{thm}[Poincar\'e Duality for non-orientable 3-manifolds]
  \label{thm:strong-duality}
  Let $N$ be a compact non-orientable $3$-manifold, and let $\no$ be a relatively oriented incompressible surface embedded in $N$.
  Then there exists $[\alpha] \in H^1(N; \ZZ)$ such that the pullback of $[\alpha]$ to $\wt{N}$ is the Poincar\'e dual of $\wt{\no}$ in $\wt{N}$.
  If $[\alpha]$ has a $1$-form representative $\alpha$ that vanishes nowhere on $N$, then $\no$ is homeomorphic to $f_{\alpha}^{-1}(\theta)$ for all $\theta \in S^1$.
\end{thm}

We will refer to the 1-form $\alpha$ given in Theorem~\ref{thm:strong-duality} as the {\it Poincar\'e dual} of the non-orientable surface $\no$.  Before proving Theorem~\ref{thm:strong-duality}, we need three lemmas.

\begin{lem}
  \label{lem:PD1}
  Let $N$ be a non-orientable 3-manifold.
  Let $\no$ be a relatively oriented embedded surface in $N$, and let $\wt{\no}=p^{-1}(\no)$ in $\wt{N}$.
  Then the Poincar\'e dual to $[\wt{\no}]$ is $\iota^{\ast}$-invariant.% where $\iota$ is the non-trivial deck transformation.
\end{lem}
\begin{proof}
  A positive vector field on $\no$ that is transverse to its tangent plane in $N$ lifts to a relative orientation of $\wt{\no}$ in $\wt{N}$.
  Since $\wt{\no}$ and $\wt{N}$ are orientable, the relative orientation of $\wt{\no}$ defines an orientation of $\wt{\no}$, and thus the homology class $[\wt{\no}]$ in $H_2(\wt{N};\RR)$ is well-defined.

  Next we show that $\iota$ reverses the orientation of $\wt{\no}$.  To do so, we first observe that because $\no$ is relatively oriented in $N$, the outward pointing transverse vector field on ${\no}$ must lift to an outward pointing transverse vector field on $\wt{\no}$.  In particular, for any outward pointing vector $\wt{v}$ on $\no$, the vector $\iota(\wt{v})$ is also outward pointing.

  Lift an outward pointing transverse vector field on $\no$ to an outward pointing transverse vector field $\wt{V}$ on $\wt{\no}$.  Let $(v_1, v_2, v_3)$ be a local frame for some point in $\wt{\no}$ such that
  $v_3$ is in $\wt{V}$.
  Since $\iota$ reverses the orientation of $\wt{N}$ but preserves the direction of $v_3$, $\iota$ must reverse the orientation of the pair $(v_1, v_2)$.
  In particular, that means $\iota$ reverses the orientation of $\wt{\no}$.

  Therefore $[\wt{\no}]$ is in the $(-1)$-eigenspace of the $\iota_{\ast}$ action on $H_2(\wt{N}; \RR)$.
  Let the cohomology class $[\wt{\alpha}]$ be the the Poincar\'e dual to $[\wt{\no}]$.
  Let $\wt{\alpha}$ be a representative $1$-form $\wt{\alpha}$ of $[\wt{\alpha}]$ (that need not be $\iota^{\ast}$-invariant).
  We use the fact that $\iota^2= \mathrm{id}$ in the first and third equalities:
  \begin{align*}
    \int_{\iota_{\ast}\wt{\no}} \omega &= \int_{\wt{\no}} \iota^{\ast}\omega &&\text{(By a change of variables)} \\
                                     &= \int_{\wt{N}} \wt{\alpha} \wedge \iota^{\ast} \omega &&\text{(Poincar\'e duality)} \\
                                     &=\int_{\wt{N}} \iota^{\ast} \left( \iota^{\ast}\wt{\alpha} \wedge \omega \right) \\
    &= \int_{\wt{N}} - \left( \iota^{\ast} \wt{\alpha} \wedge \omega \right) &&\text{($\iota$ is orientation reversing)}.
  \end{align*}
  Because $\iota_{\ast}[\wt{\no}] = -[\wt{\no}]$, we have the following:
  \begin{align*}
    \int_{\iota_{\ast}\wt{\no}} \omega &= - \int_{\wt{\no}} \omega \\
                              &= - \int_{\wt{N}} \wt{\alpha} \wedge \omega.
  \end{align*}
  Since $$\int_{\wt{N}}\wt{\alpha}\wedge\omega=\int_{\wt{N}}\iota^\ast\wt{\alpha}\wedge\omega$$ for all $\omega$, it follows that $\wt{\alpha}$ and $\iota^{\ast}\wt{\alpha}$ differ by an exact form, and therefore the cohomology class $[\wt{\alpha}]$ is
  $\iota^{\ast}$-invariant.
\end{proof}

As above, we will denote the Poincar\'e dual to $[\wt{\no}]$ by $[\wt{\alpha}]$.  The class $[\wt{\alpha}]$ is an $\iota^{\ast}$-invariant element of $H^1(\wt{N}; \ZZ)$, but it is not clear that $[\wt{\alpha}]$ is the pullback of an element of $H^1(N; \ZZ)$ under $p$.
In the next lemma, we show that is indeed the case.%, i.e. $[\wt{\alpha}]$ is the pullback of an element in $H^1(N; \ZZ)$.
\begin{lem}
  \label{lem:PD2}
Let $N$ be a non-orientable 3-manifold.  Let $[\wt{\alpha}]\in H^1(\wt{N}, Z)$ and let $\wt{S}$ be the Poincar\'e dual of $[\wt{\alpha}]$ in $\wt{N}$.  There exists $[\alpha] \in H^1(N; \ZZ)$ such that $\wt{\alpha} = p^{\ast} \alpha$.
\end{lem}
\begin{proof}
 It will suffice to show that for any simple closed curve $\gamma$ in $N$, the integral of $\wt{\alpha}$ along any path lift of $\gamma$ is an integer.  Let $x_0\in N$ be a base point of $\gamma$.  Note that $\gamma$ has two (path) lifts $\wt{\gamma}_1,\wt{\gamma}_2$ under $p$ in $\wt{N}$, one based at each element of $p^{-1}(x_0)$.  Either $\wt{\gamma}_1,\wt{\gamma}_2$ are both simple closed curves based at the each of the two preimages $p^{-1}(x_0)$ or $\wt{\gamma}_1,\wt{\gamma}_2$ are both arcs between the two points of $p^{-1}(x_0)$.
  If each lift $\wt{\gamma}_1,\wt{\gamma}_2$ of $\gamma$ is a closed curve in $\wt{N}$, the integral $\displaystyle\int_{\wt{\gamma}_i}\wt{\alpha}$ will be an integer since $[\wt{\alpha}] \in H^1(\wt{N}; \ZZ)$ for $i=1,2$.

 If each lift $\wt{\gamma}_1,\wt{\gamma}_2$ of $\gamma$ is an arc between the two preimages of $p^{-1}(x_0)$, we consider the simple closed curve $\wt{\gamma}=\wt{\gamma}_1\cup\wt{\gamma}_2$.  We note that $\iota(\wt{\gamma})=\wt{\gamma}$.
  By Lemma \ref{lem:PD1}, $\wt{\alpha}$ is $\iota^{\ast}$-invariant.  Therefore we have that $ \displaystyle\int_{\wt{\gamma}_1} \wt{\alpha} = \displaystyle\int_{\wt{\gamma}_2} \wt{\alpha}.$  Therefore $$\int_{\wt{\gamma}}\wt{\alpha}=2\int_{\wt{\gamma}_1}\wt{\alpha}.$$ It will suffice to show that $\displaystyle\int_{\wt{\gamma}}\wt{\alpha}$ is an even integer.
  Without loss of generality, we can assume all intersections of the simple closed curve $\wt{\gamma}$ with the surface $\wt{\os}$ are transverse.
  Since $\wt{\alpha}$ is a representative of the Poincar\'e dual to $[\wt{\os}]$, the integral of $\wt{\alpha}$ along $\wt{\gamma}$ is the signed intersection number of $\wt{\gamma}$ with $\wt{\os}$.
  The intersection number must be even, for if $\wt{\gamma}$ and $\wt{\os}$ intersect at a point $y$, then they also intersect at $\iota(y)$. This proves the lemma.
\end{proof}

The last lemma we need is the claim that lifts of incompressible surfaces are incompressible.
\begin{lem}
  \label{lem:lift-of-incompressible}
  Let $N$ be a non-orientable 3-manifold.  If $\no$ is a relatively oriented incompressible surface in $N$, then $\wt{\no}=p^{-1}(\no)$ is incompressible in $\wt{N}$.
\end{lem}
\begin{proof}
  Because $\no$ is incompressible in $N$, the map on fundamental groups induced by inclusion $\no\rightarrow N$ is injective.
  Since $p_\ast:\pi_1(\wt{N})\rightarrow\pi_1(N)$ is injective, the induced map $\pi_1(\wt{\no}) \to \pi_1(\wt{N})$ must also be injective.  An injective induced map on fundamental groups is equivalent to the orientable surface $\wt{\no}$ being incompressible.
\end{proof}


We now have everything we need to finish proving Theorem \ref{thm:strong-duality}.
\begin{proof}[Proof of Theorem \ref{thm:strong-duality}]
  Let $\wt{\no}=p^{-1}(\no).$
  The relative orientation of $\wt{\no}$ determines a homology class $[\wt{\no}]\in H_2(N;\ZZ)$.  Let the $1$-form $\wt{\alpha}$ be the Poincar\'e dual to $[\wt{\no}]$ in $\wt{N}$.
  By Lemma \ref{lem:PD2}, there exists a 1-form $\alpha\in H^1(N;\ZZ)$ such that $\wt{\alpha}=p^\ast\alpha$.

We define the map $f_\alpha:N\rightarrow S^1$ according to equation (\ref{form:map}).  Because $\alpha$ is non-vanishing, $f_{\alpha}$ has full rank everywhere.  Therefore $f_\alpha$ is a fibration.
The map $f_{\alpha} \circ p$ is a lift of $f_{\alpha}$ to $\wt{N}$ under $p$, and is therefore also a fibration.
  By Lemma \ref{lem:lift-of-incompressible}, $\wt{\no}$ is incompressible.
It follows from the orientable version of Poincar\'e duality that $\wt{\no}$ and $p^{-1}(f_{\alpha}^{-1}(\theta))$ are homologous surfaces in $\wt{N}$.    Theorem \ref{thm:Thur2} then tells us $\wt{\no}$ must be isotopic to a fiber of $f_{\alpha} \circ p$.
  The restriction of $p$ to the homeomorphic surfaces $\wt{\no}$ and $p^{-1}(f_{\alpha}^{-1})(\theta)$ determines two equivalent 2-fold covering maps $\wt{\no}\rightarrow \no$ and  $p^{-1}(f^{-1}_{\alpha}(\theta))\rightarrow f^{-1}_\alpha(\theta)$.  Therefore the image surfaces $\no$ and $f_{\alpha}^{-1}(\theta)$ must also be homeomorphic.
\end{proof}

  Note that the above proof does not tell us that $\no$ and $f_{\alpha}^{-1}(\theta)$ are isotopic.  Isotopy of the fibers of $N$ requires the isotopy between $\wt{\no}$ and $p^{-1}(f^{-1}_\alpha(\theta))$ to be $\iota^{\ast}$-equivariant.  However, the theorem is sufficient for our application.


We conclude the section with a non-orientable version of Theorem \ref{thm:Thur1}.
\begin{thm}
  \label{thm:classifying-fibrations}
  Let $N$ be a compact non-orientable $3$-manifold, and let $\mathcal{F}$ be the elements of $H^1(N; \ZZ)$ corresponding to fibrations of $N$ over $S^1$.
  \begin{enumerate}[(i)]
  \item Elements of $\mathcal{F}$ are in one-to-one correspondence with (non-zero) lattice points (i.e. points of $H^1(N; \ZZ)$) inside some union of cones over open faces of the unit ball in the Thurston norm.
  \item Let $\no$ be relatively oriented surface in $N$ that transverse to the suspension flow associated to some fibration $f: N \to S^1$.  Let $[\alpha]$ be the Poincar\'e dual $[\alpha]$ to $\no$.  Then $[\alpha]$ lies in the closure of the cone in $H^1(N; \RR)$ containing the $1$-form corresponding to $f$.
  \end{enumerate}
\end{thm}
\begin{proof}
For (i), we observe that by Theorem~\ref{thm:Thur1} the fibrations of $\wt{N}$ are in one-to-one correspondence with lattice points inside a union of cones over open faces of the unit ball in $H_2(\wt{N};\RR)$.  Let $\wt{\mathcal{K}}$ be the union of cones in $H_2(\wt{N};\RR)$.
 By Poincar\'e duality, $\wt{\mathcal{K}}$ is in one-to-one correspondence to a union of cones in $H^1(\wt{N};\RR)$, which we will call $\wt{\mathcal{K}}^\ast$

  Because $H^1(N;\RR)$ is isomorphic to a subspace of $H^1(\wt{N};\RR)$, we can construct a union of cones in $H^1(N; \RR)$ that map to the intersection of $p^\ast(H^1(N; \RR))$ with $\wt{\mathcal{K}}^\ast$.
  Indeed, every lattice point in $\wt{\mathcal{K}}^\ast$ corresponds to a fibration $f:N\to S^1$, since the pullback of $f$ to $H^1(\wt{N}; \ZZ)$ corresponds to a fibration of $\wt{N}$.
  Conversely, every fibration of $f:N\rightarrow S^1$ must correspond to an element of $\wt{\mathcal{K}}^\ast$, since the composition $f\circ p$ is a fibration of $\wt{N}\rightarrow S^1$.

  For (ii), assume that the surface $\no$ is transverse to the suspension flow of a fibration $f:N\rightarrow S^1$. Then $\wt{\no}$ is transverse to the suspension flow $p \circ f:\wt{N}\rightarrow S^1$.  Let $\wt{\alpha}$ be the pullback of $\alpha$ under $p$.  Then $\wt{\alpha}$ is the Poincar\'e dual of $\wt{\no}$.  By \autoref{thm:Thur1}, the 1-form $\wt{\alpha}$ lies in the closure of a component of $\wt{\mathcal{K}}^\ast$ that contains the 1-form corresponding to $f\circ p$.  Let $\wt{K}$ be this component.  Let $K\subset H^1(N;\RR)$ be the preimage of $\wt{K}$ under $p^\ast$.  The cone $K$ contains both $\alpha$ and the 1-form corresponding to $f$, as desired.
\end{proof}

\subsection{Oriented sums}
\label{sec:oriented-sums}

The next step in studying embedded non-orientable surfaces will be to describe \emph{oriented sums}.  Let $M$ be a 3-manifold.
The oriented sum of two embedded surfaces in $M$ is additive in both the Euler characteristic and $H^1(M;\RR)$.
This operation is well-known in the case of orientable $3$-manifolds (along with orientable embedded surfaces), but we will sketch the relevant details.
We then extend the construction to relatively oriented embedded surfaces.

\paragraph{Oriented sum for oriented manifolds}
Let $M$ be an orientable manifold.
Let $S$ and $S'$ be orientable embedded surfaces in $M$.
Assume that $S$ and $S'$ intersect transversally.
Thus $S \cap S'$ is a disjoint union of copies of $S^1$.
For each component of $S\cap S'$, take a tubular neighborhood that has cross section as in \autoref{fig:cross-section}.
\begin{figure}
  \centering
  \incfig[0.2]{cross-section}
  \caption{Cross section of intersection of $S$ and $S'$.}
  \label{fig:cross-section}
\end{figure}



We then perform a surgery on the leaves of $S$ and $S'$ so that the outward pointing normal vector fields match as in \autoref{fig:surgery}.
\begin{figure}[b]
  \centering
  \incfig[0.3]{surgery}
  \caption{On the left, the normal vectors on $S$ and $S'$ are consistent. On the right, they are not.}
  \label{fig:surgery}
\end{figure}

By performing this surgery at all the intersections, we get a new submanifold $S''$ of $M$ (which may have multiple components).
This new submanifold $S''$ is called the {\it oriented sum} of $S$ and $S'$.
The operation of taking oriented sums is additive on Euler characteristic, as well as the homology classes (and thus the cohomology classes of their Poincar\'e duals):
\begin{align*}
  \chi(S'') &= \chi(S) + \chi(S') \\
  [S''] &= [S] + [S'].
\end{align*}

\paragraph{Oriented sum for non-orientable manifolds} Let $N$ be a non-orientable 3-manifold and let $\no$ and $\no'$ be embedded surfaces in $N$ that are relatively oriented.
We define the oriented sum on $\no$ and $\no'$ as follows.
As above, let $p:\wt{N}\rightarrow N$ be the orientation double cover and let $\iota$ be the orientation reversing deck transformation of $\wt{N}$.
Let $\wt{\no}=p^{-1}(\no)$ and $\wt{\no}'=p^{-1}(\no')$, which are embedded oriented surfaces in $\wt{N}$.
The oriented sum of $\no$ and $\no'$ is the image under $p$ of the oriented sum of $\wt{\no}$ and $\wt{\no}'$.

To see that the operation is well-defined, we recall that $\iota$ preserves the relative orientation of $\wt{\no}$ and $\wt{\no}'$.  Therefore $\iota$ leaves the outward normal vector fields on $\wt{\no}$ and $\wt{\no}'$ invariant (see the proof of Lemma \ref{lem:PD1}).
Thus a leaf $L$ of $\wt{\no}$ is surgered with a leaf of $L'$ of $\wt{\no}'$ if and only if $\iota(L)$ and $\iota(L')$ are surgered.
Therefore surgery factors through $p$ and $[\no]+[\no']$ is well-defined for non-orientable surfaces.

\begin{example}
  \label{ex:oriented-sum}
   Let $\gamma$ be a component of $\no\cap \no'$ and $\wt{\gamma}_1$ and $\wt{\gamma}_2$ be the path lifts of $\gamma$.
  One possible orientation of $\wt{S}$ and $\wt{S}'$ is given in  \autoref{fig:consistency}.  The outward pointing normal vectors to $\wt{\no}$ and $\wt{\no'}$ determine which leaves are glued together along $\wt{\gamma}_1$ and $\wt{\gamma}_2$.

\begin{figure}
  \centering
  \incfig[0.4]{consistency}
  \caption{Neighborhoods of $\wt{\gamma}_1$ and $\wt{\gamma}_2$, with the outward pointing normal vector field.}
  \label{fig:consistency}
\end{figure}

To preserve the normal vector field, glue the left $\wt{\no}$ leaf to the bottom $\wt{\no'}$ leaf near $\wt{\gamma}_1$ and $\wt{\gamma}_2$.
Since $\iota(\wt{\gamma}_1)=\wt{\gamma_2}$, the outward pointing normal vector fields point the same (relative) directions.
\end{example}

\paragraph{Additivity}
By the consistency of the oriented sum in $N$ and $\wt{N}$, it easily follows that the oriented sum is additive in Euler characteristic, as well as in terms of Poincar\'e dual, since the Poincar\'e dual was also defined by passing to the orientation double cover.
