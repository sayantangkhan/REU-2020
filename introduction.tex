\section{Introduction}
\label{sec:introduction}


Let $S_{g,n}$ be a surface of genus $g$ with $n$ punctures.  The mapping class group of $S_{g,n}$ consists of homotopy classes of orientation preserving homeomorphisms of $S_{g,n}$.  The Nielsen--Thurston classification of mapping classes (elements of the mapping class group) says that each mapping class is periodic, preserves some multicurve, or has a representative that is pseudo-Anosov.  For each pseudo-Anosov homeomorphism $\varphi:S_{g,n}\rightarrow S_{g,n}$, the stretch factor $\lambda(\varphi)$ is an algebraic integer that describes the amount by which $\varphi$ changes the length of curves.  Arnoux--Yoccoz \cite{AY} and Ivanov \cite{ivanov} prove that the set
$$\Spec(S_{g,n})=\{\log(\lambda(\varphi)) \mid \varphi \text{ is a pseudo-Anosov homeomorphism of }S_{g,n}\}$$ is a closed discrete subset of $(0,\infty)$. The minimum of $\Spec(S_{g,n})$:
$$\ell_{g,n}=\min\{\log(\lambda(\varphi)) \mid \varphi \text{ is a pseudo-Anosov homeomorphism of }S_{g,n}\}$$ quantitatively describes both the dynamics of the mapping class group of $S_{g,n}$ and the geometry of the moduli space of $S_{g,n}$.

Penner \cite{penner1991bounds} showed that for orientable surfaces, $$\ell_{g,0}\asymp \frac{1}{g}.$$
  Penner conjectured that $\ell_{g,n}$ will have the same asymptotic behavior for  $n\geq0$ punctures.  Following Penner, substantial attention has been given to finding bounds for $\ell_{g,n}$ \cite{AD,bauer,hironaka,HK,HK20,KT,Loving,minakawa}, calculating $\ell_{g,n}$ for specific values of $(g,n)$ \cite{CH,HS,LT,SKL}, and finding asymptotic behavior of $\ell_{g,n}$ for {\it orientable} surfaces with $n\geq 0$ \cite{KT,tsai2009asymptotic,valdivia,yazdi2018pseudo}.  We adapt a result of Yazdi \cite{yazdi2018pseudo} to non-orientable surfaces.

\begin{thm}\label{thm:stretch1}
  Let $\no_{g,n}$ be a non-orientable surface of genus $g$ with $n$ punctures, and let $\ell_{g,n}'$ be the logarithm of
  the minimum stretch factor of the pseudo-Anosov mapping classes acting on $\no_{g,n}$.
  Then for any fixed $n \in \mathbb{N}$, there is a positive constant $B'_1 = B'_1(n)$ and $B'_2 = B'_2(n)$ such
  that for any $g \geq 3$,
  the quantity $\ell_{g,n}'$ satisfies the following inequalities:
  \begin{align*}
    \frac{B'_1}{g} \leq \ell'_{g,n} \leq \frac{B'_2}{g}.
  \end{align*}
\end{thm}



\paragraph{Pseudo-Anosov homeomorphisms} Let $S$ be a (possibly non-orientable) surface of finite type.  A homeomorphism $\varphi:S\rightarrow S$ is said to be {\it pseudo-Anosov} if there exist a pair of transverse measured singular foliations $\mathcal{F}_s$ and $\mathcal{F}_u$ and a real number $\lambda$ such that $$\varphi(\mathcal{F}_s)=\lambda^{-1} \cdot \mathcal{F}_s\text{ and } \varphi(\mathcal{F}_u)=\lambda \cdot \mathcal{F}_u.$$  The {\it stretch factor} of $\varphi$ is the algebraic integer $\lambda=\lambda(\varphi)$.


Endow $S$ with a Riemannian metric.  The stretch factor $\lambda(\varphi)$ measures the growth rate of the length of geodesic representatives of a simple closed curve $S$ under iteration of $\varphi$ \cite[Proposition 9.21]{FLP}.  Moreover, $\log(\lambda(\varphi))$ is the minimal topological entropy of any homeomorphism of $S$ that is isotopic to $\varphi$ \cite[Expos\'e 10]{FLP}.


\paragraph{Geometry of moduli space}
Let $\mathcal{T}_{g,n}$ denote the Teichm\"uller space of $S_{g,n}$, that is: the space of isotopy classes of hyperbolic metrics on $S_{g,n}$.
When endowed with the Teichm\"uller metric, the mapping class group of $S_{g,n}$ acts properly discontinuously on $\mathcal{T}_{g,n}$ by isometries.  The quotient of this action is the {\it moduli space} of $S_{g,n}$. The set
$\Spec(S_{g,n})$ is the length spectrum of geodesics in the moduli space of $S_{g,n}$.  Therefore the quantity $\ell_{g,n}$ is the length of the shortest geodesic in the moduli space of $S_{g,n}$.

\paragraph{Explicit bounds} In his foundational work, Penner found $\frac{\log 2}{12g-12+4n}$ to be a lower bound for $\ell_{g,n}$ for orientable surfaces \cite{penner1991bounds}.  He also determined $\frac{\log 11}{g}$ to be an upper bound for $\ell_{g,0}$.  Penner's work proves that $\ell_{g,0}\asymp \frac{1}{g}$.  McMullen  \cite{mcmullen2000polynomial} later asked:
\begin{question}[McMullen]
Does $\displaystyle\lim_{g\rightarrow\infty}g\cdot \ell_{g,0}$ exist, and if so, what does it converge to?
\end{question}
To this end, Bauer \cite{bauer} strengthened the upper bound for $g\cdot \ell_{g,0}$ to $\log 6$, and Minakawa \cite{minakawa} and Hironaka--Kin \cite{HK} further sharpened the upper bounds for $g\cdot \ell_{g,0}$ and $g\cdot \ell_{0,2g+1}$ to $\log(2+\sqrt{3})$.  Later Aaber--Dunfield \cite{AD}, Hironaka \cite{hironaka}, and Kin-Takasawa \cite{KTbounds} determined that $\log\left(\frac{3+\sqrt{5}}{2}\right)$ is an upper bound for $g\cdot \ell_{g,0}$ and conjectured it is the supremum of $g\cdot \ell_{g,0}$.

\paragraph{Asymptotic behavior of punctured surfaces}
Tsai initiated the study of asymptotic behavior of $\ell_{g,n}$ along lines in the $(g,n)$-plane \cite{tsai2009asymptotic}.  In particular, Tsai determined that for orientable surfaces of fixed genus $g\geq 2$, the asymptotic behavior in $n$ is:
$$\ell_{g,n}\asymp \frac{\log n}{n}.$$
Further, she showed that $\ell_{0,n}\asymp \frac{1}{n}.$
Later, Yazdi \cite{yazdi2018pseudo} determined that for an orientable surface with a fixed number of punctures $n\geq 0$, the asymptotic behavior in $g$ is:
$$\ell_{g,n}\asymp \frac{1}{g},$$
confirming the conjecture of Penner.

\paragraph{Non-orientable surfaces}
Let $\no_{g,n}$ be a non-orientable surface of genus $g$ with $n$ punctures.  As above, let $\ell'_{g,n}$ denote the minimum stretch factor of pseudo-Anosov homeomorphisms of $\no_{g,n}$.  For any $n\geq 0$ and $g\geq 1$, $\ell_{g-1,2n}$ is a lower bound for $\ell'_{g,n}$, which can be seen by passing to the orientation double cover of $\no_{g,n}$ (note that the definition of genus is different for orientable and non-orientable surfaces).  Because the upper bounds for $\ell_{g,n}$ are constructed by example, upper bounds for $\ell'_{g,n}$ do not follow immediately from passing to the orientation double cover.  Recently Liechti--Strenner determined $\ell'_{g,0}$ for $g\in\{4,5,6,7,8,10,12,14,16,18,20\}$ \cite{LS}.  Our work captures the asymptotic behavior for the punctured case.

%Liechti--Strenner were motivated to calculate $l'_{g,0}$ because Liechti rephrased question in


\paragraph{Techniques} To prove Theorem \ref{thm:stretch1}, we adapt the strategy of Yazdi \cite{yazdi2018pseudo} to non-orientable surfaces with punctures.  The lower bound of $\ell'_{g,n}$ is found by lifting to the orientation double cover of $\no_{g,n}$.  The upper bound (as in all prior work) is constructive.  Fix $n\geq 0$: the desired number of punctures.  Yazdi's construction is as follows.  For a sequence of genera $g_{n,k}$ (where $k$ goes from $3$ to $\infty$, and $g_{n,k} = (14k-2)n + 2$), use the Penner construction \cite{penner1988construction} to obtain a homeomorphism $f_{n,k}$ of $S_{g_{n,k},n}$ that is pseudo-Anosov and has low stretch factor.  In order to find pseudo-Anosov homeomorphisms of $S_{g,n}$ with small stretch factor for all $g$ (not just those in the sequence above), construct a mapping torus for each $f_{n,k}$.  To do this Yazdi's appeals to a technique involving the use of Thurston's theory of fibered faces.

\paragraph{Thurston norm for non-orientable 3-manifolds} In Thurston's development of what is now called the Thurston norm for 3-manifolds \cite{thurston1986norm}, his definitions and theorems required that all surfaces were orientable.  Thurston said that the theorems should still be true for non-orientable surfaces, but there are some subtleties that have not been addressed elsewhere in the literature.  In this paper, we write the details of Thurston's theory of fibered faces to non-orientable 3-manifolds.  In particular, for orientable 3-manifolds, the Thurston norm is a norm on the second homology of a 3-manifold that measures the minimum complexity of an embedded (orientable) surface; it will need to be adjusted in non-orientable 3-manifolds. Specifically, the Thurston norm does not recognize embedded non-orientable surfaces in the second homology of a non-orientable 3-manifold.  To address this limitation, we instead calculate the Thurston norm on the first cohomology of a non-orientable manifold.  We develop a (weak) version of Poincar\'e duality in Theorem \ref{thm:strong-duality} that suffices to define a Thurston norm on $H^1(M;R)$ for a non-orientable 3-manifold $M$.

\paragraph{Fibered faces} A special case of Thurston's hyperbolization theorem says that the monodromy of any fibration of a hyperbolic 3-manifold over $S^1$ is a pseudo-Anosov homeomorphism.  Therefore by finding other fibrations of the same 3-manifold, one obtains additional pseudo-Anosov homeomorphism.  Work of Fried \cite{fried1982flow,fried1983transitive}, Matsumoto \cite{matsumoto1987topological}, and Agol--Leininger--Margalit \cite{agol6983pseudo} can be used to bound the stretch factors of certain pseudo-Anosov homeomorphisms obtained in this way.

\paragraph{Outline} In Section \ref{sec:thur-norm-non-orientable} we state Thurston's theory of fibered faces and adapt it to the non-orientable setting.  In Section \ref{sec:mapping-classes-with} we show how Thurston's theory of fibered faces can be used to construct pseudo-Anosov homeomorphisms of low stretch factor for non-orientable surfaces.  Specifically, we state and prove the Nielsen--Thurston classification for non-orientable surfaces.  Then we adapt the results of Fried \cite{fried1982flow,fried1983transitive}, Matsumoto \cite{matsumoto1987topological}, and Agol--Leininger--Margalit \cite{agol6983pseudo} used to construct pseudo-Anosov homeomorphisms with low stretch factor of orientable surfaces to the non-orientable setting.  In Section \ref{sec:application}, we prove Theorem \ref{thm:stretch1}, following the strategy of Yazdi.

\paragraph{Acknowledgements}
This work is the result of an REU at the University of Michigan in summer 2020.  We are grateful to Alex Wright for organizing the REU and to Livio Liechti for suggesting the project. We are thankful to Dan Margalit for his comments and to anonymous referees for their comments.  We are also grateful to the other co-organizers, mentors, and participants in the REU: Paul Apisa, Chaya Norton, Christopher Zhang, Bradley Zykoski, Anne Larsen, and Rafael Saavedra.  The REU was partially funded by NSF Grant DMS 185615.  The third author acknowledges support of the NSF through grant 2002951.
