\section{Pushforward of fibrations along covers}
\label{sec:pushf-fibr-along}

In this section, we'll prove a criterion for when a fibration $\wt{f}: \wt{M} \to S^1$ of a
$3$-manifold $\wt{M}$ over $S^1$ (more specifically $\RR/\ZZ$) can be pushed forward to a fibration
$f$ of a $3$-manifold $M$ covered by $\wt{M}$ such that $\wt{f} = f \circ p$, where $p$ is the
covering map. More specifically, we'll be interested in the case when $M$ is a non-orientable $3$-manifold,
and $\wt{M}$ the orientation double cover.

To understand what the ingredients of the criterion ought to be, it helps to think about the
correspondence between fibrations over $\RR/\ZZ$ and $1$-forms on $\wt{M}$ of a specific kind. Given
such a fibration $\wt{f}$, we can pull back the $1$-form $d\theta$ on $S^1$ along the map $\wt{f}$
to get $1$-form $\wt{\alpha}$. Since $\wt{f}$ is regular, and $d\theta$ is nowhere vanishing, the
pullback $\wt{\alpha}$ is also nowhere vanishing, or as Thurston refers to it, a \emph{non-singular}
$1$-form. Furthermore, $\wt{\alpha}$ also has integer periods, i.e. its integral over any closed
loop lies in $\ZZ$. To see this, pushforward the loop to $S^1$, and it's clear that the integral of
$d\theta$ on an closed loop in $\RR/\ZZ$ must lie in $\ZZ$. To sum it up, the pullback $1$-form
$\wt{\alpha}$ is a non-singular $1$-form with integer periods. It turns out that this condition
describes a fibration completely. That is to say, given any other non-singular $1$-form $\beta$ with
integer periods, one gets a fibration $g: \wt{M} \to \RR/\ZZ$ from the following formula (here $x_0$
is some arbitrarily chosen basepoint).
\begin{align*}
  g(x) \coloneqq \left( \int_{x_0}^x \beta \right) \bigg/ \ZZ
\end{align*}
It's clear that pulling back $d\theta$ along this map gives $\beta$ once again, which gives us the
correspondence between fibrations over $S^1$ and non-singular $1$-forms with integer periods.

For the fibration $\wt{f}: \wt{M} \to \RR/\ZZ$ to descend to the quotient $M$, we must have the
following identity hold for any deck transformation $\iota$.
\begin{align*}
  \wt{f}(x) = \wt{f}(\iota(x))
\end{align*}
We can express the terms in the above identity as integrals of $\wt{\alpha}$: doing so tells us that
we need to have the following identity hold.
\begin{align*}
  \left(   \int_{x_0}^x \wt{\alpha} - \int_{x_0}^{\iota(x)} \wt{\alpha}\right) \in \ZZ
\end{align*}
By the change of variables formula, the first term in the expression can be written in the
following manner.
\begin{align*}
  \int_{x_0}^x \wt{\alpha} = \int_{\iota(x_0)}^{\iota(x)} (\iota^{-1})^{\ast} \wt{\alpha}
\end{align*}
If $( \iota^{-1})^{\ast}\wt{\alpha} = \wt{\alpha}$, we can substitute the right hand term into the identity we want
to prove, and simplify it to the following criterion.
\begin{align*}
  \left(   \int_{x_0}^{\iota(x_0)} \wt{\alpha}\right) \in \ZZ
\end{align*}
This gives us the following sufficient conditions for determining when a fibration can be pushed forward.

\begin{enumerate}[(i)]
    \item $\wt{\alpha}$ is $\iota^{-1}$ invariant.
    \item $\displaystyle \int_{x_0}^{\iota(x_0)} \wt{\alpha}$ is an integer.
\end{enumerate}

We can strengthen these sufficient conditions to necessary ones. To see this suppose that the fibration $\wt{f}$ from $\wt{M}$ to $\RR/\ZZ$ does descend to a fibration $f$ from $M$ to $\RR/\ZZ$. If we let $p: \wt{M} \xrightarrow[]{} M$ be the covering map, then we must have that $\wt{f} = f \circ p$. Note that this means for any deck transformation $\iota$, 
\begin{align*}
    \wt{f} \circ \iota^{-1} = f \circ p \circ \iota^{-1} = f \circ p = \wt{f}
\end{align*}
Thus if we look at our 1-form $\wt{\alpha} = \wt{f}^*d\theta$, then we have that
\begin{align*}
    (\iota^{-1})^*\wt{\alpha} = (\iota^{-1})^*(\wt{f}^*d\theta) = (\wt{f} \circ \iota^{-1})^*d\theta = \wt{f}^*d\theta = \wt{\alpha}, 
\end{align*}
so $\wt{\alpha}$ is invariant under $\iota^{-1}$.

Now that condition $(i)$ is satisfied, $(ii)$ follows naturally. Note as we mentioned above, since $\wt{f}$ descends to $f$, we know that $\wt{f} = \wt{f}(\iota(x))$ for any deck transformation $\iota$. Thus as above, this reduces to the following integrals being equivalent:
\begin{align*}
  \int_{x_0}^x \wt{\alpha} = \int_{\iota(x_0)}^{\iota(x)} (\iota^{-1})^{\ast} \wt{\alpha}
\end{align*}

Which again by the fact that condition $(i)$ is satisfied, reduces to $(ii)$ being true. So $\wt{f}$ pushing forward to a fibration $f$ implies conditions $(i)$ and $(ii)$.

This allows us to state the following lemma:

\begin{lem}
  If $\wt{f}$ is a fibration from $\wt{M}$ to $\RR/\ZZ$, and $\wt{M}$ is a finite sheeted cover of
  $M$, then the fibration $f$ can be pushed forward to a fibration from $M$ to $\RR/\ZZ$ if \textbf{and only if} the
  following conditions on $\wt{\alpha}$ (the pullback of $d\theta$) hold for all deck
  transformations $\iota$.
  \begin{enumerate}[(i)]
  \item $\wt{\alpha}$ is $\iota^{-1}$ invariant.
  \item $\displaystyle \int_{x_0}^{\iota(x_0)} \wt{\alpha}$ is an integer.
  \end{enumerate}
\end{lem}

