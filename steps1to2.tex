\section{Constructing an infinite family of psuedo-Anosovs}
\label{sec:constr-an-infin}

We will first imitate the first two steps of Yazdi's construction \cite{yazdi2018pseudo}, constructing two infinite families of non-orientable surfaces. For the most part, we will use the same notation as Yazdi to show exactly how we are copying his construction.

\textbf{Step 1:} In the first step of the construction, Yazdi constructs an infinite family of surfaces through covering actions on an infinite surface built from a fundamental building block. We will imitate this construction and construct the two infinite families of non-orientable surfaces that we will require. 

We will define two families of surfaces $P_{n,k}$ and $Q_{n,k}$, first defining $P_{n,k}$ in the following way originally done by Yazdi. Let $T$ be an orientable surface of genus 5 with 3 boundary components $c,d$ and $e$. Orient $T$ and give $c,d$ and $e$ the induced orientations from this orientation. Now add two cross-caps to $T$ but keep the boundaries of $T$ oriented. Let $p$ (respectively $q$) be a puncture (respectively a marked point) on the boundary component $e$ of $T$, with oriented arcs $r$ and $s$ connecting them on $\partial T$. See Figure X for a picture of $T$. Let $T_{i,j}$ be copies of the surface $T$, where $i,j \in \mathbb{Z}$. We will use similar notations to refer to the boundary components of $T_{i,j}$. Define the infinite surface $T_\infty$ as the quotient
$$T_\infty \coloneqq \left( \bigcup T_{i,j} \right)/\sim,$$
where $i,j \in \mathbb{Z}$. The equivalence relation $\sim$ is defined as $$c_{i,j} \sim d_{i+1,j} \hspace{1em}, \hspace{1em} r_{i,j} \sim s_{i,j+1}$$ and the gluing maps for the boundary components are by orientation-reversing homeomorphisms.

There are two natural maps $\overline{\rho_1},\overline{\rho_2}: T_\infty \xrightarrow[]{} T_\infty$ that act by shifts as follows $$\overline{\rho_1} \text{ sends } T_{i,j} \text{ to } T_{i+1,j},$$ $$\overline{\rho_2} \text{ sends } T_{i,j} \text{ to } T_{i,j+1}.$$ Note that these maps commute. Define the surface $P_{n,k}$ as the quotient of the surface $T_\infty$ by the covering action of the group generated by $(\overline{\rho_1})^n$ and $(\overline{\rho_2})^k$. Therefore, $\overline{\rho_1}$ and $\overline{\rho_2}$ induce maps on the surface $P_{n,k}$, which we denote by $\rho_1$ and $\rho_2$.

We can define an analogous family of surface $Q_{n,k}$ by starting with a different building block, $R$. Let $R$ be a non-orientable surface of genus 13 with 3 boundary components $c,d$ and $e$. We again give $c, d$ and $e$ a consistent choice of orientation as above, and let $p$,$q$,$r$ and $s$ be as above. See Figure Y for a picture of $R$. We then let $R_{i,j}$ be copies of $R$, where $i,j \in \mathbb{Z}$ and then define a surface $N_\infty$ defined exactly like $T_\infty$ above. We then analogously have maps $\overline{\rho_1},\overline{\rho_2}: N_\infty \xrightarrow[]{} N_\infty$ acting as shifts, and use them to define the family of surfaces $Q_{n,k}$.

A question that naturally arises is why did we choose the surfaces $T$ and $R$ that we did as our building blocks? It comes down to two main problems:
\begin{enumerate}
    \item The combinatorics of the curves make the associated matrix we get from the Penner construction satisfy the conditions of Lemma 1 
    \item Having a curve $\gamma$ such that it and its image under our map form the boundary of an embedded $\mathcal{N}_3$ in our surface, which we will need in later steps (\textcolor{red}{Perhaps this was outlined above}). 
\end{enumerate}

\begin{lem}
Define the sequences
\begin{align}
    g_{n,k} &= (14k - 2)n + 2, k \geq 3, n \geq 1 \\
    g'_{n,k} &= (15k - 2)n + 2, k \geq 3, n \geq 1.
\end{align}
    The genus of $P_{n,k}$ and $Q_{n,k}$ are $g_{n,k}$ and $g'_{n,k}$ respectively.
\end{lem}
\begin{proof}
    We will first prove the lemma for $P_{n,k}$. Consider the subsurface $U \subset P_{n,k}$ defined as $$U = \left( \bigcup_{i =0}^{k-1} T_{0,i} \right)/\sim.$$ Then $U$ is a compact, non-orientable surface of genus $12k$ with $2k$ boundary components, and forms a fundamental domain for the covering action of $\overline{\rho_1}$ on $T_\infty$. We have $$\chi(U) = 2 - 12k - 2k = 2 - 14k.$$ Thus $$\chi(P_{n,k}) = n \cdot \chi(U) = -n(14k - 2),$$ since $P_{n,k}$ is formed by gluing $n$ copies of $U$ together along circle boundary components. Therefore $$\chi(P_{n,k}) = 2 - g_{n,k} = -n(14k - 2) \implies g_{n,k} = n(14k - 2) + 2.$$ Likewise, the same argument holds for $Q_{n,k}$, but replacing 12 with 13 gives $g'_{n,k} = n(15k - 2) + 2$.
\end{proof}

\textbf{Step 1: (Alternate)}

We will first give an overview of how a \textit{Yazdi-example} is constructed, then specify what specific examples we will use for our construction.

In the first step of the construction, Yazdi constructs an infinite family of surfaces through covering actions on an infinite surface built from a fundamental building block. To do this, he starts with an orientable surface of genus $g$ with 3 boundary components $c$, $d$ and $e$ (decomposed into arcs $r$ and $s$), call it $S$. Then an infinite surface is constructed by gluing together copies of $S$, $S_{i,j}$, indexed by $i,j \in \mathbb{Z}$ by gluing $c_{i,j}$ to $d_{i+1,j}$ and $r_{i,j}$ to $s_{i,j+1}$. If we let $\rho_1$ denote the map on this infinite surface that shifts the $i$ index by 1
and $\rho_2$ the map on this infinite surface that shifts the $j$ index by 1, then we can be quotient our infinite surface by the covering action of the group generated $\rho_1^n$ and $\rho_2^k$ for some integers $n$ and $k$. This gives us an infinite family of closed surfaces, denoted $P_{n,k}$. Refer to section 3.1 of Yazdi's papers for details.

For our case, we want to use this construction to construct two infinite families of non-orientable surfaces. Let $T$ be an orientable surface of genus 5 with 3 boundary components $c,d$ and $e$. Orient $T$ and give $c,d$ and $e$ the induced orientations from this orientation. Now add two cross-caps to $T$ but keep the boundaries of $T$ oriented. This gives us a genus 12 non-orientable surface, see Figure X for a picture. We can now continue Yazdi's construction to get an infinite family of non-orientable surfaces we will denote $P_{n,k}$.

We can define an analogous family of surface $Q_{n,k}$ by starting with a different building block, $R$. Let $R$ be a non-orientable surface of genus 13 with 3 boundary components $c,d$ and $e$ (genus 5 orientable surface with 3 cross-caps). See Figure X for a picture of $R$. We can perform the exact same process as above to get another family of non-orientable surfaces $Q_{n,k}$, albeit with different genera. 

A question that naturally arises is why did we choose the surfaces $T$ and $R$ that we did as our building blocks? It comes down to two main problems:
\begin{enumerate}
    \item The combinatorics of the curves make the associated matrix we get from the Penner construction satisfy the conditions of Lemma 1 
    \item Having a curve $\gamma$ such that it and its image under our map form the boundary of an embedded $\mathcal{N}_3$ in our surface, which we will need in later steps (\textcolor{red}{Perhaps this was outlined above}). 
\end{enumerate}

\begin{lem}
Define the sequences
\begin{align}
    g_{n,k} &= (14k - 2)n + 2, k \geq 3, n \geq 1 \\
    g'_{n,k} &= (15k - 2)n + 2, k \geq 3, n \geq 1.
\end{align}
    The genus of $P_{n,k}$ and $Q_{n,k}$ are $g_{n,k}$ and $g'_{n,k}$ respectively.
\end{lem}
\begin{proof}
    We will first prove the lemma for $P_{n,k}$. Consider the subsurface $U \subset P_{n,k}$ defined as $$U = \left( \bigcup_{i =0}^{k-1} T_{0,i} \right)/\sim.$$ Then $U$ is a compact, non-orientable surface of genus $12k$ with $2k$ boundary components, and forms a fundamental domain for the covering action of $\overline{\rho_1}$ on $T_\infty$. We have $$\chi(U) = 2 - 12k - 2k = 2 - 14k.$$ Thus $$\chi(P_{n,k}) = n \cdot \chi(U) = -n(14k - 2),$$ since $P_{n,k}$ is formed by gluing $n$ copies of $U$ together along circle boundary components. Therefore $$\chi(P_{n,k}) = 2 - g_{n,k} = -n(14k - 2) \implies g_{n,k} = n(14k - 2) + 2.$$ Likewise, the same argument holds for $Q_{n,k}$, but replacing 12 with 13 gives $g'_{n,k} = n(15k - 2) + 2$.
\end{proof}


\textbf{Step 2:} Following Yazdi, we will now define maps $f_{n,k}: P_{n,k} \xrightarrow[]{} P_{n,k}$ and $h_{n,k}: Q_{n,k} \xrightarrow[]{} Q_{n,k}$ that are defined as a composition of specific Dehn twists, followed by a finite order mapping class. The key insight is that a power of this map will be a composition of Dehn twists that satisfy the criteria to be a Penner construction and thus pseudo-Anosov.

The following statements will hold for both $P_{n,k}$ and $Q_{n,k}$. Let $\mathcal{B}$ be the union of all $\beta$ curves except $\beta_1$ in $T_{0,0} \cup T_{0,1} \cup T_{1,0}$ (respectively $R_{0,0} \cup R_{0,1} \cup R_{1,0}$) (see figures below). Let $\rho_1(\mathcal{B})$ be the image of $\mathcal{B}$ under $\rho_1$. Define $\phi_b$ as the composition of positive Dehn twists along all the curves in the set $\overline{\mathcal{B}} \coloneqq \mathcal{B} \cup \rho_1(\mathcal{B}) \cup \dots \cup \rho_1^{n-1}(\mathcal{B})$. Since the curves in $\overline{\mathcal{B}}$ are disjoint, Dehn twists along them commute. Therefore, it is no necessary to specify the order in which we compose these Dehn twists in $\phi_b$. Let $\mathcal{R}$ be the union of all $\alpha$ curves except $\alpha_1$ in $T_{0,0}$. Define $\mathcal{R}$ and $\phi_r$ similarly but use negative Dehn twists this time.

Let $\alpha_1,\beta_1 \subset T_{0,0}$ be the curves in Figure Z. Let $\phi$ be the composition of negative dehn twists along all the curves $\alpha_1, \rho_1(\alpha_1), \dots, \rho_1^{n-1}(\alpha_1)$ followed by positive Dehn twists along all the curves $\beta_1,\rho_1(\beta_1),\dots,\rho_1^{n-1}(\beta_1)$. Define
\begin{align*}
    f_{n,k} &\coloneqq \rho_2 \circ \phi \circ \phi_b \circ \phi_r \\
    h_{n,k} & \coloneqq \rho_2 \circ \phi \circ \phi_b \circ \phi_r
\end{align*}
using the curves in $P_{n,k}$ and $Q_{n,k}$ respectively. We are using notation so the composition is from right to left. It follows from Penner's construction of pseudo-Anosov maps that $(f_{n,k})^k$ and $(h_{n,k})^k$ are pseudo-Anosov. Hence $f_{n,k}$ and $h_{n,k}$ are themselves pseudo-Anosov and invariant train tracks $\tau^1_{n,k}$ and $\tau^2_{n,k}$ for $f_[n,k]$ and $h_{n,k}$ respectively can be obtained from Penner's construction by smoothing the intersection points.

\textcolor{red}{We need to state how we know that these two sets of curves can be marked inconsistently}
