\section{Background}

\subsection{The Thurston Norm}

Thurston defined a semi-norm on the second relative homology of a compact, orientable 3-manifold with real coefficients. This Thurston norm was introduced and given a comprehensive overview in \cite{thurston1986norm}, but we will given an introduction here for the unfamiliar reader.

When looking at an orientable 3-manifold $M$, it is known that $H_2(M,\partial M)$ can be represented by embedded surfaces inside of $M$. We can use this to define the Thurston norm, $x: H_2(M,\partial M) \xrightarrow[]{} \mathbb{R}$, on the integer lattice of $H_2(M,\partial M)$. 

Let $\chi_-(S)$ denote the negative component of the Euler characteristic of a surface $S$, i.e. $\chi_-(S) = \max\{-\chi(S),0\}$, we call this the \textit{complexity} of $S$. If a surface $S$ has multiple components then its complexity is the sum of the $\chi_-$ for the individual components. If $a \in H_2(M,\partial M)$, we can define $x(a)$ as:
$$x(a) = \min\{\chi_-(S) \,\vert\, [S] = a \text{ and $S$ is compact, properly embedded and oriented}\}.$$

We can then extend $x$ to the rational points by claiming it must be linear on rays that go through the origin, and there is a unique way to extend a function defined on rational points to reals so that it is continuous.

One of the most important uses and results that follow from this norm is its connection with fibrations of fibered 3-manifolds that fiber over the circle. An interesting property of the Thurston norm is that its unit ball is a convex polyhedron and thus it makes sense to discuss faces of Thurston norm, i.e. the faces of the unit ball of Thurston norm. Thurston showed that for a closed, fibered 3-manifold $M$ that fibers over the circle, there is a striking connection between cones on open faces and possible fibers of fibrations of $M$. The following is a result of Thurston as it was restated in \cite{yazdi2018pseudo}

\begin{thm}[Thurston]
Let $\mathcal{F}$ be the set of homology classes $\in H_2(M)$ that are representable by fibers of fibrations of $M$ over the circle.
\begin{enumerate}
    \item Elements of $\mathcal{F}$ are in one-to-one correspondence with (non-zero) lattice points inside some union of cones on open faces of Thurston norm.
    \item If a surface $F$ is transverse to the suspension flow associated to some fibration of $M \xrightarrow[]{} S^1$ then $[F]$ lies in the closure of the corresponding cone in $H_2(M)$.
\end{enumerate}
\end{thm}

Note that we refer to the cones in item $(1)$ as \textit{fibered faces}.

It is one of the primary goals of this paper to extend this definition of Thurston norm to non-orientable manifolds and be able to state the non-orientable version of the theorem above. 





