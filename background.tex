\section{Background}
\label{sec:background}

\subsection{Mapping classes on non-orientable surfaces}
\label{sec:mapping-classes-non}

Compact non-orientable surfaces with marked points are classified by their genus and number of
marked points, just like compact orientable surfaces. A genus $g$ non-orientable surface is the
connect-sum of $g$ copies of $\mathbb{RP}^2$, much like how a genus $g$ surface is the connect-sum
of $g$ copies of a torus $S^1 \times S^1$. A non-orientable surface $\no_{g,n}$ of genus $g$ with
$n$ marked points has an orientable double cover, which is an orientable surface $\os_{g-1,
  2n}$. Associated to this double cover is an orientation reversing deck transformation
$\iota: \os_{g-1, 2n} \to \os_{g-1, 2n}$. The orientation double cover is often used when proving
statements about a non-orientable surface. If the genus and the number of marked points is clear
from the context, or unimportant, we will drop the subscripts, and just use $\no$ and $\os$ for the
non-orientable surface and its orientable double cover respectively.

Given any homeomorphism $\varphi: \no \to \no$ on the non-orientable surface, it is always possible
to lift it to a unique orientation preserving homeomorphism $\wt{\varphi}: \os \to \os$. This is an
easy exercise in covering space theory, but we'll give a proof here for completeness.
\begin{prop}
  For any homeomorphism $\varphi: \no \to \no$, there exists a unique orientation preserving lift
  $\wt{\varphi}: \os \to \os$. If the non-orientable surface has marked points fixed by $\varphi$,
  then the orientation preserving lift $\wt{\varphi}$ may not fix the marked points, but the lift
  of $\varphi^2$ will fix the marked points.
\end{prop}
\begin{proof}
  One can always lift a homeomorphism $\varphi: \no \to \no$ if $\varphi$ preserves the subgroup of
  $\pi_1(\no)$ corresponding to the cover $\os$. This subgroup can be concretely described as the
  subgroup generated by the two sided curves in $\no$, i.e. the curves whose tubular neighbourhoods
  are cylinders, and not M\"obius strips. Such a subgroup is clearly preserved by any homeomorphism
  $\varphi$, which means we always have a lift. There will be two choices for such a lift, since
  $\os$ is a two-sheeted cover. These two lifts $\wt{\varphi}_1$ and $\wt{\varphi}_2$ are related
  by the following identity.
  \begin{align*}
    \wt{\varphi}_1 = \iota \circ \wt{\varphi}_2
  \end{align*}
  Since $\iota$ is orientation reversing, only one of $\wt{\varphi}_1$ or $\wt{\varphi}_2$ is
  orientation preserving, which gives us a unique choice.

  If $\no$ has marked points that $\varphi$ fixes, then the lift $\wt{\varphi}$ may or may not swap
  the pre-images of the marked points. But the square of the lift will definitely fix the
  pre-images as well, which proves the second part of the proposition.
\end{proof}

A consequence of the above proposition is that one can think of the mapping class group of a
non-orientable surface as a subgroup of the mapping class group of the double cover. This inclusion
also respects the Nielsen-Thurston classification of mapping classes, both qualitatively, and
quantitatively, as the following proposition shows.
\begin{prop}
  \label{prop:2}
  If $\varphi$ is a self-homeomorphism of $\no$ and $\wt{\varphi}$ is its orientation preserving lift on $\os$, then
  the following statements are true.
  \begin{enumerate}[(i)]
  \item $\varphi$ is periodic, reducible, or pseudo-Anosov if and only if $\wt{\varphi}$ is periodic, reducible, or pseudo-Anosov
    respectively.
  \item If $\varphi$ is pseudo-Anosov with stretch factor $k$, then $\wt{\varphi}$ also has stretch factor $k$.
  \end{enumerate}
\end{prop}
\begin{proof}
  It's easy to see that if $\varphi$ is periodic, so it $\wt{\varphi}$, and the other way round. If $\varphi$ is reducible, that means
  it leaves some multi-curve on $\no$ invariant, which means $\wt{\varphi}$ leaves the lift of that multi-curve invariant as
  well. Conversely, if $\wt{\varphi}$ leaves some multi-curve $\gamma$ invariant, so does $\iota \circ \wt{\varphi}$, since they
  commute. That means the union of $\gamma$ and $\iota(\gamma)$ is also a multi-curve and thus descends
  to a multi-curve on $\no$ that is left invariant by $\varphi$. By the process of exclusion, any pseudo-Anosov
  on $\no$ must lift to a psuedo-Anosov on $\no$ and vice versa. This proves part (i) of the proposition.

  Suppose now that $\varphi$ is a psuedo-Anosov on $\no$ with stretch factor $k$ and expanding and contracting foliations
  $\lambda_e$ and $\lambda_c$ respectively. Since $\varphi$ is a pseudo-Anosov map, the following identity holds
  for all closed curves $\gamma$.
  \begin{align}
    \label{eq:1}
    i(\varphi^{-1}\gamma, \lambda_e) &= i(\gamma, \varphi(\lambda_e)) \\
                               &= k \cdot i(\gamma, \lambda_e)
  \end{align}
  A similar identity holds for $\lambda_c$.
  \begin{align}
    \label{eq:2}
    i(\varphi^{-1}\gamma, \lambda_c) &= i(\gamma, \varphi(\lambda_c)) \\
                               &= \frac{1}{k} \cdot i(\gamma, \lambda_c)
  \end{align}
  Note now that the foliations can be lifted to the double cover: call their lifts $\wt{\lambda}_e$
  and $\wt{\lambda}_c$. For any closed curve $\wt{\gamma}$ on $\os$, consider its intersection number
  with the foliations. Since computing the intersection number is a {\color{red} local  calculation}, we have the following
  identity.
  \begin{align}
    \label{eq:3}
    i(\wt{\gamma}, \wt{\lambda}_e) = i(\gamma, \lambda_e)
  \end{align}
  Combining identities \eqref{eq:1} and \eqref{eq:3}, we get the following identity for intersection numbers
  on $\os$.
  \begin{align*}
    i(\wt{\varphi}^{-1} (\wt{\gamma}), \wt{\lambda}_e) = k \cdot i(\wt{\gamma}, \wt{\lambda}_e)
  \end{align*}
  We get a similar expression for $\wt{\lambda}_c$, which proves that $\wt{\varphi}$ has the same stretch factor
  as $f$, thus proving the proposition.
\end{proof}
Part (ii) of Proposition \ref{prop:2} is going to be useful in an application of our main result,
where we'll be computing asymptotics for the minimal stretch factor of a pseudo-Anosov map on
$\no_{g,n}$.

Finally, the last thing we need to know about mapping classes on non-orientable surfaces is how to
construct examples of pseudo-Anosov maps. In the case of orientable surfaces, the Penner
construction is used to construct pseudo-Anosov maps, as well compute their stretch factors. It
turns out the Penner construction also works in the non-orientable setting, with some minor
modifications. This construction is presented in detail in Section 2 of \cite{Strenner_2017}, but
we'll give an outline of the key ideas.

The Penner construction in the orientable setting takes a pair of filling multicurves
$A = \{a_1,\dots,a_n\}$ and $B = \{b_1,\dots,b_m\}$ and claims that a composition of Dehn twists
$T_{a_i}$ and $T_{b_i}^{-1}$ that uses all curves in the multicurves at least once will be
pseudo-Anosov. The problem with making this work for non-orientable surfaces is that when defining
Dehn twists about curves on a non-orientable surface, we don't have a well-defined notion of a left
or right Dehn twist (what direction we are choosing to be the Dehn twist and which is its
inverse). The way we get around this for non-orientable surfaces is by having a collection of
filling two-sided curves that are \textit{marked inconsistently}.

Each two-sided curve $c$ on a non-orientable surface $N$ has a neighborhood homeomorphic to an
annulus $A$ by a homeomorphism $\phi: A \xrightarrow{} N$, called a \textit{marking}. In this
context, we can define the Dehn twist $T_{c,\phi}(x)$ around $(c,\phi)$ in the following manner.
\begin{align*}
  T_{c,\phi}(x) =
  \begin{cases}
    \phi \circ T \circ \phi^{-1}(x) & \text{for } x \in \phi(A) \\
    x & \text{for } x \in N - \phi(A)
  \end{cases}
\end{align*}
Here $T$ is the standard Dehn twist on $A$, i.e. $T(\theta,t) = (\theta + 2\pi t,t)$. If we fix an
orientation of $A$, then we can pushforward this orientation to $S$. We say two marked curves
$(c,\phi_c)$ and $(d,\phi_d)$ that intersect at a point $p$ are marked inconsistently if the
pushforward of the orientation of $A$ by $\phi_c$ and $\phi_d$ disagree in a neighborhood of $p$.
If all our curves are marked inconsistently and are filling, then once again a composition of Dehn
twists around them that use all the curves at least once will be pseudo-Anosov.

The Penner construction not only promises that our map is pseudo-Anosov, but it also gives a way to
compute the stretch factor of our map (see \cite{penner1988construction}).  The proof of the fact
that the composition is pseudo-Anosov, and the computation of its stretch factor works the same is
in the orientable setting.  Given the collection of curves, one smooths out the intersections of the
curves to obtain an \textit{invariant train track}, an embedded graph on the surface that remains
unchanged by the homeomorphism, which we'll call $f$. Let $\tau$ be the invariant train track and
$\mathcal{C}$ the collection of curves on our surface that are used when defining $f$. Consider now
the collection of transverse measures on our train track $\tau$. For every curve (i.e. a single
element subset of $\mathcal{C}$) $x \subset \mathcal{C}$, there is an associated transverse measure
$\mu_x$ for $\tau$ that assigns $1$ to all edges lying in $x$ and 0 to everthing else. Let $V_\tau$
be the cone of transverse measures on $\tau$, and $H$ the subspace of $V_\tau$ spanned by the
transverse measure associated to curves in $\mathcal{C}$.
\begin{align*}
  H = \mathrm{span}(\{\mu_x \mid x \text{ is a connected curve in } \mathcal{C}\})
\end{align*}
The $\mu_x$ are linearly independent and form the \textit{standard basis} for $H$. This subspace $H$
is invariant under the action of $f$ on $V_\tau$, thus $f$ has a linear action on $H$. If we let $A$
be the matrix representing this action in the standard basis, then the stretch factor of $f$,
$\lambda(f)$, is the Perron-Frobenius eigenvalue of $f$.

\subsection{Thurston's fibered faces}
\label{sec:thurst-fiber-face}

Given a surface $S$ and a self homeomorphism $f: S \to S$, one can construct a $3$-manifold $M$ via
the \emph{mapping torus} construction.
\begin{align*}
  M = \frac{S \times [0,1]}{(x,1) \sim (f(x), 0)}
\end{align*}
Inverting this construction is also an interesting problem: given a $3$-manifold $M$, is it the
mapping torus of some surface and self-homeomorphism $(S,f)$? In how many ways can one express a
$3$-manifold as a mapping torus? Another way to think about these mapping torii is by observing that
these correspond exactly to surface bundles over $S^1$, or \emph{fibrations over $S^1$}. The surface
$S$ then is just the \emph{fiber} of the fibration. Such a fibration also defines a {flow} on $M$,
called the \emph{suspension flow}, where $(x, t_0)$ gets sent to $(x, t_0 + t)$, where $x$ is a point
in the fiber $S$, and $t_0$ is a point on $S^1$. It turns out that the Thurston norm is an extremely
useful tool when studying fibrations over $S^1$ (and other related objects).

Given a orientable closed $3$-manifold $M$, the Thurston norm is a semi-norm on its second homology
group $H_1(M; \RR)$: to define the norm, we need to make some preliminary remarks and define a
function.  Let $\chi_-(S)$ denote the negative component of the Euler characteristic of a surface
$S$, i.e. $\chi_-(S) = \max\{-\chi(S),0\}$, we call this the \textit{complexity} of $S$. If a
surface $S$ has multiple components then its complexity is the sum of the $\chi_-$ for the
individual components.  Furthermore, for an \emph{orientable} 3-manifold $M$, it is known that
elements in $H_2(M ; \ZZ)$ can be represented by embedded surfaces inside of $M$. This lets us
define a norm function $x$ for every homology class $a$ in $H_2(M, \ZZ)$.
\begin{align*}
  x(a) = \min\{\chi_-(S) \mid [S] = a \text{ and $S$ is compact, properly embedded and oriented}\}
\end{align*}

One can then extend $x$ to the rational points by claiming it must be linear on rays that go through
the origin, and there is a unique way to extend a function defined on rational points to reals so
that it is continuous. This defines an $\RR$-valued function on $H_2(M; \RR)$ which is called the
Thurston norm. It turns out that the unit ball in this norm is a convex polyhedron, and thus it makes
sense to talk about the \emph{faces} of the unit ball.

The way this ties back up with the question of realizing the $3$-manifold $M$ as a fibration over
$S^1$ is via the following remarkable theorem of Thurston (the original statement appeared in
\cite{thurston1986norm}, but we use the restatement from \cite{yazdi2018pseudo} because it's
cleaner).

\begin{thm}[Thurston]
  \label{thm:Thur1}
  Let $\mathcal{F}$ be the set of homology classes in $H_2(M; \RR)$ that are representable by fibers of
  fibrations of $M$ over the circle.
\begin{enumerate}[(i)]
\item Elements of $\mathcal{F}$ are in one-to-one correspondence with (non-zero) lattice points
  inside some union of cones over open faces of the unit ball in the Thurston norm.
\item If a surface $F$ is transverse to the suspension flow associated to some fibration of
  $M \xrightarrow[]{} S^1$ then $[F]$ lies in the closure of the corresponding cone in $H_2(M)$.
\end{enumerate}
\end{thm}
Note that one needs to clarify here what the homology class $[F]$ actually is, since given an
embedded surface $F$, one has two canonical choices of $[F]$ depending upon the orientation
of $F$. The class $[F]$ referred to in the theorem comes from picking the orientation such
that the positive flow direction is pointing outwards relative to the surface.

The open faces whose cones contain the fibers of fibrations are what are referred to as
\emph{fibered faces}.  It turns out one can recover even more information about the self
homeomorphism $f$ of which $M$ is the mapping torus. The following result of Thurston tells us
precisely when the map $f$ is psuedo-Anosov.

\begin{thm}[Thurston's Hyperbolization Theorem]
    \label{thm:ThurHyp}
  If $M$ is the mapping torus of $(S, f)$, then $M$ is hyperbolic if and only if $f$ is pseudo-Anosov.
\end{thm}

In particular, a consequence of the above theorem is that if $M$ is the mapping torus of $(S, f)$
and $(S', f')$ then $f'$ is pseudo-Anosov if and only if $f$ is pseudo-Anosov. From Theorem
\ref{thm:Thur1}, we know that all such pairs $(S, f)$ correspond to lattice points in some union of
cones. A natural thing to do at this point would be to determine how the stretch factors of these
pseudo-Anosov maps vary as we move around in the cone. To this end, we have the following two
theorems, due to Fried-Matsumoto (see \cite{fried1982flow}, \cite{fried1983transitive}, and
\cite{matsumoto1987topological}) and Agol-Leininger-Margalit (see \cite{agol6983pseudo}).

\begin{thm}[Fried-Matsumoto]
  \label{thm:fm}
  Let $M$ be a hyperbolic $3$-manifold and let $\mathcal{K}$ be the union of cones in
  $H_2(M; \RR)$ whose lattice points correspond to fibrations over $S^1$.  There exists a strictly
  convex function $h: \mathcal{K} \to \RR$ satisfying the following properties.
  \begin{enumerate}[(i)]
  \item For all $t > 0$ and $u \in \mathcal{K}$, $h(tu) =  \frac{1}{t}h(u)$.
  \item For every primitive lattice point $u \in \mathcal{K}$, $h(u) = \log(k)$, where $k$ is the
    stretch factor of the pseudo-Anosov map associated to this lattice point.
  \item $h(u)$ goes to $\infty$ as $u$ approaches $\partial \mathcal{K}$.
  \end{enumerate}
\end{thm}

\begin{thm}[Agol-Leininger-Margalit]
  \label{thm:alm}
  Let $\mathcal{K}$ be a fibered cone for a mapping torus $M$ and let $\overline{\mathcal{K}}$ be its closure
  in $H_2(M;\RR)$. If $u \in \mathcal{K}$ and $v \in \overline{\mathcal{K}}$, then $h(u+v) < h(u)$.
\end{thm}
Theorems \ref{thm:fm} and \ref{thm:alm} make a quantitative link between the lattice points in the cones
over the fibered faces and the stretch factor of the associated pseudo-Anosov maps, outlining the power
of Thurston fibered face theory.

One of the primary goals of this paper to extend this definition of Thurston norm to non-orientable
manifolds and be able to state the non-orientable version of the theorems above. Most of the work in
doing so is concentrated in determining the right analog of Thurston norm for non-orientable
surfaces, and then making Theorem \ref{thm:Thur1} work with that definition. Once we have that, the
rest of the theorems follow with relatively little work. That is what we will do in Section
\ref{sec:fibered-face-theory}. Once we have the versions of the theorems for non-orientable
surfaces, we'll prove bounds on the asymptotics of minimal stretch factors for non-orientable
surfaces in Section \ref{sec:application} by adapting the methods in \cite{yazdi2018pseudo} for
non-orientable surfaces. The results of Yazdi crucially rely upon Thurston's fibered face theory,
and the non-orientable variant of fibered face theory lets us generalize Yazdi's results to the
non-orientable setting, with some work.
