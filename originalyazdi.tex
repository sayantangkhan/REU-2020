\section{Constructing an infinite family of psuedo-Anosovs}
\label{sec:constr-an-infin}

We will first imitate the first two steps of Yazdi's construction \cite{yazdi2018pseudo}, constructing two infinite families of non-orientable surfaces. For the most part, we will use the same notation as Yazdi to show exactly how we are copying his construction.

\textbf{Step 1:} In the first step of the construction, Yazdi constructs an infinite family of surfaces through covering actions on an infinite surface built from a fundamental building block. We will imitate this construction and construct the two infinite families of non-orientable surfaces that we will require. 

We will define two families of surfaces $P_{n,k}$ and $Q_{n,k}$, first defining $P_{n,k}$ in the following way originally done by Yazdi. Let $T$ be an orientable surface of genus 5 with 3 boundary components $c,d$ and $e$. Orient $T$ and give $c,d$ and $e$ the induced orientations from this orientation. Now add two cross-caps to $T$ but keep the boundaries of $T$ oriented. Let $p$ (respectively $q$) be a puncture (respectively a marked point) on the boundary component $e$ of $T$, with oriented arcs $r$ and $s$ connecting them on $\partial T$. See Figure X for a picture of $T$. Let $T_{i,j}$ be copies of the surface $T$, where $i,j \in \mathbb{Z}$. We will use similar notations to refer to the boundary components of $T_{i,j}$. Define the infinite surface $T_\infty$ as the quotient
$$T_\infty \coloneqq \left( \bigcup T_{i,j} \right)/\sim,$$
where $i,j \in \mathbb{Z}$. The equivalence relation $\sim$ is defined as $$c_{i,j} \sim d_{i+1,j} \hspace{1em}, \hspace{1em} r_{i,j} \sim s_{i,j+1}$$ and the gluing maps for the boundary components are by orientation-reversing homeomorphisms.

There are two natural maps $\overline{\rho_1},\overline{\rho_2}: T_\infty \xrightarrow[]{} T_\infty$ that act by shifts as follows $$\overline{\rho_1} \text{ sends } T_{i,j} \text{ to } T_{i+1,j},$$ $$\overline{\rho_2} \text{ sends } T_{i,j} \text{ to } T_{i,j+1}.$$ Note that these maps commute. Define the surface $P_{n,k}$ as the quotient of the surface $T_\infty$ by the covering action of the group generated by $(\overline{\rho_1})^n$ and $(\overline{\rho_2})^k$. Therefore, $\overline{\rho_1}$ and $\overline{\rho_2}$ induce maps on the surface $P_{n,k}$, which we denote by $\rho_1$ and $\rho_2$.

We can define an analogous family of surface $Q_{n,k}$ by starting with a different building block, $R$. Let $R$ be a non-orientable surface of genus 13 with 3 boundary components $c,d$ and $e$. We again give $c, d$ and $e$ a consistent choice of orientation as above, and let $p$,$q$,$r$ and $s$ be as above. See Figure Y for a picture of $R$. We then let $R_{i,j}$ be copies of $R$, where $i,j \in \mathbb{Z}$ and then define a surface $N_\infty$ defined exactly like $T_\infty$ above. We then analogously have maps $\overline{\rho_1},\overline{\rho_2}: N_\infty \xrightarrow[]{} N_\infty$ acting as shifts, and use them to define the family of surfaces $Q_{n,k}$.

A question that naturally arises is why did we choose the surfaces $T$ and $R$ that we did as our building blocks? It comes down to two main problems:
\begin{enumerate}
    \item The combinatorics of the curves make the associated matrix we get from the Penner construction satisfy the conditions of Lemma 1 
    \item Having a curve $\gamma$ such that it and its image under our map form the boundary of an embedded $\mathcal{N}_3$ in our surface, which we will need in later steps (\textcolor{red}{Perhaps this was outlined above}). 
\end{enumerate}

\begin{lem}
Define the sequences
\begin{align}
    g_{n,k} &= (14k - 2)n + 2, k \geq 3, n \geq 1 \\
    g'_{n,k} &= (15k - 2)n + 2, k \geq 3, n \geq 1.
\end{align}
    The genus of $P_{n,k}$ and $Q_{n,k}$ are $g_{n,k}$ and $g'_{n,k}$ respectively.
\end{lem}
\begin{proof}
    We will first prove the lemma for $P_{n,k}$. Consider the subsurface $U \subset P_{n,k}$ defined as $$U = \left( \bigcup_{i =0}^{k-1} T_{0,i} \right)/\sim.$$ Then $U$ is a compact, non-orientable surface of genus $12k$ with $2k$ boundary components, and forms a fundamental domain for the covering action of $\overline{\rho_1}$ on $T_\infty$. We have $$\chi(U) = 2 - 12k - 2k = 2 - 14k.$$ Thus $$\chi(P_{n,k}) = n \cdot \chi(U) = -n(14k - 2),$$ since $P_{n,k}$ is formed by gluing $n$ copies of $U$ together along circle boundary components. Therefore $$\chi(P_{n,k}) = 2 - g_{n,k} = -n(14k - 2) \implies g_{n,k} = n(14k - 2) + 2.$$ Likewise, the same argument holds for $Q_{n,k}$, but replacing 12 with 13 gives $g'_{n,k} = n(15k - 2) + 2$.
\end{proof}

\textbf{Step 1: (Alternate)}

We will first give an overview of how a \textit{Yazdi-example} is constructed, then specify what specific examples we will use for our construction.

In the first step of the construction, Yazdi constructs an infinite family of surfaces through covering actions on an infinite surface built from a fundamental building block. To do this, he starts with an orientable surface of genus $g$ with 3 boundary components $c$, $d$ and $e$ (decomposed into arcs $r$ and $s$), call it $S$. Then an infinite surface is constructed by gluing together copies of $S$, $S_{i,j}$, indexed by $i,j \in \mathbb{Z}$ by gluing $c_{i,j}$ to $d_{i+1,j}$ and $r_{i,j}$ to $s_{i,j+1}$. If we let $\rho_1$ denote the map on this infinite surface that shifts the $i$ index by 1
and $\rho_2$ the map on this infinite surface that shifts the $j$ index by 1, then we can be quotient our infinite surface by the covering action of the group generated $\rho_1^n$ and $\rho_2^k$ for some integers $n$ and $k$. This gives us an infinite family of closed surfaces, denoted $P_{n,k}$. Refer to section 3.1 of Yazdi's papers for details.

For our case, we want to use this construction to construct two infinite families of non-orientable surfaces. Let $T$ be an orientable surface of genus 5 with 3 boundary components $c,d$ and $e$. Orient $T$ and give $c,d$ and $e$ the induced orientations from this orientation. Now add two cross-caps to $T$ but keep the boundaries of $T$ oriented. This gives us a genus 12 non-orientable surface, see Figure X for a picture. We can now continue Yazdi's construction to get an infinite family of non-orientable surfaces we will denote $P_{n,k}$.

We can define an analogous family of surface $Q_{n,k}$ by starting with a different building block, $R$. Let $R$ be a non-orientable surface of genus 13 with 3 boundary components $c,d$ and $e$ (genus 5 orientable surface with 3 cross-caps). See Figure X for a picture of $R$. We can perform the exact same process as above to get another family of non-orientable surfaces $Q_{n,k}$, albeit with different genera. 

A question that naturally arises is why did we choose the surfaces $T$ and $R$ that we did as our building blocks? It comes down to two main problems:
\begin{enumerate}
    \item The combinatorics of the curves make the associated matrix we get from the Penner construction satisfy the conditions of Lemma 1 
    \item Having a curve $\gamma$ such that it and its image under our map form the boundary of an embedded $\mathcal{N}_3$ in our surface, which we will need in later steps (\textcolor{red}{Perhaps this was outlined above}). 
\end{enumerate}

\begin{lem}
Define the sequences
\begin{align}
    g_{n,k} &= (14k - 2)n + 2, k \geq 3, n \geq 1 \\
    g'_{n,k} &= (15k - 2)n + 2, k \geq 3, n \geq 1.
\end{align}
    The genus of $P_{n,k}$ and $Q_{n,k}$ are $g_{n,k}$ and $g'_{n,k}$ respectively.
\end{lem}
\begin{proof}
    We will first prove the lemma for $P_{n,k}$. Consider the subsurface $U \subset P_{n,k}$ defined as $$U = \left( \bigcup_{i =0}^{k-1} T_{0,i} \right)/\sim.$$ Then $U$ is a compact, non-orientable surface of genus $12k$ with $2k$ boundary components, and forms a fundamental domain for the covering action of $\overline{\rho_1}$ on $T_\infty$. We have $$\chi(U) = 2 - 12k - 2k = 2 - 14k.$$ Thus $$\chi(P_{n,k}) = n \cdot \chi(U) = -n(14k - 2),$$ since $P_{n,k}$ is formed by gluing $n$ copies of $U$ together along circle boundary components. Therefore $$\chi(P_{n,k}) = 2 - g_{n,k} = -n(14k - 2) \implies g_{n,k} = n(14k - 2) + 2.$$ Likewise, the same argument holds for $Q_{n,k}$, but replacing 12 with 13 gives $g'_{n,k} = n(15k - 2) + 2$.
\end{proof}


\textbf{Step 2:} Following Yazdi, we will now define maps $f_{n,k}: P_{n,k} \xrightarrow[]{} P_{n,k}$ and $h_{n,k}: Q_{n,k} \xrightarrow[]{} Q_{n,k}$ that are defined as a composition of specific Dehn twists, followed by a finite order mapping class. The key insight is that a power of this map will be a composition of Dehn twists that satisfy the criteria to be a Penner construction and thus pseudo-Anosov.

The following statements will hold for both $P_{n,k}$ and $Q_{n,k}$. Let $\mathcal{B}$ be the union of all $\beta$ curves except $\beta_1$ in $T_{0,0} \cup T_{0,1} \cup T_{1,0}$ (respectively $R_{0,0} \cup R_{0,1} \cup R_{1,0}$) (see figures below). Let $\rho_1(\mathcal{B})$ be the image of $\mathcal{B}$ under $\rho_1$. Define $\phi_b$ as the composition of positive Dehn twists along all the curves in the set $\overline{\mathcal{B}} \coloneqq \mathcal{B} \cup \rho_1(\mathcal{B}) \cup \dots \cup \rho_1^{n-1}(\mathcal{B})$. Since the curves in $\overline{\mathcal{B}}$ are disjoint, Dehn twists along them commute. Therefore, it is no necessary to specify the order in which we compose these Dehn twists in $\phi_b$. Let $\mathcal{R}$ be the union of all $\alpha$ curves except $\alpha_1$ in $T_{0,0}$. Define $\mathcal{R}$ and $\phi_r$ similarly but use negative Dehn twists this time.

Let $\alpha_1,\beta_1 \subset T_{0,0}$ be the curves in Figure Z. Let $\phi$ be the composition of negative dehn twists along all the curves $\alpha_1, \rho_1(\alpha_1), \dots, \rho_1^{n-1}(\alpha_1)$ followed by positive Dehn twists along all the curves $\beta_1,\rho_1(\beta_1),\dots,\rho_1^{n-1}(\beta_1)$. Define
\begin{align*}
    f_{n,k} &\coloneqq \rho_2 \circ \phi \circ \phi_b \circ \phi_r \\
    h_{n,k} & \coloneqq \rho_2 \circ \phi \circ \phi_b \circ \phi_r
\end{align*}
using the curves in $P_{n,k}$ and $Q_{n,k}$ respectively. We are using notation so the composition is from right to left. It follows from Penner's construction of pseudo-Anosov maps that $(f_{n,k})^k$ and $(h_{n,k})^k$ are pseudo-Anosov. Hence $f_{n,k}$ and $h_{n,k}$ are themselves pseudo-Anosov and invariant train tracks $\tau^1_{n,k}$ and $\tau^2_{n,k}$ for $f_[n,k]$ and $h_{n,k}$ respectively can be obtained from Penner's construction by smoothing the intersection points.

\textcolor{red}{We need to state how we know that these two sets of curves can be marked inconsistently}

\section{Lifts and Mapping Tori}

Let $M^1_{n,k}$ be the mapping torus of $f_{n,k}$ and $M^2_{n,k}$ be the mapping torus of $h_{n,k}$. Likewise, let $\widetilde{M^1_{n,k}}$ and $\widetilde{M^2_{n,k}}$ denote the mapping tori of $\widetilde{f_{n,k}}$ and $\widetilde{h_{n,k}}$ respectively, where $\widetilde{f_{n,k}}$ and $\widetilde{h_{n,k}}$ are lifts of $f_{n,k}$ and $h_{n,k}$ to the orientation double covers of $P_{n,k}$ and $Q_{n,k}$. Note that it follows that $\widetilde{M^i_{n,k}}$ is the orientation double cover of $M^i_{n,k}$ for $i = 1,2$. 

\textbf{Step 3:} Let $C^i_{n,k}$ denote the fibered face of $H_2(\widetilde{M^i_{n,k}},\mathbb{R})$ corresponding to the map $\widetilde{f_{n,k}}$ for $i = 1$ and $\widetilde{h_{n,k}}$ for $i = 2$. We will show that $M^i_{n,k}$ contains a closed non-orientable surface of genus 3 that lifts to a closed orientable surface of genus 2 in $\widetilde{M^i_{n,k}}$ that is contained in the closure of $C^i_{n,k}$ for $i = 1,2$.

\begin{lem}
For $i = 1,2$, there is a non-trivial homology classes $0 \neq [\widetilde{F^i_{n,k}}] \in H_2(\widetilde{M^i_{n,k}};\mathbb{Z})$ represented by orientable surfaces of genus two that is a lift of a non-orientable surfaces of genus three $F^i_{n,k}$ in $M^i_{n,k}$. Moreover, $\widetilde{F^i_{n,k}}$ is Thurston norm-minimizing and lie in the closures $\overline{\mathcal{C}^i_{n,k}}$.
\end{lem}
\begin{proof}

\end{proof}

\begin{lem}
Let $\iota: \widetilde{M^i_{n,k}} \xrightarrow[]{} \widetilde{M^i_{n,k}}$ denote the deck transformation that generates the deck group of the orientation double cover. Likewise, let $\iota_*: H_2(\widetilde{M^i_{n,k}};\mathbb{R}) \xrightarrow[]{} H_2(\widetilde{M^i_{n,k}};\mathbb{R})$ denote its action on second homology. Then $\iota_*([\wt{F^i_{n,k}}]) = -[\wt{F^i_{n,k}}]$.
\end{lem}
\begin{proof}

By the way we have defined $\wt{F^i_{n,k}}$ as a lift of an embedded subsurface in $M^i_{n,k}$, we know that $\iota$ sends $\wt{F^i_{n,k}}$ to itself. We want to see that $\iota$ is also orientation reversing when restricted to $\wt{F^i_{n,k}}$.

To begin, we first need to see what our surface $\wt{F^i_{n,k}}$ looks like embedded in $\wt{M^i_{n,k}}$. Recall the way that $F^i_{n,k}$ is defined as an embedded $\mathcal{N}_1$ with two boundary components, $\gamma$ and $f^k(\gamma) = \hat{\gamma}$in one of the fibers of $M^i_{n,k}$ union the tubes formed by following $\gamma$ $k$ times around the suspension flow in $M^i_{n,k}$. Let's observe what happens to our embedded genus 1 with 2 boundary components in a single fiber after it is lifted to the orientation double cover. The orientation double cover of a genus 1 nonorientable surface is $S^2$, and we can see here that our embedded surface will lift to a sphere with four boundary components, one can see this by imagining two copies of our embedded subsurface being glued along their single cross-cap. For ease of notation, let's denote the two lifts of $\gamma$ and $\hat{\gamma}$ as $\gamma_0,\gamma_1$ and $\hat{\gamma}_0,\hat{\gamma}_1$ respectively. These curves form the boundary of the sphere with four boundary components that is sitting in our single fiber in $\wt{M^i_{n,k}}$. 

Recall that $\wt{M^i_{n,k}}$ is not only the double orientation cover of $M^i_{n,k}$, but is also the mapping torus of $\wt{f_{n,k}}$ for $i = 1$ and $\wt{h_{n,k}}$ for $i = 2$. Looking at these maps, if we let $p$ denote the covering map for both $\wt{P_{n,k}} \xrightarrow[]{} P_{n,k}$ and $\wt{Q_{n,k}} \xrightarrow[]{} Q_{n,k}$, then we know that $p \circ \wt{f_{n,k}} = f_{n,k} \circ p$ (and likewise for $h_{n,k}$). This tells us that $\wt{f_{n,k}}$ sends $\gamma_0$ to $\hat{\gamma_0}$ and $\gamma_1$ to $\hat{\gamma}_1$. Thus we can see that the tube traced out by following the suspension flow of $\gamma$ to $\hat{\gamma}$ gets lifted to tubes following the suspension flow of $\gamma_0$ to $\hat{\gamma_0)}$ and $\gamma_1$ to $\hat{\gamma_1}$. These tubes glued to our sphere with four boundary components give us our genus 2 surface in the cover. 

We will now show that $\iota$ restricted to $\wt{F^i_{n,k}}$ is orientation reversing by showing that it is orientation reversing on the individual components, i.e. the sphere with boundary and the two tubes. First note that the boundary components of the of the sphere with boundary are also curves that lie in one of the fibers of $\wt{M^i_{n,k}}$. Suppose that we give an orientation to our fiber which induces orientations on our curves. Since $\iota$ is orientation reversing on the fiber, it must reverse the orientation of our curves, and thus reverses the orientations of the boundaries of our sphere when we restrict $\iota$. Thus $\iota$ must be orientation reversing on the whole of the sphere with boundary components. \textcolor{red}{I know you made a slightly different argument for tubes Sayantan, but we can't we just use the same exact arguement for the tubes since the tubes are bounded by these curves?}

Now that we know that $\iota$ is orientation reversing on $\wt{F^i_{n,k}}$, we know that $\iota_*: H_2(\wt{F^i_{n,k}}) \xrightarrow{} \wt{F^i_{n,k}}$ acts by sending the fundamental class $[\wt{F^i_{n,k}}]$ to its negative. We also know that $\wt{F^i_{n,k}}$ is viewed as a representative for an element of $H_2(\wt{M^i_{n,k}})$ by the image of $[\wt{F^i_{n,k}}] \in H_2(\wt{F^i_{n,k}})$ under the map on second homology induced by the inclusion $i: \wt{F^i_{n,k}} \xrightarrow[]{} \wt{M^i_{n,k}}$. Since $\iota$ can be restricted to $\wt{F^i_{n,k}}$, it is a map of the pair $(\wt{M^i_{n,k}},\wt{F^i_{n,k}})$ and thus by the naturality of the long exact sequence of a pair, $\iota_*$ and $i_*$ commute. This tells us that $\iota_*: H^2(\wt{M^i_{n,k}}) \xrightarrow[]{} H^2(\wt{M^i_{n,k}})$ acts by $\iota_*([\wt{F^i_{n,k}}]) = -[\wt{F^i_{n,k}}]$, giving us our desired result.

\end{proof}

\section{Bounding the Stretch Factor}

\textbf{Step 4:} In \cite{yazdi2018pseudo}, Yazdi shows that the family of pseudo-Anosov maps that we have constructed all have the log of their stretch factor bounded above by a similar factor. He does this by considering the associated matrix that comes along with a pseudo-Anosov homeomorphism and showing based on the combinatorics of the curves of the constructed examples, that the spectral radius is bounded by Lemma 1. We want to prove the same thing for our families of examples. Namely that if we let $\lambda_{n,k}$ be the stretch factor of $f_{n,k}$ and $\mu_{n,k}$ be the stretch factor of $h_{n,k}$, then we have the following lemma: 
\begin{lem}
There exists universal positive constants $C'$ and $C''$ such that for every $n \geq 1$ and $k \geq 3$:
$$\log(\lambda_{n,k}) \leq C'\frac{n}{g_{n,k}} \hspace{2em} \log(\mu_{n,k}) \leq C''\frac{n}{g'_{n,k}}$$
\end{lem}
\begin{proof}
We have purposefully constructed our examples so our curves are in the same "general form" as Yazdi's were and thus they will still satisfy the criteria of Lemma 1. Though we still want to explicitly show that this is the case.

We will first show that this is true for $f_{n,k}$ and the curves on $P_{n,k}$. First for consistency of notation, let
$$\mathcal{A} \coloneqq \mathcal{B} \cup \mathcal{R} \cup \{\alpha_1,\beta_1\}, \overline{\mathcal{A}} \coloneqq \mathcal{A} \cup \rho_1(\mathcal{A}) \cup \dots \cup \rho_1^{n-1}(\mathcal{A})$$
$$\hat{\mathcal{A}} \coloneqq \overline{\mathcal{A}} \cup \rho_2(\overline{\mathcal{A}}) \cup \dots \cup \rho_2^{k-1}(\overline{\mathcal{A}})$$. Thus $\hat{\mathcal{A}}$ is all the curves on our surface we are Dehn twisting around to get $f_{n,k}$. 

Now let $A$ associated to our pseudo-Anosov $f_{n,k}$ and $\gamma$ the adjacency graph of $A$. In order to bound the spectral radius of $A$, we need to show that $\gamma$ satisfies the criteria of Lemma 1. In order to do this we first need to partition the vertices of $\gamma$, which is equivalent to a partition of the curves in $\hat{\mathcal{A}}$: $$\mathcal{A} = \bigcup_{i=1}^k \rho_2^{i-2}(\overline{\mathcal{A}}).$$ Then define $V_i$ for $1 \leq i \leq k$ as the vertices of $\gamma$ corresponding to elements in $\rho_2^{i-2}(\overline{\mathcal{A}})$.

We can now check the conditions of Lemma 1, based on the combinatorics of the curves on our surface:
\begin{enumerate}
    \item 
    \item 
    \item 
    \item 
    \item All the curves corresponding to an element of $V_j$, $3 < j \leq k$ are disjoint from all the curves in $\overline{A}$. Thus $f_{n,k}$ and $h_{n,k}$ just act by rotation
\end{enumerate}

\end{proof}

\section{Filling in the Gaps}

\textbf{Step 5:}
\textcolor{red}{I'm just going to try and write out the argument for now, not worrying about format or how it sounds in context of a paper}

Recall that $\wt{P_{n,k}}$ and $\wt{Q_{n,k}}$ are the double orientation covers of our surfaces and also the fibers of $\wt{M^i_{n,k}}$ for $i = 1,2$ respectively. As in Yazdi, we are going to be considering the homology classes $[\wt{P^r_{n,k}}] \coloneqq [\wt{P_{n,k}}] + r[\wt{F^1_{n,k}}]$ and $[\wt{Q^r_{n,k}}] \coloneqq [\wt{Q_{n,k}}] + r[\wt{F^2_{n,k}}]$. Representatives for these homology classes can be found by taking the oriented sum.

At this point we should be able to cite Yazdi's Lemma 3.5 as the proof will go the exact same and say:

\begin{lem}
The surfaces $\wt{P^r_{n,k}}$ and $\wt{Q^r_{n,k}}$ are Thurston norm-minimizing, with genera equal to $\wt{g^r_{n,k}} \coloneqq \wt{g_{n,k}} + r$ and $g^{r,'}_{n,k} \coloneqq \wt{g'_{n,k}} + r$. As $r$ varies between $0$ and $14n$ (or $15n$ for $Q$), the genera of $P^r_{n,k}$ and $Q^r_{n,k}$ cover the range between $\wt{g_{n,k}}$ and $\wt{g_{n,k+1}}$ ($\wt{g'_{n,k}}$ and $\wt{g'_{n,k+1}}$ resp.). Moreover, $P^r_{n,k}$ and $Q^r_{n,k}$ are fibrations of $M^i_{n,k}$ with pseudo-Anosov monodromy that fixes $4n$ of the singularities of the invariant foliation.
\end{lem}

The only adjustment that needs to made to Yazdi's argument for the above to be true for us is in the fixing of the singularities. We just need to state that the singularities of the stable foliation of $f_{n,k}$ and $h_{n,k}$ that are fixed by the maps are the $2n$ points of intersection of the axis of $\rho_1$ with $P_{n,k}$ and $Q_{n,k}$. Thus the lifted maps $\wt{f_{n,k}}$ and $\wt{h_{n,k}}$ will fix $4n$ of the singularities of their invariant foliations.

We can now prove our version of Yazdi's Lemma 3.6:

\begin{lem}
There are constants $C,D > 0$ such that for every $n \geq 1$, $k \geq 3$, and $0 \leq r \leq 6n$ we have $$\log(\lambda^r_{n,k}) \leq C\frac{n}{\wt{g^r_{n,k}}}, \vspace{1em} \log(\mu^r_{n,k}) \leq D\frac{n}{g^{r,'}_{n,k}}$$
\end{lem}
\begin{proof}
    Let $\mathcal{C} = \wt{\mathcal{C}^i_{n,k}}$ be our fibered faces and $h: \mathcal{C} \xrightarrow[]{} \mathbb{R}$ the function described in Theorem X. Note that we have 
    $$\wt{g^r_{n,k}} = \wt{g_{n,k}} + r \leq \wt{g_{n,k}} + 14n < 2\wt{g_{n,k}} < 2g_{n,k}$$
    Thus
    $$h([\wt{P^r_{n,k}}]) < h([\wt{P_{n,k}}]) \leq C'\frac{n}{g_{n,k}} \leq 2C'\frac{n}{\wt{g^r_{n,k}}} $$
    
    The proof for $\mu^r_{n,k}$ is exactly the same.
\end{proof}

So now we have that our surfaces $\wt{P^r_{n,k}}$ and $\wt{Q^r_{n,k}}$ are fibers of fibrations of $\wt{M^i_{n,k}}$ and their monodromies are pseudo-Anosov with their stretch factors bounded. The question now is, which of these fibrations are lifts of fibrations from $M^i_{n,k}$? Recall our established criteria from the beginning of the paper though, we need to see that the 1-forms associated to $[\wt{P^r_{n,k}}]$ and $[\wt{Q^r_{n,k}}]$ are left invariant by deck transformation and that the integral from the basepoint to its image under $\iota$ along the 1-form is an integer. 

What do we know about these homology classes though? We have already shown that $[\wt{P^r_{n,k}}],[\wt{Q^r_{n,k}}]$ and $[\wt{F^i_{n,k}}]$ are all in the -1 eigenspace of the action of $\iota$ on second homology. This implies that the homology classes $[\wt{P^r_{n,k}}]$ and $[\wt{Q^r_{n,k}}]$ are also in the -1 eigenspace by linearity. Though we also have shown that if an element is in the -1 eigenspace in second homology, then its Poincar\'e dual will be in the 1 eigenspace of the action of $\iota$ on first cohomology. Thus all of our fibrations satisfy our first criteria to be a lift, but which ones satisfy the second?  

Recall that $[\wt{P^r_{n,k}}] = [\wt{P_{n,k}}] + r[\wt{F^1_{n,k}}]$ (and likewise for $\wt{Q^r_{n,k}}$), and thus the Poincar\'e dual of $[\wt{P^r_{n,k}}]$ is just a sum of the duals of $[\wt{P_{n,k}}]$ and $r$ of $[\wt{F^1_{n,k}}]$. Let $\omega$ be the dual of $[\wt{P^r_{n,k}}]$. We know that $\omega$ is the one-form associated to a fibration of $\wt{M^i_{n,k}}$ and is a lift of a fibration of $M^i_{n,k}$. Thus if we let $x_0 \in \wt{M^i_{n,k}}$ be our chosen basepoint, we know that $\int_{x_0}^{\iota(x_0)} \omega \in \mathbb{Z}$. If now let $\alpha$ be the dual of $[\wt{F^1_{n,k}}]$, we know that $\omega + r\alpha$ is a 1-form corresponding to a fibration of $\wt{M^1_{n,k}}$ for all values of $r$. Thus we know that the integral of $\omega + r\alpha$ around a loop is an integer, thus the integral of $\alpha$ around a loop must be an integer. Thus we can only guarantee that $\int_{x_0}^{\iota(x_0)} \alpha \in 0.5\mathbb{Z}$ and so $\int_{x_0}^{\iota(x_0)} \omega + r\alpha \in \mathbb{Z}$ when $r$ is even. This tells us that we can only guarantee that our fibrations $[\wt{P^r_{n,k}}]$ and $[\wt{Q^r_{n,k}}]$ are lifts when $r$ is even. This is why we need two infinite families in order to guarantee that we cover all possible cases.

Now for our fibrations that are lifts, the corresponding pseudo-Anosov monodromies $\wt{f^r_{n,k}}$ and $\wt{h^r_{n,k}}$ descend to pseudo-Anosov monodromies on $M^i_{n,k}$ with the same stretch factor. Thus our upper bound of $2C'\frac{n}{\wt{g^{r}_{n,k}}}$ still holds, but we do need to make a slight modification. This bound is in terms of $\wt{g^r_{n,k}}$, the genus on the fiber in the double orientation cover, but the genus of our fiber downstairs will be one greater, thus $2C'\frac{n}{\wt{g^r_{n,k}}} = 2C'\frac{n}{g^r_{n,k} - 1} \leq 2C'\frac{n}{\frac{1}{2}g^r_{n,k}} = 4C'\frac{n}{g^r_{n,k}}$.

We can now think of $f^r_{n,k}$ as a map on a non-orientable surface of genus $g^r_{n,k}$ and $h^r_{n,k}$ as a map on a non-orientable surface of genus $g^{r,'}_{n,k}$. Note from above we know that $g^r_{n,k}$ covers all natural numbers between $g_{n,k}$ and $g_{n,k+1}$, thus this set of genera for all $r$ covers all natural numbers larger than $g_{n,3} = 40n + 2$ or $g'_{n,3} = 43n + 2$ (\textcolor{red}{Need to check, is it still k = 3?}). Though recall, we are only dealing with genera when $r$ is even, but if $n$ is odd, then since $40n + 2$ will be even and $43n + 2$ odd, we will recover all genera greater than or equal to $43n + 2$. Also recall that all of these surfaces will have $2n$ singularities, so we can either puncture $n$ or $n + 1$ to account for all possible number of punctures.
