\section{Fibered face theory for non-orientable $3$-manifolds}
\label{sec:fibered-face-theory}

In what follows, we will restrict our attention to mapping
tori of surfaces with a pseudo-Anosov homeomorphism.  %When the surface is orientable, we will call the resulting mapping torus orientable and when the surface is non-orientable, we will call the resulting mapping torus non-orientable.  
The goal of this section is to prove the necessary results for mapping tori of non-orientable surfaces with a pseudo-Anosov map.% (these will be the $3$-manifolds we will be referring to when talking about orientable $3$-manifolds), and the mapping tori of non-orientable $3$-manifolds, again with a pseudo-Anosov map (these will be the manifolds we will be referring to when talking about non-orientable $3$-manifolds). While a lot of our statements will hold more generally for compact non-orientable $3$-manifolds, it will be easier to describe examples when working in this restricted setting; additionally, our application will only involve mapping tori of pseudo-Anosov maps.

\subsection{The problem with homology in non-orientable $3$-manifolds}
\label{sec:probl-with-homol}

A first attempt at defining the Thurston norm for a compact non-orientable $3$-manifold might be $M$ as follows.
Let $S$ be an embedded surface.  Define the complexity function $\chi_-$,
much like in the case of orientable $3$-manifolds, and then define the norm of a homology class
$a \in H^2(M; \ZZ)$ by minimizing $\chi_-(S')$ over all $S'$ representing $a$. This may work, but is quite
unsatisfying: this construction assigns zero norm to all embedded non-orientable surfaces, since their
fundamental classes are trivial, and thus map to $0$ in $H_2(M; \ZZ)$. But we would like the
incompressible surfaces in non-orientable $3$-manifolds to have a positive norm! There are plenty of
incompressible surfaces even in non-orientable $3$-manifolds, namely fibers of fibrations over $S^1$, and a useful definition of Thurston norm should find them.

It turns out that the fundamental problem with non-orientable $3$-manifolds is that homology is a
very coarse invariant: too coarse to detect embedded non-orientable surfaces. Our workaround will
be to deal with the first cohomology $H^1(M)$ rather than the second homology $H_2(M)$. By Poincar\'e duality they are the same in orientable 3-manifolds, but the same is not true for non-orientable $3$-manifolds.

To see why Poincar\'e duality for non-orientable 3-manifolds fails, consider an orientable $3$-manifold $M$. We can explicitly work out the map from
$H^1(M; \ZZ)$ to $H_2(M; \ZZ)$ given by Poincar\'e duality.
To do so, we set up a correspondence between elements of $H^1(M; \ZZ)$ and homotopy classes of maps from
$M$ to $S^1$ as follows. Given a cohomology class $[\alpha]$ in $H^1(M; \ZZ)$, choose a representative $1$-form $\alpha$,
and a basepoint $y_0$ in $M$. The associated map $f_{\alpha}$ is given by the following formula
\begin{align}\label{form:map}
  f_{\alpha}(y) \coloneqq  \int_{y_0}^y \alpha \mod \ZZ
\end{align}
Changing the basepoint or the representative $1$-form gives a different map to $S^1$ that is homotopic to the
original map (see Section 5.1 of \cite{calegari2007foliations} for the details). One can recover the $1$-form
$\alpha$ from the map $f_{\alpha}$ by pulling back the canonical volume form $d\theta$ on $S^1$ along $f_{\alpha}$.

Let $q\in S^1$ be a regular value and let $S = f_{\alpha}^{-1}(q)$ be a surface. To construct a homology class, we choose an orientation on $S$ by declaring
that the outwards pointing normal vectors on $S$ are assigned a positive value by the form
$\alpha$. Then $S$ inherits an orientation from the orientation on $M$, and we have defined a
fundamental class $[S]$. We claim that $[S]$ is the Poincar\'e dual to $\alpha$.
\begin{lem}
  Let $q$ and $q'$ be two regular values of the function $f_{\alpha}$ and let $S=f_\alpha^{-1}(q)$ and $S'=f_\alpha^{-1}(q')$. Then for any closed $2$-form $\omega$ on $M$,
  the following identity holds:
  \begin{align*}
    \int_{S} \omega = \int_{S'} \omega.
  \end{align*}
  Furthermore, the following identity also holds:
  \begin{align*}
    \int_S \omega = \int_M \alpha \wedge \omega.
  \end{align*}
  In particular, the homology class of $S$ is Poincar\'e dual to $\alpha$.
\end{lem}
\begin{proof}
  The first part of the lemma follows from the fact that $S$ and $S'$ are homologous,
  i.e. $f^{-1}_{\alpha}([q, q'])$ is a singular $3$-chain that has $S$ and $S'$ as boundaries. From Stokes'
  theorem, we get the following:
  \begin{align*}
    \int_{S - S'} \omega &= \int_{f_{\alpha}^{-1}([q, q'])} d\omega \\
                         &= 0.
  \end{align*}

  To prove the second claim, observe that we can break up the second integral as a product integral:
  \begin{align*}
    \int_M \alpha \wedge \omega &= \int_{S^1} \left(   \int_{f_{\alpha}^{-1}(\theta)} \omega \right) d\theta.
  \end{align*}
  The above equation is true because $\alpha$ is the pullback of $d\theta$ along the map $f_{\alpha}$. Observe
  that the inner integral only makes sense when $\theta$ is a regular value, but by Sard's theorem, almost
  every $\theta \in [0,1]$ is a regular value, so the right hand side is well-defined. By the first claim, the inner integral is a constant function, as we vary over the $\theta$ which are regular values of $f_{\alpha}$.
  Then the integral of $d\theta$ over $S^1$ is $1$, giving us the identity we want:
  \begin{align*}
    \int_M \alpha \wedge \omega = \int_S \omega.
  \end{align*}
\end{proof}
What we have here is an explicit formula for the Poincar\'e duality map. For orientable $3$-manifolds, this
is an isomorphism, and more specifically the following theorem is true.
\begin{thm}[Poincar\'e duality for orientable $3$-manifolds]
  Let $M$ be an orientable $3$-manifold, and let $S$ be an oriented embedded surface. Then there exists a $1$-form
  $\alpha$ and a regular value $q\in M$ such that $S = f_{\alpha}^{-1}(q)$.
\end{thm}

Note that the maps from the $1$-form to a homology class of an embedded surface still makes sense for a non-orientable $3$-manifold $M$. However in that case the map from $H^1(M; \ZZ)$ to $H_2(M; \ZZ)$ has a nontrivial kernel.

\p{Failure of Poincar\'e duality for non-orientable 3-manifolds}
  Let $\no$ be a non-orientable surface, and let $\varphi$ be any homeomorphism of $\no$. Let $M$ be the mapping torus
  of $(\no, \varphi)$. We then have a map $f: M \to S^1$ given by mapping to the base of the mapping torus.
  Pulling back the form $d\theta$ under $f$, we get a closed but not exact $1$-form $\alpha$ on $M$. Observe
  that $f_{\alpha} = f$, because of how we constructed $\alpha$. Furthermore $f_{\alpha}^{-1}(0)$ is $\no$ inside
  $M$. Thus, the ``Poincar\'e duality map'' for $M$ maps a non-trivial element $\alpha \in H^1(M; \ZZ)$ to the
  zero element $[\no] \in H_2(M; \ZZ)$. In particular, we end up losing information
  going from $H^1(M)$ to $H_2(M)$.


The above example also suggests an alternative strategy of defining the Thurston norm for non-orientable
$3$-manifolds: rather than working with $H_2(M; \RR)$, we can instead work with $H^1(M; \RR)$. We will also be interested in getting a partial
inverse for this map: given a non-orientable surface $\no$ inside a non-orientable
$3$-manifold $M$, we would like to understand if $M$ can be realized as the mapping torus of some homeomorphism of
$\no$.

\subsection{Thurston norm for non-orientable $3$-manifolds}
\label{sec:thurston-norm-non}

For this section, we'll use $M$ to denote a non-orientable $3$-manifold, and $\wt{M}$ to denote its
orientation double cover. We will denote by $\iota$ the orientation reversing deck transformation
of $\wt{M}$, and the covering map $\wt{M} \to M$ by $p$. If $M=M_\varphi$ is the mapping torus of the
non-orientable surface $\no$ and a pseudo-Anosov map $\varphi: \no \to \no$, then $\wt{M}$ is the
mapping torus of $(\os, \wt{\varphi})$, where $\os$ is the orientable double cover of $\no$, and
$\wt{\varphi}$ is the orientation preserving lift of $\varphi$.

Since we have already concluded that the first cohomology is the ``right'' space on which to define the
Thurston norm, we need to relate $H^1(M; \RR)$ and $H^1(\wt{M}; \RR)$.  In particular, we will pullback $H^1(M;\RR)$ to $H^1(\wt{M};\RR)$ under $p$.

\begin{lem}
  \label{lem:injective}
  The pullback $p^{\ast}:H^1(M;\RR)\rightarrow H^1(\wt{M};\RR)$ maps $H^1(M; \RR)$ bijectively to the $\iota^{\ast}$-invariant subspace of
  $H^1(\wt{M}; \RR)$.
\end{lem}
\begin{proof}
  For any $1$-form $\alpha$ on $M$, $p^{\ast}(\alpha)$ will be
  $\iota^{\ast}$-invariant. %This means that the image of $p^{\ast}$ lands inside the $\iota^{\ast}$-invariant subspace. 
  To check that $p^\ast$ is injective, consider a $1$-form $\alpha$ on $M$ such that $p^{\ast}\alpha$
  is exact. We thus have a smooth function $g:\wt{M}\to\RR$ such that the following relation holds:
    \begin{align*}
        dg = p^{\ast} \alpha.
    \end{align*}
    But since $p^{\ast}\alpha$ is $\iota^{\ast}$-invariant, we must have $dg = \iota^{\ast} dg$,
    Because $\iota^\ast$ commutes with the exterior derivative, we have $dg = d(\iota^{\ast}g)$. That means $g$
    and $\iota^{\ast}g$ differ by a constant, but that constant must be $0$ since $\iota$ is finite
    order. Thus $g$ is $p^*$-equivariant and therefore descends to a function on $M$. Therefore $\alpha$ is exact, which
    proves injectivity of $p^{\ast}$. 
    
    To show surjectivity, let $[\alpha]$ be an element in
    $H^1(\wt{M}; \RR)$ that is $\iota^{\ast}$-invariant and let $\alpha$ be a representative.  Define $\beta\in H^1(M;\RR)$ in local coordinates such that $\beta$ takes on the value of $\alpha-\frac{df}{2}$.  Then $\beta$ is well-defined because $\alpha-\frac{df}{2}$ is $\iota^\ast$ invariant.  Indeed, for some smooth function $g$:
    \begin{align*}
        \alpha - \iota^{\ast}(\alpha) = dg.
    \end{align*}
    Applying $\iota^\ast$ to both sides of the equality, we have $\iota^{\ast}dg = -dg$. Then $\alpha$ is the pullback of $\beta$ under %Then we have that the $1$-form $\alpha - \frac{dg}{2}$ is $\iota^{\ast}$-invariant, and thus in the image of
    $p^{\ast}$. This proves surjectivity, and the lemma.
\end{proof}

Lemma \ref{lem:injective} tells us that $H^1(M; \RR)$ is a subspace of $H^1(\wt{M}; \RR)$, so we define the Thurston norm on $H^1(M; \RR)$ by restricting the Thurston norm on the orientable 3-manifold $\widetilde{M}$ to the subspace $p^*(H^1(M;\RR))$ of $H^1(\widetilde{M};\RR)$.
%\begin{cor}
%The pullback $p^\ast:H^(M;\ZZ)\rightarrow H^1(\wt{M};\ZZ)$ maps bijectively to a finite-index subgroup of $\iota^\ast$-invariant subspace of $H^1(\wt{M};\ZZ)$.
%\end{cor}

\p{Thurston norm for non-orientable 3-manifolds}
Let $M$ be a non-orientable 3-manifold and $\wt{M}$ its orientation double cover.  Let $\wt{x}$ be the Thurston norm on $H^1(\wt{M};\RR)$ defined in Section \ref{sec:thurst-fiber-face}.
  Let $\alpha\in H^1(M;\RR).$ 
  The {\it Thurston norm on $H^1(M; \RR)$}, is the norm $x: H^1(M;\RR)\rightarrow \RR$ defined:
  \begin{align*}
    x(\alpha) \coloneqq \wt{x}(p^{\ast}\alpha).
  \end{align*}

We now extend properties of the Thurston norm for orientable manifolds to the Thurston norm on $H^1(M;\RR)$.
\begin{thm}
  The unit ball with respect to the dual Thurston norm on $\left( H^1(M; \RR) \right)^{\ast}$ is a polyhedron in $(H^1(M,\RR))^\ast$
  whose vertices are lattice points $\{\pm \beta_1, \ldots \pm \beta_k\}$. The unit ball $B_1$ with respect to
  Thurston norm is a polyhedron given by the following inequalities.
  \begin{align*}
    B_1 = \left\{ a\in H^1(M,\RR) \mid \left| \beta_i(a) \right| \leq 1 \text{ for $1\leq i \leq k$} \right\}
  \end{align*}
\end{thm}

\begin{proof}
  The proof is identical to the original proof of Thurston
  \cite[Theorem 2]{thurston1986norm}. Because the norm of an element of
  $H^1(M; \ZZ)$ is the Thurston norm of the corresponding element in $H^1(\wt{M}; \ZZ)$, the norm of any element in
  $H^1(M; \ZZ)$ is also an integer.  The linear algebra follows identically.
\end{proof}

Observe that the way we defined the Thurston norm for non-orientable $3$-manifolds is lacking in two
ways. First of all, in the orientable case, the Thurston norm is a norm on the second homology, and thus also
embedded surfaces. We have already seen how working with second homology does not quite work, which is why we defined the analogue of the Thurston norm for non-orientable surfaces on the
first cohomology. %We would still like to talk about the norm of an embedded surface though, even if the homology class of that surface may be trivial.  This is something we'll see in Section
In Section \ref{sec:invert-poincare}, we will develop a version of Poincar\'e duality for non-orientable 3-manifolds to better understand embedded non-orientable surfaces.

The second shortcoming of the definition of Thurston norms for non-orientable manifolds is that Lemma \ref{lem:injective} gives a bijection between $H^1(M;\RR)$ and a subspace of $H^1(\wt{M};\RR)$.  But the Thurston norm describes the relationship between fibrations an orientable manifold $\wt{M}$ over $S^1$ and {\it lattice points} $H^1(\wt{M}; \ZZ)$. % that when working with fibrations over $S^1$, the elements of $H^1(M; \ZZ)$ are the elements of interest, rather than the elements of $H^1(M; \RR)$. %Lemma \ref{lem:injective} tells us that elements of $H^1(M; \RR)$ are precisely the $\iota^{\ast}$-invariant elements of $H^1(\wt{M}; \RR)$. 
However there are $\iota^{\ast}$-invariant
elements on $H^1(\wt{M}; \ZZ)$ that are not pullbacks of elements of $H^1(M; \ZZ)$.  So there is not a bijection between $H^1(M; \ZZ)$ and fibrations of $M$ over $S^1$.

\p{Failure of surjectivity}
  Let $\no$ be a non-orientable surface, $\os$ its orientation double cover. Let $\gamma$ be a one-sided curve
  on $\no$, i.e. a curve whose preimage $\wt{\gamma}$ in $\os$ has a single component. Let the $3$-manifolds $M$ be the mapping torus of $\no$ with some pseudo-Anosov $\varphi$ and $\wt{M}$ the mapping torus of $\os$ with the orientation preserving lift of $\varphi$.  We can then
  consider $\gamma$ and $\wt{\gamma}$ as curves in the $3$-manifolds $M$ and $\wt{M}$.

  Extend $\wt{\gamma}$ to a basis $\mathcal{B}$ of $H_1(\wt{M}; \ZZ)$. We can construct an element
  of $H^1(\wt{M}; \ZZ)$ by simply assigning integer values to the elements of $\mathcal{B}$. Define $\alpha\in H^1(\wt{M};\ZZ)$
  that assigns $0.5$ to $\wt{\gamma}$ and an integer value to every element of $\mathcal{B}$. Consider the
  cohomology class $\alpha + \iota^{\ast}\alpha$. Because $\wt{\gamma}$ is the pre-image of a curve of $M$, we have that $\iota \wt{\gamma} = \wt{\gamma}$. Therefore  $\alpha + \iota^{\ast}\alpha$ is an
  $\iota^{\ast}$-invariant element of $H^1(\wt{M}; \ZZ)$ that assigns $1$ to $\wt{\gamma}$. Such a cohomology
  class cannot be a pullback of a class on $M$ since the pullback of a cohomology class on $M$ would assign an
  even value to $\wt{\gamma}$.

What the above example does show is that for any $\alpha\in H^1(\wt{M}; \ZZ)$ that is  $\iota^{\ast}$-invariant, the class $2\alpha$ definitely is a pullback of class in $H^1(M; \ZZ)$.

\subsection{Inverting the Poincar\'e duality map for embedded surfaces}
\label{sec:invert-poincare}

For either an orientable or non-orientable 3-manifold $M$, given $\alpha\in H^1(M; \ZZ)$, we can construct a dual map $f_\alpha$.  The preimage of a regular value $q\in S^1$ $f_{\alpha}^{-1}(q)$, will be an embedded surface. %In the orientable setting, the homology class of this embedded surface is well-defined, independent both of the choice of representative $1$-form in its cohomology class and the choice of regular value. While the homology class is also well defined in the non-orientable setting, the homology class is trivial when the embedded surface is non-orientable. 
We will invert this construction: given an embedded surface $S$, we want a
closed $1$-form $\alpha$ such that the surface $S$ comes from $\alpha$ in the manner described
above.

When $M$ is orientable, Poincar\'e duality determines a closed 1-form corresponding any embedded surface.  In this section, we create an ad hoc version of Poincar\'e duality for non-orientable surfaces in Theorem \ref{thm:Poincare-duality}.
%associating embedded surfaces to $1$-forms. 
However, we need a version of the orientability condition for embedded non-orientable surfaces that we call \emph{relative orientability}.

\p{Relative orientability}
  Let $M$ be a $3$-manifold, and $S$ an embedded surface in $M$. The surface $S$ is said to be {\it relatively
  oriented with respect to $M$} if there is a nowhere vanishing normal vector field on $S$. Two
  such normal vector fields are said to induce the same orientation if locally they induce the
  same orientation after picking a local frame for $S$. A surface $S$ is \emph{relatively oriented}
  if both $S$ and the choice of positive normal vector field are specified.

Note that relative orientability is a strictly weaker notion than orientability. If $S$ and $M$ are
orientable, then $S$ is relatively orientable with respect to $M$. But even if $M$ is
non-orientable, a non-orientable embedded surface $S$ may be relatively orientable with respect to $M$. For instance, let $S$ be
the fiber of a non-orientable mapping torus $M$.  The pre-image under the bundle map of a non-vanishing vector field on $S^1$ is a non-orientable vector field on $M$.
%It is not the case that every embedded surface in a non-orientable $3$-manifold is relatively orientable.

\p{A surface that is not relatively orientable in a 3-manifold}
  Let $S$ be the standard torus $\RR^2/\ZZ^2$, and let $\varphi$ map $(x,y)$ to $(-x, y)$. Then $\varphi$ is an
  orientation-reversing homeomorphism.  Therefore the mapping torus $M_\varphi$ is non-orientable. Consider a vertical line $\gamma$ in $S$ preserved by $\varphi$, i.e. the line
  $x = 0$. The image of $\gamma$ in $S$ under the suspension flow in $M$ is a subsurface of $M$,
  which we'll call $S'$. The normal direction to $S'$ when restricted to $S$ is $\frac{\partial}{\partial x}$. Because the suspension flow reverses the direction of $\gamma$, the 
  the normal vector field cannot be continuously extended to all of $M$.
  This means that the surface $S'$ is not relatively orientable in $M$ (despite being orientable itself.)

However, if both $M$ and an embedded surface are non-orientable, the surface will be relatively orientable.
\begin{prop}
  \label{prop:relative-orientability}
  Let $M$ be a non-orientable $3$-manifold, and let $S$ be an embedded connected non-orientable surface in $M$.
  Then $S$ is relatively orientable with respect to $M$.
\end{prop}
\begin{proof}
  Let $\wt{M}$ be the orientation double cover of $M$, and $\wt{S}$ be the pre-image of $S$ under the double cover. The
  restriction of the orientation reversing deck transformation $\iota:\wt{M}\rightarrow\wt{M}$ to $\wt{S}$ is an orientation reversing homeomorphism of $\wt{S}$.
  Let $(v_1, v_2)$ be positively oriented local frame for the tangent space to $\wt{S}$. Let $n$ be an outward pointing normal vector to $\wt{S}$ so the local frame $(v_1, v_2, n)$ is positively oriented. Since $\iota$ reverses the orientation of both $\wt{S}$ and $\wt{M}$, $(\iota(v_1), \iota(v_2))$ and $(\iota(v_1), \iota(v_2), \iota(n))$ are both negatively oriented. Then $\iota(n)$ is outward pointing. 
  %, since the quoti(nent $S$ is non-orientable. 
  %That means $S$ leaves the outward pointing normal
  %direction from $\wt{S}$ invariant, and that descends to an outward pointing normal direction on $S$. This
  %shows that $S$ is relatively orientable with respect to $M$.
  Therefore the outward pointing normal direction on $\wt{S}$ descends to an outward pointing normal direction on $S$, and $S$ is relatively orientable in $M$.
\end{proof}

We care about relatively orientable surfaces because for these surfaces can be mapped to cohomology classes.
\begin{thm}[Poincar\'e duality for non-orientable $3$-manifolds]
  \label{thm:Poincare-duality}
  Let $M$ be a non-orientable $3$-manifold, and let $S$ be a relatively oriented embedded
  surface. Then there exists a cohomology class $[\alpha]$ in $H^1(M; \ZZ)$ and a regular value $q\in S^1$ such that for some
  representative $\alpha$, $S=f_{\alpha}^{-1}(q)$. Furthermore, $\alpha$ assigns positive values to the positively oriented normal vector
  field on $S$.
\end{thm}

The idea of the proof of this theorem is fairly straightforward. Starting with the embedded surface
$S$ in $M$, we look at the pre-image $\wt{S}$ in the orientation double cover $\wt{M}$. We show
that the Poincar\'e dual to $\wt{S}$ is $\iota^{\ast}$-invariant.
\begin{lem}
  \label{lem:PD1}
  Let $S$ be a relatively oriented embedded surface in $M$, and $\wt{S}$ its pre-image in
  $\wt{M}$. Then the Poincar\'e dual to $[\wt{S}]$ is $\iota^{\ast}$-invariant.
\end{lem}
\begin{proof}
  If $S$ is relatively oriented with respect to $M$, then the relative orientation lifts to a relative orientation of $\wt{S}$ with respect to $\wt{M}$. Since $\wt{S}$ and $\wt{M}$ are orientable, this defines an orientation on $\wt{S}$,
  and thus the homology class $[\wt{S}]$ is well defined.

  The deck transformation $\iota$ reverses the orientation on $\wt{S}$. Indeed, let $(v_1, v_2, v_3)$ be a local frame for some point in $\wt{S}$ such that
  $v_3$ is the outward pointing normal vector field. Since the outward pointing normal vector
  field descends to the quotient by the orientation reversing map $\iota$.  Therefore $\iota(v_3)$
  must also be outward pointing. Since $\iota$ reverses the
  orientation on $\wt{M}$ but preserves the direction of $\iota(v_3)$, $\iota$ must reversing the orientation on the pair
  $(v_1, v_2)$.\becca[inline]{Is orientation really the best term in reference to the pair?} In particular, that means $\iota$ reverses the orientation on $\wt{S}$.

  This means $[\wt{S}]$ is in the $-1$-eigenspace of the $\iota_{\ast}$ action on
  $H_2(\wt{M}; \RR)$. Let $\wt{\alpha}$ be the the Poincar\'e dual to $[\wt{S}]$. The 1-form $\wt{\alpha}$ is $\iota^{\ast}$-invariant. This follows from
  the following chain of equalities which hold for all closed $2$-forms $\omega$.  We use the fact that $\iota^2=id$ in the first and third equalities.
  \begin{align*}
    \int_{\iota_{\ast}\wt{S}} \omega &= \int_{\wt{S}} \iota^{\ast}\omega &&\text{(By a change of variables)} \\
                                     &= \int_{\wt{M}} \wt{\alpha} \wedge \iota^{\ast} \omega &&\text{(Poincar\'e duality)} \\
                                     &=\int_{\wt{M}} \iota^{\ast} \left( \iota^{\ast}\wt{\alpha} \wedge \omega \right) \\
    &= \int_{\wt{M}} - \left( \iota^{\ast} \wt{\alpha} \wedge \omega \right) &&\text{($\iota$ is orientation reversing)}
  \end{align*}
  On the other hand, the following equalities follow from the fact that
  $\iota_{\ast}[\wt{S}] = -[\wt{S}]$.
  \begin{align*}
    \int_{\iota_{\ast}\wt{S}} \omega &= - \int_{\wt{S}} \omega \\
                              &= - \int_{\wt{M}} \wt{\alpha} \wedge \omega
  \end{align*}
  Because $$\int_{\wt{M}}\wt{\alpha}\wedge\omega=\int_{\wt{M}}\iota^\ast\wt{\alpha}\wedge\omega$$ for all $\omega$, it follows that $\wt{\alpha}$ is
  $\iota^{\ast}$-invariant.
\end{proof}

We now have an $\iota^{\ast}$-invariant $1$-form $\wt{\alpha}$.  Then we construct the map $f_{\widetilde{\alpha}}:\wt{M}\rightarrow S^1$ such that for a regular value $p\in S^1$, the surface $\wt{S}=f_{\wt{\alpha}}^{-1}(p)$. The next claim we want to make is that the map $f_{\wt{\alpha}}: \wt{M} \to S^1$ factors through the quotient $M$.
\begin{lem}
  \label{lem:PD2}
Let $p:\wt{M}\rightarrow M$ be the orientation double cover. For all points $y \in \wt{M}$, $f_{\wt{\alpha}}(y) = f_{\wt{\alpha}}(\iota (y))$.
\end{lem}
\begin{proof}
  Recall that $f_{\wt{\alpha}}(y)$ is given by the following integral formula.
  \begin{align*}
    f_{\wt{\alpha}}(y) = \int_{x_0}^y \wt{\alpha} \mod \ZZ,
  \end{align*}
  where $x_0$ is a basepoint in $\wt{M}$. Since $f_{\wt{\alpha}}(y)$ is equal to
  $f_{\wt{\alpha}}(\iota(y))$ for all $y$, we have the following:
  \begin{align*}
    \left(  \int_{x_0}^y \wt{\alpha} - \int_{x_0}^{\iota(y)} \wt{\alpha} \right) \in \ZZ.
  \end{align*}
  By a change of variables, and using the $\iota^{\ast}$-invariance of $\wt{\alpha}$, the left hand
  side of the above condition can be transformed, giving us the following condition.
  \begin{align*}
    \label{cond:integer}
    \left( \int_{x_0}^{\iota(x_0)} \wt{\alpha} \right) \in \ZZ
  \end{align*}
  %In other words, we want the integral of $\wt{\alpha}$ along any curve $\gamma$ from $x_0$ to $\iota(x_0)$ to be an integer. Equivalently, 
  Let $\gamma$ be a simple one-sided curve based at $p(x_0)$. The preimage $p^{-1}(\gamma)$ in $\widetilde{M}$ is a simple closed curve based at $x_0$, call it $\delta$.  It will suffice to show that the integral of
  $\wt{\alpha}$ along $\delta$ is an even integer.

  The parity of $\displaystyle \int_{\delta} \wt{\alpha}$ is precisely the parity
  of the intersection number of $\delta$ and $\wt{S}$.  But the intersection of $\delta$ and $\wt{S}$ is even because it is twice the intersection of $p(\delta)=\gamma$ and $p(\widetilde{S})=S$.   %Furthermore, both $\delta$ and $\wt{S}$ are lifts of a curve and surface from $M$. Which means the number of intersections they have in $\wt{M}$ is twice the number of intersections have in $M$. But the latter number must be an integer, and thus the former number must be an even integer, showing that condition \eqref{cond:integer} holds. 
  In particular,
  this shows that the map $f_{\wt{\alpha}}$ factors through, proving the lemma.
\end{proof}
We now have everything we need to finish proving Theorem \ref{thm:Poincare-duality}.
\begin{proof}[Proof of Theorem \ref{thm:Poincare-duality}]
  Starting with a relatively oriented surface $S$ in $M$, we look at its pre-image $\wt{S}$ in
  $\wt{M}$ under the orientation double cover. The relative orientation of the preimage gives us the homology class $[\wt{S}]$, and
  we get a $1$-form $\wt{\alpha}$, which is Poincar\'e dual to the homology class of $\wt{S}$.
  More specifically, we have a map $f_{\wt{\alpha}}$ and a regular value $q\in S^1$ such that $\wt{S}=f^{-1}_{\wt{\alpha}}(q)$.
  By Lemma \ref{lem:PD1}, $\wt{\alpha}$ is $\iota^{\ast}$-invariant, and by Lemma \ref{lem:PD2}, the map $f_{\wt{\alpha}}$ factors through $M$ to a map $f_{\alpha}:M\to S^1$.  The map $f_\alpha$ has the property that $f_{\alpha}^{-1}(q) =
  S$. By pulling back $d\theta$ on $S^1$ under $f_\alpha$, we obtain the desired 1-form $\alpha$ in $H^1(M; \ZZ)$.
\end{proof}

\subsection{Oriented sums of surfaces}
\label{sec:orient-sums-surf}

%We now have a way of going from an embedded surface to an element of $H^1(M; \ZZ)$. To make this mapping even
%more useful, we'll describe a way of adding two surfaces via the operation of taking 
The next step in studying embedded non-orientable surfaces will be to describing
\emph{oriented sums}.  The oriented sum of two surfaces embedded in a manifold $M$ indeed is additive in both the Euler characteristic and $H^1(M;\RR)$. This
operation is well-known in the case of orientable $3$-manifolds (along with orientable embedded
surfaces), but we will sketch out the relevant details.  We then extend the construction relatively orientable embedded surfaces. %The same construction works for relatively orientable surfaces; one just needs to verify consistency.

\p{Oriented sum for oriented manifolds}
Let $S$ and $S'$ be oriented embedded surfaces in an oriented manifold $M$. Assume that $S$ and $S'$ intersect
trasversally. Thus, $S \cap S'$ is a disjoint union of copies of $S^1$. For each component $S\cap S'$, take a tubular neighborhood that has cross section as in Figure \ref{fig:cross-section}.

%\autoref{fig:cross-section}.
\begin{figure}
  \centering
  \incfig[0.2]{cross-section}
  \caption{Cross section of intersection of $S$ and $S'$.}
  \label{fig:cross-section}
\end{figure}

We then perform a surgery on the leaves of $S$ and $S'$ so that the outward pointing normal vector fields match as in Figure \ref{fig:surgery}.% We have two possible
%choices: we could join the left $S$ leaf to either the top or the bottom $S'$ leaf. Since both $S$ and $S'$ are oriented submanifolds of $M$, there is an outward pointing normal vector field on $S$ and $S'$. Suppose the outward normal vector field on $S$ points upwards and the outward normal vector on $S'$ points to the right. In that case, we'd glue the left $S$ leaf to the bottom $S'$ leaf to maintain a consistent outward normal vector field. See \autoref{fig:surgery} to see how the choice affects orientability.
\begin{figure}
  \centering
  \incfig[0.3]{surgery}
  \caption{On the left, the normal vectors on $S$ and $S'$ are consistent. On the right, they aren't.}
  \label{fig:surgery}
\end{figure}

By performing this surgery at all the intersections, we get a new submanifold $S''$ (which may have
multiple components). This new submanifold $S''$ is the oriented sum of $S$ and $S'$. The operation
of taking oriented sums is additive on Euler characteristic, as well as the homology classes (and thus
the cohomology classes of their Poincar\'e duals).
\begin{align*}
  \chi(S'') &= \chi(S) + \chi(S') \\
  [S''] &= [S] + [S'] \\
\end{align*}

\p{Oriented sum for non-orientable manifolds}
Let $M$ be a non-orientable manifold and let $S$ and $S'$ be embedded surfaces in $M$ that are relatively oriented.
%Observe that in order to canonically choose the right leaves to join, all we need is a relative orientation for both $S$ and $S'$. %That suggests that the same construction ought to work.
%Like in the case of an orientable ambient manifold, at every transversal intersection, we perform surgery based on the outwards pointing normal vector field.
%We need to verify that this construction is consistent with the covering map: i.e. taking
%the oriented sum of $S$ and $S'$ is the same as taking the oriented sum of $\wt{S}$ and $\wt{S'}$
%and then taking the quotient by $\iota$.

%Let $\gamma$ be a component of $S\cap S'$ in $M$.  %, and let $\wt{\gamma}_1$ and $\wt{\gamma}_2$ be the distinct path lifts of $\gamma$ in $\wt{M}$ under the orientation double cover $p$. 
We will define the oriented sum on $S$ and $S'$ as follows.  Let $p:\wt{M}\rightarrow M$ be the orientation double cover and let $\iota$ be the orientation reversing deck transformation of $\wt{M}$.  As above, let $\wt{S}=p^{-1}(S)$ and $\wt{S}'=p^{-1}(S')$, which are embedded oriented surfaces in $\wt{M}$.  The oriented sum of $S$ and $S'$ is the image under $p$ of the oriented sum of $\wt{S}$ and $\wt{S}'$ (as defined above for oriented surfaces in oriented manifolds).  We need to justify that this operation is well-defined.

As in the proof of Lemma \ref{lem:PD1}, $\iota$ preserves the relative orientation, and thus leaves the outward
normal vector fields on $\wt{S}$ and $\wt{S}'$ invariant. Therefore a leaf $\ell$ of $\wt{S}$ is surgered with a leaf of $\ell'$ of $\wt{S}'$ if and only if $\iota(\ell)$ and $\iota(\ell')$ are surgered.  Therefore surgery factors through $p$ and the oriented double sum is well-defined.  %The oriented sum of $S$ and %The oriented sum of embeddein $M$ that is consistent with the oriented sum on the orientation double cover.%We need to show that when we surger the leaves of $\wt{S}$ to a leaf of $\wt{S'}$ along $\wt{\gamma}_1$, . 

\p{Example} Let $\gamma$ be a component of $S\cap S'$ and $\wt{\gamma}_1$ and $\wt{\gamma}_2$ be the path lifts of $\gamma$.  Consider
\autoref{fig:consistency}, which shows the outward point normal vectors to $\wt{S}$ and $\wt{S'}$,
which determine which leaves are glued together along $\wt{\gamma}_1$ and $\wt{\gamma}_2$.
\begin{figure}
  \centering
  \incfig[0.4]{consistency}
  \caption{Neighborhoods of $\wt{\gamma}_1$ and $\wt{\gamma}_2$, with the outward pointing normal vector field.}
  \label{fig:consistency}
\end{figure}

The normal vector field tells us that the left $\wt{S}$ leaf gets glued to the bottom $\wt{S'}$
leaf near $\wt{\gamma}_1$ and $\wt{\gamma}_2$. Since $\iota(\wt{\gamma}_1)=\wt{\gamma_2})$, the outward pointing normal vector fields point the same (relative) directions.  %Consider now the deck transformation $\iota$. %Note that $\iota$ is an orientation reversing self map for $\wt{M}$, $\wt{S}$ and $\wt{S'}$. We've


\p{Additivity} By the consistency of the oriented sum in $M$ and $\wt{M}$, it easily follows that the oriented sum
is additive in Euler characteristic, as well as in terms of Poincar\'e dual, since the Poincar\'e
dual was also defined by passing to the orientation double cover.

\subsection{Relating $1$-forms and fibrations over $S^1$}
\label{sec:relating-1-forms}
%While we have informally described what a fibration over $S^1$ is prior to this section, it will be useful to formally define a fibration at this stage.
%\begin{defn}[Fibration over $S^1$]
  Given a $3$-manifold $M$, a fibration (or a fiber bundle) over $S^1$ is a map $f: M \to S^1$ such
  that the derivative of $f$ has full rank at all points in $M$. The pre-image of every point in
  $S^1$ is an relatively oriented embedded surface in $M$, where the positive normal direction is
  the pre-image of the positive direction in $S^1$. This surface is called the fiber of the
  fibration.
%\end{defn}
%Note that any $3$-manifold that admits a fibration over $S^1$ is a mapping torus of the fiber, along with the homeomorphism that comes from the transition map when changing coordinate charts on $S^1$. We can instead look at homotopy classes of fibrations, and every equivalence class will correspond to a homotopy class of a homeomorphism of the fiber, i.e. a mapping class. Since we're mostly interested in mapping tori of mapping classes rather than mapping classes of specific homeomorphisms in those mapping classes, we'll be focusing on homotopy classes of fibrations.

%For the purposes of our work, we will  \emph{non-singular integer $1$-forms}.

\p{Non-singular integer 1-forms}  A {\it non-singular integer $1$-form} on a $3$-manifold $M$ is a smooth nowhere vanishing $1$-form $\alpha$ on $M$
  such that for any closed loop $\gamma$, the integral of $\alpha$ along $\gamma$ lies in $\ZZ$.
  \begin{align*}
    \int_{\gamma} \alpha \in \ZZ
  \end{align*}

Given a non-singular integer $1$-form $\alpha$, the map
\begin{align*}
  f_{\alpha}(x) \coloneqq \int_{x_0}^x \alpha \mod \ZZ
\end{align*}
is a fibration over $S^1.$

As above, given a fibration $f: M \to S^1$, we obtain a non-singular integer $1$-form by
pulling back $d\theta$ under $f$. %The correct $1$-form to pull back is $d\theta$, i.e. the non-vanishing $1$-form on $S^1$ such that $\int_{S^1} d\theta = 1$ (note that despite the notation, this is not an exact form). These two constructions are inverses of each other, which is fairly easy to verify. Furthermore, 
If we change $\alpha$ to $\alpha + df$, where $df$ is an exact form, then the associated map to $S^1$ is not the same, but homotopic to the original map. Conversely, if we pull back $d\theta$ along a map homotopic to $f$ rather than $f$, we get a form that differs from the original form by an exact form (see Section 5.2.1 of \cite{calegari2007foliations} for the details). %The takeaway here is that if we only care about the mapping torus structure of the mapping classes, we can focus our attention to the elements of $H^1(M; \ZZ)$ that admit a non-singular $1$-form representative.

We now have all we need to prove a version of Theorem \ref{thm:Thur1} for non-orientable
$3$-manifolds.

\begin{thm}
  \label{thm:NOThur1}
  Let $M$ be a non-orientable $3$-manifold, and let $\mathcal{F}$ be the set of all possible
  ways $M$ fibers over $S^1$ (up to homotopy). Then the following results hold for $\mathcal{F}$.
  \begin{enumerate}[(i)]
  \item Elements of $\mathcal{F}$ are in a one-to-one correspondence with (non-zero) lattice points
    inside a union of cones over open faces of the unit ball with respect to the Thurston norm
    in $H^1(M; \RR)$.
  \item If an embedded relatively oriented surface $S$ is transverse to the suspension flow
    associated to some fibration $f$ such that the flow direction is the outwards normal direction,
    then the Poincar\'e dual to $S$ lies in the closure of the cone in $H^1(M;\RR)$ corresponding to $f$.
  \end{enumerate}
\end{thm}
\begin{proof}
  We proceed by reducing to the the orientable case. Let $p:\wt{M}\rightarrow M$ be the orientation double cover of $M$.  %To get the union of cones $\mathcal{K}$ in $H^1(M; \RR)$ corresponding to fibrations, we
 % look at the corresponding union of cones
 Let $\wt{\mathcal{K}}$ be the union of cones over the open faces of the unit Thurston norm ball in $H^1(\wt{M};\RR)$. Recall that $H^1(M; \RR)$ bijectively maps into
  $H^1(\wt{M}; \RR)$ as a subspace. Let $\mathcal{K}$ be the preimage of $\mathcal{\wt{K}}\cap p^\ast(H_1(M;\ZZ))$ under $p^\ast$.  %We define $\mathcal{K}$ to be the restriction of $\wt{\mathcal{K}}$ to the subspace $p^\ast(H^1(M; \RR))$.

 %observe that 
 Define a map $\mathcal{L}:\mathcal{F}\rightarrow \mathcal{K}$ as follows.  % Let $f: M \to S^1$ be a fibration. Define $\mathcal{L}(f)$ to be the pullback of $d\theta$ under $f^\ast$.  To see the injectivity of $\mathcal{L}$,
 The map $f\circ p$ is a fibration of $\wt{M}\to S^1$.  Then by Theorem \ref{thm:Thur1} and Poincar\'e duality, $f\circ p$ corresponds to an element $\wt{\alpha}\in \wt{K}$.  The 1-form $\wt{\alpha}$ is the pullback of $d\theta$ under $(f\circ p)^\ast$.  But $\wt{\alpha}$ is also the pullback of $f^\ast(d\theta)$ under $p^\ast$ and therefore $\wt{\alpha}$ lies not only in $\wt{K}$ but also in the image of $H^1(\wt{M};\ZZ)$ under $p^\ast$.  So we define $\mathcal{L}(f)$ to be $f^\ast(d\theta)$, the element of $\mathcal{K}$ that $p^\ast$ maps to $\wt{\alpha}$.
 %the composition $f \circ p: \wt{M} \to S^1$ is a fibration of $\wt{M}$ over $S^1$. Let $\widetilde{\alpha}$ be the pullback of $d\theta$ under $f\circ p$.  The 1-form $\wt{\alpha}$ lies in $\wt{\mathcal{K}}\subset H^1(\wt{M};\ZZ)$ by Theorem \ref{thm:Thur1}. Moreover, $\wt{\alpha}$ is also a pullback of $f^{\ast}(d\theta) \in H^1(M; \ZZ)$ under $p^\ast$, and
  %therefore $\wt{\alpha}$ lies in $p^\ast(H^1(M; \ZZ))$. % We define $\mathcal{L}(f)$ to be  \becca{Why is it injective?}Therefore $p^\ast$ maps every homotopy class of fibrations injectively maps into $\mathcal{K}$. 
  
  Let $\alpha\in\mathcal{K}$ and let $\wt{\alpha}$ be the pullback of $\alpha$ under $p^\ast$. Then $\wt{\alpha}$ lies in $\wt{\mathcal{K}}$,
  and therefore corresponds to a fibration $f_{\wt{\alpha}}: \wt{M} \to S^1$. We would like
  to pushforward this map to a map from $M$ to $S^1$. By Lemma \ref{lem:PD2}, this means that $\wt{\alpha}$ satisfies the following condition
  for any basepoint $x_0$ in $\wt{M}$:
  \begin{align*}
    \int_{x_0}^{\iota(x_0)} \wt{\alpha} \in \ZZ.
  \end{align*}
  Any path from $x_0$ to $\iota(x_0)$ is a lift of a closed curve $\gamma$
  on $M$ from $p(x_0)$ to $p(\iota(x_0)) = p(x_0)$, and $\wt{\alpha}$ is the pullback of
  the $\alpha \in H^1(M; \ZZ)$, the above integral is equal to an integral on $M$.
  \begin{align*}
    \int_{x_0}^{\iota(x_0)} \wt{\alpha} = \int_{\gamma} \alpha
  \end{align*}
  The right hand side term is clearly an integer, since $\alpha$ is an integer $1$-form. This
  shows that the map $f_{\wt{\alpha}}$ descends to a map on $M$, and therefore $\alpha$ corresponds
  to a fibration. This proves part (i) of the theorem.

  For part (ii), let $\alpha\in H^1(M;\RR)$ be the Poincar\'e dual of $S$. Let $\wt\alpha$ be the pullback of $\alpha$ under $p^\ast$.  Then $\wt{\alpha}$ is the Poincar\'e dual to the $\wt{S}=p^{-1}(S)$.  Let $\widetilde{f}$ be the orientation preserving lift of $f$ under $p$.  Because $S$ is
  transverse to the suspension flow direction associated to $f$ with outward flow direction, $\wt{S}$ must be transverse to the suspension flow on the suspension flow direction associated to $\wt{f}$ with outward flow direction. Therefore $\wt{\alpha}$ lies in the closure of the cone of $H^1(\widetilde{M};\ZZ)$ associated to $\wt{f}$. Since $\wt{\alpha}$ is also a pullback of $\alpha$, it must be that $\wt{\alpha}\in p^\ast(H^1(M;\RR))$. Therefore $\alpha$ lies in the restriction to $H^1(M;\RR)$ of the closure of the cone associated to $f$.
\end{proof}

Part (ii) of the above theorem (and Theorem \ref{thm:Thur1}) is especially useful when trying to
decompose a $3$-manifold into a mapping torus. Let
$M_\varphi = (\no, \varphi)$ and $f:M\rightarrow S^1$ the bundle map. We can construct another relatively oriented surface $\no'$ inside $M$ such that
$\no'$ is transverse to the suspension flow direction associated to $f$. Let $\alpha$ be the Poincar\'e dual of $\no$ and $\alpha'$ be the Poincar\'e dual of $\no'$.  By Theorem \ref{thm:NOThur1} $\alpha$ lies in a cone $\mathcal{C}_\varphi$ associated to $\varphi$ with other $1$-forms also coming from fibrations. Furthermore, $\alpha'$ lies in $\mathcal{C}_\varphi$. All positive
integer linear combinations of $\alpha$ and $\alpha'$ are elements of $\mathcal{C}_\varphi$. Each linear combination of $\alpha$ and $\alpha'$ is Poincar\'e dual to an oriented sum of $\no$ and $\no'$ in $M$. Under reasonably mild conditions on $\no$ and $\no'$, we can actually realize their oriented sum as
fibers of a fibration.

\begin{thm}
  \label{thm:oriented-sum}
  Let $\no$ be a non-orientable surface and $\varphi$ a homeomorphism of $\no$.  Let $M_\varphi$ be the mapping torus $(\no,\varphi)$ and $f:M\rightarrow S^1$ the bundle map.  Let $\no'$ be an incompressible surface embedded in $M$ that is transverse to the suspension flow direction associated to $f$.  Let $\alpha$ be the Poincar\'e dual of $\no$ and $\alpha'$ the Poincar\'e dual of $\no'$. If the oriented sum of $\no$ and $\no'$ is connected, then
  $\no + \no'$ is isotopic to the fiber of the fibration given by $\alpha + \alpha'$.
\end{thm}
\begin{proof}
  %The first step is to observe that in this case, one can compute the Thurston norm of $\alpha$ and $\alpha'$ using $\no$ and $\no'$. 
  Let $p:\wt{M}\rightarrow M$ be the orientation double cover of $M$.  %Let $\widetilde{f}:\wt{M}\rightarrow S^1$ be the orientation preserving lift of $f$. \becca[inline]{Is this the right fibration of $\wt{M}$?}
  The surface $\no$ is incompressible because it is a fiber of $f$; therefore its preimage under $p$ is also incompressible.  By Theorem \ref{thm:ThurHyp}, the Thurston norm of $\alpha$ is 
  $2\chi_-(\no)$.  Since $\no'$ is also incompressible, the Thurston norm of $\alpha'$ is $2\chi_-(\no')$.

  Since $\alpha$ and $\alpha'$ lie in a cone over a fibered face, the Thurston norm $x$ on $H^1(M;\ZZ)$ is linear.  Indeed:
  %We can thus compute the Thurston norm of $\alpha + \alpha'$ in terms of $\no + \no'$.
  \begin{align*}
    x(\alpha + \alpha') &= x(\alpha) + x(\alpha') \\
                        &= 2\chi_-(\no) + 2\chi_-(\no') \\
                        &= 2\chi_-(\no + \no').
  \end{align*}
  The last equality follows from the linearity of the oriented sum. Because the preimage of $\no + \no'$ in the orientation double cover of $M$ must be Thurston norm minimizing.  Therefore the surface $\no+\no'$ is incompressible.

  By Theorem \ref{thm:NOThur1}, we have that $\alpha + \alpha'$ corresponds to some other fibration
  of $f'':M\rightarrow S^1$. Let $\wt{f}''$ be the orientation-preserving lift of $f''$ to $\wt{M}$.\becca{Is this the right choice?}  Since $M$ is non-orientable, there are two possibilities:
  \begin{enumerate}[(i)]
  \item $f''$ is the mapping torus of a non-orientable surface and a homeomorphism.
  \item $f''$ is the mapping torus of an orientable surface and an orientation
    reversing homeomorphism.
  \end{enumerate}
  In the first case, the fiber of $f''$ is a non-orientable surface homologous to $\no + \no'$. The preimage of $\no+\no'$ under $p$ is two homologous orientable surfaces, both of which minimize the Thurston norm. By
  Theorem \ref{thm:ThurIsotope}, we have that preimage of $\no + \no'$ under $p$ is isotopic to a fiber of $\wt{f}''$.   Therefore $\no + \no'$ is isotopic to the fiber in $M$, and $M$ can be realized as the mapping torus of some
  homeomorphism on $\no + \no'$.

  For the second case, let $M$ be the mapping
  torus of an orientable surface $S$ with an orientation reversing homeomorphism.  A fiber of $\wt{f}$ is two homologous copies of $S$. But
  that is homologous to the preimage of $\no + \no'$ under $p$, which will have a single component since
  $\no + \no'$ is non-orientable. Theorem \ref{thm:NOThur1} says these two surfaces must be
  isotopic, but that is a contradiction since they have a different number of connected components.
\end{proof}

The non-orientable versions of Theorems \ref{thm:fm} and \ref{thm:alm} follow in a straightforward
manner from the orientable versions.

\begin{thm}
  \label{thm:NOfm}
  Let $M$ be a non-orientable hyperbolic $3$-manifold and let $\mathcal{K}$ be the union of cones
  in $H^1(M; \RR)$ whose lattice points correspond to fibrations over $S^1$. There exists a
  strictly convex function $h: \mathcal{K} \to \RR$ satisfying the following properties.
  \begin{enumerate}[(i)]
  \item For all $t > 0$ and $u \in \mathcal{K}$, $h(tu) = \frac{1}{t} h(u)$.
  \item For every primitive lattice point $u \in \mathcal{K}$, $h(u) = \log(\lambda)$, where $\lambda$ is
    the stretch factor of the pseudo-Anosov map associated to this lattice point.
  \item $h(u)$ goes to $\infty$ as $u$ approaches $\partial \mathcal{K}$.
  \end{enumerate}
\end{thm}

\begin{proof}
  By Theorems \ref{thm:fm} and \ref{thm:alm}, there is a function $\wt{h}$ on $H^1(\wt{M}; \RR)$ that satisfies properties (i)-(iii). Restricting $\wt{h}$
  to the subspace corresponding to $H^1(M; \RR)$, we get a convex function satisfying properties
  (i) and (iii). By Proposition \ref{prop:2}, the stretch factor of a pseudo-Anosov
  map on a non-orientable surface is the same as the stretch factor of the unique lift to its
  double cover. By Theorem \ref{thm:fm}, (ii) also holds.
\end{proof}

The exact statement of Theorem \ref{thm:alm} holds for the non-orientable setting too: one just
restricts the function $h$ on $H^1(\wt{M}; \RR)$ to the subspace corresponding to $H^1(M; \RR)$.