\section{Introduction}

Let $M$ be a closed, fibered 3-manifold that fibers over $S^1$. It is a well known fact that these surface bundles over the circle all have a separate description, that of the mapping torus of some homeomorphism of a closed surface. Not only that, but a single 3-manifold can have many, possibly infinite, descriptions as a mapping torus. Thus when studying these 3-manifolds, an important question is whether one can understand or give a description of all these possible fibrations. 

In 1986, Thurston gave a way to answer this question, a semi-norm on the second homology of an orientable 3-manifold that was able to "detect" when an embedded surface of the 3-manifold was the fiber of a fibration of said manifold. This \textit{Thurston norm} is given a full treatment in \cite{thurston1986norm}, in which Thurston shows that not only does this norm have unti ball a polyhedron, but the fibers of fibrations of one of these 3-manifolds $M$ were in one-to-one correpsondence with cones on open faces of this polyhedron unit ball. This almost combinatorial description of the fibers turns out to be a powerful tool in studying these fibered 3-manifolds.

There is a small catch to this norm, note above we said that Thurston defines his norm on the second homology of an \textit{orientable} 3-manifold. Now while in most cases of its use, this orientibilty issue hasn't caused too much of a problem, yet it does beg the question of whether this norm works in the non-orientable case as well. Thurston himself gives a comment on this in \cite{thurston1986norm}, writing "Most of this paper also works for non-oriented manifolds, but simplicity we deal only with the oriented case." It is the goal of this paper to deal with the other case, to see what works and what, if anything, possibly goes wrong when trying to extend the Thurston norm and its consequences for fibered 3-manifolds to the non-orientable setting. 

As we stated above...