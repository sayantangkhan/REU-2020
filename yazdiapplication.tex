\section{Minimal stretch factors for non-orientable surfaces with marked points}
\label{sec:application}

Recall that associated to every pseudo-Anosov homeomorphism there is a number is the \textit{dilatation} or \textit{stretch factor}, the amount that the stable and unstable foliations of the pseudo-Anosov change by. Given a surface $S$, it is natural to ask what we can say about the set of all possible stretch factors, i.e.
\begin{align*}
    \{\log(\lambda(f)) | f \in \text{Mod}(S) \text{ is pseudo-Anosov}\}
\end{align*}

We call this set the \textit{spectrum} of $\mathcal{S}_{g,n}$. We define the spectrum in terms of the logarithm of the stretch factors, as this is equivalent to the \textit{topological entropy} of a pseudo-Anosov homeomorphism. Topological entropy is a natural measure of complexity that one can assign to any topological map. A first step at understanding this set of entropies associated to a surface is considering the following quantity
\begin{align*}
    l_{g,n} =\min\{\log(\lambda(f)) | f \in \text{Mod}(\mathcal{S}_{g,n}) \text{ is pseudo-Anosov}\}.
\end{align*}

The study of this minimal stretch factor $l_{g,n}$ was initiated by Penner in his work \cite{penner1991bounds}. In this work Penner studied the asymptotic behavior of minimal stretch factors of non punctured orientable surfaces, i.e. studied the behavior of $l_{g,0}$. He showed that there exists positive constants $A_1$ and $A_2$ such that for any $g \geq 2$
\begin{align*}
    \frac{A_1}{g} \leq l_{g,0} \leq \frac{A_2}{g}.
\end{align*}
Hence showing that asymptotically $l_{g,0}$ behaves like $\frac{1}{g}$ for $g = 2$. The work by Yazdi in \cite{yazdi2018pseudo} that we aim to generalize using our results on the Thurston norm for non-orientable manifolds is a generalization of these first steps by Thurston. Yazdi asks what the asymptotic behavior of $l_{g,n}$ is when we look at rays in the $(g,n)$ plane with $n$ being constant. To this extent he proves the following result.

\begin{thm}[Yazdi]
\label{thm:yazdi1}
For any fixed $n \in \mathbb{N}$, there are positive constants $B_1 = B_1(n)$ and $B_2 = B_2(n)$ such that for any $g \geq 2$
\begin{align*}
    \frac{B_1}{g} \leq l_{g,n} \leq \frac{B_2}{g}
\end{align*}.
\end{thm}

Yazdi's result is one of many recent results in studying the aymptotics of $l_{g,n}$ for different subsets of the $(g,n)$ plane. See the introductions of \cite{yazdi2018pseudo} and \cite{tsai2009asymptotic} for more examples of results of this form. Yazdi proves an additional result along these lines for a large subset of the $(g,n)$ plane, one containing balls of arbitrary large radii.

\begin{thm}[Yazdi]
    \label{thm:yazdi2}
    There exists positive constants $A$, $B$ and $C$ such that for any $n \geq 1$ and $g \geq Cn\log^2(n)$ we have
    \begin{align*}
        \frac{B}{g} \leq l_{g,n} \leq \frac{A}{g}
    \end{align*}

\end{thm}

Our goal is to show these results also hold for non-orientable surfaces, albeit with possibly different constants. That is, if we let $\mathcal{N}_{g,n}$ be the genus $g$ non-orientable surface with $n$ punctures and let
$$l'_{g,n} = \min\{\log(\lambda(f)) \, \vert \, f \in \text{Mod}(\mathcal{N}_{g,n})\ \text{ is pseudo-Anosov}\}$$
Then we prove the following results
\begin{thm}
\label{thm:stretch}
~\begin{enumerate}
    \item For any fixed $n \in \mathbb{N}$, there are positive constants $B'_1 = B'_1(n)$ and $B'_2 = B'_2(n)$ such that for any $g \geq 2$ $$\frac{B'_1}{g} \leq l'_{g,n} \leq \frac{B'_2}{g}$$.
    \item There exists positive constants $A'$, $B'$ and $C'$ such that for any $n \geq 1$ and $g \geq C'n\log^2(n)$ $$\frac{B'}{g} \leq l'_{g,n} \leq \frac{A'}{g}$$
\end{enumerate}
\end{thm}

Recall that Proposition \ref{prop:2} states a pseudo-Anosov on a non-orientable surface lifts to a pseudo-Anosov with the same stretch factor on the double orientation cover. Thus, overlooking a few details we will describe at the end of this section, Yazdi's lower bound ``lifts" to a lower bound for non-orientable surfaces. The work in Yazdi's paper is done in constructing a family of pseudo-Anosovs, one for each surface $S_{g,n}$, that have 'small' stretch factors. Our goal, and the primary part of the proof of the first part of Theorem \ref{thm:stretch}, will be to replicate Yazdi's construction on non-orientable surfaces, using our extension of the Thurston norm.

\subsection{The Yazdi Construction}

Yazdi provides a 5 step construction in order to produce a pseudo-Anosov map for every punctured orientable surface that has stretch factor bounded above by the desired quantity. We will reproduce each step here, giving our version of Yazdi's construction for non-orientable surfaces.

\begin{center}
\textbf{Step 1: Constructing the Surfaces}
\end{center}

The first step in the construction is defining a family of surfaces that exhibit a sort of rotational symmetry. Using this symmetry, if one shows that a power of some homeomorphism is pseudo-Ansov, then so was the original homeomorphism. Yazdi cites this insight as being due to Penner in his construction in \cite{penner1991bounds}.

Note that we will try to follow Yazdi's notation as close as we can, in order to make it clear to the reader how our construction replicates his.

We will define a family of surfaces $P_{n,k}$ in the way originally done by Yazdi. Let $T$ be an orientable surface of genus 5 with 3 boundary components $c,d$ and $e$. Orient $T$ and give $c,d$ and $e$ the induced orientations from this orientation. Now add two cross-caps to $T$ but keep the boundaries of $T$ oriented. Let $p$ (respectively $q$) be a puncture (respectively a marked point) on the boundary component $e$ of $T$, with oriented arcs $r$ and $s$ connecting them on $\partial T$. See Figure \ref{fig:buildingblock} for a picture of $T$.

\begin{figure}[]
    \centering
    \resizebox{.25\totalheight}{!}{\incfig{YazdiTypeSurface}}
    \caption{The surface $T$, the building block for our Yazdi construction}
    \label{fig:buildingblock}
\end{figure}

Let $T_{i,j}$ be copies of the surface $T$, where $i,j \in \mathbb{Z}$. We will use similar notations to refer to the boundary components of $T_{i,j}$. Define an infinite surface $S_\infty$ as the quotient
$$S_\infty \coloneqq \left( \bigcup T_{i,j} \right)/\sim,$$
where $i,j \in \mathbb{Z}$. The equivalence relation $\sim$ is defined as
\begin{align*}
    c_{i,j} \sim d_{i+1,j} \hspace{1em}, \hspace{1em} r_{i,j} \sim s_{i,j+1}
\end{align*}
and the gluing maps for the boundary components are by orientation-reversing homeomorphisms.

We have two natural shift maps $\overline{\rho_1},\overline{\rho_2}: T_\infty \xrightarrow[]{} T_\infty$ that act by $$\overline{\rho_1} \text{ sends } T_{i,j} \text{ to } T_{i+1,j},$$ $$\overline{\rho_2} \text{ sends } T_{i,j} \text{ to } T_{i,j+1}.$$ Note that these maps commute. Define the surface $P_{n,k}$ as the quotient of the surface $T_\infty$ by the covering action of the group generated by $(\overline{\rho_1})^n$ and $(\overline{\rho_2})^k$. Therefore, $\overline{\rho_1}$ and $\overline{\rho_2}$ induce maps on the surface $P_{n,k}$, which we denote by $\rho_1$ and $\rho_2$.

A question that naturally arises is why did we choose the surface $T$ for our building block? It comes down to two main problems:
\begin{enumerate}
    \item The combinatorics of the curves make the associated matrix we get from the Penner construction satisfy the conditions of Lemma Y
    \item Having a curve $\gamma$ such that it and its image under our map form the boundary of an embedded $\mathcal{N}_3$ in our surface, which we will need in later steps (\textcolor{red}{Perhaps this was outlined above}).
\end{enumerate}

\begin{lem}
Define the sequence
\begin{align}
    g_{n,k} &= (14k - 2)n + 2, k \geq 3, n \geq 1
\end{align}
    The genus of $P_{n,k}$ is $g_{n,k}$.
\end{lem}
\begin{proof}
    Consider the subsurface $U \subset P_{n,k}$ defined as $$U = \left( \bigcup_{i =0}^{k-1} T_{0,i} \right)/\sim.$$ Then $U$ is a compact, non-orientable surface of genus $12k$ with $2k$ boundary components, and forms a fundamental domain for the covering action of $\overline{\rho_1}$ on $T_\infty$. We have $$\chi(U) = 2 - 12k - 2k = 2 - 14k.$$ Thus $$\chi(P_{n,k}) = n \cdot \chi(U) = -n(14k - 2),$$ since $P_{n,k}$ is formed by gluing $n$ copies of $U$ together along circle boundary components. Therefore $\chi(P_{n,k}) = 2 - g_{n,k} = -n(14k - 2) \implies g_{n,k} = n(14k - 2) + 2$.
\end{proof}

\begin{center}
\textbf{Step 2: Constructing the Maps}
\end{center}


Following Yazdi, we will now define maps $f_{n,k}: P_{n,k} \xrightarrow[]{} P_{n,k}$ that are defined as a composition of specific Dehn twists followed by a finite order mapping class. The key insight is that a power of this map will be a composition of Dehn twists that satisfy the criteria to be a Penner construction and thus pseudo-Anosov. This is how we take advantage of the rotational symmetry given to us by how the $P_{n,k}$ are constructed.

Recall that for non-orientable surfaces, we don't initially have a well-defined notion of a positive or negative Dehn twist. In order to do a Penner construction, we need to ensure that the curves we are working with are marked inconsistently. Then do Dehn twists around the curves according to these markings. Note that our labeling of the curves already gives us an inconsistent marking though. For any alpha curve $\alpha_i$, we let the marking $\phi_{\alpha_i}$ be orientation preserving and for beta curves $\beta_j$ let $\phi_{\beta_j}$ be orientation reversing. Since alpha curves only intersect beta curves (and vice versa), we have an inconsistent marking at each point of intersection.

Let $\mathcal{B}$ be the union of all $\beta$ curves except $\beta_1$ in $T_{0,0} \cup T_{0,1} \cup T_{1,0}$ (see figures below). Let $\rho_1(\mathcal{B})$ be the image of $\mathcal{B}$ under $\rho_1$. Define $\phi_b$ as the composition of Dehn twists along all the curves in the set $\overline{\mathcal{B}} \coloneqq \mathcal{B} \cup \rho_1(\mathcal{B}) \cup \dots \cup \rho_1^{n-1}(\mathcal{B})$. Since the curves in $\overline{\mathcal{B}}$ are disjoint, Dehn twists along them commute. Therefore, it is not necessary to specify the order in which we compose these Dehn twists in $\phi_b$. Let $\mathcal{R}$ be the union of all $\alpha$ curves except $\alpha_1$ in $T_{0,0}$. Define $\mathcal{R}$ and $\phi_r$ in the exact same way.

Let $\alpha_1,\beta_1 \subset T_{0,0}$ be the curves in Figure Z. Let $\phi$ be the composition of Dehn twists along all the curves $\alpha_1, \rho_1(\alpha_1), \dots, \rho_1^{n-1}(\alpha_1)$ followed by Dehn twists along all the curves $\beta_1,\rho_1(\beta_1),\dots,\rho_1^{n-1}(\beta_1)$. Define
\begin{align*}
    f_{n,k} &\coloneqq \rho_2 \circ \phi \circ \phi_b \circ \phi_r
\end{align*}
We are using the same notation as Yazdi, so the composition is from right to left. It follows from the Penner construction that $(f_{n,k})^k$ is pseudo-Anosov. Hence $f_{n,k}$ itself is pseudo-Anosov and invariant train tracks $\tau^1_{n,k}$ for $f_{n,k}$ can be obtained from Penner's construction that we described above.

\begin{center}
\textbf{Step 3: The Mapping Torus}
\end{center}

We have now constructed an infinite family of non-orientable surfaces and pseudo-Anosov maps, but there's an alarming issue with it. Looking back at Lemma Z, the genera of this family of surfaces does not encapsulate every positive integer. We are going to use our extension of the Thurston norm to "fill in the gaps", by constructing fibers for fibrations of the mapping tori of the pseudo-Anosov maps we defined above.

Let $M_{n,k}$ be the mapping torus of $f_{n,k}$ .Likewise, let $\widetilde{M_{n,k}}$ denote the mapping tori of $\widetilde{f_{n,k}}$, where $\widetilde{f_{n,k}}$ is the orientation preserving lift of $f_{n,k}$ to the orientation double cover of $P_{n,k}$. Note that it follows that $\widetilde{M_{n,k}}$ is the orientation double cover of $M_{n,k}$.

Let $C_{n,k}$ denote the fibered face of $H_2(\widetilde{M_{n,k}},\mathbb{R})$ corresponding to the map $\widetilde{f_{n,k}}$. We will show that $M_{n,k}$ contains a closed non-orientable surface of genus 3 that lifts to a closed orientable surface of genus 2 in $\widetilde{M_{n,k}}$ that is contained in the closure of $C_{n,k}$.

\textcolor{red}{Will come back to this section later, first I need to go and do the oriented sum}

\begin{lem}
There is a non-trivial homology classes $0 \neq [\widetilde{F_{n,k}}] \in H_2(\widetilde{M_{n,k}};\mathbb{Z})$ represented by orientable surfaces of genus two that is a lift of a non-orientable surfaces of genus three $F_{n,k}$ in $M_{n,k}$. Moreover, $\widetilde{F_{n,k}}$ is Thurston norm-minimizing and lie in the closures $\overline{\mathcal{C}_{n,k}}$.
\end{lem}
\begin{proof}

\end{proof}

\begin{lem}
Let $\iota: \widetilde{M_{n,k}} \xrightarrow[]{} \widetilde{M_{n,k}}$ denote the deck transformation that generates the deck group of the orientation double cover. Likewise, let $\iota_*: H_2(\widetilde{M_{n,k}};\mathbb{R}) \xrightarrow[]{} H_2(\widetilde{M_{n,k}};\mathbb{R})$ denote its action on second homology. Then $\iota_*([\wt{F_{n,k}}]) = -[\wt{F_{n,k}}]$.
\end{lem}
\begin{proof}

By the way we have defined $\wt{F_{n,k}}$ as a lift of an embedded subsurface in $M_{n,k}$, we know that $\iota$ sends $\wt{F_{n,k}}$ to itself. We want to see that $\iota$ is also orientation reversing when restricted to $\wt{F_{n,k}}$.

To begin, we first need to see what our surface $\wt{F_{n,k}}$ looks like embedded in $\wt{M_{n,k}}$. Recall the way that $F_{n,k}$ is defined as an embedded $\mathcal{N}_1$ with two boundary components, $\gamma$ and $f^k(\gamma) = \hat{\gamma}$ in one of the fibers of $M_{n,k}$ union the tubes formed by following $\gamma$ $k$ times around the suspension flow in $M_{n,k}$. Let's observe what happens to our embedded genus 1 with 2 boundary components in a single fiber after it is lifted to the orientation double cover. The orientation double cover of a genus 1 nonorientable surface is $S^2$, and we can see here that our embedded surface will lift to a sphere with four boundary components, one can see this by imagining two copies of our embedded subsurface being glued along their single cross-cap. For ease of notation, let's denote the two lifts of $\gamma$ and $\hat{\gamma}$ as $\gamma_0,\gamma_1$ and $\hat{\gamma}_0,\hat{\gamma}_1$ respectively. These curves form the boundary of the sphere with four boundary components that is sitting in our single fiber in $\wt{M_{n,k}}$.

Recall that $\wt{M_{n,k}}$ is not only the double orientation cover of $M_{n,k}$, but is also the mapping torus of $\wt{f_{n,k}}$. Looking at these maps, if we let $p$ denote the covering map for $\wt{P_{n,k}} \xrightarrow[]{} P_{n,k}$, then we know that $p \circ \wt{f_{n,k}} = f_{n,k} \circ p$. This tells us that $\wt{f_{n,k}}$ sends $\gamma_0$ to $\hat{\gamma_0}$ and $\gamma_1$ to $\hat{\gamma}_1$. Thus we can see that the tube traced out by following the suspension flow of $\gamma$ to $\hat{\gamma}$ gets lifted to tubes following the suspension flow of $\gamma_0$ to $\hat{\gamma_0)}$ and $\gamma_1$ to $\hat{\gamma_1}$. These tubes glued to our sphere with four boundary components give us our genus 2 surface in the cover.

We will now show that $\iota$ restricted to $\wt{F_{n,k}}$ is orientation reversing by showing that it is orientation reversing on the individual components, i.e. the sphere with boundary and the two tubes. First note that the boundary components of the of the sphere with boundary are also curves that lie in one of the fibers of $\wt{M_{n,k}}$. Suppose that we give an orientation to our fiber which induces orientations on our curves. Since $\iota$ is orientation reversing on the fiber, it must reverse the orientation of our curves, and thus reverses the orientations of the boundaries of our sphere when we restrict $\iota$. Thus $\iota$ must be orientation reversing on the whole of the sphere with boundary components. \textcolor{red}{I know you made a slightly different argument for tubes Sayantan, but we can't we just use the same exact arguement for the tubes since the tubes are bounded by these curves?}

Now that we know that $\iota$ is orientation reversing on $\wt{F_{n,k}}$, we know that $\iota_*: H_2(\wt{F_{n,k}}) \xrightarrow{} \wt{F^i_{n,k}}$ acts by sending the fundamental class $[\wt{F_{n,k}}]$ to its negative. We also know that $\wt{F_{n,k}}$ is viewed as a representative for an element of $H_2(\wt{M_{n,k}})$ by the image of $[\wt{F_{n,k}}] \in H_2(\wt{F_{n,k}})$ under the map on second homology induced by the inclusion $i: \wt{F_{n,k}} \xrightarrow[]{} \wt{M_{n,k}}$. Since $\iota$ can be restricted to $\wt{F_{n,k}}$, it is a map of the pair $(\wt{M_{n,k}},\wt{F_{n,k}})$ and thus by the naturality of the long exact sequence of a pair, $\iota_*$ and $i_*$ commute. This tells us that $\iota_*: H^2(\wt{M_{n,k}}) \xrightarrow[]{} H^2(\wt{M_{n,k}})$ acts by $\iota_*([\wt{F_{n,k}}]) = -[\wt{F_{n,k}}]$, giving us our desired result.
\end{proof}

\begin{center}
\textbf{Step 4: Bounding the Stretch Factor}
\end{center}


In \cite{yazdi2018pseudo}, Yazdi shows that the family of pseudo-Anosov maps that we have constructed all have the log of their stretch factor bounded above by a similar factor. In order to to do this, recall in Section \ref{sec:background} we saw that pseudo-Anosov maps give rise to matrices whose Perron-Frobenius eigenvalue is our stretch factor. So a way to find an upper bound of the stretch factor of the maps we have constructed is to bound the spectral radius of the associated matrices. The following lemma by Yazdi does just this for a specific class of matrices that our examples are based off of.

\begin{lem}[Yazdi]
\label{lem:spectral}
Let $A$ be a non-negative integral matrix, $\Gamma$ be the adjacency graph of $A$, and $V(\Gamma)$ the set of vertices of $\Gamma$. For each $v \in V(\Gamma)$, define $v^+$ to be the set of vertices $u$ such that there is an oriented edge from $v$ to $u$. Let $D$ and $k$ be fixed natural numbers. Assume the following conditions hold for $\Gamma$: \begin{enumerate}
    \item For each $v \in V(\Gamma)$ we have $\deg_{\text{out}}(v) \leq D$.
    \item There is a partition $V(\Gamma) = V_1 \cup \dots \cup V_k$ such that for each $v \in V_i$ we have $v^+ \subset V_{i+1}$, for any $1 \leq i \leq k$ except possibly when $i = 1$ or 3 (indices are mod $k$).
    \item For each $v \in V_1$, we have $v^+ \subset V_2 \cup V_3$.
    \item For each $v \in V_3$ we have $v^+ \subset V_3 \cup V_4$, and for $u \in V^+ \cap V_3$ we have $u^+ \subset V_4$.
    \item For all $3 < j \leq k$ and each $v \in V_j$, the set $v^+$ consists of a single element.
\end{enumerate}
\end{lem}

With this result in hand, we can now show that in the same way as Yazdi, the stretch factors for our main family of examples are all bounded above in the way we hope.

\begin{lem}
There exists a universal positive constants $C'$ and such that for every $n \geq 1$ and $k \geq 3$:
$$\log(\lambda_{n,k}) \leq C'\frac{n}{g_{n,k}}$$
\end{lem}

\begin{proof}
We have purposefully constructed our examples so our curves are in the same ``general form" as Yazdi's were and thus they will still satisfy the criteria of Lemma 1. Though we still want to explicitly show that this is the case.

First for consistency of notation, let
\begin{align*}
    \mathcal{A} \coloneqq \mathcal{B} \cup \mathcal{R} &\cup \{\alpha_1,\beta_1\}, \overline{\mathcal{A}} \coloneqq \mathcal{A} \cup \rho_1(\mathcal{A}) \cup \dots \cup \rho_1^{n-1}(\mathcal{A}) \\
    &\hat{\mathcal{A}} \coloneqq \overline{\mathcal{A}} \cup \rho_2(\overline{\mathcal{A}}) \cup \dots \cup \rho_2^{k-1}(\overline{\mathcal{A}}).
\end{align*}
Thus $\hat{\mathcal{A}}$ is all the curves on our surface we are Dehn twisting around to get $f_{n,k}$.

Recall from above that we stated we need to find the eigenvalue of the matrix that represents the action of $f_{n,k}$ on the subspace of the cone of transverse measures that is spanned by the measures assigning $1$ to single curves in $\hat{\mathcal{A}}$ and 0 to everything else. Let $A$ be said matrix and $\Gamma$ the adjacency graph of $A$. In order to bound the spectral radius of $A$, we need to show that $\Gamma$ satisfies the criteria of Lemma 1. To do this we first need to partition the vertices of $\gamma$, which is equivalent to a partition of the curves in $\hat{\mathcal{A}}$: $$\mathcal{A} = \bigcup_{i=1}^k \rho_2^{i-2}(\overline{\mathcal{A}}).$$ Then define $V_i$ for $1 \leq i \leq k$ as the vertices of $\Gamma$ corresponding to elements in $\rho_2^{i-2}(\overline{\mathcal{A}})$.

We can now check the conditions of Lemma 1, based on the combinatorics of the curves on our surface:
\begin{enumerate}
    \item
    \item As above, we now have a partition of our vertices where $V_i \coloneqq \rho_2^{i-2}(\overline{\mathcal{A}})$. So suppose that $v \in V_i$, $i \neq 1,3$, is a vertex that correpsonds to a curve $c \in \hat{\mathcal{A}}$. By the partitioning $c$ must be a curve in $\rho_2^{i-2}(\overline{\mathcal{A}})$, for $i \neq 1,3$. Note in order for all vertices of this form to have $v^+ \subset V_{i+1}$, we need to see that $f_{n,k}$
    \item
    \item
    \item All the curves corresponding to an element of $V_j$, $3 < j \leq k$ are disjoint from all the curves in $\overline{A}$. Thus $f_{n,k}$ just acts by rotation.
\end{enumerate}

Setting $\lambda = \lambda_{n,k}$, Lemma \ref{lem:spectral} implies that
\begin{gather*}
    \lambda^{k-1} = \rho(A)^{k-1} = \rho(A^{k-1}) \leq 4D^4 \implies (k-1)\cdot \log(\lambda) \leq \log(4D^4) \\
    \implies \frac{k}{2}\log(\lambda) \leq (k-1)\log(\lambda) \leq \log(4D^4)
\end{gather*}

On the other hand, we know $g_{n,k} = (14k - 2)n + 2 \leq 14kn$. Therefore
\begin{align*}
    \log(\lambda) \leq 2\log(4D^4)\cdot\frac{1}{k} \leq 2\log(4D^4)\cdot \frac{14n}{g_{n,k}} = C'\frac{n}{g_{n,k}}
\end{align*}
where $C' \coloneqq 28\log(4D^4)$.
\end{proof}

\begin{center}
\textbf{Step 5: Filling in the Gaps}
\end{center}

Recall that $\wt{P_{n,k}}$ is the double orientation covers of $P_{n,k}$ and also the fibers of $\wt{M_{n,k}}$. As in Yazdi, we are going to be considering the homology classes $[\wt{P^r_{n,k}}] \coloneqq [\wt{P_{n,k}}] + r[\wt{F_{n,k}}]$. Representatives for these homology classes can be found by taking the oriented sum.

At this point we should be able to cite Yazdi's Lemma 3.5 as the proof will go the exact same and say:

\begin{lem}
The surfaces $\wt{P^r_{n,k}}$ are Thurston norm-minimizing, with genera equal to $\wt{g^r_{n,k}} \coloneqq g_{n,k} + r - 1$. As $r$ varies between $0$ and $14n$, the genera of $\wt{P^r_{n,k}}$ cover the range between $\wt{g_{n,k}}$ and $\wt{g_{n,k+1}}$. Moreover, $\wt{P^r_{n,k}}$ are fibers of fibrations of $\wt{M_{n,k}}$ with pseudo-Anosov monodromy that fixes $4n$ of the singularities of the invariant foliation and descend to fibrations of $M_{n,k}$.
\end{lem}

\begin{proof}
    \textcolor{red}{For now I'm going to write this section as if the oriented sum section above didn't exist, because I'm not sure how I want to structure that section yet.}
    All statements of this lemma except for the last follow in the same way as Lemma 3.5 in \cite{yazdi2018pseudo}. We know that
    $$\chi(\wt{P^r_{n,k}}) = \chi(\wt{P_{n,k}}) + r\cdot\chi(\wt{F_{n,k}}) = (-2(g_{n,k} - 1) + 2)-2r = -2(g_{n,k} + r - 1) + 2.$$ This proves the identity for the genus of $\wt{P^r_{n,k}}.$ To see that $\wt{P^r_{n,k}}$ is Thurston norm-minimizing, note that $[\wt{P^r_{n,k}}] \subset C_{n,k}$ due to $P_{n,k} \subset C_{n,k}$ and $F_{n,k} \subset \overline{C_{n,k}}$. Note that from Theorem 3 in \cite{thurston1986norm}, one can deduce that the fiber of a fibration of a 3-manifold is Thurston norm-minimizing in its second homology class. This along with Lemma\textcolor{red}{F is minimizing} and the linearity of the Thurston norm on a fibered faace tells us
    $$x([\wt{P^r_{n,k}}]) = x([\wt{P_{n,k}}]) + rx([\wt{F_{n,k}}]) = \chi_-(\wt{P_{n,k}}) + 2r = 2(g_{n,k} + r - 1) - 2$$ so $\wt{P^r_{n,k}}$ is also norm minimizing.

    Just as Yazdi says, the homology class $[\wt{P^r_{n,k}}] = [P_{n,k}] + r[F_{n,k}]$ is clearly integral. It is also primitive since there is a curve in $\wt{M_{n,k}}$ that intersects $P_{n,k}$ transversely and exactly once, while avoiding $F_{n,k}$. (?). Since $[P^r_{n,k}]$ is integral, primitive and lies in the fibered face $C_{n,k}$, by Theorem X it is the fiber of a fibration of $\wt{M_{n,k}}$. Since the monodromy $\wt{f_{n,k}}$ is pseudo-Anosov, all monodromies that correspond to this face are pseudo-Anosov \textcolor{red}{CITE}, so in particular $\wt{f^r_{n,k}}$ is pseudo-Anosov.

    As we see in Yazdi, the singularities of the stable foliation of $f_{n,k}$ that are fixed are the $2n$ intersection points of the axis of $\rho_1$ with $P_{n,k}$. Thus $\wt{f_{n,k}}$ has $4n$ singularities of its stable foliation. Furthermore we have already seen that the surface $F_{n,k}$ can be isotoped to be transverse to the suspension flow and disjoint from the orbit of the $2n$ singularities of $f_{n,k}$, thus $\wt{F_{n,k}}$ can be isotoped to be disjoint from the orbit of these $4n$ singularities of $\wt{f_{n,k}}$. Thus the monodromy $\wt{f^r_{n,k}}$ still fixes the corresponding $4n$ singularities on $P^r_{n,k}$.

    This all suffices for the orietable case, but for the non-orientable case we need to also show that our fibrations of $\wt{M_{n,k}}$ defined by our maps $\wt{f^r_{n,k}}$ and surfaces $P^r_{n,k}$ descend to fibrations on $M_{n,k}$. Recall that Lemma 3 gave us the exact criteria for a fibration of $\wt{M_{n,k}}$ to descend to a fibration of $M_{n,k}$. It is precisely when the corresponding 1-form $\wt{\alpha}$ is the pullback of a 1-form on $M_{n,k}$ and the integral $\int_{x_0}^{\iota(x_0)} \alpha \in \mathbb{Z}$ for some chosen basepoint $x_0 \in \wt{M_{n,k}}$. For our situation, this corresponding 1-form is exactly the Poincar\'e dual of $[P^r_{n,k}]$, which we will denote $\wt{\alpha}$. Since $[\wt{P^r_{n,k}}] = [\wt{P_{n,k}}] + r\cdot[\wt{F_{n,k}}]$ and both of the latter homology classes come from lifting surfaces in $M$, we know that $\wt{\alpha}$ satisifes both of these conditions by Lemmas 4 and 7..
\end{proof}

We can now prove our version of Yazdi's Lemma 3.6:

\begin{lem}
There exists a constant $C > 0$ such that for every $n \geq 1$, $k \geq 3$, and $0 \leq r \leq 6n$ we have $$\log(\lambda^r_{n,k}) \leq C\frac{n}{\wt{g^r_{n,k}}}$$
\end{lem}
\begin{proof}
    Let $\mathcal{C} = \wt{\mathcal{C}^i_{n,k}}$ be our fibered faces and $h: \mathcal{C} \xrightarrow[]{} \mathbb{R}$ the function described in Theorem X. Note that we have
    $$\wt{g^r_{n,k}} = \wt{g_{n,k}} + r \leq \wt{g_{n,k}} + 14n < 2\wt{g_{n,k}} < 2g_{n,k}$$
    Thus
    $$h([\wt{P^r_{n,k}}]) < h([\wt{P_{n,k}}]) \leq C'\frac{n}{g_{n,k}} \leq 2C'\frac{n}{\wt{g^r_{n,k}}} $$
\end{proof}

So now we have that our surfaces $\wt{P^r_{n,k}}$ are fibers of fibrations of $\wt{M_{n,k}}$ that descend to fibrations of $M_{n,k}$ and their monodromies are pseudo-Anosov with their stretch factors bounded. The pseudo-Anosov monodromies $\wt{f^r_{n,k}}$ descend to pseudo-Anosov monodromies on $M_{n,k}$ with the same stretch factor. Thus our upper bound of $2C'\frac{n}{\wt{g^{r}_{n,k}}}$ still holds, but we do need to make a slight modification. This bound is in terms of $\wt{g^r_{n,k}}$, the genus on the fiber in the double orientation cover, but the genus of our fiber downstairs will be one greater, thus $2C'\frac{n}{\wt{g^r_{n,k}}} = 2C'\frac{n}{g^r_{n,k} - 1} \leq 2C'\frac{n}{\frac{1}{2}g^r_{n,k}} = 4C'\frac{n}{g^r_{n,k}}$.

We can now think of $f^r_{n,k}$ as a map on a non-orientable surface of genus $g^r_{n,k}$. Note from above we know that $g^r_{n,k}$ covers all natural numbers between $g_{n,k}$ and $g_{n,k+1}$, thus this set of genera for all $r$ covers all natural numbers larger than $g_{n,3} = 40n + 2$. Recall that all of these surfaces will have $2n$ singularities, so we can either puncture $n$ or $n + 1$ to account for all possible number of punctures.

We can now give a proof of the first half of Theorem 9:

\begin{thm}[Theorem 9.1]
For any fixed $n \in \mathbb{N}$, there are positive constants $B_1' = B_1'(n)$ and $B_2' = B_2'(n)$ such that for any $g \geq 2$
$$\frac{B_1'}{g} \leq l'_{g,n} \leq \frac{B_2'}{g}$$
\end{thm}
\begin{proof}
    ~
    We begin by proving the upper bound. By Lemma 18 and above, we have that there exists a number $C' > 0$ such that for $g \geq 40n + 2$, $l'_{g,n} \leq 4C'\frac{n}{g}$. So we take $B_2(n)$ to be:
    $$B_2(n) = \max{4C'n, l'_{1,n}, 2l'_{2,n}, \dots, (40n + 1)l'_{40n+1,n}}$$
    For the lower bound, first recall that the lift of any pseudo-Anosov on a non-orientable surface $N_{g,n}$ to the double cover $S_{g-1,2n}$ is still pseudo-Anosov with the same stretch factor. Thus any lower bound on $l_{g-1,2n}$ is a lower bound for $l'_{g,n}$. In \cite{penner1991bounds}, Penner proved that
    $$l_{g,n} \geq \frac{\log(2)}{12g - 12 + 4n}$$
    And thus $$l'_{g,n} \geq l_{g-1,2n} \geq \frac{log(2)}{12g - 24 + 8n} > \frac{\log(2)}{12ng}$$
    And so like Yazdi, we can set $B_2'(n) = \frac{\log(2)}{12n}$
\end{proof}