\section{Fibered face theory for non-orientable $3$-manifolds}
\label{sec:fibered-face-theory}

In what follows, we will restrict our attention our attention on two kinds of compact $3$-manifolds:
mapping tori of orientable surfaces with a pseudo-Anosov map (these will be the $3$-manifolds we
will be referring to when talking about orientable $3$-manifolds), and the mapping tori of
non-orientable $3$-manifolds, again with a psuedo-Anosov map (these will be the manifolds we will be
referring to when talking about non-orientable $3$-manifolds). While a lot of our statements will
hold more generally for compact non-orientable $3$-manifolds, it will be easier to describe examples
when working in this restricted setting; additionally, our application will only involve mapping
tori of pseudo-Anosov maps.

\subsection{The problem with homology in non-orientable $3$-manifolds}
\label{sec:probl-with-homol}

A first attempt at defining the Thurston norm for a compact non-orientable $3$-manifold might go as
follows: for any embedded surface $S$ (whether it's orientable or not), one defines the $\chi_-$
function, much like in the case of orientable $3$-manifolds, and then define the norm of a homology
class $a \in H^2(M; \ZZ)$ by minimizing $\chi_-(S')$ over all $S'$ representing $a$. This may work,
but is quite unsatisfying: this construction assigns $0$ norm to all embedded non-orientable
surfaces, since their fundamental classes are trivial, and thus map to $0$ in $H_2(M; \ZZ)$. This is bad,
because for orientable $3$-manifolds, it is known that Thurston norm minimizing surfaces are
precisely the ones that are closed leaves of foliations (see Theorem 5.5 in \cite{gabai1983}). On
the other hand, it is possible to realize embedded non-orientable surfaces as closed leaves of
foliations, namely if the original $3$-manifold is the mapping torus of a non-orientable surface,
then one can think of that surface embedded inside the $3$-manifold.

It turns out that the fundamental problem with non-orientable $3$-manifolds is that homology is a
very coarse invariant: too coarse to detect embedded non-orientable surfaces. Our workaround will
be to deal with the first cohomology $H^1(M)$ rather than the second homology $H_2(M)$. The two are
same in the case of orientable $3$-manifolds, by Poincar\'e duality, but that fails in the case
of non-orientable $3$-manifolds.

To see why it fails, consider an orientable $3$-manifold $M$. We can explicitly work out the map
from $H^1(M; \ZZ)$ to $H_2(M; \ZZ)$ given by Poincar\'e duality. Given any $1$-form $\alpha$ such
that $[\alpha]$ is in $H^1(M; \ZZ)$, one can construct a map $f_{\alpha}$ from $M$ to $\RR/\ZZ$ in
the following manner.
\begin{align*}
  f_{\alpha}(x) \coloneqq \left. \int_{x_0}^x \alpha \right/ \ZZ
\end{align*}
Here $x_0$ is a basepoint on $M$ we can choose arbitrarily. The fact that this map is well defined
independent of the choice of path from $x_0$ to $x$ is a consequence of the fact that $[\alpha]$
lies in $H^1(M; \ZZ)$, i.e.  its integral around any closed loop is an integer. Consider now the
surface $S = f_{\alpha}^{-1}(p)$ for some regular value $p \in \RR/\ZZ$. Note that we have $S$
merely as a subset of $M$ right now. To get a homology class, we pick an orientation by declaring
that the outwards pointing normal vectors on $S$ are the ones which get assigned a positive value
by the form $\alpha$. This defines an orientation of $S$ because we already have an orientation on
$M$, and thus defines a fundamental class $[S]$. We claim that $[S]$ is the Poincar\'e dual to
$\alpha$. More precisely, we claim the following.
\begin{claim}
  Let $p$ and $p'$ be two regular values of the function $f_{\alpha}$ and let $S$ and $S'$ be
  $f_{\alpha}^{-1}(p)$ and $f^{-1}_{\alpha}(p')$ respectively. Then for any closed $2$-form $\omega$ on $M$, the following
  identity holds.
  \begin{align*}
    \int_{S} \omega = \int_{S'} \omega
  \end{align*}
  Furthermore, the following identity also holds.
  \begin{align*}
    \int_S \omega = \int_M \alpha \wedge \omega
  \end{align*}
  In particular, the homology class of $S$ is Poincar\'e dual to $\alpha$.
\end{claim}
\begin{proof}
  The first part of the claim follows from the fact that $S$ and $S'$ are homologous, i.e. $f^{-1}_{\alpha}([p, p'])$
  is a singular $3$-chain that has $S$ and $S'$ as boundaries. From Stokes' theorem, we get the following.
  \begin{align*}
    \int_{S - S'} \omega &= \int_{f_{\alpha}^{-1}([p, p'])} d\omega \\
                         &= 0
  \end{align*}

  To prove the second claim, observe that we can break up the second integral as a product integral.
  \begin{align*}
    \int_M \alpha \wedge \omega &= \int_{\RR/\ZZ} \left(   \int_{f_{\alpha}^{-1}(x)} \omega \right) dx
  \end{align*}
  The above equation is true because $\alpha$ is the pullback of $dx$ along the map $f_{\alpha}$. Observe
  that the inner integral only makes sense when $x$ is a regular value, but by Sard's theorem, almost
  every $x \in [0,1]$ is a regular value, so the right hand side is well-defined. By the first part of the
  claim, the inner integral is a constant function, as we vary over the $x$ which are regular values of $f_{\alpha}$,
  and the integral of $dx$ over $\RR/\ZZ$ is $1$, giving us the identity we want.
  \begin{align*}
    \int_M \alpha \wedge \omega = \int_S \omega
  \end{align*}
\end{proof}
What we have here is an explicit formula for the Poincar\'e duality map. For orientable $3$-manifolds, this
is an isomorphism, but the formula still makes sense for non-orientable manifolds. In that case, one can see
that the map from $H^1(M; \ZZ)$ to $H_2(M; \ZZ)$ has a kernel.
\begin{example}[Failure of Poincar\'e duality for non-orientable $3$-manifolds]
  Let $\no$ be a non-orientable surface, and let $f$ be any self homeomorphism. Let $M$ be the mapping torus
  of $(\no, f)$. We then have a map $f: M \to \RR/\ZZ$ given by mapping to the base of the mapping torus.
  Pulling back the form $dx$ along this map, we get a closed but not exact $1$-form $\alpha$ on $M$. Observe
  that $f_{\alpha} = f$, because of how we constructed $\alpha$. Furthermore $f_{\alpha}^{-1}(0)$ is $\no$ inside
  $M$. Thus, the ``Poincar\'e duality map'' for $M$ maps a non-trivial element $\alpha \in H^1(M; \ZZ)$ to the
  zero element $[\no] \in H_2(M; \ZZ)$, since $\no$ is non-orientable. In particular, we end up losing information
  going from $H^1(M)$ to $H_2(M)$.
\end{example}

The above example also suggests an alternative method of defining the Thurston norm for
non-orientable $3$-manifolds: rather than working with $H_2$, we can work with $H^1$ instead,
because we don't want to lose information by going to $H_2$. We'll also be interested in getting
a partial inverse for this map. More specifically, given a non-orientable surface $\no$ inside
a non-orientable $3$-manifold $M$, we'd like to understand if $M$ can be realized as the mapping torus
of some homeomorphism of $\no$.

\subsection{Thurston norm for non-orientable $3$-manifolds}
\label{sec:thurston-norm-non}

For this section, we'll use $M$ to denote a non-orientable $3$-manifold, and $\wt{M}$ to denote its
orientation double cover. We will denote by $\iota$ the orientation reversing deck transformation of
$\wt{M}$, and the covering map $\wt{M} \to M$ by $p$. If $M$ is the mapping torus of the non-orientable
surface $\no$ and a pseudo-Anosov map $\varphi: \no \to \no$, then $\wt{M}$ is the mapping torus of $(\os, \wt{\varphi})$,
where $\os$ is the orientable double cover of $\no$, and $\wt{\varphi}$ is the orientation preserving lift of $\varphi$.

Since we've already concluded that the first cohomology is the ``right'' space to define the
Thurston norm on, we need to relate $H^1(M; \RR)$ and $H^1(\wt{M}; \RR)$. The obvious thing to do is
to look at the pullback via $p$.

\begin{lem}
  \label{lem:injective}
  Let $M$ be a non-orientable $3$-manifold, and $\wt{M}$ its orientation double cover. Let
  $\iota: \wt{M} \to \wt{M}$ be the orientation reversing deck transformation, and $p: \wt{M} \to M$
  be the covering map. Then $p^{\ast}$ maps $H^1(M; \RR)$ bijectively to the
  $\iota^{\ast}$-invariant subspace of $H^1(\wt{M}; \RR)$.
\end{lem}
\begin{proof}
    Clearly, for any $1$-form $\alpha$ on $M$, $p^{\ast}(\alpha)$ will be $\iota^{\ast}$-invariant. This means that the image of $p^{\ast}$ lands inside the $\iota^{\ast}$-invariant subspace. To see that the map is injective at the level of $H^1$ (rather than at the level of $1$-forms), consider a $1$-form $\alpha$ on $M$ such that $p^{\ast}\alpha$ is exact. We thus have a smooth function $f$ on $\wt{M}$ such that the following relation holds.
    \begin{align*}
        df = p^{\ast} \alpha
    \end{align*}
    But since $p^{\ast}\alpha$ is $\iota^{\ast}$-invariant, we must have $df = \iota^{\ast} df$, and by pushing the $\iota^{\ast}$ inside, we get that $df = d(\iota^{\ast}f)$. That means $f$ and $\iota^{\ast}f$ differ by a constant, but that constant must be $0$ since $\iota$ is finite order. This shows that $f$ descends to a function on $M$, and thus $\alpha$ is exact, which proves injectivity of $p^{\ast}$. Now we show surjectivity. Let $[\alpha]$ be an element in $H^1(\wt{M}, \RR)$ that is $\iota^{\ast}$-invariant. That means if we pick a $1$-form $\alpha$ in this equivalence class, the following identity holds for some smooth function $f$.
    \begin{align*}
        \alpha - \iota^{\ast}(\alpha) = df
    \end{align*}
    Note that this means $\iota^{\ast}df = -df$. Using these two identities, it's easy to verify that the $1$-form $\alpha - \frac{df}{2}$ is $\iota^{\ast}$-invariant, and thus in the image of $p^{\ast}$. This proves surjectivity, and the lemma.
\end{proof}

The above lemma tells us that $H^1(M; \RR)$ is a subspace of $H^1(\wt{M}; \RR)$ (and we know
precisely what that subspace is), and thus the Thurston norm on $H^1(\wt{M}; \RR)$ can be restricted
to a norm on $H^1(M; \RR)$.

\begin{defn}[Thurston norm for non-orientable $3$-manifolds]
  Let $M$ is a non-orientable $3$-manifold, and $\wt{M}$ is its orientation double cover, with the covering
  map $p: \wt{M} \to M$. The Thurston norm $x$ is a norm on $H^1(M; \RR)$, defined using the Thurston norm $\wt{x}$
  on $H^1(\wt{M}; \RR)$ (identified with $H_2(M; \RR)$ via Poincar\'e duality) in the following manner.
  \begin{align*}
    x(\alpha) \coloneqq \wt{x}(p^{\ast}\alpha)
  \end{align*}
\end{defn}
Now that we have a Thurston norm on $H^1(M; \RR)$, we need to describe some of its properties. All of these properties
follow fairly easily from the orientable version.
\begin{thm}
  The unit ball with respect to the dual Thurston norm on $\left( H^1(M; \RR) \right)^{\ast}$ is a polyhedron
  whose vertices are lattice points $\{\pm \beta_1, \ldots \pm \beta_k\}$. The unit ball $B_1$ with respect to Thurston
  norm is a polyhedron given by the following inequalities.
  \begin{align*}
    B_1 = \left\{ a \mid \left| \beta_i(a) \right| \leq 1 \text{ for $1\leq i \leq k$} \right\}
  \end{align*}
\end{thm}

\begin{proof}
  The proof is identical to the original proof of Theorem $2$ of Thurston in
  \cite{thurston1986norm}. The key ingredient of the proof is that the norm of any element in
  $H^1(M; \ZZ)$ is an integer. That is true in our case because the norm of an element of
  $H^1(M; \ZZ)$ is the Thurston norm of the corresponding element in $H^1(\wt{M}; \ZZ)$, which is an
  integer because it's the negative Euler characteristic of an embedded surface. The rest of the
  proof is just a matter of linear algebra, and works just as well in our setting.
\end{proof}

Observe that the way we defined the Thurston norm for non-orientable $3$-manifolds is lacking in two
ways. First of all, in the orientable case, the Thurston norm is a norm on the second homology, and
thus also embedded surfaces. In other words, the Thurston norm tells us something about embedded
surfaces in the manifold. We've already seen how working with second homology doesn't quite work,
which is why we had to go to first cohomology instead. We would still like to talk about the
norm of an embedded surface though, even if the homology class of that surface may be trivial.
This is something we'll see in subsection \ref{sec:invert-poincare}.

The other issue is that when working with fibrations over $S^1$, the elements of $H^1(M; \ZZ)$ are
the elements of interest, rather than things in $H^1(M; \RR)$. Lemma \ref{lem:injective} tells us
that elements of $H^1(M; \RR)$ are precisely the $\iota^{\ast}$-invariant elements of
$H^1(\wt{M}; \RR)$. That does not hold for $\ZZ$-coefficients: there are $\iota^{\ast}$-invariant
elements on $H^1(\wt{M}; \ZZ)$ that are not pullbacks of elements of $H^1(M; \ZZ)$.
\begin{example}[Failure of surjectivity]
  {\color{red} Turn this into a theorem.}
  Let $\no$ be a non-orientable surface, $\os$ its orientation double cover. Let $\gamma$ be a
  one-sided curve on $\no$, i.e. a curve whose lift to $\os$ is an arc, and the pre-image is a
  single closed curve, which is twice as ``long''. Call the pre-image $\wt{\gamma}$. Let the
  $3$-manifolds $M$ and $\wt{M}$ we're considering be the mapping tori of $\no$ and $\os$ with
  respect to some pseudo-Anosov map. We can then consider $\gamma$ and $\wt{\gamma}$ as curves
  inside the $3$-manifolds $M$ and $\wt{M}$.

  Pick a basis of $H_1(\wt{M}; \ZZ)$ containing $\wt{\gamma}$. Using this basis, we can construct an
  element of $H^1(\wt{M}; \ZZ)$ by simply assigning values to the basis elements. \caleb[margin]{Is this not an element of $H^1(\wt{M}, \RR)$?} Pick an element
  $\alpha$ that assigns $0.5$ to $\wt{\gamma}$ and an integer value to every other basis
  element. Consider the cohomology class $\alpha + \iota^{\ast}\alpha$. Since
  $\iota \wt{\gamma} = \wt{\gamma}$ (because $\wt{\gamma}$ is the pre-image of a curve of $M$),
  $\alpha + \iota^{\ast}\alpha$ is an $\iota^{\ast}$-invariant element of $H^1(\wt{M}; \ZZ)$ that
  assigns $1$ to $\wt{\gamma}$. Such a cohomology class cannot be a pullback of a class on $M$ since
  the pullback of a cohomology class on $M$ would assign an even value to $\wt{\gamma}$.
\end{example}

What the above example does show is that for any $\iota^{\ast}$-invariant $\alpha$ in
$H^1(\wt{M}; \ZZ)$, $2\alpha$ definitely is a pullback of class in $H^1(M; \ZZ)$.

\subsection{Inverting the Poincar\'e duality map for embedded surfaces}
\label{sec:invert-poincare}

In both the orientable and non-orientable setting, we have a way of assigning an embedded surface
to an element $\alpha$ in $H^1(M; \ZZ)$ by looking at $f_{\alpha}^{-1}(p)$, where $p$ is a regular
value. In the orientable setting, the homology class of this embedded surface is well defined,
independent of the choice of representative $1$-form in its cohomology class, as well as the choice
of regular value. That's true in the non-orientable setting as well, but the homology class is not
very useful. We'd like to invert this construction, i.e. given an embedded surface $S$, we want a
closed $1$-form $\alpha$ such that the surface $S$ comes from $\alpha$ in the manner described
above.

For an orientable $3$-manifold $M$, this is just a consequence of Poincar\'e duality, so the only
work we need to do is prove this for non-orientable manifolds. If we do so, we can talk about the
norm of an embedded surface, and more generally, have an ad hoc version of Poincar\'e duality,
associating embedded surfaces to $1$-forms.
\begin{thm}[Poincar\'e duality for non-orientable $3$-manifolds]
  \label{thm:Poincare-duality}
  Let $M$ be a non-orientable $3$-manifold, and let $S$ be an embedded surface. Then
  there exists a cohomology class $[\alpha]$ in $H^1(M; \ZZ)$ such that for some representative
  $\alpha$, $S$ is $f_{\alpha}^{-1}(p)$ for some regular value $p \in S^1$.
\end{thm}

The idea of the proof of this theorem is fairly straightforward. Starting with the embedded surface
$S$ in $M$, we look at the pre-image $\wt{S}$ in the orientation double cover $\wt{M}$. We show
that the Poincar\'e dual to $\wt{S}$ is $\iota^{\ast}$-invariant.
\begin{lem}
  \label{lem:PD1}
  Let $S$ be an embedded surface in $M$, and $\wt{S}$ its pre-image in $\wt{M}$. Then the
  Poincar\'e dual to $[\wt{S}]$, after arbitrarily picking an orientation on $\wt{S}$, is
  $\iota^{\ast}$-invariant.
\end{lem}
\begin{proof}
  Note that even if we have a parameterization of $S$ in $M$, i.e. a map from a reference surface,
  we do not automatically get a parameterization of $\wt{S}$ in $\wt{M}$, so it doesn't make sense
  to talk about the homology class of $\wt{S}$ in $\wt{M}$. However, we can pick a reference
  surface $S$, and consider its orientation double cover (in the case where $S$ it self is
  orientable, the double cover is just two copies of $S$ with opposite orientations). We then
  have two choices of maps from our reference double cover to $\wt{S}$, and by picking one of
  those two choices, we get the class $[\wt{S}]$.

  Note that the deck transformation $\iota$ reverses the orientation on $\wt{S}$, which means
  $[\wt{S}]$ is in the $-1$-eigenspace of the $\iota_{\ast}$ action on $H_2(\wt{M}; \RR)$. This
  means that the Poincar\'e dual to $[\wt{S}]$, which we'll call $\wt{\alpha}$, is in the
  $1$-eigenspace, i.e. $\iota^{\ast}$-invariant. This just follows from the following
  chain of equalities which hold for all closed $2$-forms $\omega$.
  \begin{align*}
    \int_{\iota_{\ast}\wt{S}} \omega &= \int_{\wt{S}} \iota^{\ast}\omega &&\text{(By a change of variables)} \\
                                     &= \int_{\wt{M}} \wt{\alpha} \wedge \iota^{\ast} \omega &&\text{(Poincar\'e duality)} \\
                                     &=\int_{\wt{M}} \iota^{\ast} \left( \iota^{\ast}\wt{\alpha} \wedge \omega \right) \\
    &= \int_{\wt{M}} - \left( \iota^{\ast} \wt{\alpha} \wedge \omega \right) &&\text{($\iota$ is orientation reversing)}
  \end{align*}
  On the other hand, the following equalities follow from the fact that
  $\iota_{\ast}[\wt{S}] = -[\wt{S}]$.
  \begin{align*}
    \int_{\iota_{\ast}\wt{S}} &= - \int_{\wt{S}} \omega \\
                              &= - \int_{\wt{M}} \wt{\alpha} \wedge \omega
  \end{align*}
  Comparing the right hand side of the two equations, it follows that $\wt{\alpha}$ is
  $\iota^{\ast}$-invariant.
\end{proof}

We now have an $\iota^{\ast}$-invariant $1$-form $\wt{\alpha}$ such that $\wt{S}$ is
$f_{\wt{\alpha}}^{-1}(p)$ for some regular value $p$ on $S^1$. The next claim we want to make is that
the map $f_{\wt{\alpha}}: \wt{M} \to S^1$ factors through the quotient $M$.
\begin{lem}
  \label{lem:PD2}
  The map $f_{\wt{\alpha}}$ factors through $M$, i.e. for all points $x \in \wt{M}$, $f_{\wt{\alpha}}(x) = f_{\wt{\alpha}}(\iota (x))$.
\end{lem}
\begin{proof}
  Recall that $f_{\wt{\alpha}}(x)$ is given by the following integral formula.
  \begin{align*}
    f_{\wt{\alpha}}(x) = \left. \int_{x_0}^x \wt{\alpha} \right/ \ZZ
  \end{align*}
  Here $x_0$ is some arbitrarily chosen basepoint on $\wt{M}$. For $f_{\wt{\alpha}}(x)$ to equal
  $f_{\wt{\alpha}}(\iota(x))$ for all $x$, we must have the following.
  \begin{align*}
    \left(  \int_{x_0}^x \wt{\alpha} - \int_{x_0}^{\iota(x)} \wt{\alpha} \right) \in \ZZ
  \end{align*}
  By a change of variables, and using the $\iota^{\ast}$-invariance of $\wt{\alpha}$, the left hand
  side of the above condition can be transformed, giving us the following condition.
  \begin{align}
    \label{cond:integer}
    \left( \int_{x_0}^{\iota(x_0)} \wt{\alpha} \right) \in \ZZ
  \end{align}
  In other words, we want the integral of $\wt{\alpha}$ along any curve $\gamma$ from $x_0$ to
  $\iota(x_0)$ to be an integer. Equivalently, it will suffice to show that the integral of
  $\wt{\alpha}$ along $\delta$ is an even integer, where $\delta$ is the closed curve obtained by taking
  the union of $\gamma$ and $\iota(\gamma)$.

  Recall now that the $2$-parity of $\displaystyle \int_{\delta} \wt{\alpha}$ is precisely the $2$-parity
  of the intersection number of $\delta$ and $\wt{S}$ as long as all the intersections are
  transversal. Furthermore, both $\delta$ and $\wt{S}$ are lifts of a curve and surface from
  $M$. Which means the number of intersections they have in $\wt{M}$ is twice the number of
  intersections have in $M$. But the latter number must be an integer, and thus the former number
  must be an even integer, showing that condition \eqref{cond:integer} holds. In particular,
  this shows that the map $f_{\wt{\alpha}}$ factors through, proving the lemma.
\end{proof}
We now have everything we need to finish proving Theorem \ref{thm:Poincare-duality}.
\begin{proof}[Proof of Theorem \ref{thm:Poincare-duality}]
  Starting with a surface $S$ in $M$, we look at its pre-image $\wt{S}$ in $\wt{M}$. After
  arbitrarily choosing a parameterization on $\wt{S}$, we get a homology class, and we get a
  $1$-form $\wt{\alpha}$, which is Poincar\'e dual to the homology class of $\wt{S}$.  More
  specifically, we have that $\wt{S}$ is $f^{-1}_{\wt{\alpha}}(p)$ for some regular value $p$.
  By lemma \ref{lem:PD1}, $\wt{\alpha}$ is $\iota^{\ast}$-invariant, and by lemma \ref{lem:PD2},
  we have that $f_{\wt{\alpha}}$ factors through $M$, which means $f_{\alpha}^{-1}(p) = S$. Furthermore,
  we also have that $\alpha$ lies in $H^1(M; \ZZ)$. That just follows from the fact that $\alpha$
  is the pullback of the form $d\theta$ on $S^1$ along the map $f_{\alpha}$. This proves the result.
\end{proof}

\subsection{Relating $1$-forms and fibrations over $\RR/\ZZ$}
\label{sec:relating-1-forms}
While we have informally described what a fibration over $S^1$ is prior to this section, it will be
useful to formally define a fibration at this stage.
\begin{defn}[Fibration over $S^1$]
  Given a $3$-manifold $M$, a fibration (or a fiber bundle) over $S^1$ is a map $f: M \to S^1$ such
  that the derivative of $f$ has full rank at all points in $M$. The pre-image of every point in
  $S^1$ is an embedded surface in $M$, which is called the fiber of the fibration.
\end{defn}
Note that any $3$-manifold that admits a fibration over $S^1$ is a mapping torus of the fiber (which
is a surface), along with the homeomorphism that comes from the transition map when changing
coordinate charts on $S^1$. We can instead look at homotopy classes of fibrations, and every
equivalence class will correspond to a homotopy class of a homeomorphism of the fiber, i.e. a
mapping class. Since we're mostly interested in mapping tori of mapping classes rather than mapping
classes of specific homeomorphisms in those mapping classes, we'll be focusing on homotopy classes
of fibrations.

It turns out that the ``right'' way to think about fibrations $f: M \to S^1$ is by studying a specific kind
of $1$-form on $M$, namely \emph{non-singular integer $1$-forms}.

\begin{defn}
  A non-singular integer $1$-form on a $3$-manifold $M$ is a smooth nowhere vanishing $1$-form $\alpha$ on
  $M$ such that for any closed loop $\gamma$, the integral of $\alpha$ along $\gamma$ lies in $\ZZ$.
  \begin{align*}
    \int_{\gamma} \alpha \in \ZZ
  \end{align*}
\end{defn}

Given a non-singular integer $1$-form, there is a canonical way of getting a fibration $f_{\alpha}$ over
$S^1$, as we've seen before. The map $f_{\alpha}$ is given by the following formula.
\begin{align*}
  f_{\alpha}(x) \coloneqq \left. \int_{x_0}^x \alpha \right/ \ZZ
\end{align*}
Conversely, given a fibration $f: M \to S^1$, one can obtain a non-singular integer $1$-form by
pulling back a $1$-form along the map $f$. The correct $1$-form to pull back is $d\theta$, i.e.  the
non-vanishing $1$-form on $S^1$ such that $\int_{S^1} d\theta = 1$ (note that despite the notation,
this is not an exact form). These two constructions are inverses of each other, which is fairly easy
to verify. Furthermore, if we change $\alpha$ to $\alpha + df$, where $df$ is an exact form, then
the associated map to $S^1$ is not the same, but homotopic to the original map. Conversely, if we
pull back $d\theta$ along a map homotopic to $f$ rather than $f$, we get a form that differs from
the original form by an exact form (see Section 5.2.1 of \cite{calegari2007foliations} for the
details). The takeaway here is that if we only care about the mapping torus structure of the mapping
classes, we can focus our attention to the elements of $H^1(M; \ZZ)$ that admit a non-singular $1$-form
representative.
