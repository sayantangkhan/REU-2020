\section{Thurston norm for non-orientable $3$-manifolds}
\label{sec:thur-norm-non-orientable}

\subsection{Background on Thurston norm and mapping tori}
\label{sec:backgr-thurst-norm}

Given a orientable closed $3$-manifold $M$, the Thurston norm is a semi-norm on its second homology group $H_2(M; \RR)$; to define the norm, we need to make some preliminary remarks and define a function.
Let $S$ be a connected surface: define the complexity of $S$ to be $\chi_-(S) = \max\left\{-\chi(S),0\right\}$.
If the surface $S$ has multiple components $S_1, \ldots, S_m$ then $\chi_-(S) = \displaystyle\sum_{i=1}^m\chi_-(S_i)$.
Furthermore, for an \emph{orientable} 3-manifold $M$, it is known that elements in $H_2(M ; \ZZ)$ can be represented by embedded surfaces inside of $M$.
This lets us define a norm function $x$ for every homology class $a$ in $H_2(M; \ZZ)$:
\begin{align*}
  x(a) = \min\{\chi_-(S) \mid [S] = a \text{ and $S$ is compact, properly embedded and oriented}\}.
\end{align*}

We then linearly extend $x$ to the rational points.
There is then a is a unique continuous extension of $x$ to $\RR$: an $\RR$-valued function on $H_2(M; \RR)$ called the \emph{Thurston norm}.
The unit ball for the Thurston norm is a convex polyhedron, and thus it makes
sense to talk about the \emph{faces} of the unit ball.

In this paper, we'll focus on $3$-manifolds arising from mapping tori constructed from surfaces and their homeomorphisms.
Given a surface $S$ and a homeomorphism $\varphi: S \to S$, one can construct a $3$-manifold $M_\varphi$ via the \emph{mapping torus} construction.
\begin{align*}
  M_\varphi \coloneqq \frac{S \times [0,1]}{(x,1) \sim (\varphi(x), 0)}
\end{align*}
Mapping tori are surface bundles over $S^1$, or \emph{fibrations over $S^1$}, denoted $S\rightarrow M\rightarrow S^1$.
A fibration defines a flow on $M$, called the \emph{suspension flow}, where for any $x_0\in S$ and $t_0\in S^1$ the pair $(x_0,t_0)$ is sent to $(x_0,t_0+t)$.
The fiber of a fibration is the pre-image of any point $\theta \in S^1$ under the projection map from $M_{\varphi} \to S^1$.
The fiber as a subset of $M$ is only well defined up to isotopy, since we don't specify what $\theta$ we pick, but the homology class of the fiber in $H_2(M_{\varphi}; \RR)$ is well defined.

The theory of Thurston norm becomes useful for answering the following question: Given a compact oriented $3$-manifold $M$, in how many ways can $M$ be expressed as a mapping torus of a surface and a homeomorphism?
This question was answered by the following remarkable theorem of Thurston \cite{thurston1986norm}.
We use the restatement from \cite{yazdi2018pseudo}.
\begin{thm}[Thurston]
  \label{thm:Thur1}
  Let $\mathcal{F}$ be the set of homology classes in $H_2(M; \RR)$ that are representable by fibers of fibrations of $M$ over the circle.
\begin{enumerate}[(i)]
\item Elements of $\mathcal{F}$ are in one-to-one correspondence with (non-zero) lattice points inside some union of cones over open faces of the unit ball in the Thurston norm.
\item If a surface $F$ is transverse to the suspension flow associated to some fibration of
  $M \xrightarrow[]{} S^1$ then $[F]$ lies in the closure of the corresponding cone in $H_2(M;\RR)$.
\end{enumerate}
\end{thm}
The class $[F]$ referred to in the theorem has orientation such that the positive flow direction is pointing outwards relative to the surface.
The open faces whose cones contain the fibers of fibrations are what are referred to as \emph{fibered faces}.
The takeaway from this theorem is that if we want to determine if a surface $S$ embedded in a $3$-manifold is homologous to the fiber of some fibration, it suffices to check whether the homology class $[S]$ lies in one of the open cones corresponding to the fibrations.

The goal for the rest of this section is to prove a version of Theorem \ref{thm:Thur1} for compact non-orientable $3$-manifolds.
Most of the work in the proof will involve reducing the version for non-orientable $3$-manifolds to the orientable version by passing to the double cover.

\subsection{Thurston norm on cohomology of non-orientable mapping tori}
\label{sec:thurst-norm-cohom}

\subsubsection*{The problem with homology in non-orientable $3$-manifolds}
A na\"ive first attempt at defining the Thurston norm would be to define it on the second homology group, like in the orientable case.
However, if the norm is defined in that fashion, the non-orientable version of Theorem \ref{thm:Thur1} will not be true.
Consider a compact non-orientable surface $\no$, a homeomorphism $\varphi: \no \to \no$, and the associated mapping torus $M_{\varphi}$.
Clearly, $M_{\varphi}$ fibers over $S^1$, and $\no$ is the fiber of this fibration: however, the homology class associated to $\no$ is the $0$ homology class, since the top-dimensional homology of non-orientable compact surfaces is $0$-dimensional.

Our workaround for this problem will be to deal with the first cohomology $H^1(M)$ rather than the second homology $H_2(M)$.
By Poincar\'e duality they are the same for orientable 3-manifolds, but that is not true for non-orientable $3$-manifolds.

To see why Poincar\'e duality for non-orientable 3-manifolds fails, consider an orientable $3$-manifold $M$.
We can explicitly work out the map from $H^1(M; \ZZ)$ to $H_2(M; \ZZ)$ given by Poincar\'e duality.
To do so, we set up a correspondence between elements of $H^1(M; \ZZ)$ and homotopy classes of maps from $M$ to $S^1$ as follows.
Given a cohomology class $[\alpha]$ in $H^1(M; \ZZ)$, choose a representative $1$-form $\alpha$, and a basepoint $y_0$ in $M$.
The associated map $f_{\alpha}$ is given by the following formula.
\begin{align}\label{form:map}
  f_{\alpha}(y) \coloneqq  \int_{y_0}^y \alpha \mod \ZZ
\end{align}
Changing the basepoint or the representative $1$-form gives a different map to $S^1$ that is homotopic to the original map (see Section 5.1 of \cite{calegari2007foliations} for the details).
One can recover the $1$-form $\alpha$ from the map $f_{\alpha}$ by pulling back the canonical length form $d\theta$ on $S^1$ along $f_{\alpha}$.

Let $q\in S^1$ be a regular value and let $S = f_{\alpha}^{-1}(q)$ be a surface.
To construct the associated homology class, we choose an orientation on $S$ by declaring that the outwards pointing normal vectors on $S$ are assigned a positive value by the form $\alpha$.
Then $S$ inherits an orientation from the orientation on $M$, and we have defined a
fundamental class $[S]$.
We claim that $[S]$ is the Poincar\'e dual to $\alpha$.
\begin{lem}
  Let $q$ and $q'$ be two regular values of the function $f_{\alpha}$ and let $S=f_\alpha^{-1}(q)$ and $S'=f_\alpha^{-1}(q')$.
  Then for any closed $2$-form $\omega$ on $M$, the following identity holds:
  \begin{align*}
    \int_{S} \omega = \int_{S'} \omega.
  \end{align*}
  Furthermore, the following identity also holds:
  \begin{align*}
    \int_S \omega = \int_M \alpha \wedge \omega.
  \end{align*}
  In particular, the homology class of $S$ is Poincar\'e dual to $\alpha$.
\end{lem}
\begin{proof}
  The first part of the lemma follows from the fact that $S$ and $S'$ are homologous, i.e. $f^{-1}_{\alpha}([q, q'])$ is a singular $3$-chain that has $S$ and $S'$ as boundaries.
  From Stokes' theorem, we get the following:
  \begin{align*}
    \int_{S - S'} \omega &= \int_{f_{\alpha}^{-1}([q, q'])} d\omega \\
                         &= 0.
  \end{align*}

  To prove the second claim, observe that we can break up the second integral as a product integral:
  \begin{align*}
    \int_M \alpha \wedge \omega &= \int_{S^1} \left(   \int_{f_{\alpha}^{-1}(\theta)} \omega \right) d\theta.
  \end{align*}
  The above equation is true because $\alpha$ is the pullback of $d\theta$ along the map $f_{\alpha}$.
  Observe that the inner integral only makes sense when $\theta$ is a regular value, but by Sard's theorem, almost every $\theta \in [0,1]$ is a regular value, so the right hand side is well-defined.
  By the first claim, the inner integral is a constant function, as we vary over the $\theta$ which are regular values of $f_{\alpha}$.
  Then the integral of $d\theta$ over $S^1$ is $1$, giving us the identity we want:
  \begin{align*}
    \int_M \alpha \wedge \omega = \int_S \omega.
  \end{align*}
\end{proof}
What we have here is an explicit formula for the Poincar\'e duality map from $H^1(M; \RR)$ to $H_2(M; \RR)$.
For orientable $3$-manifolds, this is an isomorphism, and more specifically the following theorem is true.
\begin{thm}[Poincar\'e duality for orientable $3$-manifolds]
  \label{thm:orientable-poincare-duality}
  Let $M$ be an orientable $3$-manifold, and let $S$ be an oriented embedded surface. Then there exists a $1$-form
  $\alpha$ and a regular value $q\in M$ such that $S$ and $f_{\alpha}^{-1}(q)$ are homologous surfaces.
\end{thm}

Note that the map from the space of $1$-forms to homology classes of an embedded surface still makes sense for a non-orientable $3$-manifold $M$.
However in that case the map from $H^1(M; \ZZ)$ to $H_2(M; \ZZ)$ has a nontrivial kernel, as we saw in the example where $M$ is the mapping torus of a non-orientable surface.

\subsubsection*{Defining the norm on cohomology}
For this section, we'll use $M$ to denote a non-orientable $3$-manifold, and $\wt{M}$ to denote its orientation double cover.
We will denote by $\iota$ the orientation reversing deck transformation
of $\wt{M}$, and the covering map $\wt{M} \to M$ by $p$.
If $M=M_\varphi$ is the mapping torus of the non-orientable surface $\no$ and a self-homeomorphism $\varphi: \no \to \no$, then $\wt{M}$ is the mapping torus of $(\os, \wt{\varphi})$, where $\os$ is the orientable double cover of $\no$, and $\wt{\varphi}$ is the orientation preserving lift of $\varphi$.

Since we have already concluded that the first cohomology is the right space on which to define the Thurston norm, we need to relate $H^1(M; \RR)$ and $H^1(\wt{M}; \RR)$. We do so by pulling back $H^1(M;\RR)$ to $H^1(\wt{M};\RR)$ via $p$.

\begin{lem}
  \label{lem:injective}
  The pullback $p^{\ast}:H^1(M;\RR)\rightarrow H^1(\wt{M};\RR)$ maps $H^1(M; \RR)$ bijectively to the $\iota^{\ast}$-invariant subspace of   $H^1(\wt{M}; \RR)$.
\end{lem}
\begin{proof}
  For any $1$-form $\alpha$ on $M$, $p^{\ast}(\alpha)$ will be $\iota^{\ast}$-invariant.
  To check that $p^\ast$ is injective, consider a $1$-form $\alpha$ on $M$ such that $p^{\ast}\alpha$ is exact.
  Then there exists a smooth function $g:\wt{M} \to \RR$ such that the following relation holds:
    \begin{align*}
        dg = p^{\ast} \alpha.
    \end{align*}
    But since $p^{\ast}\alpha$ is $\iota^{\ast}$-invariant, we must have $dg = \iota^{\ast} dg$.
    Because $\iota^\ast$ commutes with the exterior derivative, we have $dg = d(\iota^{\ast}g)$.
    That means $g$ and $\iota^{\ast}g$ differ by a constant, but that constant must be $0$ since $\iota^2$ is the identity map.
    Thus $g$ is $\iota$-equivariant and therefore descends to a function on $M$ and $\alpha$ must be exact, which proves injectivity of $p^{\ast}$.

    To show surjectivity, let $[\alpha]$ be an element in $H^1(\wt{M}; \RR)$ that is $\iota^{\ast}$-invariant and let $\alpha$ be a representative.
    Since the cohomology class $[\alpha]$ is $\iota$-invariant, $\alpha$ and $\iota^{\ast}\alpha$ must differ by an exact form.
    \begin{align*}
        \alpha - \iota^{\ast}(\alpha) = dg
    \end{align*}
    Applying $\iota^\ast$ to both sides of the equality, we have $\iota^{\ast}dg = -dg$.
    This means that the $1$-form $\beta = \alpha - \frac{dg}{2}$ is an $\iota^{\ast}$-invariant representative of the cohomology class $[\alpha]$.
    Since $\beta$ is $\iota^{\ast}$-invariant, we can push it forward by the projection map $p$, defining the value of its pushforward locally.
    This gives a well defined $1$-form on $M$, whose pullback is $\beta$, and proves surjectivity of $p^{\ast}$.
\end{proof}
\begin{rem}
  Note that if we change the coefficients in the statement of this lemma from $\RR$ to $\ZZ$, the proof of injectivity follows through, but the proof of surjectivity does not.
  {\color{red} In general, one does not have surjectivity with $\ZZ$ coefficients, }
\end{rem}

Lemma \ref{lem:injective} tells us that $H^1(M; \RR)$ is a subspace of $H^1(\wt{M}; \RR)$, so we define the Thurston norm on $H^1(M; \RR)$ by restricting the Thurston norm on the orientable 3-manifold $\widetilde{M}$ to the subspace $p^*(H^1(M;\RR))$ of $H^1(\widetilde{M};\RR)$.

\begin{defn}[Thurston norm for non-orientable $3$-manifolds]
  Let $M$ be a non-orientable 3-manifold and $\wt{M}$ its orientation double cover, $\wt{x}$ be the Thurston norm on $H^1(\wt{M};\RR)$ defined in Section \ref{sec:backgr-thurst-norm}, and let $\alpha \in H^1(M;\RR)$.
  The \emph{Thurston norm on $H^1(M; \RR)$}, is the norm $x: H^1(M;\RR) \rightarrow \RR$ defined:
  \begin{align*}
    x(\alpha) \coloneqq \wt{x}(p^{\ast}\alpha).
  \end{align*}
\end{defn}

We now extend properties of the Thurston norm for orientable manifolds to the Thurston norm on $H^1(M;\RR)$.
\begin{thm}
  The unit ball with respect to the dual Thurston norm on $\left( H^1(M; \RR) \right)^{\ast}$ is a polyhedron in $(H^1(M,\RR))^\ast$ whose vertices are lattice points $\{\pm \beta_1, \ldots \pm \beta_k\}$.
  The unit ball $B_1$ with respect to Thurston norm is a polyhedron given by the following inequalities.
  \begin{align*}
    B_1 = \left\{ a\in H^1(M,\RR) \mid \left| \beta_i(a) \right| \leq 1 \text{ for $1\leq i \leq k$} \right\}
  \end{align*}
\end{thm}

\begin{proof}
  The proof is identical to the original proof of Thurston \cite[Theorem 2]{thurston1986norm}.
  Because the norm of an element of $H^1(M; \ZZ)$ is the Thurston norm of the corresponding element in $H^1(\wt{M}; \ZZ)$, the norm of any element in $H^1(M; \ZZ)$ is also an integer.
  The linear algebra follows identically.
\end{proof}

Note that defining the Thurston norm on $H^1(M; \RR)$ rather than $H_2(M; \RR)$ is not quite satisfactory.
In particular, fibers of fibrations are embedded surfaces in $M$, and in the orientable case, the norm on the second homology relates these fibers to the Thurston norm.
In Section \ref{sec:weak-inverse-poinc}, we develop a version of Poincar\'e duality for non-orientable 3-manifolds that will let us translate a family of embedded surfaces into $1$-forms, allowing us to talk about their Thurston norm.

\subsection{Weak inverse to the Poincar\'e duality map}
\label{sec:weak-inverse-poinc}

\subsection{Oriented sums}
\label{sec:oriented-sums}
