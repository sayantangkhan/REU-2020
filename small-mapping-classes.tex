\section{Mapping classes with small stretch factors}
\label{sec:mapping-classes-with}

In this section, we provide a strategy to compute pseudo-Anosov homeomorphisms with small stretch factors.
%The key tools from \autoref{sec:thur-norm-non-orientable} are Theorems \ref{thm:strong-duality} and \ref{thm:classifying-fibrations}, and the operation of oriented sums for non-orientable surfaces.

\subsection{Mapping class groups of non-orientable surfaces}
\label{sec:backgr-mapp-class}
Let $\no$ be a non-orientable surface and let $\wt{\no}$ and the covering map $p:\wt{\no}\rightarrow \no$ be its orientation double covering space.
Every homeomorphism $\varphi: \no \to \no$, has a unique orientation-preserving lift $\wt{\varphi}: \wt{\no} \to \wt{\no}$.


A consequence is that lifting homeomorphisms induces a monomorphism between homeomorphisms of $\no$ and orientation-preserving homeomorphisms of $\wt{\no}$.
Every homotopy of $\no$ lifts to a homotopy of $\wt{\no}$.
%Further, if $f,g:\no\to\no$ are homeomorphisms such that their orientation preserving lifts are homotopic, then $f$ and $g$ are homotopic.
%\becca[inline]{Do we need this (the previous) sentence?  It does not follow naturally from the sentence before it and actually requires proof/citation.  But it is only required if we want the inclusion to be injective.}
%\sayantan[inline]{We don't really need this to be injective, but the fact is probably folklore: we could prove it if we needed to, but it would be more elementary covering space stuff, and wouldn't add anything useful.}
%\becca[inline]{That's what I thought (we could also just cite it, there's a paper that proves it somewhere out there).  But I think we should just remove that sentence.}
Therefore there is an inclusion from the mapping class group of $\no$ to the (orientation-preserving) mapping class group of $\wt{\no}$.
This inclusion also respects the Nielsen-Thurston classification of mapping classes, both qualitatively, and quantitatively, as the following proposition shows.
\begin{prop}
  \label{prop:2}
  Let $\varphi:\no\rightarrow\no$ be a homeomorphism and let $\wt{\varphi}:\wt{\no}\rightarrow\wt{\no}$ be the orientation-preserving lift of $\varphi$.  Then:
  \begin{enumerate}[(i)]
  \item $\varphi$ is periodic if and only if $\wt{\varphi}$ is periodic,
  \item $\varphi$ is reducible if and only if $\wt{\varphi}$ is reducible, and
  \item $\varphi$ is pseudo-Anosov if and only if $\wt{\varphi}$ is pseudo-Anosov.  Moreover if $\varphi$ has stretch factor $\lambda$, then $\wt{\varphi}$ also has stretch factor $\lambda$.
  \end{enumerate}
\end{prop}
\begin{proof}
 % It's easy to see that if $\varphi$ is periodic, so it $\wt{\varphi}$, and the other way round.
  %If $\varphi$ is reducible, that means it leaves some multicurve $\gamma$ in $\no$ invariant, which means $\wt{\varphi}$ leaves the pre-image of $\gamma$ invariant as well.
  %Conversely, if $\wt{\varphi}$ leaves some multicurve $\wt\gamma$ invariant, so does $\iota \circ \wt{\varphi}$, since they commute.\becca[inline]{Need more here.  I'll think about it (See Aramayona--Leininger--Souto ``Injections on mapping class groups": there exist (branched) covering spaces where pA lift to reducible.  This is not such a case, but why?)}
  %That means the union of $\wt\gamma$ and $\iota(\wt\gamma)$, where $\iota$ is the orientation reversing deck transformation, is also a multi-curve and thus descends to a multi-curve on $\no$ that is left invariant by $\varphi$.\becca[inline]{The image of the multicurve may be non-simple or trivial}
  %If $\varphi$ is neither periodic nor reducible, it must be pseudo-Anosov.  By exclusion, $\wt{\varphi}$ must also be pseudo-Anosov.
  The fact that the map from $\Mod(\no)$ to $\Mod(\wt{\no})$ is type-preserving follows from Aramayona--Leininger--Souto \cite[Lemma 10]{aramayona2009injections} (while the statement of the Lemma is for orientable surfaces, the argument, which we will skip, is identical for non-orientable surfaces).

  Suppose now that $\varphi:\no\rightarrow\no$ is a psuedo-Anosov homeomorphism with stretch factor $\lambda$ and stable and unstable foliations $\mathcal{F}_s$ and $\mathcal{F}_u$ respectively.
  Let $\wt{\mathcal{F}_s}$ and $\wt{\mathcal{F}_u}$ denote the lifts of the stable and unstable foliations to the orientation double cover.  Let $\gamma$ be a simple closed curve in $\wt{\no}$.
  We need to show that the following identities hold for all $\gamma$ (see \cite[Expos\'e 5]{FLP} for the definition of intersection number with measured foliations; the fact that these identities suffice follows from \cite[Lemma 9.15]{FLP}):
%  \sayantan[inline]{Provide citations for stretch factor claim and also intersection number with foliations definition.}
  \begin{align}
      \label{eq:unstable-foliation}
      i(\gamma, \wt{\varphi}(\wt{\mathcal{F}_u})) &= \lambda \cdot i(\gamma, \wt{\mathcal{F}_u}) \\
      \label{eq:stable-foliation}
      i(\gamma, \wt{\varphi}(\wt{\mathcal{F}_s})) &= \frac{1}{\lambda} \cdot i(\gamma, \wt{\mathcal{F}_s}).
  \end{align}

  To see that (\ref{eq:unstable-foliation}) holds, we partition $\gamma$ into short arcs $\{\gamma_i\}$ such that the restriction of the covering map $p$ to a neighborhood of each arc is a homeomorphism.
  %The local homeomorphism lets us compute the intersection number for each arc $\gamma_i$ by instead computing the intersection number on the surface $\no$:
  Then we have:
  \begin{align}
  \label{eq:push1}
    i(\gamma_i, \wt{\mathcal{F}_u}) &= i(p(\gamma_i), \mathcal{F}_u) \\
  \label{eq:push2}
    i(\gamma_i, \wt{\varphi}(\wt{\mathcal{F}_u})) &= i(p(\gamma_i), \varphi(\mathcal{F}_u)).
  \end{align}
  Since we know that $\mathcal{F}_u$ is the unstable foliation for $\varphi$ with stretch factor $\lambda$, we can compute the ratio of the right hand side of \eqref{eq:push1} and \eqref{eq:push2}:
  \begin{align}
      \label{eq:ratio}
      i(p(\gamma_i), \varphi(\mathcal{F}_u)) = \lambda \cdot i(p(\gamma_i), \mathcal{F}_u).
  \end{align}
  Combining \eqref{eq:push1}, \eqref{eq:push2}, and \eqref{eq:ratio}, and summing over all $\gamma_i$ gives us that \eqref{eq:unstable-foliation} holds. A similar argument also proves that \eqref{eq:stable-foliation} holds.
\end{proof}



\subsection{Constructing pseudo-Anosov maps using oriented sums}
\label{sec:constr-psuedo-anos}
%\becca[inline]{I rewrote the section.  The old version is in comments.}
The goal of this section is to prove that the stretch factor of any pseudo-Anosov homeomorphism provides an asymptotic upper bound for the minimum stretch factor.  We do this in Proposition \ref{prop:asymptotic}.


\begin{prop}\label{prop:asymptotic}
Let $\no_g$ be a non-orientable surface of genus $g$ and let $\varphi:\no_g\rightarrow \no_g$ be a pseudo-Anosov homeomorphism with stretch factor $\lambda$.  Let $N_\varphi$ be the mapping torus of $\no_g$ by $\varphi$.  Let $\no_{g'}$ be a genus $3$ non-orientable relatively orientable surface embedded in $N_\varphi$ that is transverse to the suspension flow associated to $\varphi$.  Then for all $k\in\ZZ^+$, there is a pseudo-Anosov homeomorphism of the oriented sum $\no_{g}+k\no_{g'}$ with stretch factor at most $\lambda$.
\end{prop}

Our strategy for proving Proposition \ref{prop:asymptotic} is to find fibrations of $N_{\varphi}$ over $S^1$ that have fiber $\no_g+k\no_{g'}$.  We then apply a special case of Thurston's hyperbolization theorem, which says that the mapping torus of an orientable surface $S$ by a homeomorphism $\varphi$ is hyperbolic if and only if $\varphi$ is pseudo-Anosov \cite[Theorem 0.1]{thurston_hyp}.  In particular, Thurston's theorem implies that if $M=M_\varphi$ fibers over $S^1$ in two ways, either both mondromies are pseudo-Anosov or neither monodromy is pseudo-Anosov.  Finally, we adapt theorems of Fried and Matsumoto (Theorem \ref{thm:fm}) and Agol--Leininger--Margalit (Theorem \ref{thm:alm}) to work for mapping tori with non-orientable fibers.


\medskip
We will use the following two facts for orientable surfaces and hyperbolic 3-manifolds that fiber over $S^1$:
\begin{enumerate}
 % \label{thm:norm-minimizing}
 \item A Thurston norm-minimizing surface $S$ is incompressible \cite[Lemma 5.7]{calegari2007foliations}.
\item  The fiber of any fibration over $S^1$ minimizes the Thurston norm in its homology class \cite[Corollary 2]{thurston1986norm}.
\end{enumerate}


%As a consequence, if we can decompose the mapping torus of a pseudo-Anosov map as the mapping torus of some other homeomorphism, then the other homeomorphism must also be pseudo-Anosov.


%Let $\no$ be a non-orientable surface, and $\varphi$ be a pseudo-Anosov homeomorphism.
%Let $N_\varphi$ be the associated mapping torus and $f:N\rightarrow S^1$ the associated fibration.
%Suppose we can construct another relatively oriented surface $\no'$ inside $N$ such that $\no'$ is transverse to the suspension flow direction associated to $f$. Let $\alpha$ be the Poincar\'e dual of $\no$ and $\alpha'$ be the Poincar\'e dual of $\no'$.  By Theorem \ref{thm:classifying-fibrations} $\alpha$ lies in a cone $\mathcal{C}_\varphi \subset H^1(N_\varphi; \ZZ)$ associated to $\varphi$ with other $1$-forms also coming from fibrations.
%Furthermore, $\alpha'$ lies in the closure of $\mathcal{C}_\varphi$.
%All positive integer linear combinations of $\alpha$ and $\alpha'$ are lattice elements of $\mathcal{C}_\varphi$.
%Each linear combination of $\alpha$ and $\alpha'$ is Poincar\'e dual to an oriented sum of $\no$ and $\no'$ in $N$.  Under reasonably mild conditions on $\no$ and $\no'$, we can show that the fiber of $f$ is homeomorphic to the oriented sum $\no + \no'$ of $\no$ and $\no'$.

\begin{prop}
  \label{thm:oriented-sum}
  Let $\no'$ be a genus $3$ non-orientable relatively orientable surface embedded in $N$ that is transverse to the suspension flow associated to $\varphi$.
  Let $\alpha$ be the Poincar\'e dual of $\no$ and $\alpha'$ the Poincar\'e dual of $\no'$.
  If the oriented sum of $\no$ and $\no'$ is connected, then $\no + \no'$ is homeomorphic to the fiber of the fibration given by $\alpha + \alpha'$.
\end{prop}
\begin{proof}
  We first need to show that $\no^{\prime}$ is incompressible to consider its Poincar\'e dual: this follows from the fact that the pre-image $\wt{\no^{\prime}}$ in the orientation double cover is a genus $2$ surface, and minimizes the Thurston norm in its homology class.
  If it did not minimize the Thurston norm in the homology class, then the norm minimizing surface in its homology class would have to be a torus or a sphere, but that would contradict the fact the $3$-manifold is the mapping torus of a pseudo-Anosov map. By \cite[Lemma 5.7]{calegari2007foliations}, we have that it must be incompressible, and therefore $\no^{\prime}$ is incompressible too.

  Let $p:\wt{N}\rightarrow N$ be the orientation double cover of $N$.
  The surface $\no$ minimizes Thurston norm because it is a fiber of $f$; for the same re $p^{-1}(\no)$ is also minimizes Thurston norm, thus the Thurston norm of $\alpha$ is $2\chi_-(\no)$.  Likewise, the Thurston norm of $\alpha'$ is $2\chi_-(\no')$.

By \autoref{thm:classifying-fibrations} (ii), $\alpha'$ lies in the same cone in $H^1(N;\ZZ)$ as $\alpha$.  The Thurston norm $x$ on $H^1(N;\ZZ)$ is linear function on that cone.
 Since the Thurston norm is also linear on oriented sums of $\no$ and $\no'$, we have:
  \begin{align*}
    x(\alpha + \alpha') &= x(\alpha) + x(\alpha') \\
                        &= 2\chi_-(\no) + 2\chi_-(\no') \\
                        &= 2\chi_-(\no + \no').
  \end{align*}

  Because $2\chi_-(\no+\no')$ achieves the Thurston norm of $\alpha+\alpha'$, the preimage $p^{-1}(\no+\no')$ achieves the Thurston norm of the pullback of $\alpha+\alpha'$ under $p$.  Therefore $p^{-1}(\no+\no')$ is incompressible.  Then $\no+\no'$ is also incompressible.


  By Theorem \ref{thm:classifying-fibrations} (i), we have that $\alpha + \alpha'$ corresponds to some other fibration $f'':N\rightarrow S^1$.
  By Theorem \ref{thm:strong-duality}, the fiber of $f''$ must be homeomorphic to $\no + \no'$.
\end{proof}

In the proof of Proposition \ref{prop:asymptotic}, we will use Proposition \ref{thm:oriented-sum} along with a theorem of Thurston to obtain a pseudo-Anosov homemorphism $\varphi_k$ of the surface of genus of genus $g+kg'$.  We the use Theorems \ref{thm:fm} and \ref{thm:alm} to obtain a upper bound on the stretch factor of $\varphi_k$.


\begin{thm}[Fried \cite{fried1982flow,fried1983transitive}, Matsumoto \cite{matsumoto1987topological}]
  \label{thm:fm}
  Let $M$ be an orientable hyperbolic $3$-manifold and let $\mathcal{K}$ be the union of cones in $H^1(M; \RR)$ whose lattice points correspond to fibrations over $S^1$.
  There exists a strictly convex function $h: \mathcal{K} \to \RR$ satisfying the following properties:
  \begin{enumerate}[(i)]
  \item For all $c > 0$ and $u \in \mathcal{K}$, $h(cu) =  \frac{1}{c}h(u)$,
  \item For every primitive lattice point $u \in \mathcal{K}$, $h(u) = \log(\lambda)$, where $\lambda$ is the
    stretch factor of the pseudo-Anosov map associated to this lattice point, and
  \item $h(u)$ goes to $\infty$ as $u$ approaches $\partial \mathcal{K}$.
  \end{enumerate}
\end{thm}

\begin{thm}[Agol-Leininger-Margalit]
  \label{thm:alm}
  Let $\mathcal{K}$ be a fibered cone for a mapping torus $M$ and let $\overline{\mathcal{K}}$ be its closure
  in $H^1(M;\RR)$. If $u \in \mathcal{K}$ and $v \in \overline{\mathcal{K}}$, then $h(u+v) < h(u)$.
\end{thm}

\begin{proof}[Proof of Proposition \ref{prop:asymptotic}]
The oriented sum $\mathcal{S}=\no_g+k\no_{g'}$ constructed in Proposition \ref{thm:oriented-sum} is a surface of genus $g+kg'$, and $\mathcal{S}$ is homeomorphic to a fiber of $N_\varphi$ given by $\alpha+k\alpha'$.  Let $\varphi_{k}:\mathcal{S}\rightarrow\mathcal{S}$ be the monodromy of $N_\varphi$ over $\mathcal{S}$.  By Thurston's theorem, $\varphi_k$ is pseudo-Anosov.  We claim that $\varphi_k$ has stretch factor at most $\lambda$.


Let $p:\wt{N}\rightarrow N_\varphi$ be the orientation double cover of $N_\varphi$. Let $h\mid_{N}$ be the restriction of $h$ to the pullback  $p^\ast(H^1(N_\varphi; \RR))$ in $H^1(\wt{N}; \RR)$.
The restriction $h\mid_N$ satisfies all the properties of Theorems \ref{thm:fm} and \ref{thm:alm}.

 Let $\wt{\varphi}$ be the orientation preserving lift of $\varphi$ to $p^{-1}(\no)$.  Since $\wt{\alpha}$ is the pullback of $\alpha$, the map $\wt{\varphi}$ is the pseudo-Anosov homeomorphism associated to $\wt{\alpha}$.  By Proposition \ref{prop:2}, the stretch factor of $\wt{\varphi}$ is $\lambda$.

Let $\mathcal{K}$ be the cone in $H^1(N_\varphi;\RR)$ that contains $\alpha$.  Since $\no_{g'}$ is transverse to the suspension flow in the direction of $\varphi$, we have that $\alpha'$ is in the closure of $\mathcal{K}$ in $H^1(N;\RR)$.  Let $\wt{\alpha}$ be the pullback of $\alpha$ under $p$ and let $\wt{\alpha}'$ be the pullback of $\alpha'$ under $p$.  Then $h\mid_N(\wt{\alpha}+\wt{\alpha}')<h\mid_N(\wt{\alpha})$.  By Theorem \ref{thm:fm}, $h(\wt{\alpha})$ is equal to the stretch factor of the pseudo-Anosov homeomorphism associated to $\wt{\alpha}$.  Therefore we have $h\mid_N(\wt{\alpha}+\wt{\alpha}')<\log(\lambda)$. It follows that the stretch factor of $\varphi_k$ is less than $\lambda$.
\end{proof}
