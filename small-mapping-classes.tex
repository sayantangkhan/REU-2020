\section{Mapping classes with small stretch factors}
\label{sec:mapping-classes-with}

In this section, we'll see how to construct mapping classes with small stretch factor on non-orientable surfaces.
The key tools from \autoref{sec:thur-norm-non-orientable} are Theorems \ref{thm:strong-duality} and \ref{thm:classifying-fibrations}, and the operation of oriented sums for non-orientable surfaces.

\subsection{Background on mapping class groups of non-orientable surfaces}
\label{sec:backgr-mapp-class}

We will visualize non-orientable surfaces by thinking of them as orientable surfaces with \emph{crosscaps} attached.
We attach a crosscap to a surface $S$ by first deleting a small open disc $D\subset S$, and identifying the boundary of that disc (on the surface) via the antipodal map.
In pictures, this is often denoted by an X inscribed in a circle, see \autoref{fig:buildingblock} for an example of a surface with two crosscaps attached.
Let $\no_{g,n}$ be a non-orientable surface obtained by attaching $g$ crosscaps to $S^2$ and marking $n$ points in $S^2$.
The integer $g$ is referred to as the genus of $\no_{g,n}$.
Compact non-orientable surfaces are classified by the triple $(g,n,b)$ where $g$ is the genus, $n$ is the number of marked points and $b$ is the boundary.

Every homeomorphism $\varphi: \no \to \no$, has a unique orientation preserving lift, that is a homeomorphism $\wt{\varphi}: \os \to \os$, where $\os$ is the orientation double cover of $\no$, with $p\wt{\varphi}=\varphi p$, where $p$ is the covering map.

A consequence is that lifting homeomorphisms induces a monomorphism between orientation preserving homeomorphisms of $\no$ and (orientation preserving) homeomorphisms of $\os$.
Every homotopy of $\no$ lifts to a homotopy of $\os$.
Moreover, if $f,g:\no\to\no$ are homeomorphisms such that their orientation preserving lifts $\widetilde{f},\widetilde{g}$ of $\os$ are homotopic, then $f$ and $g$ are homotopic.
Therefore there is an inclusion from the mapping class group of $\no$ to the (orientation preserving) mapping class group of $\os$.
This inclusion also respects the Nielsen-Thurston classification of mapping classes, both qualitatively, and quantitatively, as the following proposition shows.
\begin{prop}
  \label{prop:2}
  If $\varphi$ is a self-homeomorphism of $\no$ and $\wt{\varphi}$ is its orientation preserving lift on $\os$, then:
  \begin{enumerate}[(i)]
  \item $\varphi$ is periodic if and only if $\wt{\varphi}$ is periodic,
  \item $\varphi$ is reducible if and only if $\wt{\varphi}$,
  \item $\varphi$ is pseudo-Anosov if and only if $\wt{\varphi}$ is pseudo-Anosov.  Moreover if $\varphi$ has stretch factor $\lambda$, then $\wt{\varphi}$ also has stretch factor $\lambda$.
  \end{enumerate}
\end{prop}
\begin{proof}
  It's easy to see that if $\varphi$ is periodic, so it $\wt{\varphi}$, and the other way round.
  If $\varphi$ is reducible, that means it leaves some multicurve $\gamma$ in $\no$ invariant, which means $\wt{\varphi}$ leaves the pre-image of $\gamma$ invariant as well.
  Conversely, if $\wt{\varphi}$ leaves some multicurve $\wt\gamma$ invariant, so does $\iota \circ \wt{\varphi}$, since they commute.
  That means the union of $\wt\gamma$ and $\iota(\wt\gamma)$, where $\iota$ is the orientation reversing deck transformation, is also a multi-curve and thus descends to a multi-curve on $\no$ that is left invariant by $\varphi$.
  Since any mapping class of $\no$ that is neither periodic nor reducible must be pseudo-Anosov on $\no$ must lift to a pseudo-Anosov on $\no$ and vice versa.

  Suppose now that $\varphi$ is a psuedo-Anosov on $\no$ with stretch factor $\lambda$ and expanding and contracting foliations $\mu_e$ and $\mu_c$ respectively.
  Since $\varphi$ is a pseudo-Anosov map, the following identity involving the intersection form $i$ holds for all closed curves $\gamma$ in $\no$.
  \begin{align}
    \label{eq:1}
    i(\varphi^{-1}\gamma, \mu_e) &= i(\gamma, \varphi(\mu_e)) \\
                               &= \lambda \cdot i(\gamma, \mu_e)
  \end{align}
  A similar identity holds for $\mu_c$.
  \begin{align}
    \label{eq:2}
    i(\varphi^{-1}\gamma, \mu_c) &= i(\gamma, \varphi(\mu_c)) \\
                               &= \frac{1}{\lambda} \cdot i(\gamma, \mu_c)
  \end{align}
  Note now that the foliations can be lifted to the double cover: call their lifts $\wt{\mu}_e$ and $\wt{\mu}_c$.
  For any closed curve $\wt{\gamma}$ of $\os$, consider its intersection number with the foliations.
  Observe that computing the intersection number is a local calculation.
  Start by picking an open cover $U$ on $\no$ such that all the open sets in $U$ are homeomorphic to the connected components of their pre-image in $\os$.
  By picking a partition of unity subordinate to this cover, one can compute the intersection number by restricting computation on each open set in the cover.
  This calculation lifts to the orientation double cover, giving us the following identity.
  \begin{align}
    \label{eq:3}
    i(\wt{\gamma}, \wt{\mu}_e) = i(\gamma, \mu_e)
  \end{align}
  Combining identities \eqref{eq:1} and \eqref{eq:3}, we get the following identity for intersection numbers on $\os$.
  \begin{align*}
    i(\wt{\varphi}^{-1} (\wt{\gamma}), \wt{\mu}_e) = \lambda \cdot i(\wt{\gamma}, \wt{\mu}_e)
  \end{align*}
  We get a similar expression for $\wt{\mu}_c$, which proves that $\wt{\varphi}$ has the same stretch factor as $\varphi$, thus proving the proposition.
\end{proof}

In the case of orientable surfaces, the Penner construction is used to construct pseudo-Anosov maps, as well compute their stretch factors. It turns out the Penner construction also works in the non-orientable setting, with some minor modifications. This construction is presented in detail in Section 2 of \cite{Strenner_2017}, but we give an outline of the key ideas below.

\p{The Penner construction} The Penner construction in the orientable setting starts with a pair of filling multicurves $A = \{a_1,\dots,a_n\}$
and $B = \{b_1,\dots,b_m\}$.  A Penner construction is a composition of positive Dehn twists around curves in $A$ and negative Dehn twists about curves in $B$ that uses a Dehn twist about each curve in $A\cup B$ at least once.  Penner proves that this construction is pseudo-Anosov \cite{penner1988construction}. The problem with making this work for
non-orientable surfaces is that when defining Dehn twists about curves on a non-orientable surface, there is not a well-defined notion of a left or right Dehn twist. For non-orientable surfaces we will use a set of filling two-sided curves that are \textit{marked inconsistently}.

Each two-sided curve $c$ on a non-orientable surface $N$ has a neighborhood homeomorphic to an
annulus $A$ by a homeomorphism $\phi: A \xrightarrow{} N$, called a \textit{marking}. In this
context, we can define the Dehn twist $T_{c,\phi}(x)$ around $(c,\phi)$ in the following manner.
\begin{align*}
  T_{c,\phi}(x) =
  \begin{cases}
    \phi \circ T \circ \phi^{-1}(x) & \text{for } x \in \phi(A) \\
    x & \text{for } x \in N - \phi(A)
  \end{cases}
\end{align*}
Here $T$ is the standard Dehn twist on $A$, i.e. $T(\theta,t) = (\theta + 2\pi t,t)$. If we fix an
orientation of $A$, then we can pushforward this orientation to $S$. We say two marked curves
$(c,\phi_c)$ and $(d,\phi_d)$ that intersect at a point $p$ are marked inconsistently if the
pushforward of the orientation of $A$ by $\phi_c$ and $\phi_d$ disagree in a neighborhood of $p$.
If all our curves are marked inconsistently and are filling, then once again a composition of Dehn
twists around them that use all the curves at least once will be pseudo-Anosov.

\p{Train tracks} The Penner construction not only promises that our map is pseudo-Anosov, but it also gives a way to
compute the stretch factor of our map (see \cite{penner1988construction}).  The proof of the fact
that the composition is pseudo-Anosov, and the computation of its stretch factor works the same is
in the orientable setting.  Let $\varphi$ be a pseudo-Anosov homeomorphism of $\no$.  A {\it train track} is an embedded graph in $\no$ such that for every vertex of valence three or greater, all adjacent edges have the same tangent vector.  An {\it invariant train track for $\varphi$} is a train track track $\tau$ such that $\varphi(\tau)$ is homotopic to $\tau$.  Let $\mathcal{C}$ be a collection of curves in $\no$. %Consider now the collection of transverse measures on our train track $\tau$.
For every curve $\gamma \in\mathcal{C}$, there is an associated transverse measure
$\mu_\gamma$ for $\tau$ that assigns $1$ to all edges lying in $\gamma$ and 0 to everything else. Let $V_\tau$
be the cone of transverse measures on $\tau$, and $H$ the subspace of $V_\tau$ spanned by the
transverse measure associated to curves in $\mathcal{C}$.
%\begin{align*}
 % H = \mathrm{span}(\{\mu_\gamma \mid \gamma \text{ is a connected curve in } \mathcal{C}\}).
%\end{align*}
The measures $\mu_\gamma$ are linearly independent and form the \textit{standard basis} for $H$. The subspace $H$ is invariant under the action of $\varphi$ on $V_\tau$, thus $\varphi$ has a linear action on $H$. If we let $A$
be the matrix representing this action in the standard basis, then the stretch factor of $\varphi$,
$\lambda(\varphi)$, is the Perron-Frobenius eigenvalue of $\varphi$.

\subsection{Constructing pseudo-Anosov maps on nearby surfaces using oriented sums}
\label{sec:constr-psuedo-anos}

In this section, we answer the following question.
\begin{question}
  If $\no_g$ has a pseudo-Anosov map with stretch factor $\lambda$, does there exist a pseudo-Anosov map on $\no_{g'}$ with stretch factor $\lambda'$, where the difference between $\lambda$ and $\lambda'$ is controlled by the difference between $g$ and $g'$?
\end{question}

We begin by stating two useful characterizations of incompressible surfaces in orientable $3$-manifolds.
\begin{fact}
  \label{thm:norm-minimizing}
  A surface $S$ minimizes the Thurston norm in its homology class if and only if it is incompressible.
\end{fact}

\begin{fact}
  If $M$ fibers over $S^1$, then the fiber is incompressible.
\end{fact}

We also have the following theorem of Thurston relating hyperbolicity of mapping tori and pseudo-Anosov maps.
\begin{thm}[Thurston's Hyperbolization Theorem]
  If $M$ is the mapping torus of a surface $S$ and a homeomorphism $\varphi$, then $M$ is hyperbolic if and only if $\varphi$ is pseudo-Anosov.
\end{thm}

A consequence of the above theorem is that if we can decompose the mapping torus of a pseudo-Anosov map as the mapping torus of some other homeomorphism, then the other homeomorphism must also be pseudo-Anosov.

Let $\no$ be a non-orientable surface, and $\varphi$ be a pseudo-Anosov homeomorphism.
Let $M_\varphi$ be the associated mapping torus and $f:M\rightarrow S^1$ the associated fibration.
Suppose we can construct another relatively oriented surface $\no'$ inside $M$ such that $\no'$ is transverse to the suspension flow direction associated to $f$.
Let $\alpha$ be the Poincar\'e dual of $\no$ and $\alpha'$ be the Poincar\'e dual of $\no'$.
By Theorem \ref{thm:classifying-fibrations} $\alpha$ lies in a cone $\mathcal{C}_\varphi \subset H^1(M_\varphi; \ZZ)$ associated to $\varphi$ with other $1$-forms also coming from fibrations.
Furthermore, $\alpha'$ lies in the closure of $\mathcal{C}_\varphi$.
All positive integer linear combinations of $\alpha$ and $\alpha'$ are lattice elements of $\mathcal{C}_\varphi$.
Each linear combination of $\alpha$ and $\alpha'$ is Poincar\'e dual to an oriented sum of $\no$ and $\no'$ in $M$.
Under reasonably mild conditions on $\no$ and $\no'$, we can show that the fiber is homeomorphic to the oriented sum $\no + \no'$ of $\no$ and $\no'$.

\begin{thm}
  \label{thm:oriented-sum}
  Let $\no$ be a non-orientable surface and $\varphi$ a homeomorphism of $\no$.
  Let $M_\varphi$ be the mapping torus of $(\no,\varphi)$ and $f:M\rightarrow S^1$ the fibration.
  Let $\no'$ be an incompressible surface embedded in $M$ that is transverse to the suspension flow direction associated to $f$.
  Let $\alpha$ be the Poincar\'e dual of $\no$ and $\alpha'$ the Poincar\'e dual of $\no'$.
  If the oriented sum of $\no$ and $\no'$ is connected, then $\no + \no'$ is homeomorphic to the fiber of the fibration given by $\alpha + \alpha'$.
\end{thm}
\begin{proof}
  Let $p:\wt{M}\rightarrow M$ be the orientation double cover of $M$.
  The surface $\no$ is incompressible because it is a fiber of $f$; therefore its pre-image under $p$ is also incompressible.
  Since the surfaces $\no$ and $\no'$ are incompressible, the Thurston norms of  their duals $\alpha$ and $\alpha'$ are $2\chi_-(\no)$ and $2\chi_{-}(\no')$.

  $\alpha$ and $\alpha'$ lie in a cone over a fibered face, which means the Thurston norm $x$ on $H^1(M;\ZZ)$ is linear function on that cone.
  This lets us compute the Thurston norm of the oriented sum $\no + \no'$, which must be twice the Euler characteristic.
  \begin{align*}
    x(\alpha + \alpha') &= x(\alpha) + x(\alpha') \\
                        &= 2\chi_-(\no) + 2\chi_-(\no') \\
                        &= 2\chi_-(\no + \no').
  \end{align*}
  The last equality follows from the linearity of the oriented sum.
  Because the pre-image of $\no + \no'$ in the orientation double cover of $M$ must be Thurston norm minimizing.
  Therefore the surface $\no+\no'$ is incompressible.

  By Theorem \ref{thm:classifying-fibrations}, we have that $\alpha + \alpha'$ corresponds to some other fibration $f'':M\rightarrow S^1$.
  But the fiber of this fibration must be homeomorphic to $\no + \no'$, by Theorem \ref{thm:strong-duality}, since the Poincar\'e dual to $\no + \no'$ is $\alpha + \alpha'$, and $\no + \no'$ is incompressible.
\end{proof}

Theorem \ref{thm:oriented-sum} is the first step in answering the question posed at the beginning of this subsection.
Given a pseudo-Anosov map on some non-orientable surface $\no$, we have a pseudo-Anosov map on a different non-orientable surface $\no'$, and we know the difference between the Euler characteristics, therefore genus, of $\no$ and $\no'$.
The next step is to estimate the difference in their stretch factors.

That is done using the following two theorems, due to Fried-Matsumoto (see \cite{fried1982flow}, \cite{fried1983transitive}, and \cite{matsumoto1987topological}) and Agol-Leininger-Margalit (see \cite{agol6983pseudo}) in the orientable case.

\begin{thm}[Fried-Matsumoto]
  \label{thm:fm}
  Let $M$ be a hyperbolic $3$-manifold and let $\mathcal{K}$ be the union of cones in $H^1(M; \RR)$ whose lattice points correspond to fibrations over $S^1$.
  There exists a strictly convex function $h: \mathcal{K} \to \RR$ satisfying the following properties.
  \begin{enumerate}[(i)]
  \item For all $c > 0$ and $u \in \mathcal{K}$, $h(cu) =  \frac{1}{c}h(u)$.
  \item For every primitive lattice point $u \in \mathcal{K}$, $h(u) = \log(\lambda)$, where $\lambda$ is the
    stretch factor of the pseudo-Anosov map associated to this lattice point.
  \item $h(u)$ goes to $\infty$ as $u$ approaches $\partial \mathcal{K}$.
  \end{enumerate}
\end{thm}

\begin{thm}[Agol-Leininger-Margalit]
  \label{thm:alm}
  Let $\mathcal{K}$ be a fibered cone for a mapping torus $M$ and let $\overline{\mathcal{K}}$ be its closure
  in $H^1(M;\RR)$. If $u \in \mathcal{K}$ and $v \in \overline{\mathcal{K}}$, then $h(u+v) < h(u)$.
\end{thm}

\begin{proof}[Proof of Theorems \ref{thm:fm} and \ref{thm:alm} for non-orientable $M$]
Consider the pullback of $H^1(M; \RR)$ to $H^1(\wt{M}; \RR)$, where $\wt{M}$ is the orientation double cover, and restrict the function $h$ to the image of the pullback.
It is easy to verify that the restriction satisfies all the listed properties as well.
\end{proof}

This answers the question we posed at the beginning of this subsection, and we use the results of this section in Section \ref{sec:application} to construct pseudo-Anosov maps with small stretch factors.
