\section{Fibered face theory for non-orientable $3$-manifolds}
\label{sec:fibered-face-theory}

In what follows, we will restrict our attention our attention on two kinds of compact $3$-manifolds:
mapping tori of orientable surfaces with a pseudo-Anosov map (these will be the $3$-manifolds we
will be referring to when talking about orientable $3$-manifolds), and the mapping tori of
non-orientable $3$-manifolds, again with a psuedo-Anosov map (these will be the manifolds we will be
referring to when talking about non-orientable $3$-manifolds). While a lot of our statements will
hold more generally for compact non-orientable $3$-manifolds, it will be easier to describe examples
when working in this restricted setting; additionally, our application will only involve mapping
tori of pseudo-Anosov maps.

\subsection{The problem with homology in non-orientable $3$-manifolds}
\label{sec:probl-with-homol}

A first attempt at defining the Thurston norm for a compact non-orientable $3$-manifold might go as
follows: for any embedded surface $S$ (whether it's orientable or not), one defines the $\chi_-$
function, much like in the case of orientable $3$-manifolds, and then define the norm of a homology
class $a \in H^2(M; \ZZ)$ by minimizing $\chi_-(S')$ over all $S'$ representing $a$. This may work,
but is quite unsatisfying: this construction assigns $0$ norm to all embedded non-orientable
surfaces, since their fundamental classes are trivial, and thus map to $0$ in $H_2(M; \ZZ)$. This is bad,
because for orientable $3$-manifolds, it is known that Thurston norm minimizing surfaces are
precisely the ones that are closed leaves of foliations (see Theorem 5.5 in \cite{gabai1983}). On
the other hand, it is possible to realize embedded non-orientable surfaces as closed leaves of
foliations, namely if the original $3$-manifold is the mapping torus of a non-orientable surface,
then one can think of that surface embedded inside the $3$-manifold.

It turns out that the fundamental problem with non-orientable $3$-manifolds is that homology is a
very coarse invariant: too coarse to detect embedded non-orientable surfaces. Our workaround will
be to deal with the first cohomology $H^1(M)$ rather than the second homology $H_2(M)$. The two are
same in the case of orientable $3$-manifolds, by Poincar\'e duality, but that fails in the case
of non-orientable $3$-manifolds.

To see why it fails, consider an orientable $3$-manifold $M$. We can explicitly work out the map
from $H^1(M; \ZZ)$ to $H_2(M; \ZZ)$ given by Poincar\'e duality. Given any $1$-form $\alpha$ such
that $[\alpha]$ is in $H^1(M; \ZZ)$, one can construct a map $f_{\alpha}$ from $M$ to $\RR/\ZZ$ in
the following manner.
\begin{align*}
  f_{\alpha}(x) \coloneqq \left. \int_{x_0}^x \alpha \right/ \ZZ
\end{align*}
Here $x_0$ is a basepoint on $M$ we can choose arbitrarily. The fact that this map is well defined
independent of the choice of path from $x_0$ to $x$ is a consequence of the fact that $[\alpha]$
lies in $H^1(M; \ZZ)$, i.e.  its integral around any closed loop is an integer. Consider now the
surface $S = f_{\alpha}^{-1}(p)$ for some regular value $p \in \RR/\ZZ$. Note that we have $S$
merely as a subset of $M$ right now. To get a homology class, we pick an orientation on $S$ by
declaring that the outwards pointing normal vectors on $S$ are the ones which get assigned a
positive value by the form $\alpha$. This defines an orientation of $S$ because we already have an
orientation on $M$, and thus defines a fundamental class $[S]$. We claim that $[S]$ is the
Poincar\'e dual to $\alpha$. More precisely, we claim the following.
\begin{claim}
  Let $p$ and $p'$ be two regular values of the function $f_{\alpha}$ and let $S$ and $S'$ be
  $f_{\alpha}^{-1}(p)$ and $f^{-1}_{\alpha}(p')$ respectively. Then for any closed $2$-form $\omega$ on $M$, the following
  identity holds.
  \begin{align*}
    \int_{S} \omega = \int_{S'} \omega
  \end{align*}
  Furthermore, the following identity also holds.
  \begin{align*}
    \int_S \omega = \int_M \alpha \wedge \omega
  \end{align*}
  In particular, the homology class of $S$ is Poincar\'e dual to $\alpha$.
\end{claim}
\begin{proof}
  The first part of the claim follows from the fact that $S$ and $S'$ are homologous, i.e. $f^{-1}_{\alpha}([p, p'])$
  is a singular $3$-chain that has $S$ and $S'$ as boundaries. From Stokes' theorem, we get the following.
  \begin{align*}
    \int_{S - S'} \omega &= \int_{f_{\alpha}^{-1}([p, p'])} d\omega \\
                         &= 0
  \end{align*}

  To prove the second claim, observe that we can break up the second integral as a product integral.
  \begin{align*}
    \int_M \alpha \wedge \omega &= \int_{\RR/\ZZ} \left(   \int_{f_{\alpha}^{-1}(x)} \omega \right) dx
  \end{align*}
  The above equation is true because $\alpha$ is the pullback of $dx$ along the map $f_{\alpha}$. Observe
  that the inner integral only makes sense when $x$ is a regular value, but by Sard's theorem, almost
  every $x \in [0,1]$ is a regular value, so the right hand side is well-defined. By the first part of the
  claim, the inner integral is a constant function, as we vary over the $x$ which are regular values of $f_{\alpha}$,
  and the integral of $dx$ over $\RR/\ZZ$ is $1$, giving us the identity we want.
  \begin{align*}
    \int_M \alpha \wedge \omega = \int_S \omega
  \end{align*}
\end{proof}
What we have here is an explicit formula for the Poincar\'e duality map. For orientable $3$-manifolds, this
is an isomorphism, but the formula still makes sense for non-orientable manifolds. In that case, one can see
that the map from $H^1(M; \ZZ)$ to $H_2(M; \ZZ)$ has a kernel.
\begin{example}[Failure of Poincar\'e duality for non-orientable $3$-manifolds]
  Let $\no$ be a non-orientable surface, and let $f$ be any self homeomorphism. Let $M$ be the mapping torus
  of $(\no, f)$. We then have a map $f: M \to \RR/\ZZ$ given by mapping to the base of the mapping torus.
  Pulling back the form $dx$ along this map, we get a closed but not exact $1$-form $\alpha$ on $M$. Observe
  that $f_{\alpha} = f$, because of how we constructed $\alpha$. Furthermore $f_{\alpha}^{-1}(0)$ is $\no$ inside
  $M$. Thus, the ``Poincar\'e duality map'' for $M$ maps a non-trivial element $\alpha \in H^1(M; \ZZ)$ to the
  zero element $[\no] \in H_2(M; \ZZ)$, since $\no$ is non-orientable. In particular, we end up losing information
  going from $H^1(M)$ to $H_2(M)$.
\end{example}

The above example also suggests an alternative method of defining the Thurston norm for
non-orientable $3$-manifolds: rather than working with $H_2(M)$, we can work with $H^1(M)$ instead,
because we don't want to lose information by going to $H_2(M)$. We'll also be interested in getting
a partial inverse for this map. More specifically, given a non-orientable surface $\no$ inside a
non-orientable $3$-manifold $M$, we'd like to understand if $M$ can be realized as the mapping
torus of some homeomorphism of $\no$.

\subsection{Thurston norm for non-orientable $3$-manifolds}
\label{sec:thurston-norm-non}

For this section, we'll use $M$ to denote a non-orientable $3$-manifold, and $\wt{M}$ to denote its
orientation double cover. We will denote by $\iota$ the orientation reversing deck transformation
of $\wt{M}$, and the covering map $\wt{M} \to M$ by $p$. If $M$ is the mapping torus of the
non-orientable surface $\no$ and a pseudo-Anosov map $\varphi: \no \to \no$, then $\wt{M}$ is the
mapping torus of $(\os, \wt{\varphi})$, where $\os$ is the orientable double cover of $\no$, and
$\wt{\varphi}$ is the orientation preserving lift of $\varphi$.

Since we've already concluded that the first cohomology is the ``right'' space to define the
Thurston norm on, we need to relate $H^1(M; \RR)$ and $H^1(\wt{M}; \RR)$. The obvious thing to do
is to look at the pullback via $p$.

\begin{lem}
  \label{lem:injective}
  Let $M$ be a non-orientable $3$-manifold, and $\wt{M}$ its orientation double cover. Let
  $\iota: \wt{M} \to \wt{M}$ be the orientation reversing deck transformation, and
  $p: \wt{M} \to M$ be the covering map. Then $p^{\ast}$ maps $H^1(M; \RR)$ bijectively to the
  $\iota^{\ast}$-invariant subspace of $H^1(\wt{M}; \RR)$.
\end{lem}
\begin{proof}
  Clearly, for any $1$-form $\alpha$ on $M$, $p^{\ast}(\alpha)$ will be
  $\iota^{\ast}$-invariant. This means that the image of $p^{\ast}$ lands inside the
  $\iota^{\ast}$-invariant subspace. To see that the map is injective at the level of $H^1$ (rather
  than at the level of $1$-forms), consider a $1$-form $\alpha$ on $M$ such that $p^{\ast}\alpha$
  is exact. We thus have a smooth function $f$ on $\wt{M}$ such that the following relation holds.
    \begin{align*}
        df = p^{\ast} \alpha
    \end{align*}
    But since $p^{\ast}\alpha$ is $\iota^{\ast}$-invariant, we must have $df = \iota^{\ast} df$,
    and by pushing the $\iota^{\ast}$ inside, we get that $df = d(\iota^{\ast}f)$. That means $f$
    and $\iota^{\ast}f$ differ by a constant, but that constant must be $0$ since $\iota$ is finite
    order. This shows that $f$ descends to a function on $M$, and thus $\alpha$ is exact, which
    proves injectivity of $p^{\ast}$. Now we show surjectivity. Let $[\alpha]$ be an element in
    $H^1(\wt{M}, \RR)$ that is $\iota^{\ast}$-invariant. That means if we pick a $1$-form $\alpha$
    in this equivalence class, the following identity holds for some smooth function $f$.
    \begin{align*}
        \alpha - \iota^{\ast}(\alpha) = df
    \end{align*}
    Note that this means $\iota^{\ast}df = -df$. Using these two identities, it's easy to verify
    that the $1$-form $\alpha - \frac{df}{2}$ is $\iota^{\ast}$-invariant, and thus in the image of
    $p^{\ast}$. This proves surjectivity, and the lemma.
\end{proof}

The above lemma tells us that $H^1(M; \RR)$ is a subspace of $H^1(\wt{M}; \RR)$ (and we know
precisely what that subspace is), and thus the Thurston norm on $H^1(\wt{M}; \RR)$ can be
restricted to a norm on $H^1(M; \RR)$.

\begin{defn}[Thurston norm for non-orientable $3$-manifolds]
  Let $M$ is a non-orientable $3$-manifold, and $\wt{M}$ is its orientation double cover, with the covering
  map $p: \wt{M} \to M$. The Thurston norm $x$ is a norm on $H^1(M; \RR)$, defined using the Thurston norm $\wt{x}$
  on $H^1(\wt{M}; \RR)$ (identified with $H_2(M; \RR)$ via Poincar\'e duality) in the following manner.
  \begin{align*}
    x(\alpha) \coloneqq \wt{x}(p^{\ast}\alpha)
  \end{align*}
\end{defn}
Now that we have a Thurston norm on $H^1(M; \RR)$, we need to describe some of its properties. All of these properties
follow fairly easily from the orientable version.
\begin{thm}
  The unit ball with respect to the dual Thurston norm on $\left( H^1(M; \RR) \right)^{\ast}$ is a polyhedron
  whose vertices are lattice points $\{\pm \beta_1, \ldots \pm \beta_k\}$. The unit ball $B_1$ with respect to Thurston
  norm is a polyhedron given by the following inequalities.
  \begin{align*}
    B_1 = \left\{ a \mid \left| \beta_i(a) \right| \leq 1 \text{ for $1\leq i \leq k$} \right\}
  \end{align*}
\end{thm}

\begin{proof}
  The proof is identical to the original proof of Theorem $2$ of Thurston in
  \cite{thurston1986norm}. The key ingredient of the proof is that the norm of any element in
  $H^1(M; \ZZ)$ is an integer. That is true in our case because the norm of an element of
  $H^1(M; \ZZ)$ is the Thurston norm of the corresponding element in $H^1(\wt{M}; \ZZ)$, which is an
  integer because it's the negative Euler characteristic of an embedded surface. The rest of the
  proof is just a matter of linear algebra, and works just as well in our setting.
\end{proof}

Observe that the way we defined the Thurston norm for non-orientable $3$-manifolds is lacking in two
ways. First of all, in the orientable case, the Thurston norm is a norm on the second homology, and
thus also embedded surfaces. In other words, the Thurston norm tells us something about embedded
surfaces in the manifold. We've already seen how working with second homology doesn't quite work,
which is why we had to go to first cohomology instead. We would still like to talk about the
norm of an embedded surface though, even if the homology class of that surface may be trivial.
This is something we'll see in subsection \ref{sec:invert-poincare}.

The other issue is that when working with fibrations over $S^1$, the elements of $H^1(M; \ZZ)$ are
the elements of interest, rather than things in $H^1(M; \RR)$. Lemma \ref{lem:injective} tells us
that elements of $H^1(M; \RR)$ are precisely the $\iota^{\ast}$-invariant elements of
$H^1(\wt{M}; \RR)$. That does not hold for $\ZZ$-coefficients: there are $\iota^{\ast}$-invariant
elements on $H^1(\wt{M}; \ZZ)$ that are not pullbacks of elements of $H^1(M; \ZZ)$.
\begin{example}[Failure of surjectivity]
  {\color{red} Turn this into a theorem.}
  Let $\no$ be a non-orientable surface, $\os$ its orientation double cover. Let $\gamma$ be a
  one-sided curve on $\no$, i.e. a curve whose lift to $\os$ is an arc, and the pre-image is a
  single closed curve, which is twice as ``long''. Call the pre-image $\wt{\gamma}$. Let the
  $3$-manifolds $M$ and $\wt{M}$ we're considering be the mapping tori of $\no$ and $\os$ with
  respect to some pseudo-Anosov map. We can then consider $\gamma$ and $\wt{\gamma}$ as curves
  inside the $3$-manifolds $M$ and $\wt{M}$.

  Pick a basis of $H_1(\wt{M}; \ZZ)$ containing $\wt{\gamma}$. Using this basis, we can construct an element
  of $H^1(\wt{M}; \ZZ)$ by simply assigning integer values to the basis elements. Pick an element $\alpha$ that assigns $0.5$ to $\wt{\gamma}$ and an
  integer value to every other basis element. Consider the cohomology class $\alpha +
  \iota^{\ast}\alpha$. Since $\iota \wt{\gamma} = \wt{\gamma}$ (because $\wt{\gamma}$ is the pre-image of a
  curve of $M$), $\alpha + \iota^{\ast}\alpha$ is an $\iota^{\ast}$-invariant element of $H^1(\wt{M}; \ZZ)$
  that assigns $1$ to $\wt{\gamma}$. Such a cohomology class cannot be a pullback of a class on $M$ since the
  pullback of a cohomology class on $M$ would assign an even value to $\wt{\gamma}$.
\end{example}

What the above example does show is that for any $\iota^{\ast}$-invariant $\alpha$ in
$H^1(\wt{M}; \ZZ)$, $2\alpha$ definitely is a pullback of class in $H^1(M; \ZZ)$.

\subsection{Inverting the Poincar\'e duality map for embedded surfaces}
\label{sec:invert-poincare}

In both the orientable and non-orientable setting, we have a way of assigning an embedded surface
to an element $\alpha$ in $H^1(M; \ZZ)$ by looking at $f_{\alpha}^{-1}(p)$, where $p$ is a regular
value. In the orientable setting, the homology class of this embedded surface is well defined,
independent of the choice of representative $1$-form in its cohomology class, as well as the choice
of regular value. That's true in the non-orientable setting as well, but the homology class is not
very useful. We'd like to invert this construction, i.e. given an embedded surface $S$, we want a
closed $1$-form $\alpha$ such that the surface $S$ comes from $\alpha$ in the manner described
above.

For an orientable $3$-manifold $M$, this is just a consequence of Poincar\'e duality, so the only
work we need to do is prove this for non-orientable manifolds. If we do so, we can talk about the
norm of an embedded surface, and more generally, have an ad hoc version of Poincar\'e duality,
associating embedded surfaces to $1$-forms. However, we need a version of the orientability
hypothesis to keep things well defined. For an embedded surface $S$ in a non-orientable
$3$-manifold $M$, we define the notion of \emph{relative orientability}.

\begin{defn}[Relative orientability]
  Let $M$ be a $3$-manifold, and $S$ an embedded surface in $M$. $S$ is said to be relatively
  oriented with respect to $M$ if there is a nowhere vanishing normal vector field on $S$. Two
  such normal vector fields are said to induce the same orientation if locally they induce the
  same orientation after picking a local frame for $S$. A surface $S$ is \emph{relatively oriented}
  if the positive normal vector field is specified along with the data of $S$.
\end{defn}

Note that relative orientability is a strictly weaker notion than orientability. If $S$ and $M$ are
orientable, then $S$ is relatively orientable with respect to $M$. But even if $M$ is
non-orientable, $S$ may be relatively orientable with respect to $M$. For instance, if $S$ is
the fiber of a non-orientable mapping torus, then one can get a non-vanishing normal vector field
by looking at the pre-image of a non-vanishing vector field on $S^1$. For relatively oriented
surfaces in $M$, it turns out one can go from surfaces to $1$-forms.

\begin{thm}[Poincar\'e duality for non-orientable $3$-manifolds]
  \label{thm:Poincare-duality}
  Let $M$ be a non-orientable $3$-manifold, and let $S$ be a relatively oriented embedded
  surface. Then there exists a cohomology class $[\alpha]$ in $H^1(M; \ZZ)$ such that for some
  representative $\alpha$, $S$ is $f_{\alpha}^{-1}(p)$ for some regular value $p \in
  S^1$. Furthermore, $\alpha$ assigns positive values to the positively oriented normal vector
  field on $S$.
\end{thm}

The idea of the proof of this theorem is fairly straightforward. Starting with the embedded surface
$S$ in $M$, we look at the pre-image $\wt{S}$ in the orientation double cover $\wt{M}$. We show
that the Poincar\'e dual to $\wt{S}$ is $\iota^{\ast}$-invariant.
\begin{lem}
  \label{lem:PD1}
  Let $S$ be a relatively oriented embedded surface in $M$, and $\wt{S}$ its pre-image in
  $\wt{M}$. Then the Poincar\'e dual to $[\wt{S}]$ is $\iota^{\ast}$-invariant.
\end{lem}
\begin{proof}
  Observe that if $S$ comes with a relative orientation, then $\wt{S}$ inherits that relative
  orientation. Since $\wt{S}$ and $\wt{M}$ are orientable, this defines an orientation on $\wt{S}$,
  and thus the homology class $[\wt{S}]$ is well defined.

  Observe now that the deck transformation $\iota$ reverses the orientation on $\wt{S}$. To see why
  this is the case, pick a local frame $(v_1, v_2, v_3)$ around some point in $\wt{S}$ such that
  $v_3$ is the outwards pointing normal vector field. Since the outwards pointing normal vector
  field descends to the quotient by the orientation reversing map $\iota$, that means $\iota(v_3)$
  must also be outwards pointing (and not inwards pointing). If $\iota$ has to reverse the
  orientation on $\wt{M}$, it must do so by reversing the orientation on the sub-basis
  $(v_1, v_2)$. In particular, that means $\iota$ reverses the orientation on $\wt{S}$.

  This means $[\wt{S}]$ is in the $-1$-eigenspace of the $\iota_{\ast}$ action on
  $H_2(\wt{M}; \RR)$. Therefore the Poincar\'e dual to $[\wt{S}]$, which we'll call
  $\wt{\alpha}$, is in the $1$-eigenspace, i.e. $\iota^{\ast}$-invariant. This just follows from
  the following chain of equalities which hold for all closed $2$-forms $\omega$.
  \begin{align*}
    \int_{\iota_{\ast}\wt{S}} \omega &= \int_{\wt{S}} \iota^{\ast}\omega &&\text{(By a change of variables)} \\
                                     &= \int_{\wt{M}} \wt{\alpha} \wedge \iota^{\ast} \omega &&\text{(Poincar\'e duality)} \\
                                     &=\int_{\wt{M}} \iota^{\ast} \left( \iota^{\ast}\wt{\alpha} \wedge \omega \right) \\
    &= \int_{\wt{M}} - \left( \iota^{\ast} \wt{\alpha} \wedge \omega \right) &&\text{($\iota$ is orientation reversing)}
  \end{align*}
  On the other hand, the following equalities follow from the fact that
  $\iota_{\ast}[\wt{S}] = -[\wt{S}]$.
  \begin{align*}
    \int_{\iota_{\ast}\wt{S}} \omega &= - \int_{\wt{S}} \omega \\
                              &= - \int_{\wt{M}} \wt{\alpha} \wedge \omega
  \end{align*}
  Comparing the right hand side of the two equations, it follows that $\wt{\alpha}$ is
  $\iota^{\ast}$-invariant.
\end{proof}

We now have an $\iota^{\ast}$-invariant $1$-form $\wt{\alpha}$ such that $\wt{S}$ is
$f_{\wt{\alpha}}^{-1}(p)$ for some regular value $p$ on $S^1$. The next claim we want to make is that
the map $f_{\wt{\alpha}}: \wt{M} \to S^1$ factors through the quotient $M$.
\begin{lem}
  \label{lem:PD2}
  The map $f_{\wt{\alpha}}$ factors through $M$, i.e. for all points $x \in \wt{M}$, $f_{\wt{\alpha}}(x) = f_{\wt{\alpha}}(\iota (x))$.
\end{lem}
\begin{proof}
  Recall that $f_{\wt{\alpha}}(x)$ is given by the following integral formula.
  \begin{align*}
    f_{\wt{\alpha}}(x) = \left. \int_{x_0}^x \wt{\alpha} \right/ \ZZ
  \end{align*}
  Here $x_0$ is some arbitrarily chosen basepoint on $\wt{M}$. For $f_{\wt{\alpha}}(x)$ to equal
  $f_{\wt{\alpha}}(\iota(x))$ for all $x$, we must have the following.
  \begin{align*}
    \left(  \int_{x_0}^x \wt{\alpha} - \int_{x_0}^{\iota(x)} \wt{\alpha} \right) \in \ZZ
  \end{align*}
  By a change of variables, and using the $\iota^{\ast}$-invariance of $\wt{\alpha}$, the left hand
  side of the above condition can be transformed, giving us the following condition.
  \begin{align}
    \label{cond:integer}
    \left( \int_{x_0}^{\iota(x_0)} \wt{\alpha} \right) \in \ZZ
  \end{align}
  In other words, we want the integral of $\wt{\alpha}$ along any curve $\gamma$ from $x_0$ to
  $\iota(x_0)$ to be an integer. Equivalently, it will suffice to show that the integral of
  $\wt{\alpha}$ along $\delta$ is an even integer, where $\delta$ is the closed curve obtained by taking
  the union of $\gamma$ and $\iota(\gamma)$.

  Recall now that the $2$-parity of $\displaystyle \int_{\delta} \wt{\alpha}$ is precisely the $2$-parity
  of the intersection number of $\delta$ and $\wt{S}$ as long as all the intersections are
  transversal. Furthermore, both $\delta$ and $\wt{S}$ are lifts of a curve and surface from
  $M$. Which means the number of intersections they have in $\wt{M}$ is twice the number of
  intersections have in $M$. But the latter number must be an integer, and thus the former number
  must be an even integer, showing that condition \eqref{cond:integer} holds. In particular,
  this shows that the map $f_{\wt{\alpha}}$ factors through, proving the lemma.
\end{proof}
We now have everything we need to finish proving Theorem \ref{thm:Poincare-duality}.
\begin{proof}[Proof of Theorem \ref{thm:Poincare-duality}]
  Starting with a relatively oriented surface $S$ in $M$, we look at its pre-image $\wt{S}$ in
  $\wt{M}$. The relative orientation of the pre-image gives us the homology class $[\wt{S}]$, and
  we get a $1$-form $\wt{\alpha}$, which is Poincar\'e dual to the homology class of $\wt{S}$.
  More specifically, we have that $\wt{S}$ is $f^{-1}_{\wt{\alpha}}(p)$ for some regular value $p$.
  By lemma \ref{lem:PD1}, $\wt{\alpha}$ is $\iota^{\ast}$-invariant, and by lemma \ref{lem:PD2}, we
  have that $f_{\wt{\alpha}}$ factors through $M$, which means $f_{\alpha}^{-1}(p) =
  S$. Furthermore, we also have that $\alpha$ lies in $H^1(M; \ZZ)$. That just follows from the
  fact that $\alpha$ is the pullback of the form $d\theta$ on $S^1$ along the map
  $f_{\alpha}$. This proves the result.
\end{proof}

\subsection{Oriented sums of surfaces}
\label{sec:orient-sums-surf}

We now have a way of going from an embedded surface to an element of $H^1(M; \ZZ)$. To make this
mapping even more useful, we'll define an operation the operation so taking \emph{oriented sums} of
surfaces which is additive in two senses: the Euler characteristic, and the cohomology class of the
dual. This operation is standard knowledge in the case of orientable $3$-manifolds (along with
orientable embedded surfaces), but we will sketch out the relevant details for completeness. The
same construction works for relatively orientable surfaces; one just needs to verify consistency.

\paragraph{Oriented sum for oriented manifolds}
Let $S$ and $S'$ be oriented embedded surfaces in an oriented manifold $M$. Assume that $S$ and
$S'$ intersect transversely. Thus, $S \cap S'$ is a disjoint union of copies of $S^1$. Pick a small
neighbourhood of each component of the intersection such that the cross section looks like the
following.
\begin{figure}[h]
  \centering
  \incfig[0.2]{cross-section}
  \caption{Cross section of intersection of $S$ and $S'$.}
  \label{fig:cross-section}
\end{figure}

We then perform a local surgery such that each $S$ leaf joins an $S'$ leaf. We have two possible
choices: we could join the left $S$ leaf to either the top or the bottom $S'$ leaf. Since both $S$
and $S'$ are oriented submanifolds of $M$, there is an outward pointing normal vector field on $S$
and $S'$. Suppose the outward normal vector field on $S$ points upwards and the outward normal
vector on $S'$ points to the right. In that case, we'd glue the left $S$ leaf to the bottom $S'$
leaf to maintain a consistent outward normal vector field. See \autoref{fig:surgery} to see how the
choice affects orientability.
\begin{figure}[h]
  \centering
  \incfig[0.3]{surgery}
  \caption{On the left, the normal vectors on $S$ and $S'$ are consistent. On the right, they aren't.}
  \label{fig:surgery}
\end{figure}

By performing this surgery at all the intersections, we get a new submanifold $S''$ (which may have
multiple components). This new submanifold $S''$ is the oriented sum of $S$ and $S'$. The operation
of taking oriented sums adds the Euler characteristic, as well as the homology classes (and thus
the cohomology classes of their Poincar\'e duals).
\begin{align*}
  \chi(S'') &= \chi(S) + \chi(S') \\
  [S''] &= [S] + [S'] \\
\end{align*}

\paragraph{Oriented sum for non-orientable manifolds}
Observe that in order to canonically choose the right leaves to join, all we needed was a relative
orientation for both $S$ and $S'$. That suggests that the same construction ought to work. Like in the case of an orientable ambient manifold, at every transversal intersection, we perform surgery
based on the outwards pointing normal vector field.
We just need to verify that this construction is consistent with the covering map: i.e. taking
the oriented sum of $S$ and $S'$ is the same as taking the oriented sum of $\wt{S}$ and $\wt{S'}$
and then quotienting with $\iota$.

It boils down to verifying that if we glue a leaf of $\wt{S}$ and $\wt{S'}$, then their images
under the orientation reversing deck transformation $\iota$ also get glued together. Consider
\autoref{fig:consistency}, which shows the outward point normal vectors to $\wt{S}$ and $\wt{S'}$,
which dictate which leaves are glued together.
\begin{figure}[h]
  \centering
  \incfig[0.4]{consistency}
  \caption{Neighbourhoods of $\wt{\gamma}_1$ and $\wt{\gamma}_2$, with the outward pointing normal vector field.}
  \label{fig:consistency}
\end{figure}

The normal vector field tells us that the left $\wt{S}$ leaf gets glued to the bottom $\wt{S'}$
leaf near $\wt{\gamma}_1$ and $\wt{\gamma}_2$. Consider now the deck transformation $\iota$. Note
that $\iota$ is an orientation reversing self map for $\wt{M}$, $\wt{S}$ and $\wt{S'}$. We've
already seen that $\iota$ preserves the relative orientation, and thus leaves both the outwards
normal vector fields invariant. This means the gluing is $\iota$-invariant, i.e.  two leaves get
glued iff their images under $\iota$ get glued. This shows that the gluing descends to $M$, and we
can thus define an oriented sum operation on embedded surfaces in $M$ that is consistent with the
oriented sum on the orientation double cover.

By the consistency of the oriented sum in $M$ and $\wt{M}$, it easily follows that the oriented sum
is additive in Euler characteristic, as well as in terms of Poincar\'e dual, since the Poincar\'e
dual was also defined by going to the orientation double cover.

\subsection{Relating $1$-forms and fibrations over $\RR/\ZZ$}
\label{sec:relating-1-forms}
While we have informally described what a fibration over $S^1$ is prior to this section, it will be
useful to formally define a fibration at this stage.
\begin{defn}[Fibration over $S^1$]
  Given a $3$-manifold $M$, a fibration (or a fiber bundle) over $S^1$ is a map $f: M \to S^1$ such
  that the derivative of $f$ has full rank at all points in $M$. The pre-image of every point in
  $S^1$ is an relatively oriented embedded surface in $M$, where the positive normal direction is
  the pre-image of the positive direction in $S^1$. This surface is called the fiber of the
  fibration.
\end{defn}
Note that any $3$-manifold that admits a fibration over $S^1$ is a mapping torus of the fiber
(which is a surface), along with the homeomorphism that comes from the transition map when changing
coordinate charts on $S^1$. We can instead look at homotopy classes of fibrations, and every
equivalence class will correspond to a homotopy class of a homeomorphism of the fiber, i.e. a
mapping class. Since we're mostly interested in mapping tori of mapping classes rather than mapping
classes of specific homeomorphisms in those mapping classes, we'll be focusing on homotopy classes
of fibrations.

It turns out that the ``right'' way to think about fibrations $f: M \to S^1$ is by studying a
specific kind of $1$-form on $M$, namely \emph{non-singular integer $1$-forms}.

\begin{defn}
  A non-singular integer $1$-form on a $3$-manifold $M$ is a smooth nowhere vanishing $1$-form $\alpha$ on $M$
  such that for any closed loop $\gamma$, the integral of $\alpha$ along $\gamma$ lies in $\ZZ$.
  \begin{align*}
    \int_{\gamma} \alpha \in \ZZ
  \end{align*}
\end{defn}

Given a non-singular integer $1$-form, there is a canonical way of getting a fibration $f_{\alpha}$
over $S^1$, as we've seen before. The map $f_{\alpha}$ is given by the following formula.
\begin{align*}
  f_{\alpha}(x) \coloneqq \left. \int_{x_0}^x \alpha \right/ \ZZ
\end{align*}
Conversely, given a fibration $f: M \to S^1$, one can obtain a non-singular integer $1$-form by
pulling back a $1$-form along the map $f$. The correct $1$-form to pull back is $d\theta$, i.e.
the non-vanishing $1$-form on $S^1$ such that $\int_{S^1} d\theta = 1$ (note that despite the
notation, this is not an exact form). These two constructions are inverses of each other, which is
fairly easy to verify. Furthermore, if we change $\alpha$ to $\alpha + df$, where $df$ is an exact
form, then the associated map to $S^1$ is not the same, but homotopic to the original
map. Conversely, if we pull back $d\theta$ along a map homotopic to $f$ rather than $f$, we get a
form that differs from the original form by an exact form (see Section 5.2.1 of
\cite{calegari2007foliations} for the details). The takeaway here is that if we only care about the
mapping torus structure of the mapping classes, we can focus our attention to the elements of
$H^1(M; \ZZ)$ that admit a non-singular $1$-form representative.

We now have all we need to prove a version of Theorem \ref{thm:Thur1} for non-orientable
$3$-manifolds.

\begin{thm}
  \label{thm:NOThur1}
  Let $M$ be a non-orientable $3$-manifold, and let $\mathcal{F}$ be the set of all possible
  ways $M$ fibers over $S^1$ (up to homotopy). Then the following results hold for $\mathcal{F}$.
  \begin{enumerate}[(i)]
  \item Elements of $\mathcal{F}$ are in a one-to-one correspondence with (non-zero) lattice points
    inside some union of cones over open faces of the unit ball with respect to the Thurston norm
    in $H^1(M; \RR)$.
  \item If a embedded relatively oriented surface $S$ is transverse to the suspension flow
    associated to some fibration $f$ such that the flow direction is the outwards normal direction,
    then the Poincar\'e dual to $S$ lies in the closure of the cone corresponding to $f$.
  \end{enumerate}
\end{thm}
\begin{proof}
  We've already done most of the work in reducing the proof of this result to the orientable
  version. To get the union of cones $\mathcal{K}$ in $H^1(M; \RR)$ corresponding to fibrations, we
  look at the corresponding union of cones $\wt{\mathcal{K}}$ in $H^1(\wt{M}; \RR)$ (i.e. the first
  cohomology of the orientable double cover). Recall that $H^1(M; \RR)$ bijectively maps into
  $H^1(\wt{M}; \RR)$ as a subspace. We define $\mathcal{K}$ to be the restriction of
  $\wt{\mathcal{K}}$ to the subspace corresponding to $H^1(M; \RR)$.

  Consider now any fibration $f: M \to S^1$. By composing it with the covering map
  $p: \wt{M} \to M$, we get a fibration of $\wt{M}$: $f \circ p: \wt{M} \to S^1$. By pulling back
  $d\theta$ along this map, we get an element of $H^1(\wt{M}; \ZZ)$, which lies in
  $\wt{\mathcal{K}}$. This element is a pullback of $f^{\ast}(d\theta) \in H^1(M; \ZZ)$, and
  therefore lies in the subspace corresponding to $H^1(M; \ZZ)$. This shows that every homotopy
  class of fibrations injectively maps into $\mathcal{K}$. Conversely, suppose we have some
  element $\alpha$ in $\mathcal{K}$. Its pullback $\wt{\alpha}$ lies in $\wt{\mathcal{K}}$,
  and therefore corresponds to a fibration $f_{\wt{\alpha}}: \wt{M} \to S^1$. We would like
  to pushforward this map to a map from $M$ to $S^1$. Recall the proof of Lemma \ref{lem:PD2}
  in which we saw that for this to happen, $\wt{\alpha}$ must satisfy the following condition
  for any basepoint $x_0$ in $\wt{M}$.
  \begin{align*}
    \int_{x_0}^{\iota(x_0)} \wt{\alpha} \in \ZZ
  \end{align*}
  But observe that since any path from $x_0$ to $\iota(x_0)$ is a lift of a closed curve $\gamma$
  on $M$ from $p(x_0)$ to $p(\iota(x_0)) = p(x_0)$, and $\wt{\alpha}$ is the pullback of
  the $\alpha \in H^1(M; \ZZ)$, the above integral is equal to an integral on $M$.
  \begin{align*}
    \int_{x_0}^{\iota(x_0)} \wt{\alpha} = \int_{\gamma} \alpha
  \end{align*}
  The right hand side term is clearly an integer, since $\alpha$ is an integer $1$-form. This
  shows that the map $f_{\wt{\alpha}}$ descends to a map on $M$, and therefore $\alpha$ corresponds
  to a fibration. This proves part (i) of the theorem.

  Let $\alpha$ be the Poincar\'e dual to $S$. Because of the way we defined $\alpha$, the pullback
  $\wt{\alpha}$ is the Poincar\'e dual to the pre-image $\wt{S}$ of $S$. Furthermore, since $S$ is
  transverse to the suspension flow direction, with the flow direction pointing outwards, the same
  must hold for $\wt{S}$. Therefore the Poincar\'e dual to $\wt{S}$ lies in the closure of the
  corresponding cone. But since the dual also is a pullback of $\alpha$, it lies in the subspace
  corresponding to $H^1(M; \RR)$. This shows that $\alpha$ lies in the closure of the cone
  restricted to $H^1(M; \RR)$ and proves part (ii) of the result.
\end{proof}

Part (ii) of the above theorem (and Theorem \ref{thm:Thur1}) is especially useful when one is
trying to decompose a $3$-manifold into a mapping torus. Suppose one starts off with a mapping
torus $M = (\no, \varphi)$. One then constructs another relatively oriented surface $\no'$ inside
$M$ such that $\no'$ is transverse to the suspension flow direction. By the above theorem, the
Poincar\'e dual to $\no$, which we'll call $\alpha$, lies in a cone with other $1$-forms also
coming from fibrations. Furthermore, the Poincar\'e dual to $\no'$, which we'll call $\alpha'$,
lies in the closure of said cone. By taking positive integer linear combinations of $\alpha$ and
$\alpha'$, we can get lots of other elements of the cone. At the level of surfaces, that
corresponds to taking oriented sums of $\no$ and $\no'$ to get new relatively oriented surfaces in
$M$. Under reasonably mild conditions on $\no$ and $\no'$, we can actually realize their oriented
sums as fibers of fibrations. To see this, we'll need to use a theorem due to Thurston.

\begin{thm}[Theorem 4 of \cite{thurston1986norm}]
  \label{thm:ThurIsotope}
  Let $\wt{M}$ be an orientable $3$-manifold that fibers over a circle with fiber $S$. If
  $S'$ is another embedded surface in $\wt{M}$ homologous to $S$ and also Thurston norm
  minimizing, then $S'$ is isotopic to $S$.
\end{thm}

\begin{thm}
  Let $M$, $\no$ and $\no'$ be as described above. Suppose that $\no'$ is incompressible, and the
  oriented sum of $\no$ and $\no'$, which we'll denote by $\no + \no'$, is connected. Then
  $\no + \no'$ is isotopic to the fiber of the fibration given by $\alpha + \alpha'$ (the
  Poincar\'e duals to $\no$ and $\no'$).
\end{thm}
\begin{proof}
  The first step is to observe that in this case, one can compute the Thurston norm of $\alpha$ and
  $\alpha'$ using $\no$ and $\no'$. For $\alpha$ and $\alpha'$, the Thurston norms are
  $2\chi_-(\no)$ and $2\chi_-(\no')$ respectively. This follows by passing to the orientation
  double cover, and noting that incompressible surfaces minimize the Thurston norm in their
  homology class. The first surface $\no$ is incompressible by virtue of being the fiber of a
  fibration, and the second surface $\no'$ is incompressible by hypothesis.

  Since $\alpha$ and $\alpha'$ lie in a cone over a fibered face, the Thurston norm $x$ is linear.
  We can thus compute the Thurston norm of $\alpha + \alpha'$ in terms of $\no + \no'$.
  \begin{align*}
    x(\alpha + \alpha') &= x(\alpha) + x(\alpha') \\
                        &= 2\chi_-(\no) + 2\chi_-(\no') \\
                        &= 2\chi_-(\no + \no')
  \end{align*}
  The last equality follows from the properties of the oriented sum. This tells us that the
  pre-image of $\no + \no'$ in the double cover must be Thurston norm minimizing, and thus
  incompressible.

  By Theorem \ref{thm:NOThur1}, we have that $\alpha + \alpha'$ corresponds to some other fibration
  of $M$. Since $M$ is non-orientable, there are two kinds of fiber possible.
  \begin{enumerate}[(i)]
  \item The fibration is the mapping torus of a non-orientable surface along with a homeomorphism.
  \item The fibration is the mapping torus of an orientable surface along with an orientation
    reversing homeomorphism.
  \end{enumerate}
  In the first case, the fiber is a non-orientable surface homologous to $\no + \no'$. By passing
  to the double cover, we get two homologous non-orientable surfaces, both of which minimize the
  Thurston norm. By Theorem \ref{thm:ThurIsotope}, we have that $\no + \no'$ is isotopic to the
  fiber, which means $M$ can be realized as the mapping torus of some homeomorphism on
  $\no + \no'$.

  The second case can be ruled out using a similar argument. In the case that $M$ is the mapping
  torus of an orientable surface $S$ with an orientation reversing homeomorphism, we can pass to
  the double cover. The fiber of a point in the double cover is two disjoint copies of $S$. But
  that is homologous to the double cover of $\no + \no'$ which will have a single component, since
  $\no + \no'$ is non-orientable. Theorem \ref{thm:NOThur1} says these two surfaces must be
  isotopic, but that can't happen since they have a different number of connected components.
\end{proof}

The non-orientable versions of Theorems \ref{thm:fm} and \ref{thm:alm} follow in a straightforward
manner from the orientable versions.

\begin{thm}
  \label{thm:NOfm}
  Let $M$ be a non-orientable hyperbolic $3$-manifold and let $\mathcal{K}$ be the union of cones
  in $H^1(M; \RR)$ whose lattice points correspond to fibrations over $S^1$. There exists a
  strictly convex function $h: \mathcal{K} \to \RR$ satisfying the following properties.
  \begin{enumerate}[(i)]
  \item For all $t > 0$ and $u \in \mathcal{K}$, $h(tu) = \frac{1}{t} h(u)$.
  \item For every primitive lattice point $u \in \mathcal{K}$, $h(u) = \log(k)$, where $k$ is
    the stretch factor of the pseudo-Anosov map associated to this lattice point.
  \item $h(u)$ goes to $\infty$ as $u$ approaches $\partial \mathcal{K}$.
  \end{enumerate}
\end{thm}

\begin{proof}
  We already have such a function $\wt{h}$ on $H^1(\wt{M}; \RR)$. Restricting that function
  to the subspace corresponding to $H^1(M; \RR)$, we get a convex function satisfying properties
  (i) and (iii). To verify property (ii), we need verify that the stretch factor of a pseudo-Anosov
  map on a non-orientable surface is the same as the stretch factor of the unique lift to its
  double cover. This follows from Proposition \ref{prop:2}.
\end{proof}

The exact statement of Theorem \ref{thm:alm} holds for the non-orientable setting too: one just
restricts the function $h$ on $H^1(\wt{M}; \RR)$ to the subspace corresponding to $H^1(M; \RR)$.