\section{Minimal stretch factors for non-orientable surfaces with marked points}
\label{sec:application}

Recall that associated to every pseudo-Anosov homeomorphism $f$ there is a number $\lambda(f)$, the
\textit{dilatation} or \textit{stretch factor}, the amount that the stable and unstable foliations of the
pseudo-Anosov change by. Given a surface $S$, it is natural to ask what we can say about the set of all
possible stretch factors, i.e.
\begin{align*}
    \left\{\log(\lambda(f)) \mid f \in \text{Mod}(S) \text{ is pseudo-Anosov}\right\}
\end{align*}

We call this set the \textit{spectrum} of $S$. A first step at understanding this set of all stretch factors
associated to a surface is considering the following quantity
\begin{align*}
    l_{g,n} =\min\{\log(\lambda(f)) \mid f \in \text{Mod}(\mathcal{S}_{g,n}) \text{ is pseudo-Anosov}\}.
\end{align*}

The study of this minimal stretch factor $l_{g,n}$ was initiated by Penner in his work
\cite{penner1991bounds}. In this paper Penner studied the asymptotic behavior of minimal stretch factors of orientable surfaces without punctures, i.e. the behavior of $l_{g,0}$. He showed that there exist positive constants
$A_1$ and $A_2$ such that the following inequalities held for any $g \geq 2$.
\begin{align*}
    \frac{A_1}{g} \leq l_{g,0} \leq \frac{A_2}{g}
\end{align*}
This showed that asymptotically $l_{g,0}$ behaves like $\frac{1}{g}$ for $g \geq 2$.
It turns out something similar is true when one starts adding punctures. Yazdi showed
that for any fixed $n$, $l_{g,n}$ behaves like $\frac{1}{g}$. More precisely, he proved the following theorem.
\begin{thm}[Theorem 1.2 of \cite{yazdi2018pseudo}]
\label{thm:yazdi1}
For any fixed $n \in \mathbb{N}$, there are positive constants $B_1 = B_1(n)$ and $B_2 = B_2(n)$ such that the following inequalities hold for
any $g \geq 2$.
\begin{align*}
    \frac{B_1}{g} \leq l_{g,n} \leq \frac{B_2}{g}.
\end{align*}
\end{thm}

Yazdi's result is one of many recent results in studying the asymptotics of $l_{g,n}$ for different subsets of
the $(g,n)$ plane. See the introductions of \cite{yazdi2018pseudo} and \cite{tsai2009asymptotic} for more
examples of results of this form. Yazdi proves an additional result along these lines for a large subset of
the $(g,n)$ plane, one containing balls of arbitrary large radii.

\begin{thm}[Yazdi]
    \label{thm:yazdi2}
    There exists positive constants $A$, $B$ and $C$ such that for any $n \geq 1$ and $g \geq Cn\log^2(n)$ such that
    the following inequalities hold.
    \begin{align*}
        \frac{B}{g} \leq l_{g,n} \leq \frac{A}{g}
    \end{align*}

\end{thm}
A key tool in Yazdi's proof of both these theorems was the fibered face theory of Thurston. With the non-orientable analog of
Thurston's fibered face theory we developed in the previous section, it's possible to prove an analogous
theorem for non-orientable punctured surfaces.  Let $\mathcal{N}_{g,n}$ be the genus $g$ non-orientable
surface with $n$ punctures and let $l_{g,n}'$ be the minimal stretch factor of $\no_{g,n}$.
\begin{align*}
  l'_{g,n} = \min\left\{\log(\lambda(f)) \mid f \in \text{Mod}(\mathcal{N}_{g,n})\ \text{is pseudo-Anosov}\right\}
\end{align*}
Then we have the following result, analogous to Yazdi's results.
\begin{thm}
  \label{thm:stretch1}
  For any fixed $n \in \mathbb{N}$, there are positive constants $B'_1 = B'_1(n)$ and $B'_2 = B'_2(n)$ such that
  for any $g \geq 2$, the stretch factor satisfies the following inequalities.
  \begin{align*}
    \frac{B'_1}{g} \leq l'_{g,n} \leq \frac{B'_2}{g}
  \end{align*}
\end{thm}

Observe that the lower bound for the non-orientable case follows easily from the lower bound for the orientable case.
Let $f$ be the pseudo-Anosov map with the minimal stretch factor on $\no_{g,n}$. Then, by Proposition \ref{prop:2},
this map lifts to a map $\wt{f}$ on $\os_{g-1, 2n}$ (possibly after squaring). Furthermore, $\wt{f}$ has the same
stretch factor as $f$. The former is bounded below by $\frac{B}{g}$, and thus the stretch factor of $f$ is bounded
below as well. The more challenging part of the proof is showing the upper bound holds. This will be done by explicitly
constructing pseudo-Anosov maps with small stretch factors, adapting Yazdi's techniques to the non-orientable setting.

The construction of Yazdi proceeds in five steps: in steps 1 and 2, a family of small dilatation psuedo-Anosov
maps is constructed on $\os_{g_i,n}$, where $\{g_i\}$ is a sequence of genera going off to infinity, by not
containing every element of $\mathbb{N}$, i.e. there are plenty of gaps. Steps 3 and 5 deal with filling in the
gaps, i.e. constructing small dilatation pseudo-Anosov maps on the missing surfaces, and this is where
Thurston's fibered face theory enters the picture. In this section, we will adapt the steps to work for
non-orientable surfaces.

\paragraph{Step 1: Constructing the surfaces}

The first step in the construction is defining a family of surfaces that exhibit a sort of rotational
symmetry. Using this symmetry, if one shows that a power of some homeomorphism is pseudo-Ansov, then so is
the original homeomorphism. Yazdi cites this insight as being due to Penner in his construction in
\cite{penner1991bounds}.

Note that we will try to follow Yazdi's notation as close as we can, in order to make it clear to the reader
how our construction replicates his.

We begin by defining a family of surfaces $P_{n,k}$. Let $T$ be an orientable surface of genus 5 with 3
boundary components $c,d$ and $e$. Orient $T$ and give $c,d$ and $e$ the induced orientations from this
orientation. Now add two cross-caps to $T$ but keep the boundaries of $T$ oriented. Let $p$ (respectively $q$)
be a puncture (respectively a marked point) on the boundary component $e$ of $T$, with oriented arcs $r$ and
$s$ connecting them on $\partial T$. See Figure \ref{fig:buildingblock} for a picture of $T$.

\begin{figure}[]
    \centering
    \incfig[0.4]{YazdiTypeSurface}
    \caption{The surface $T$, which will be the building block of the construction.}
    \label{fig:buildingblock}
\end{figure}

Let $T_{i,j}$ be copies of the surface $T$, where $i,j \in \mathbb{Z}$. We will use similar notation to refer
to the boundary components of $T_{i,j}$. Define an infinite surface $S_\infty$ as the following quotient:
\begin{align*}
  S_\infty \coloneqq \left. \left( \bigcup T_{i,j} \right)\right/\sim
\end{align*}
Here, $i$ and $j$ are integers. The gluing $\sim$ is given by the following two families of
identifications.
\begin{align*}
  c_{i,j} &\sim d_{i+1,j} \\
  r_{i,j} &\sim s_{i,j+1}
\end{align*}
Furthermore, the boundary components are glued by an orientation-reversing homeomorphisms.  We have two
natural shift maps $\overline{\rho_1},\overline{\rho_2}: T_\infty \to T_\infty$ that act in the
following manner.
\begin{align*}
  \overline{\rho_1}: T_{i,j} &\mapsto T_{i+1, j} \\
  \overline{\rho_2}: T_{i,j} &\mapsto T_{i, j+1}
\end{align*}

Note that these maps commute. Define the surface $P_{n,k}$ as the quotient of the surface $T_\infty$ by the
covering action of the group generated by $(\overline{\rho_1})^n$ and $(\overline{\rho_2})^k$. Therefore,
$\overline{\rho_1}$ and $\overline{\rho_2}$ induce maps on the surface $P_{n,k}$, which we denote by $\rho_1$
and $\rho_2$.

A natural question at this point is why did we choose the surface $T$ for our building block? It comes down to
two main problems:
\begin{itemize}
\item The combinatorics of the curves make the associated matrix we get from the Penner construction satisfy
  the conditions of Lemma \ref{lem:spectral}. This is used to prove our family of pseudo-Anosov maps have
  stretch factors bounded above by the quantity we desire.
\item Having a curve $\gamma$ such that it and its image under our map form the boundary of an embedded
  $\RR P^2$ with two boundary components in the mapping torus, which will come into play when extending our family of surfaces in Step 3.
\end{itemize}

\begin{lem}
\label{lem:genera}
Define the sequence $g_{n,k}$ in the following manner for $n \geq 1$ and $k \geq 3$.
\begin{align*}
    g_{n,k} &= (14k - 2)n + 2
\end{align*}
    The genus of $P_{n,k}$ is $g_{n,k}$.
\end{lem}
\begin{proof}
  Consider the subsurface $U \subset P_{n,k}$ defined as
  \begin{align*}
    U = \left. \left( \bigcup_{i =0}^{k-1} T_{0,i} \right)\right/\sim
  \end{align*}
  Then $U$ is a compact, non-orientable surface of
  genus $12k$ with $2k$ boundary components, and forms a fundamental domain for the covering action of
  $\overline{\rho_1}$ on $T_\infty$. We have a formula for the Euler characteristic of $U$.
  \begin{align*}
    \chi(U) &= 2 - 12k - 2k \\
            &= 2 - 14k
  \end{align*}
  This gives us a formula for the Euler characteristic of $P_{n,k}$.
  \begin{align*}
    \chi(P_{n,k}) &= n \cdot \chi(U)\\
                  &= -n(14k - 2)
  \end{align*}
  since $P_{n,k}$ is formed by gluing $n$ copies of $U$ together along circle boundary components. By the
  relation between genus and Euler characteristic, we have the claimed formula for genus.
  \begin{align*}
    g_{n,k} = n(14k-2) + 2
  \end{align*}
\end{proof}

\paragraph{Step 2: Constructing the maps}

We now construct maps $f_{n,k}: P_{n,k} \to P_{n,k}$ that are defined as a composition of specific Dehn twists
followed by a finite order mapping class. The key insight is that a power of this map will be a composition of
Dehn twists that satisfy the criteria to be a Penner construction and thus pseudo-Anosov. This is how we take
advantage of the rotational symmetry of the $P_{n,k}$.

\begin{figure}[h]
    \centering
    \incfig[0.65]{CurvesOnSurface}
    \caption{The curves that lie on $T_{0,0}$}
    \label{fig:curves}
\end{figure}

\begin{figure}[h]
    \centering
    \incfig[0.4]{ExtraCurves}
    \caption{The parts of curves $\beta_2$ and $\beta_7$ on $T_{0,1}$ and $T_{1,0}$}
    \label{fig:extracurves}
\end{figure}

Recall that for non-orientable surfaces, we don't initially have a well-defined notion of a positive or
negative Dehn twist. As we saw in Section \ref{sec:mapping-classes-non}, in order to perform the Penner
construction, we need to ensure that the curves we are working with are marked inconsistently. Note that our
labeling of the curves already gives us an inconsistent marking. For any alpha curve $\alpha_i$, we let the
marking $\phi_{\alpha_i}$ be orientation preserving and for beta curves $\beta_j$ let $\phi_{\beta_j}$ be
orientation reversing. Since alpha curves only intersect beta curves (and vice versa), we have an inconsistent
marking at each point of intersection.

Let $\mathcal{B}$ be the union of all $\beta$ curves except $\beta_1$ in $T_{0,0} \cup T_{0,1} \cup T_{1,0}$
(see figures below). Let $\rho_1(\mathcal{B})$ be the image of $\mathcal{B}$ under $\rho_1$. Define $\phi_b$
as the composition of Dehn twists along all the curves in the set
$\overline{\mathcal{B}} \coloneqq \mathcal{B} \cup \rho_1(\mathcal{B}) \cup \dots \cup
\rho_1^{n-1}(\mathcal{B})$. Since the curves in $\overline{\mathcal{B}}$ are disjoint, Dehn twists along them
commute and therefore it is not necessary to specify the order in which we compose these Dehn twists in
$\phi_b$. Let $\mathcal{R}$ be the union of all $\alpha$ curves except $\alpha_1$ in $T_{0,0}$. Define
$\overline{\mathcal{R}}$ and $\phi_r$ in the exact same way.

Let $\alpha_1,\beta_1 \subset T_{0,0}$ be the curves in Figure \ref{fig:curves}. Let $\phi$ be the composition
of Dehn twists along all the curves $\alpha_1, \rho_1(\alpha_1), \dots, \rho_1^{n-1}(\alpha_1)$ followed by
Dehn twists along all the curves $\beta_1,\rho_1(\beta_1),\dots,\rho_1^{n-1}(\beta_1)$. Define the map $f_{n,k}$
in the following manner.
\begin{align*}
    f_{n,k} &\coloneqq \rho_2 \circ \phi \circ \phi_b \circ \phi_r
\end{align*}
It follows from the Penner construction that $(f_{n,k})^k$ is pseudo-Anosov. Hence $f_{n,k}$ itself is
pseudo-Anosov and an invariant train track $\tau_{n,k}$ for $f_{n,k}$ can be obtained from Penner's
construction that we described in Section \ref{sec:mapping-classes-non}.

\paragraph{Step 3: The Mapping Torus}

We have now constructed an infinite family of non-orientable surfaces and pseudo-Anosov maps, but this isn't
enough. Looking back at Lemma \ref{lem:genera}, the genera of this family of surfaces do not include every
positive integer. In fact, it misses infinitely many integers. We will use our extension of the Thurston's
fibered face theory to fill in the gaps, by constructing fibers for fibrations of the mapping tori of the
pseudo-Anosov maps we defined above.

Let $M_{n,k}$ be the mapping torus of $f_{n,k}$. Likewise, let $\mathcal{K}_{n,k}$ denote the fibered cone of
$H^1(M_{n,k},\mathbb{R})$ corresponding to the map $f_{n,k}$. We will show that $M_{n,k}$ contains a closed,
relatively orientable, incompressible surface homeomorphic to $\mathcal{N}_3$ that is transverse to the
suspension flow direction. This will allow us to apply Theorem \ref{thm:oriented-sum} to construct
new fibrations of $M_{n,k}$.

\begin{figure}[h]
    \centering
    \incfig[0.4]{GammaCurves}
    \caption{The curves $\gamma$ and $\hat{\gamma}$ bound an a non-orientable surface of genus 1.}
    \label{fig:gammacurves}
\end{figure}

\begin{lem}
\label{lem:genus3}
There is a relatively orientable incompressible surface $F_{n,k}$ in $M_{n,k}$ that is homeomorphic to $\mathcal{N}_3$.
Moreover it is transverse to the suspension flow direction given by $f_{n,k}$ and its Poincar\'e dual is in
the closure $\overline{\mathcal{K}_{n,k}}$.
\end{lem}
\begin{proof}
  Let $\gamma \subset T_{0,0}$ be the curve as shown in Figure \ref{fig:curves}. Note that as we mentioned in Step 1,
  $\gamma$ was specifically chosen so that $\gamma$ and $\phi(\gamma)$ bound a non-orientable surface
  $\hat{F}$ of genus 1 with boundary, i.e. cutting along them creates an an annulus with a cross-cap. For
  convenience, we will denote $\phi(\gamma)$ by $\hat{\gamma}$. We are going to follow the image of $\gamma$
  under iterations of our pseudo-Anosov map $f_{n,k}$, this will allow us to attach tubes (annuli) to the
  boundary of $\hat{F}$ to get a closed $\mathcal{N}_3$. Following $\gamma$ under $f_{n,k}$ gives us the
  following.
  \begin{align*}
    f_{n,k}(\gamma) &= \rho_2 \circ \phi \circ \phi_b \circ \phi_r(\gamma) \\
                    &= \rho_2 \circ \phi(\gamma) \\
                    &= \rho_2(\hat{\gamma}) \\
  \end{align*}
  Applying $f_{n,k}$ repeatedly to ${\gamma}$ gives us the following.
  \begin{align*}
    f^2_{n,k}(\gamma) &= \rho_2^2(\hat{\gamma}) \\
                      &\vdots \\
    f^k_{n,k}(\gamma) &= \rho_2^k(\hat{\gamma})\\
                      &= \hat{\gamma}
  \end{align*}
  Let $T_i$ be a tube (annulus) that connects $f_{n,k}^{i-1}(\gamma)$ to $f_{n,k}^i(\gamma)$
  (which as we saw above will just be $\gamma$ or copies of $\hat{\gamma}$ on a $\rho_2$ rotation of $T_{0,0}$
  ) in the mapping torus $M_{n,k}$, for $1 \leq i \leq k$. We obtain these tubes by following the suspension
  flow of $f_{n,k}$ around $M_{n,k}$. We can now obtain our embedded surface $F_{n,k}$ by taking the union of
  $T_1,T_2,\dots,T_k$ and $\hat{F}$. Since we are ``adding an orientable genus" to a non-orientable surface of
  genus 1, we get that $F_{n,k}$ is homeomorphic to $\mathcal{N}_3$.

  Since the resulting surface is an embedded non-orientable surface in a non-orientable $3$-manifold, we have
  relative orientability by Proposition \ref{prop:relative-orientability}.

  The proof of the fact that $F_{n,k}$ can be isotoped to be transverse to the suspension flow is the same as the
  proof in \cite{yazdi2018pseudo}, which in turn follows the proof in \cite{leininger2013number}. The proof goes
  through even in this setting essentially because of the local nature of the proof.

  Finally, $F_{n,k}$ is incompressible in $M_{n,k}$ because $M_{n,k}$ is hyperbolic, and $F_{n,k}$ is genus $3$, the
  lowest possible genus for a hyperbolic non-orientable surface.
\end{proof}

\paragraph{Step 4: Bounding the Stretch Factor}

In \cite{yazdi2018pseudo}, Yazdi shows that the family of pseudo-Anosov maps that we have constructed all have
the log of their stretch factor bounded above by a similar factor. In order to to do this, recall in Section
\ref{sec:background} we saw that pseudo-Anosov maps give rise to matrices whose Perron-Frobenius eigenvalue is
our stretch factor. So a way to find an upper bound of the stretch factor of the maps we have constructed is
to bound the spectral radius of the associated matrices. The following lemma by Yazdi does just this for a
specific class of matrices that our examples are based off of.

\begin{lem}[Lemma 2.3 of \cite{yazdi2018pseudo}]
\label{lem:spectral}
Let $A$ be a non-negative integral matrix, $\Gamma$ be the adjacency graph of $A$, and $V(\Gamma)$ the set of
vertices of $\Gamma$. For each $v \in V(\Gamma)$, define $v^+$ to be the set of vertices $u$ such that there
is an oriented edge from $v$ to $u$. Let $D$ and $k$ be fixed natural numbers. Assume the following conditions
hold for $\Gamma$.
\begin{enumerate}[(i)]
\item For each $v \in V(\Gamma)$ we have $\deg_{\text{out}}(v) \leq D$.
\item There is a partition $V(\Gamma) = V_1 \cup \dots \cup V_k$ such that for each $v \in V_i$ we have
  $v^+ \subset V_{i+1}$, for any $1 \leq i \leq k$ except possibly when $i = 1$ or 3 (indices are mod $k$).
\item For each $v \in V_1$, we have $v^+ \subset V_2 \cup V_3$.
\item For each $v \in V_3$ we have $v^+ \subset V_3 \cup V_4$, and for $u \in v^+ \cap V_3$ we have
  $u^+ \subset V_4$.
\item For all $3 < j \leq k$ and each $v \in V_j$, the set $v^+$ consists of a single element.
\end{enumerate}
\end{lem}
With this result in hand, we can now show that the stretch factors for our main family of examples are all
bounded above in the way we hope.

\begin{lem}
  Let $\lambda_{n,k}$ be the stretch factor of $f_{n,k}$. Then there exists a universal positive constant $C'$
  and such that for every $n \geq 1$ and $k \geq 3$, we have the following upper bound on $\log(\lambda_{n,k})$.
  \begin{align*}
   \log(\lambda_{n,k}) \leq C'\frac{n}{g_{n,k}}
  \end{align*}
\end{lem}

\begin{proof}
  We deliberately constructed our examples so our curves are in the same ``general form" as the ones in
  \cite{yazdi2018pseudo} and thus they will still satisfy the criteria of Lemma \ref{lem:spectral}. All
  intersections between the curves happen inside the building block $T$, except for the intersections between
  building blocks given by the beta curves $\beta_3$ and $\beta_8$. Though we still will explicitly show that
  it is the case our set of curves will satisfy the above lemma by Yazdi as well.

  We define the following multi-curves.
\begin{align*}
  \mathcal{A} &\coloneqq \mathcal{B} \cup \mathcal{R} \cup \{\alpha_1,\beta_1\} \\
  \overline{\mathcal{A}} &\coloneqq \mathcal{A} \cup \rho_1(\mathcal{A}) \cup \dots \cup \rho_1^{n-1}(\mathcal{A}) \\
  \widehat{\mathcal{A}} &\coloneqq \overline{\mathcal{A}} \cup \rho_2\left(\overline{\mathcal{A}}\right) \cup \dots \cup \rho_2^{k-1}\left(\overline{\mathcal{A}}\right).
\end{align*}
Thus $\hat{\mathcal{A}}$ is all the curves on our surface we are Dehn twisting around to get $f_{n,k}$.

Recall from above that we stated we need to find the eigenvalue of the matrix that represents the action of
$f_{n,k}$ on the subspace of the cone of transverse measures that is spanned by the measures assigning $1$ to
single curves in $\hat{\mathcal{A}}$ and 0 to everything else. Let $A$ be said matrix and $\Gamma$ the
adjacency graph of $A$. In order to bound the spectral radius of $A$, we need to show that $\Gamma$ satisfies
the criteria of Lemma 1. To do this we first need to partition the vertices of $\gamma$, which is equivalent
to a partition of the curves in $\hat{\mathcal{A}}$:
$$\mathcal{A} = \bigcup_{i=1}^k \rho_2^{i-2}(\overline{\mathcal{A}}).$$ Then define $V_i$ for
$1 \leq i \leq k$ as the vertices of $\Gamma$ corresponding to elements in
$\rho_2^{i-2}(\overline{\mathcal{A}})$.

We can now check the conditions of Lemma 1, based on the combinatorics of the curves on our surface:
\begin{enumerate}[(i)]
\item From the way that the curves are constructed, there will exist a constant $D'$, independent of $n$ and
  $k$, such that for every connected curve (single element subset) $c \subset \hat{A}$, the geometric
  intersection number between $c$ and $\overline{A}$ is at most $D'$.  Recall from Section
  \ref{sec:mapping-classes-non} that we refer to the linear action of $f_{n,k}$ on the subspace of the cone of
  transverse measures on our invariant train track corresponding to connected curves in $\hat{A}$ as
  $A$. Following Yazdi we express $A$ as the following product.
  \begin{align*}
    A = M_4M_3M_2M_1
  \end{align*}
  Here each $M_i$ is the linear action of $\rho_2,\phi,\phi_b$ and $\phi_r$ respectively. For a connected
  curve $x \in \hat{A}$, the $L^1$-norm of $A(\mu_x)$ is bounded above by the geometric intersection of
  $f_{n,k}(x)$ with the curves in $\overline{A}$, thus each of $M_1$, $M_2$ and $M_3$ will change the norm by
  a factor of at most $(1 + D')$. Since $\rho_2$ won't change intersection numbers, $M_4$ will preserve the
  $L^1$-norm. If we let $D = (1 + D')^3$, then the outward degree of each vertex in $\Gamma$ is at most $D$.
\item As above, we now have a partition of our vertices where
  $V_i \coloneqq \rho_2^{i-2}(\overline{\mathcal{A}})$. So suppose that $v \in V_i$, $i \neq 1,3$, is a vertex
  that corresponds to $\mu_c$ for a curve $c \in \hat{\mathcal{A}}$. By the partitioning $c$ must be a curve
  in $\rho_2^{i-2}(\overline{\mathcal{A}})$, for $i \neq 1,3$. Recall that $f_{n,k}$ is defined as
  $f_{n,k} = \rho_2 \circ \phi \circ \phi_b \circ \phi_r$. The action of $\phi \circ \phi_b \circ \phi_r$ will
  send $\mu_c$ to a sum of $\mu_y$ where $y$ corresponds to elements of $V_i$, since
  $\phi \circ \phi_b \circ \phi_r$ will send curves in these partitions to curves that just intersect curves
  in the same partition. Then $\rho_2$ will rotate all the curves to the next partition, thus sending $\mu_y$
  to $\mu_z$, where $z$ corresponds to an element of $V_{i+1}$.
\item We need to see which vertices in $v \in V_1$ have $v^+ \not\subset V_2$. As in Yazdi, this is precisely
  the vertices corresponding to the following set.
  \begin{align*}
    X = \{\mu_y \mid \exists i\ \text{s.t. $y$ is a connected curve in } \rho_1(\rho_2^{-1}(\beta_8))\}
  \end{align*}
  One can see from the picture of the curves that elements corresponding to $X$ will have
  $v^+ \subset V_2 \cup V_3$.
\item The elements $v \in V_3$ such that $v^+ \not\subset V_4$ are the ones that correspond to the elements of
  the following set.
  \begin{align*}
    Y = \{\mu_y \mid \exists i \text{ s.t. $y$ is a connected curve in } \rho_1^i(\rho_2(\alpha_8))\}
  \end{align*}
  This is due to the intersection of the curves $\rho_1^i(\rho_2(\alpha_8))$ with the curves
  $\rho_1^i(\beta_8)$. Moreover, for any element $v \in V_3$ corresponding to $Y$ and any
  $u \in v^+ \cap V_3$, $u$ no longer correpsonds to $Y$ and hence $u^+ \subset V_4$.
\item All the curves corresponding to an element of $V_j$, $3 < j \leq k$ are disjoint from all the curves in
  $\overline{A}$. Thus $f_{n,k}$ just acts by rotation.
\end{enumerate}

Setting $\lambda = \lambda_{n,k}$, Lemma \ref{lem:spectral} implies the following.
\begin{gather*}
    \lambda^{k-1} = \rho(A)^{k-1} = \rho(A^{k-1}) \leq 4D^4 \implies (k-1)\cdot \log(\lambda) \leq \log(4D^4) \\
    \implies \frac{k}{2}\log(\lambda) \leq (k-1)\log(\lambda) \leq \log(4D^4)
\end{gather*}

On the other hand, we know $g_{n,k} = (14k - 2)n + 2 \leq 14kn$. Therefore
\begin{align*}
    \log(\lambda) \leq 2\log(4D^4)\cdot\frac{1}{k} \leq 2\log(4D^4)\cdot \frac{14n}{g_{n,k}} = C'\frac{n}{g_{n,k}}
\end{align*}
where $C' \coloneqq 28\log(4D^4)$.
\end{proof}

\paragraph{Step 5: Filling in the Gaps}

We now want to use the mapping tori $M_{n,k}$ of our maps $f_{n,k}$ to construct pseudo-Anosov maps with small
stretch factors on the surfaces of the genera we our missing from our family $P_{n,k}$. To do this we consider
the following surfaces: $P^r_{n,k} = P_{n,k} + r(F_{n,k})$, that is taking the oriented sum of $P^r_{n,k}$ and
$F_{n,k}$ $r$ times as defined in Theorem $\ref{thm:oriented-sum}$.

\begin{lem}
  The surface $P^r_{n,k}$ have genus equal to $g^r_{n,k} = g_{n,k} + r$. In particular as $r$ varies between
  $0$ and $6n$, the genera of $P^r_{n,k}$ cover the range between $g_{n,k}$ and $g_{n,k+1}$. Moreover,
  $P^r_{n,k}$ is isotopic to a fiber of a fibration of $M_{n,k}$ with pseudo-Anosov monodromy that fixes $2n$
  of the singularities of its invariant foliation.
\end{lem}

\begin{proof}
  We know that the Euler characteristic of an oriented sum is the sum of the Euler characteristics of the summands.
  \begin{align*}
    \chi(P^r_{n,k}) &= \chi(P_{n,k}) + r\cdot\chi(F_{n,k}) \\
                    &= (-2(g_{n,k} - 1) + 2)-2r \\
                    &= -2(g_{n,k} + r - 1) + 2
  \end{align*}
  This proves the identity for the genus of $P^r_{n,k}.$

  From Theorem \ref{thm:oriented-sum}, we know that $P^r_{n,k}$ is isotopic to a fiber of a fibration of
  $M_{n,k}$ since we showed that $F_{n,k}$ is incompressible and transverse to the suspension flow in Lemma
  \ref{lem:genus3}. If we now let $f^r_{n,k}$ be the first return map of this new fibration, since one
  monodromy of $M_{n,k}$ is pseudo-Anosov, all monodromies are and $f^r_{n,k}$ is a pseudo-Anosov map.

  As in the proof of Lemma 3.5 of \cite{yazdi2018pseudo}, the singularities of the stable foliation of
  $f_{n,k}$ that are fixed are the $2n$ intersection points of the axis of $\rho_1$ with
  $P_{n,k}$. Furthermore we have already seen that the surface $F_{n,k}$ can be isotoped to be transverse to
  the suspension flow and disjoint from the orbit of the $2n$ singularities of $f_{n,k}$, hence the monodromy
  $f^r_{n,k}$ still fixes the corresponding $2n$ singularities on $P^r_{n,k}$.
\end{proof}

We can now prove the non-orientable version of Lemma 3.6 of \cite{yazdi2018pseudo}.
\begin{lem}
\label{lem:bound}
Let $\lambda_{n,k}^r$ be the stretch factor of $f_{n,k}^r$.Then there exists a constant $C > 0$ such that for every $n \geq 1$, $k \geq 3$, and $0 \leq r \leq 6n$ we have the following upper bound on $\log(\lambda_{n,k}^r)$.
\begin{align*}
  \log(\lambda^r_{n,k}) \leq C\frac{n}{g^r_{n,k}}
\end{align*}
\end{lem}
\begin{proof}
  Let $\mathcal{K} = \mathcal{K}_{n,k}$ be our fibered faces and $h: \mathcal{K} \xrightarrow[]{} \mathbb{R}$
  the function described in Theorem \ref{thm:NOfm}. Note that we have the following bounds on $g_{n,k}^r$.
  \begin{align*}
    g^r_{n,k} &= g_{n,k} + r \\
              &\leq g_{n,k} + 14n \\
              &< 2g_{n,k}
  \end{align*}
  Now if we let $\omega$ be the Poincar\'e dual of $P^r_{n,k}$, $\alpha$ the Poincar\'e dual of $P_{n,k}$ and
  $\alpha'$ the Poincar\'e dual of $F_{n,k}$, we have the following bounds by the convexity of the entropy
  function $h$.
  \begin{align*}
    h([\omega]) &< h([\alpha]) \\
                &\leq C'\frac{n}{g_{n,k}} \\
                &\leq 2C'\frac{n}{g^r_{n,k}}
  \end{align*}
\end{proof}

So now we have that our surfaces $P^r_{n,k}$ are isotopic to fibers of fibrations of $M_{n,k}$ with
pseudo-Anosov monodromies with bounded stretch factors.

We can now think of $f^r_{n,k}$ as a map on a non-orientable surface of genus $g^r_{n,k}$. Note from above we
know that $g^r_{n,k}$ covers all natural numbers between $g_{n,k}$ and $g_{n,k+1}$, thus this set of genera
for all $r$ covers all natural numbers larger than $g_{n,3} = 40n + 2$. Recall that all of these surfaces will
have $2n$ singularities, so we can either puncture $n$ or $n + 1$ to account for all possible number of
punctures.

We can now give a proof of Theorem \ref{thm:stretch1}.

\begin{manualtheorem}{\ref{thm:stretch1}}
For any fixed $n \in \mathbb{N}$, there are positive constants $B'_1 = B'_1(n)$ and $B'_2 = B'_2(n)$ such that
  for any $g \geq 2$, the stretch factor satisfies the following inequalities.
  \begin{align*}
    \frac{B'_1}{g} \leq l'_{g,n} \leq \frac{B'_2}{g}
  \end{align*}
\end{manualtheorem}
\begin{proof}
  We begin by proving the upper bound. By Lemma \ref{lem:bound} and above, we have that there exists a number
  $C' > 0$ such that for $g \geq 40n + 2$, $l'_{g,n} \leq 2C'\frac{n}{g}$. So we take $B'_2(n)$ to be the
  following quantity.
  \begin{align*}
    B'_2(n) = \max\{4C'n, l'_{1,n}, 2l'_{2,n}, \dots, (40n + 1)l'_{40n+1,n}\}
  \end{align*}

  The lower bound follows easily from the lower bound in the orientable setting, as demonstrated in the
  discussion following Theorem \ref{thm:stretch1}. For ease of reading, we replicate that here.  Let $f$ be
  the pseudo-Anosov map with the minimal stretch factor on $\no_{g,n}$. Then, by Proposition \ref{prop:2},
  this map lifts to a map $\wt{f}$ on $\os_{g-1, 2n}$ (possibly after squaring). Furthermore, $\wt{f}$ has the
  same stretch factor as $f$. The former is bounded below by $\frac{B}{g}$, and thus the stretch factor of $f$
  is bounded below as well.
\end{proof}
