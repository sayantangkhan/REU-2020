\documentclass[11pt, notitlepage]{amsart}
\usepackage[utf8]{inputenc}

% \usepackage[nottoc]{tocbibind}
\usepackage{datetime}
\usepackage{graphicx}
\usepackage[nointegrals]{wasysym}
\usepackage{mathtools}
\usepackage{tikz}
\usepackage{tikz-cd}
\usepackage{enumerate}
\usepackage{amsmath}
\usepackage{amsfonts, amssymb, amsthm}
\usepackage{bbm}
\usepackage[normalem]{ulem}
\usepackage{calc}
\usepackage{hyperref}
\hypersetup{
  colorlinks   = true, %Colours links instead of ugly boxes
  urlcolor     = blue, %Colour for external hyperlinks
  linkcolor    = blue, %Colour of internal links
  citecolor   = blue %Colour of citations
}
\newcommand{\p}[1]{\bigskip\noindent{\bf#1.}}

\newtheorem{thm}{Theorem}[section]
\newtheorem*{unthm}{Theorem}

\newtheorem{lem}[thm]{Lemma}
\newtheorem{claim}[thm]{Claim}
\newtheorem{prop}[thm]{Proposition}
\newtheorem{cor}[thm]{Corollary}
\newtheorem{conj}[thm]{Conjecture}

\newtheorem{manualtheoreminner}{Theorem}
\newenvironment{manualtheorem}[1]{%
  \renewcommand\themanualtheoreminner{#1}%
  \manualtheoreminner
}{\endmanualtheoreminner}

\theoremstyle{definition}
\newtheorem{question}[thm]{Question}

\theoremstyle{definition}
\newtheorem{example}[thm]{Example}

\theoremstyle{definition}
\newtheorem{fact}[thm]{Fact}

\theoremstyle{remark}
\newtheorem*{rem}{Remark}

\theoremstyle{definition}
\newtheorem{defn}[thm]{Definition}

\usepackage[top=2cm, bottom=2cm, inner=2cm, outer=2cm]{geometry}

\usepackage{enumerate}
\usepackage{xcolor}

\usepackage{import}
\usepackage{xifthen}
\usepackage{pdfpages}
\usepackage{transparent}

% Making paragraphs bolder
\makeatletter
\def\paragraph{\@startsection{paragraph}{4}%
  \z@\z@{-\fontdimen2\font}%
  {\normalfont\bfseries}}
\makeatother

% Spacing for paragraph
\makeatletter
\renewcommand{\paragraph}{%
  \@startsection{paragraph}{4}%
  {\z@}{2ex \@plus 1ex \@minus .2ex}{-1em}%
  {\normalfont\normalsize\bfseries}%
}
\makeatother

\newcommand*{\incfig}[2][1]{%
    \def\svgscale{#1}
    \import{./images/}{#2.pdf_tex}
}
\graphicspath{{images/}}

% Notation macros
\newcommand{\RR}{\mathbb{R}}
\newcommand{\ZZ}{\mathbb{Z}}
\newcommand{\wt}[1]{\widetilde{#1}}
\newcommand{\no}{\mathcal{N}}
\newcommand{\os}{\mathcal{S}}
\DeclareMathOperator{\Mod}{Mod}
\DeclareMathOperator{\Spec}{Spec}

\title[Pseudo-Anosov homeomorphisms of non-orientable surfaces]{Pseudo-Anosov homeomorphisms of punctured non-orientable surfaces with small stretch factor}

\author{Sayantan Khan}
\address{Department of Mathematics, University of Michigan, 530 Church Street, Ann Arbor, MI 48109}
\email{saykhan@umich.edu}
\urladdr{\url{https://www-personal.umich.edu/~saykhan/}}

\author{Caleb Partin}
\address{Department of Mathematics, Georgia Institute of Technology, 686 Cherry Street, Atlanta, GA 30332}
\email{ctpartin@gmail.com}

\author{Rebecca R. Winarski}
\address{ Department of Mathematics and Computer Science, College of the Holy Cross, 1 College Street, Worcester, MA 01610}
\email{rwinarski@holycross.edu}
\urladdr{\url{https://sites.google.com/site/rebeccawinarski/}}

% give a separate \keyword and \subject line for each keyword/phrase or
% subject class eg \keyword{framed link} \subject{primary}{msc2010}{57M25}

% \keyword{pseudo-Anosov}
% \keyword{mapping class group}
% \keyword{dilatation}
% \keyword{stretch factor}
% \keyword{non-orientable surfaces}
% \subject{primary}{msc2010}{37E30}
% \subject{secondary}{msc2010}{37B40}

%  Fill in the reference number if your article is stored on the arXiv
%  eg \arxivreference{math.GT/0512347} or \arxivreference{1203.4984}.
%  The newer style reference numbers (with a period) do not require the
%  prefix arxiv: or math.NT/ or anything else. Just the reference
%  number is sufficient.

% \arxivreference{2107.04068}

\def\sectionautorefname{Section}
\def\thmautorefname{Theorem}
\def\lemautorefname{Lemma}
\def\quesqutorefname{Question}

\makeatletter

\makeatother

\begin{document}
\maketitle

\begin{abstract}
We prove that for non-orientable surface the minimal stretch factor of a pseudo-Anosov homeomorphism of a surface of genus $g$ with a fixed number of punctures is asympototically on the order of $\frac{1}{g}$.  Our result adapts the work of Yazdi to non-orientable surfaces.  We include the details of Thurston's theory of fibered faces for non-orientable 3-manifolds.
 % We generalize Thurston norm and the related theory of fibered faces to the setting of non-orientable $3$-manifolds. This lets us construct examples of pseudo-Anosov maps on non-orientable surfaces with small stretch factors. Using this, we prove that for a fixed number of punctures, the minimal stretch factor of a genus $g$ non-orientable surface behaves like $\frac{1}{g}$, generalizing the techniques and a result of Yazdi.
\end{abstract}

\section{Introduction}
\label{sec:introduction}
%Let $S_{g,n}$ be a surface of genus $g$ with $n$ punctures.  The mapping class group of $S_{g,n}$ consists of homotopy classes of orientation preserving homeomorphisms of $\Mod(S_{g,n})$.  The Nielsen--Thurston classification of mapping class groups says that each element of the mapping class is either reducible (preserves a multi-curve), periodic, or is pseudo-Anosov.  Pseudo-Anosov mapping classes play a critical role in Thurston's Hyperbolization Theorem: a mapping torus of a surface and a homeomorphism $\varphi$ is a hyperbolic 3-manifold if and only if $\varphi$ is pseudo-Anosov.

Let $S_{g,n}$ be a surface of genus $g$ with $n$ punctures.  The mapping class group of $S_{g,n}$ consists of homotopy classes of orientation preserving homeomorphisms of $S_{g,n}$.  The Nielsen--Thurston classification of mapping classes (elements of the mapping class group) says that each mapping class is periodic, preserves some multicurve, or has a representative that is pseudo-Anosov.  For each pseudo-Anosov homeomorphism $\varphi:S_{g,n}\rightarrow S_{g,n}$, the stretch factor $\lambda(\varphi)$ is an algebraic integer that describes the amount by which $\varphi$ changes the length of curves.  Arnoux--Yaccoz \cite{AY} and Ivanov \cite{ivanov} prove that the set
$$\Spec(S_{g,n})=\{\log(\lambda(\varphi)) \mid \varphi \text{ is a pseudo-Anosov homeomorphism of }S_{g,n}\}$$ is a closed discrete subset of $(0,\infty)$. The minimum of $\Spec(S_{g,n})$:
$$\ell_{g,n}=\min\{\log(\lambda(\varphi)) \mid \varphi \text{ is a pseudo-Anosov homeomorphism of }S_{g,n}\}$$ quantitatively describes both the dynamics of the mapping class group of $S_{g,n}$ and the geometry of the moduli space of $S_{g,n}$.

Penner \cite{penner1991bounds} showed that for orientable surfaces, $$\ell_{g,0}\asymp \frac{1}{g}.$$ %bounded $$\frac{A}{g}\leq l_{g,0}\leq \frac{B}{g},$$ where $A,B$ are real constants
  Penner conjectured that $\ell_{g,n}$ will have the same asymptotic behavior for  $n\geq0$ punctures.  Following Penner, substantial attention has been given to finding bounds for $\ell_{g,n}$ \cite{AD,bauer,hironaka,HK,HK20,KT,Loving,minakawa}, calculating $\ell_{g,n}$ for specific values of $(g,n)$ \cite{CH,HS,LT,SKL}, and finding asymptotic behavior of $\ell_{g,n}$ for {\it orientable} surfaces with $n\geq 0$ \cite{KT,tsai2009asymptotic,valdivia,yazdi2018pseudo}.  We adapt a result of Yazdi \cite{yazdi2018pseudo} to non-orientable surfaces. %In particular, Tsai proved that when $S_{g,n}$ is orientable surface of fixed genus $g\geq 2$, $l_{g,n}$ is on the order of $\frac{\log(n)}{n}$ Much of this work has focused on orientable surfaces.  Let $\mathcal{N}_{g,n}$ be a non-orientable surface of genus $g$ with $n$ punctures.  In this paper we study the asympototic behavior of $l_{g,n}$ as $n$ increases.

\begin{thm}\label{thm:stretch1}
  Let $\no_{g,n}$ be a non-orientable surface of genus $g$ with $n$ punctures, and let $\ell_{g,n}'$ be the logarithm of
  the minimum stretch factor of the pseudo-Anosov mapping classes acting on $\no_{g,n}$.
  Then for any fixed $n \in \mathbb{N}$, there is a positive constant $B'_1 = B'_1(n)$ and $B'_2 = B'_2(n)$ such
  that for any $g \geq 3$,
  %\becca[inline]{This should be 3, right?  When we are talking about non-orientable genus?}
  %\sayantan[inline]{Yes, that is correct. I have made the changes elsewhere in the document to reflect this correction.}
  the quantity $\ell_{g,n}'$ satisfies the following inequalities:
  \begin{align*}
    \frac{B'_1}{g} \leq \ell'_{g,n} \leq \frac{B'_2}{g}.
  \end{align*}
\end{thm}

%Our proof follows that of Yazdi \cite{yazdi2018pseudo}, who proves the same result for orientable surfaces.

\p{Pseudo-Anosov homeomorphisms} Let $S$ be a (possibly non-orientable) surface of finite type.  A homeomorphism $\varphi:S\rightarrow S$ is said to be {\it pseudo-Anosov} if there exist a pair of transverse measured singular foliations $\mathcal{F}_s$ and $\mathcal{F}_u$ and a real number $\lambda$ such that $$\varphi(\mathcal{F}_s)=\lambda^{-1} \cdot \mathcal{F}_s\text{ and } \varphi(\mathcal{F}_u)=\lambda \cdot \mathcal{F}_u.$$  The {\it stretch factor} of $\varphi$ is the algebraic integer $\lambda=\lambda(\varphi)$.

%\p{Dynamics of the mapping class group}
%Let $S_{g,n}$ be a surface and let $\varphi:S_{g,n}\rightarrow S_{g,n}$ be a pseudo-Anosov mapping class.
Endow $S$ with a Riemannian metric.  The stretch factor $\lambda(\varphi)$ measures the growth rate of the length of geodesic representatives of a simple closed curve $S$ under iteration of $\varphi$ \cite[Proposition 9.21]{FLP}.  Moreover, $\log(\lambda(\varphi))$ is the minimal topological entropy of any homeomorphism of $S$ that is istopic to $\varphi$ \cite[Expos\'e 10]{FLP}.

%\p{Volume of mapping tori} Let $S$ be a surface and $\varphi:S\rightarrow S$ be a homeomorphism.  Let $M$ be the mapping torus of $S$ by $\varphi$.  By Thurston's hyperbolization theorem, $M$ is a hyperbolic 3-manifold if and only if $\varphi$ is hyperbolic.

\p{Geometry of moduli space}
Let $\mathcal{T}_{g,n}$ denote the Teichm\"uller space of $S_{g,n}$, that is: the space of isotopy classes of hyperbolic metrics on $S_{g,n}$.
When endowed with the Teichm\"uller metric, the mapping class group of $S_{g,n}$ acts properly discontinuously on $\mathcal{T}_{g,n}$ by isometries.  The quotient of this action is the {\it moduli space} of $S_{g,n}$. The set
$\Spec(S_{g,n})$ is the length spectrum of geodesics in the moduli space of $S_{g,n}$.  Therefore the quantity $\ell_{g,n}$ is the length of the shortest geodesic in the moduli space of $S_{g,n}$.

\p{Explicit bounds} In his foundational work, Penner found $\frac{\log 2}{12g-12+4n}$ to be a lower bound for $\ell_{g,n}$ for orientable surfaces \cite{penner1991bounds}.  He also determined $\frac{\log 11}{g}$ to be an upper bound for $\ell_{g,0}$.  Penner's work proves that $\ell_{g,0}\asymp \frac{1}{g}$.  McMullen  \cite{mcmullen2000polynomial} later asked:
\begin{question}[McMullen]
Does $\displaystyle\lim_{g\rightarrow\infty}g\cdot \ell_{g,0}$ exist, and if so, what does it converge to?
\end{question}
To this end, Bauer \cite{bauer} strengthened the upper bound for $g\cdot \ell_{g,0}$ to $\log 6$, and Minakawa \cite{minakawa} and Hironaka--Kin \cite{HK} further sharpened the upper bounds for $g\cdot \ell_{g,0}$ and $g\cdot \ell_{0,2g+1}$ to $\log(2+\sqrt{3})$.  Later Aaber--Dunfield \cite{AD}, Hironaka \cite{hironaka}, and Kin-Takasawa \cite{KTbounds} determined that $\log\left(\frac{3+\sqrt{5}}{2}\right)$ is an upper bound for $g\cdot \ell_{g,0}$ and conjectured it is the supremum of $g\cdot \ell_{g,0}$.

\p{Asymptotic behavior of punctured surfaces}
Tsai initiated the study of asymptotic behavior of $\ell_{g,n}$ along lines in the $(g,n)$-plane \cite{tsai2009asymptotic}.  In particular, Tsai determined that for orientable surfaces of fixed genus $g\geq 2$, the asymptotic behavior in $n$ is:
$$\ell_{g,n}\asymp \frac{\log n}{n}.$$
Further, he showed that $\ell_{0,n}\asymp \frac{1}{n}.$
Later, Yazdi \cite{yazdi2018pseudo} determined that for an orientable surface with a fixed number of punctures $n\geq 0$, the asymptotic behavior in $g$ is:
$$\ell_{g,n}\asymp \frac{1}{g},$$
confirming the conjecture of Penner.

\p{Non-orientable surfaces}
Let $\no_{g,n}$ be a non-orientable surface of genus $g$ with $n$ punctures.  As above, let $\ell'_{g,n}$ denote the minimum stretch factor of pseudo-Anosov homeomorphisms of $\no_{g,n}$.  For any $n\geq 0$ and $g\geq 1$, $\ell_{g-1,2n}$ is a lower bound for $\ell'_{g,n}$, which can be seen by passing to the orientation double cover of $\no_{g,n}$ (note that the definition of genus is different for orientable and non-orientable surfaces).  Because the upper bounds for $\ell_{g,n}$ are constructed by example, upper bounds for $\ell'_{g,n}$ do not follow immediately from passing to the orientation double cover.  Recently Liechti--Strenner determined $\ell'_{g,0}$ for $g\in\{4,5,6,7,8,10,12,14,16,18,20\}$ \cite{LS}.  Our work captures the asymptotic behavior for the punctured case.

%Liechti--Strenner were motivated to calculate $l'_{g,0}$ because Liechti rephrased question in


\p{Techniques} To prove Theorem \ref{thm:stretch1}, we adapt the strategy of Yazdi \cite{yazdi2018pseudo} to non-orientable surfaces with punctures.  The lower bound of $\ell'_{g,n}$ is found by lifting to the orientation double cover of $\no_{g,n}$.  The upper bound (as in all prior work) is constructive.  Fix $n\geq 0$: the desired number of punctures.  Yazdi's construction is as follows.  For a sequence of genera $g_{n,k}$ (where $k$ goes from $3$ to $\infty$, and $g_{n,k} = (14k-2)n + 2$), use the Penner construction \cite{penner1988construction} to obtain a homeomorphism $f_{n,k}$ of $S_{g_{n,k},n}$ that is pseudo-Anosov and has low stretch factor.  In order to find pseudo-Anosov homeomorphisms of $S_{g,n}$ with small stretch factor for all $g$ (not just those in the sequence above), construct a mapping torus for each $f_{n,k}$.  To do this Yazdi's appeals to a technique involving the use of Thurston's theory of fibered faces.  %Therefore a secondary goal of this paper is to adapt Thurston's theory of
 %Most of the work in doing so is concentrated in determining the right analog of Thurston norm for non-orientable surfaces, and then making Theorem \ref{thm:Thur1} work with that definition, which is what we will do in Section \ref{sec:thur-norm-non-orientable}. Once we have the versions of the theorems for non-orientable surfaces, we'll generalize a trick due to McMullen that lets one construct pseudo-Anosov maps with small stretch factor to non-orientable surfaces. Finally we will prove bounds on the asymptotics of minimal stretch factors for non-orientable surfaces in Section \ref{sec:application} by adapting the methods in \cite{yazdi2018pseudo} for non-orientable surfaces.

\p{Thurston norm for non-orientable 3-manifolds} In Thurston's development of what is now called the Thurston norm for 3-manifolds \cite{thurston1986norm}, his definitions and theorems required that all surfaces were orientable.  Thurston said that the theorems should still be true for non-orientable surfaces, but there are some subtleties that have not been addressed elsewhere in the literature.  In this paper, we write the details of Thurston's theory of fibered faces to non-orientable 3-manifolds.  In particular, the Thurston norm is a norm on the second homology of a 3-manifold, that measures the minimum complexity of an embedded (orientable) surface. However the Thurston norm does not recognize embedded non-orientable surfaces in the second homology of a non-orientable 3-manifold.  To address this limitation, we instead calculate the Thurston norm on the first cohomology of a non-orientable manifold.  We develop a (weak) version of Poincar\'e duality in Theorem \ref{thm:strong-duality} that suffices to define a Thurston norm on $H^1(M;R)$ for a non-orientable 3-manifold $M$.

\p{Fibered faces} A special case of Thurston's hyperbolization theorem says that the monodromy of any fibration of a hyperbolic 3-manifold over $S^1$ is a pseudo-Anosov homeomorphism.  Therefore by finding other fibrations of the same 3-manifold, one obtains additional pseudo-Anosov homeomorphism.  Work of Fried \cite{fried1982flow,fried1983transitive}, Matsumoto \cite{matsumoto1987topological}, and Agol--Leininger--Margalit \cite{agol6983pseudo} can be used to find a bound on the stretch factors of certain pseudo-Anosov homeomorphisms obtained in this way.


%In Section \ref{sec:backgr-thurst-norm} we recall Thurston's theory of fibered faces for orientable 3-manifolds. Then in Section \ref{sec:thurst-norm-cohom} we define the Thurston norm on the first cohomology of a non-orientable 3-manifold.  In order to use the Thurston norm to detect non-orientable surfaces, we will need a version of Poincar\'e duality for a pair consisting of a non-orientable 3-manifold and an embedded non-orientable surface, which we state and prove in Section \ref{sec:weak-inverse-poinc}.
In Section \ref{sec:thur-norm-non-orientable} we state Thurston's theory of fibered faces and adapt it to the non-orientable setting.  In Section \ref{sec:mapping-classes-with} we show how Thurston's theory of fibered faces can be used to construct pseudo-Anosov homeomorphisms of low stretch factor for non-orientable surfaces.  Specifically, we state and prove the Nielsen--Thurston classification for non-orientable surfaces.  Then we adapt the results of Fried \cite{fried1982flow,fried1983transitive}, Matsumoto \cite{matsumoto1987topological}, and Agol--Leininger--Margalit \cite{agol6983pseudo} used to construct pseudo-Anosov homeomorphisms with low stretch factor of orientable surfaces to the non-orientable setting.  In Section \ref{sec:application}, we prove Theorem \ref{thm:stretch1}, following the strategy of Yazdi.

\p{Acknowledgements}
This work is the result of an REU at the University of Michigan in summer 2020.  We are grateful to Alex Wright for organizing the REU and to Livio Liechti for suggesting the project. The REU was partially funded by NSF Grant No. DMS 185615. We are also grateful to the other co-organizers, mentors, and participants in the REU: Paul Apisa, Chaya Norton, Christopher Zhang, Bradley Zykoski, Anne Larsen, and Rafael Saavedra.  The third author acknowledges support of the NSF through grant 2002951.
%\begin{manualtheorem}{\ref{thm:NOThur1}}
 % The unit ball with respect to the dual Thurston norm on $\left( H^1(M; \RR) \right)^{\ast}$ is a polyhedron in $(H^1(M,\RR))^\ast$ whose vertices are lattice points $\{\pm \beta_1, \ldots \pm \beta_k\}$. The unit ball $B_1$ with respect to Thurston norm is a polyhedron given by the following inequalities.
%  \begin{align*}
 %   B_1 = \left\{ a\in H^1(M,\RR) \mid \left| \beta_i(a) \right| \leq 1 \text{ for $1\leq i \leq k$} \right\}
 % \end{align*}
%\end{manualtheorem}

%This paper has two main goals: to understand the asymptotic behavior of the stretch factor of pseudo-Anosov homeomorphisms of non-orientable surfaces and to write down the details of the Thurston norm and Thurston's theory of fibered faces for non-orientable surfaces.  In the study of orientable surfaces, Thurston's theory of fibered faces is an important tool in studying dilations.

%Let $M$ be a closed, fibered 3-manifold that fibers over $S^1$. It is a well known fact that these surface bundles over the circle all have a separate description, that of the mapping torus of some homeomorphism of a closed surface. Not only that, but a single 3-manifold can have many, possibly infinite, descriptions as a mapping torus. Thus when studying these 3-manifolds, an important question is whether one can understand or give a description of all these possible fibrations.

%In 1986, Thurston gave a way to answer this question, a semi-norm on the second homology of an orientable 3-manifold that was able to ``detect'' when an embedded surface of the 3-manifold was the fiber of a fibration of said manifold. This \textit{Thurston norm} is given a full treatment in \cite{thurston1986norm}, in which Thurston shows that not only does this norm have unit ball which is a polyhedron, but the fibers of fibrations of one of these 3-manifolds $M$ were in one-to-one correspondence with cones on open faces of this polyhedron unit ball. This almost combinatorial description of the fibers turns out to be a powerful tool in studying these fibered 3-manifolds.

%All this also plays a key role in understanding mapping class groups of surfaces, since a surface and a mapping class on it is associated to a $3$-manifold, namely, the mapping torus of the mapping class.  By relating the different ways a given mapping torus can fiber, one is able to relate mapping classes on different surfaces, and in doing so, construct mapping classes of interest.

%However, there is a small issue with using Thurston norm based techniques: Thurston defines his norm on the second homology of an \textit{orientable} 3-manifold. This leads to the question of whether the results that depend on the Thurston norm work in the non-orientable setting as well. While Thurston himself does comment on this in \cite{thurston1986norm}, writing, ``(m)ost of this paper works also for non-oriented manifolds, but for simplicity we deal only with the oriented case.'' It is the goal of this paper to deal with the other case, to see what works and what, if anything, possibly goes wrong when trying to extend the Thurston norm and its consequences for fibered 3-manifolds to the non-orientable setting.

%One of the ways Thurston's results are used in the study of mapping classes is via the operation of \emph{oriented sum}. Given the mapping torus $M$ of some surface $S$ and some homeomorphism $\varphi$, one can identify $S$ with an embedded surface in $M$. By picking another embedded surface $S'$ in an appropriate manner, one can perform local surgery to combine $S$ and $S'$: the resulting surface is called the oriented sum of $S$ and $S'$ and denoted $S+S'$. Under the appropriate hypothesis, the oriented sum is also the fiber of some other fibration, and thus has a mapping class $\varphi'$ on it. It turns out one can relate the stretch factors of $\varphi$ (which is a mapping class on $S$) and $\varphi'$ (which is a mapping class on $S + S'$). We generalize this operation of oriented sum for non-orientable surfaces in Theorem \ref{thm:oriented-sum}.

%To show that this paper isn't just generalization for the sake of generalization, we use this generalization of Thurston's results and oriented sums to study the asymptotic behaviour of the minimal stretch factor of punctured non-orientable surfaces. The result for orientable punctured surfaces was proven by Yazdi in 2019, and relied heavily on Thurston's result, among others. We are able to essentially ``plug in'' the non-orientable version of Thurston's results to get a non-orientable version of Yazdi's results. This is one of the main theorems of this paper.



%One of the primary goals of this paper to extend this definition of Thurston norm to non-orientable manifolds and be able to state the non-orientable version of the theorems above. Most of the work in doing so is concentrated in determining the right analog of Thurston norm for non-orientable surfaces, and then making Theorem \ref{thm:Thur1} work with that definition, which is what we will do in Section \ref{sec:thur-norm-non-orientable}. Once we have the versions of the theorems for non-orientable surfaces, we'll generalize a trick due to McMullen that lets one construct pseudo-Anosov maps with small stretch factor to non-orientable surfaces. Finally we will prove bounds on the asymptotics of minimal stretch factors for non-orientable surfaces in Section \ref{sec:application} by adapting the methods in \cite{yazdi2018pseudo} for non-orientable surfaces.
\section{Thurston norm for non-orientable $3$-manifolds}
\label{sec:thur-norm-non-orientable}

\subsection{Background on non-orientable surfaces and their mapping tori}
\label{sec:backgr-non-orient}

A genus $g$ non-orientable surface is the connect-sum of $g$ copies of
$\mathbb{RP}^2$, analogous to how a genus $g$ surface is the connect-sum of $g$ copies of a torus
$S^1 \times S^1$.  A common way of visualizing non-orientable surfaces is to think of them as orientable
surfaces with \emph{crosscaps} attached. We attach a crosscap to a surface $S$ by first deleting a small open disc $D\subset S$, and
identifying the boundary of that disc (on the surface) via the antipodal map. In pictures, this is often denoted by
an X inscribed in a circle, see \autoref{fig:buildingblock} for an example of a surface with two crosscaps
attached.  Let $\no_{g,n}$ be a non-orientable surface obtained by attaching $g$ crosscaps to $S^2$ and marking $n$ points in $S^2$.  The integer $g$ is referred to as the genus of $\no_{g,n}$.  The compact non-orientable surfaces are classified by the triple $(g,n,b)$ where $g$ is the genus, $n$ is the number of marked points and $b$ is the boundary.

\p{The orientation double cover} The orientation double cover of $\no_{g,n}$ is an orientable surface $\os_{g-1, 2n}$ of genus $g-1$ and $2n$ marked points and a covering map $p$ defined as follows. The surface $\os_{g-1,2n}$ has an orientation reversing deck transformation $\iota: \os_{g-1, 2n} \to \os_{g-1,
  2n}$ of order 2. The covering map from $p:\os_{g-1,2n}\to \no_{g,n}$ is the quotient of $\os_{g-1,2n}$. If the genus and the
number of marked points is clear from the context, we will drop the subscripts and just use
$\no$ and $\os$ for the non-orientable surface and its orientation double cover respectively.

Every homeomorphism $\varphi: \no \to \no$, has a unique orientation preserving lift, that is a homeomorphism $\wt{\varphi}: \os \to \os$ with $p\wt{\varphi}=\varphi p$. %This is an easy exercise in covering space theory, but we'll give a proof here for completeness.
%\begin{prop}
 % For any homeomorphism $\varphi: \no \to \no$, there exists a unique orientation preserving lift
 % $\wt{\varphi}: \os \to \os$. If the non-orientable surface has marked points fixed by $\varphi$,
  %then the orientation preserving lift $\wt{\varphi}$ may not fix the marked points, but the lift
  %of $\varphi^2$ will fix the marked points.
%\end{prop}
%\begin{proof}
 % One can always lift a homeomorphism $\varphi: \no \to \no$ if $\varphi$ preserves the subgroup of
  %$\pi_1(\no)$ corresponding to the cover $\os$. This subgroup can be concretely described as the
  %subgroup generated by the two sided curves in $\no$, i.e. the curves whose tubular neighbourhoods
  %are cylinders, and not M\"obius strips. Such a subgroup is clearly preserved by any homeomorphism
  %$\varphi$, which means we always have a lift. There will be two choices for such a lift, since
  %$\os$ is a two-sheeted cover. These two lifts $\wt{\varphi}_1$ and $\wt{\varphi}_2$ are related
  %by the following identity.
  %\begin{align*}
%\wt{\varphi}_1 = \iota \circ \wt{\varphi}_2
 % \end{align*}
  %Since $\iota$ is orientation reversing, only one of $\wt{\varphi}_1$ or $\wt{\varphi}_2$ is
  %orientation preserving, which gives us a unique choice.
%
 % If $\no$ has marked points that $\varphi$ fixes, then the lift $\wt{\varphi}$ may or may not swap
 % the pre-images of the marked points. But the square of the lift will definitely fix the
 % pre-images as well, which proves the second part of the proposition.
%\end{proof}

A consequence is that lifting homeomorphisms induces a monomorphism between orientation preserving homeomorphisms of $\no$ and (orientation preserving) homeomorphisms of $\os$.  Every homotopy of $\no$ lifts to a homotopy of $\os$.  Moreover, if $f,g:\no\to\no$ are homeomomorphisms such that their orientation preserving lifts $\widetilde{f},\widetilde{g}$ of $\os$ are homotopic, then $f$ and $g$ are homotopic.  Therefore there is an inclusion from the mapping class group of $\no$ to the (orientation preserving) mapping class group of $\os$.  This inclusion also respects the Nielsen-Thurston classification of mapping classes, both qualitatively, and
quantitatively, as the following proposition shows.
\begin{prop}
  \label{prop:2}
  If $\varphi$ is a self-homeomorphism of $\no$ and $\wt{\varphi}$ is its orientation preserving lift on $\os$, then:
  \begin{enumerate}[(i)]
  \item $\varphi$ is periodic if and only if $\wt{\varphi}$ is periodic,
  \item $\varphi$ is reducible if and only if $\wt{\varphi}$,
  \item $\varphi$ is pseudo-Anosov if and only if $\wt{\varphi}$ is pseudo-Anosov.  Moreover if $\varphi$ has stretch factor $\lambda$, then $\wt{\varphi}$ also has stretch factor $\lambda$.
  \end{enumerate}
\end{prop}
\begin{proof}
  It's easy to see that if $\varphi$ is periodic, so it $\wt{\varphi}$, and the other way round. If $\varphi$
  is reducible, that means it leaves some multicurve $\gamma$ in $\no$ invariant, which means $\wt{\varphi}$ leaves
  the preimage of $\gamma$ invariant as well. Conversely, if $\wt{\varphi}$ leaves some multicurve
  $\wt\gamma$ invariant, so does $\iota \circ \wt{\varphi}$, since they commute. That means the union of $\wt\gamma$
  and $\iota(\wt\gamma)$ is also a multi-curve and thus descends to a multi-curve on $\no$ that is left invariant
  by $\varphi$. Since any mapping class of $\no$ that is neither periodic nor reducible must be pseudo-Anosov on $\no$ must lift to a pseudo-Anosov on $\no$
  and vice versa.

  Suppose now that $\varphi$ is a psuedo-Anosov on $\no$ with stretch factor $\lambda$ and expanding and contracting
  foliations $\mu_e$ and $\mu_c$ respectively. Since $\varphi$ is a pseudo-Anosov map, the following
  identity involving the intersection form $i$ holds for all closed curves $\gamma$ in $\no$.
  \begin{align}
    \label{eq:1}
    i(\varphi^{-1}\gamma, \mu_e) &= i(\gamma, \varphi(\mu_e)) \\
                               &= \lambda \cdot i(\gamma, \mu_e)
  \end{align}
  A similar identity holds for $\mu_c$.
  \begin{align}
    \label{eq:2}
    i(\varphi^{-1}\gamma, \mu_c) &= i(\gamma, \varphi(\mu_c)) \\
                               &= \frac{1}{\lambda} \cdot i(\gamma, \mu_c)
  \end{align}
  Note now that the foliations can be lifted to the double cover: call their lifts $\wt{\mu}_e$ and
  $\wt{\mu}_c$. For any closed curve $\wt{\gamma}$ of $\os$, consider its intersection number with the
  foliations. Observe that computing the intersection number is a local calculation. Start by picking an open
  cover $U$ on $\no$ such that all the open sets in $U$ are homeomorphic to the connected components of their
  pre-image in $\os$. By picking a partition of unity subordinate to this cover, one can compute the intersection
  number by restricting computation on each open set in the cover. This calculation lifts to the orientation double
  cover, giving us the following identity.
  \begin{align}
    \label{eq:3}
    i(\wt{\gamma}, \wt{\mu}_e) = i(\gamma, \mu_e)
  \end{align}
  Combining identities \eqref{eq:1} and \eqref{eq:3}, we get the following identity for intersection numbers
  on $\os$.
  \begin{align*}
    i(\wt{\varphi}^{-1} (\wt{\gamma}), \wt{\mu}_e) = \lambda \cdot i(\wt{\gamma}, \wt{\mu}_e)
  \end{align*}
  We get a similar expression for $\wt{\mu}_c$, which proves that $\wt{\varphi}$ has the same stretch factor
  as $\varphi$, thus proving the proposition.
\end{proof}
% Part (ii) of Proposition \ref{prop:2} is going to be useful in an application of our main result,
% where we'll be computing asymptotics for the minimal stretch factor of a pseudo-Anosov map on
% $\no_{g,n}$.

%Finally, the last thing we need to know about mapping classes on non-orientable surfaces is how to construct examples of pseudo-Anosov maps.
In the case of orientable surfaces, the Penner construction is used to construct pseudo-Anosov maps, as well compute their stretch factors. It turns out the Penner construction also works in the non-orientable setting, with some minor modifications. This construction is presented in detail in Section 2 of \cite{Strenner_2017}, but we give an outline of the key ideas below.

\p{The Penner construction} The Penner construction in the orientable setting starts with a pair of filling multicurves $A = \{a_1,\dots,a_n\}$
and $B = \{b_1,\dots,b_m\}$.  A Penner construction is a composition of positive Dehn twists around curves in $A$ and negative Dehn twists about curves in $B$ that uses a Dehn twist about each curve in $A\cup B$ at least once.  Penner proves that this construction is pseudo-Anosov \cite{penner1988construction}. The problem with making this work for
non-orientable surfaces is that when defining Dehn twists about curves on a non-orientable surface, there is not a well-defined notion of a left or right Dehn twist. For non-orientable surfaces we will use a set of filling two-sided curves that are \textit{marked inconsistently}.

Each two-sided curve $c$ on a non-orientable surface $N$ has a neighborhood homeomorphic to an
annulus $A$ by a homeomorphism $\phi: A \xrightarrow{} N$, called a \textit{marking}. In this
context, we can define the Dehn twist $T_{c,\phi}(x)$ around $(c,\phi)$ in the following manner.
\begin{align*}
  T_{c,\phi}(x) =
  \begin{cases}
    \phi \circ T \circ \phi^{-1}(x) & \text{for } x \in \phi(A) \\
    x & \text{for } x \in N - \phi(A)
  \end{cases}
\end{align*}
Here $T$ is the standard Dehn twist on $A$, i.e. $T(\theta,t) = (\theta + 2\pi t,t)$. If we fix an
orientation of $A$, then we can pushforward this orientation to $S$. We say two marked curves
$(c,\phi_c)$ and $(d,\phi_d)$ that intersect at a point $p$ are marked inconsistently if the
pushforward of the orientation of $A$ by $\phi_c$ and $\phi_d$ disagree in a neighborhood of $p$.
If all our curves are marked inconsistently and are filling, then once again a composition of Dehn
twists around them that use all the curves at least once will be pseudo-Anosov.

\p{Train tracks} The Penner construction not only promises that our map is pseudo-Anosov, but it also gives a way to
compute the stretch factor of our map (see \cite{penner1988construction}).  The proof of the fact
that the composition is pseudo-Anosov, and the computation of its stretch factor works the same is
in the orientable setting.  Let $\varphi$ be a pseudo-Anosov homeomorphism of $\no$.  A {\it train track} is an embedded graph in $\no$ such that for every vertex of valence three or greater, all adjacent edges have the same tangent vector.  An {\it invariant train track for $\varphi$} is a train track track $\tau$ such that $\varphi(\tau)$ is homotopic to $\tau$.  Let $\mathcal{C}$ be a collection of curves in $\no$. %Consider now the collection of transverse measures on our train track $\tau$.
For every curve $\gamma \in\mathcal{C}$, there is an associated transverse measure
$\mu_\gamma$ for $\tau$ that assigns $1$ to all edges lying in $\gamma$ and 0 to everything else. Let $V_\tau$
be the cone of transverse measures on $\tau$, and $H$ the subspace of $V_\tau$ spanned by the
transverse measure associated to curves in $\mathcal{C}$.
%\begin{align*}
 % H = \mathrm{span}(\{\mu_\gamma \mid \gamma \text{ is a connected curve in } \mathcal{C}\}).
%\end{align*}
The measures $\mu_\gamma$ are linearly independent and form the \textit{standard basis} for $H$. The subspace $H$ is invariant under the action of $\varphi$ on $V_\tau$, thus $\varphi$ has a linear action on $H$. If we let $A$
be the matrix representing this action in the standard basis, then the stretch factor of $\varphi$,
$\lambda(\varphi)$, is the Perron-Frobenius eigenvalue of $\varphi$.

Another tool that is used to study mapping classes is the associated \emph{mapping torus}, a $3$-manifold, constructed using a surface $\no$ and a mapping class $[\varphi]$.


In what follows, we will restrict our attention to mapping
tori of surfaces with a pseudo-Anosov homeomorphism.  %When the surface is orientable, we will call the resulting mapping torus orientable and when the surface is non-orientable, we will call the resulting mapping torus non-orientable.
The goal of this section is to prove the necessary results for mapping tori of non-orientable surfaces with a pseudo-Anosov map.% (these will be the $3$-manifolds we will be referring to when talking about orientable $3$-manifolds), and the mapping tori of non-orientable $3$-manifolds, again with a pseudo-Anosov map (these will be the manifolds we will be referring to when talking about non-orientable $3$-manifolds). While a lot of our statements will hold more generally for compact non-orientable $3$-manifolds, it will be easier to describe examples when working in this restricted setting; additionally, our application will only involve mapping tori of pseudo-Anosov maps.

\subsection{The problem with homology in non-orientable $3$-manifolds}
\label{sec:probl-with-homol}

A first attempt at defining the Thurston norm for a compact non-orientable $3$-manifold might be $M$ as follows.
Let $S$ be an embedded surface.  Define the complexity function $\chi_-$,
much like in the case of orientable $3$-manifolds, and then define the norm of a homology class
$a \in H^2(M; \ZZ)$ by minimizing $\chi_-(S')$ over all $S'$ representing $a$. This may work, but is quite
unsatisfying: this construction assigns zero norm to all embedded non-orientable surfaces, since their
fundamental classes are trivial, and thus map to $0$ in $H_2(M; \ZZ)$. But we would like the
incompressible surfaces in non-orientable $3$-manifolds to have a positive norm! There are plenty of
incompressible surfaces even in non-orientable $3$-manifolds, namely fibers of fibrations over $S^1$, and a useful definition of Thurston norm should find them.

It turns out that the fundamental problem with non-orientable $3$-manifolds is that homology is a
very coarse invariant: too coarse to detect embedded non-orientable surfaces. Our workaround will
be to deal with the first cohomology $H^1(M)$ rather than the second homology $H_2(M)$. By Poincar\'e duality they are the same in orientable 3-manifolds, but the same is not true for non-orientable $3$-manifolds.

To see why Poincar\'e duality for non-orientable 3-manifolds fails, consider an orientable $3$-manifold $M$. We can explicitly work out the map from
$H^1(M; \ZZ)$ to $H_2(M; \ZZ)$ given by Poincar\'e duality.
To do so, we set up a correspondence between elements of $H^1(M; \ZZ)$ and homotopy classes of maps from
$M$ to $S^1$ as follows. Given a cohomology class $[\alpha]$ in $H^1(M; \ZZ)$, choose a representative $1$-form $\alpha$,
and a basepoint $y_0$ in $M$. The associated map $f_{\alpha}$ is given by the following formula
\begin{align}\label{form:map}
  f_{\alpha}(y) \coloneqq  \int_{y_0}^y \alpha \mod \ZZ
\end{align}
Changing the basepoint or the representative $1$-form gives a different map to $S^1$ that is homotopic to the
original map (see Section 5.1 of \cite{calegari2007foliations} for the details). One can recover the $1$-form
$\alpha$ from the map $f_{\alpha}$ by pulling back the canonical volume form $d\theta$ on $S^1$ along $f_{\alpha}$.

Let $q\in S^1$ be a regular value and let $S = f_{\alpha}^{-1}(q)$ be a surface. To construct a homology class, we choose an orientation on $S$ by declaring
that the outwards pointing normal vectors on $S$ are assigned a positive value by the form
$\alpha$. Then $S$ inherits an orientation from the orientation on $M$, and we have defined a
fundamental class $[S]$. We claim that $[S]$ is the Poincar\'e dual to $\alpha$.
\begin{lem}
  Let $q$ and $q'$ be two regular values of the function $f_{\alpha}$ and let $S=f_\alpha^{-1}(q)$ and $S'=f_\alpha^{-1}(q')$. Then for any closed $2$-form $\omega$ on $M$,
  the following identity holds:
  \begin{align*}
    \int_{S} \omega = \int_{S'} \omega.
  \end{align*}
  Furthermore, the following identity also holds:
  \begin{align*}
    \int_S \omega = \int_M \alpha \wedge \omega.
  \end{align*}
  In particular, the homology class of $S$ is Poincar\'e dual to $\alpha$.
\end{lem}
\begin{proof}
  The first part of the lemma follows from the fact that $S$ and $S'$ are homologous,
  i.e. $f^{-1}_{\alpha}([q, q'])$ is a singular $3$-chain that has $S$ and $S'$ as boundaries. From Stokes'
  theorem, we get the following:
  \begin{align*}
    \int_{S - S'} \omega &= \int_{f_{\alpha}^{-1}([q, q'])} d\omega \\
                         &= 0.
  \end{align*}

  To prove the second claim, observe that we can break up the second integral as a product integral:
  \begin{align*}
    \int_M \alpha \wedge \omega &= \int_{S^1} \left(   \int_{f_{\alpha}^{-1}(\theta)} \omega \right) d\theta.
  \end{align*}
  The above equation is true because $\alpha$ is the pullback of $d\theta$ along the map $f_{\alpha}$. Observe
  that the inner integral only makes sense when $\theta$ is a regular value, but by Sard's theorem, almost
  every $\theta \in [0,1]$ is a regular value, so the right hand side is well-defined. By the first claim, the inner integral is a constant function, as we vary over the $\theta$ which are regular values of $f_{\alpha}$.
  Then the integral of $d\theta$ over $S^1$ is $1$, giving us the identity we want:
  \begin{align*}
    \int_M \alpha \wedge \omega = \int_S \omega.
  \end{align*}
\end{proof}
What we have here is an explicit formula for the Poincar\'e duality map. For orientable $3$-manifolds, this
is an isomorphism, and more specifically the following theorem is true.
\begin{thm}[Poincar\'e duality for orientable $3$-manifolds]
  Let $M$ be an orientable $3$-manifold, and let $S$ be an oriented embedded surface. Then there exists a $1$-form
  $\alpha$ and a regular value $q\in M$ such that $S = f_{\alpha}^{-1}(q)$.
\end{thm}

Note that the maps from the $1$-form to a homology class of an embedded surface still makes sense for a non-orientable $3$-manifold $M$. However in that case the map from $H^1(M; \ZZ)$ to $H_2(M; \ZZ)$ has a nontrivial kernel.

\p{Failure of Poincar\'e duality for non-orientable 3-manifolds}
  Let $\no$ be a non-orientable surface, and let $\varphi$ be any homeomorphism of $\no$. Let $M$ be the mapping torus
  of $(\no, \varphi)$. We then have a map $f: M \to S^1$ given by mapping to the base of the mapping torus.
  Pulling back the form $d\theta$ under $f$, we get a closed but not exact $1$-form $\alpha$ on $M$. Observe
  that $f_{\alpha} = f$, because of how we constructed $\alpha$. Furthermore $f_{\alpha}^{-1}(0)$ is $\no$ inside
  $M$. Thus, the ``Poincar\'e duality map'' for $M$ maps a non-trivial element $\alpha \in H^1(M; \ZZ)$ to the
  zero element $[\no] \in H_2(M; \ZZ)$. In particular, we end up losing information
  going from $H^1(M)$ to $H_2(M)$.


The above example also suggests an alternative strategy of defining the Thurston norm for non-orientable
$3$-manifolds: rather than working with $H_2(M; \RR)$, we can instead work with $H^1(M; \RR)$. We will also be interested in getting a partial
inverse for this map: given a non-orientable surface $\no$ inside a non-orientable
$3$-manifold $M$, we would like to understand if $M$ can be realized as the mapping torus of some homeomorphism of
$\no$.

\subsection{Thurston norm for non-orientable $3$-manifolds}
\label{sec:thurston-norm-non}

For this section, we'll use $M$ to denote a non-orientable $3$-manifold, and $\wt{M}$ to denote its
orientation double cover. We will denote by $\iota$ the orientation reversing deck transformation
of $\wt{M}$, and the covering map $\wt{M} \to M$ by $p$. If $M=M_\varphi$ is the mapping torus of the
non-orientable surface $\no$ and a pseudo-Anosov map $\varphi: \no \to \no$, then $\wt{M}$ is the
mapping torus of $(\os, \wt{\varphi})$, where $\os$ is the orientable double cover of $\no$, and
$\wt{\varphi}$ is the orientation preserving lift of $\varphi$.

Since we have already concluded that the first cohomology is the ``right'' space on which to define the
Thurston norm, we need to relate $H^1(M; \RR)$ and $H^1(\wt{M}; \RR)$.  In particular, we will pullback $H^1(M;\RR)$ to $H^1(\wt{M};\RR)$ under $p$.

\begin{lem}
  \label{lem:injective}
  The pullback $p^{\ast}:H^1(M;\RR)\rightarrow H^1(\wt{M};\RR)$ maps $H^1(M; \RR)$ bijectively to the $\iota^{\ast}$-invariant subspace of
  $H^1(\wt{M}; \RR)$.
\end{lem}
\begin{proof}
  For any $1$-form $\alpha$ on $M$, $p^{\ast}(\alpha)$ will be
  $\iota^{\ast}$-invariant. %This means that the image of $p^{\ast}$ lands inside the $\iota^{\ast}$-invariant subspace.
  To check that $p^\ast$ is injective, consider a $1$-form $\alpha$ on $M$ such that $p^{\ast}\alpha$
  is exact. We thus have a smooth function $g:\wt{M}\to\RR$ such that the following relation holds:
    \begin{align*}
        dg = p^{\ast} \alpha.
    \end{align*}
    But since $p^{\ast}\alpha$ is $\iota^{\ast}$-invariant, we must have $dg = \iota^{\ast} dg$,
    Because $\iota^\ast$ commutes with the exterior derivative, we have $dg = d(\iota^{\ast}g)$. That means $g$
    and $\iota^{\ast}g$ differ by a constant, but that constant must be $0$ since $\iota$ is finite
    order. Thus $g$ is $p^*$-equivariant and therefore descends to a function on $M$. Therefore $\alpha$ is exact, which
    proves injectivity of $p^{\ast}$.

    To show surjectivity, let $[\alpha]$ be an element in
    $H^1(\wt{M}; \RR)$ that is $\iota^{\ast}$-invariant and let $\alpha$ be a representative.  Define $\beta\in H^1(M;\RR)$ in local coordinates such that $\beta$ takes on the value of $\alpha-\frac{df}{2}$.  Then $\beta$ is well-defined because $\alpha-\frac{df}{2}$ is $\iota^\ast$ invariant.  Indeed, for some smooth function $g$:
    \begin{align*}
        \alpha - \iota^{\ast}(\alpha) = dg.
    \end{align*}
    Applying $\iota^\ast$ to both sides of the equality, we have $\iota^{\ast}dg = -dg$. Then $\alpha$ is the pullback of $\beta$ under %Then we have that the $1$-form $\alpha - \frac{dg}{2}$ is $\iota^{\ast}$-invariant, and thus in the image of
    $p^{\ast}$. This proves surjectivity, and the lemma.
\end{proof}

Lemma \ref{lem:injective} tells us that $H^1(M; \RR)$ is a subspace of $H^1(\wt{M}; \RR)$, so we define the Thurston norm on $H^1(M; \RR)$ by restricting the Thurston norm on the orientable 3-manifold $\widetilde{M}$ to the subspace $p^*(H^1(M;\RR))$ of $H^1(\widetilde{M};\RR)$.
%\begin{cor}
%The pullback $p^\ast:H^(M;\ZZ)\rightarrow H^1(\wt{M};\ZZ)$ maps bijectively to a finite-index subgroup of $\iota^\ast$-invariant subspace of $H^1(\wt{M};\ZZ)$.
%\end{cor}

\p{Thurston norm for non-orientable 3-manifolds}
Let $M$ be a non-orientable 3-manifold and $\wt{M}$ its orientation double cover.  Let $\wt{x}$ be the Thurston norm on $H^1(\wt{M};\RR)$ defined in Section \ref{sec:thurst-fiber-face}.
  Let $\alpha\in H^1(M;\RR).$
  The {\it Thurston norm on $H^1(M; \RR)$}, is the norm $x: H^1(M;\RR)\rightarrow \RR$ defined:
  \begin{align*}
    x(\alpha) \coloneqq \wt{x}(p^{\ast}\alpha).
  \end{align*}

We now extend properties of the Thurston norm for orientable manifolds to the Thurston norm on $H^1(M;\RR)$.
\begin{thm}
  The unit ball with respect to the dual Thurston norm on $\left( H^1(M; \RR) \right)^{\ast}$ is a polyhedron in $(H^1(M,\RR))^\ast$
  whose vertices are lattice points $\{\pm \beta_1, \ldots \pm \beta_k\}$. The unit ball $B_1$ with respect to
  Thurston norm is a polyhedron given by the following inequalities.
  \begin{align*}
    B_1 = \left\{ a\in H^1(M,\RR) \mid \left| \beta_i(a) \right| \leq 1 \text{ for $1\leq i \leq k$} \right\}
  \end{align*}
\end{thm}

\begin{proof}
  The proof is identical to the original proof of Thurston
  \cite[Theorem 2]{thurston1986norm}. Because the norm of an element of
  $H^1(M; \ZZ)$ is the Thurston norm of the corresponding element in $H^1(\wt{M}; \ZZ)$, the norm of any element in
  $H^1(M; \ZZ)$ is also an integer.  The linear algebra follows identically.
\end{proof}

Observe that the way we defined the Thurston norm for non-orientable $3$-manifolds is lacking in two
ways. First of all, in the orientable case, the Thurston norm is a norm on the second homology, and thus also
embedded surfaces. We have already seen how working with second homology does not quite work, which is why we defined the analogue of the Thurston norm for non-orientable surfaces on the
first cohomology. %We would still like to talk about the norm of an embedded surface though, even if the homology class of that surface may be trivial.  This is something we'll see in Section
In Section \ref{sec:invert-poincare}, we will develop a version of Poincar\'e duality for non-orientable 3-manifolds to better understand embedded non-orientable surfaces.

The second shortcoming of the definition of Thurston norms for non-orientable manifolds is that Lemma \ref{lem:injective} gives a bijection between $H^1(M;\RR)$ and a subspace of $H^1(\wt{M};\RR)$.  But the Thurston norm describes the relationship between fibrations an orientable manifold $\wt{M}$ over $S^1$ and {\it lattice points} $H^1(\wt{M}; \ZZ)$. % that when working with fibrations over $S^1$, the elements of $H^1(M; \ZZ)$ are the elements of interest, rather than the elements of $H^1(M; \RR)$. %Lemma \ref{lem:injective} tells us that elements of $H^1(M; \RR)$ are precisely the $\iota^{\ast}$-invariant elements of $H^1(\wt{M}; \RR)$.
However there are $\iota^{\ast}$-invariant
elements on $H^1(\wt{M}; \ZZ)$ that are not pullbacks of elements of $H^1(M; \ZZ)$.  So there is not a bijection between $H^1(M; \ZZ)$ and fibrations of $M$ over $S^1$.

\p{Failure of surjectivity}
  Let $\no$ be a non-orientable surface, $\os$ its orientation double cover. Let $\gamma$ be a one-sided curve
  on $\no$, i.e. a curve whose preimage $\wt{\gamma}$ in $\os$ has a single component. Let the $3$-manifolds $M$ be the mapping torus of $\no$ with some pseudo-Anosov $\varphi$ and $\wt{M}$ the mapping torus of $\os$ with the orientation preserving lift of $\varphi$.  We can then
  consider $\gamma$ and $\wt{\gamma}$ as curves in the $3$-manifolds $M$ and $\wt{M}$.

  Extend $\wt{\gamma}$ to a basis $\mathcal{B}$ of $H_1(\wt{M}; \ZZ)$. We can construct an element
  of $H^1(\wt{M}; \ZZ)$ by simply assigning integer values to the elements of $\mathcal{B}$. Define $\alpha\in H^1(\wt{M};\ZZ)$
  that assigns $0.5$ to $\wt{\gamma}$ and an integer value to every element of $\mathcal{B}$. Consider the
  cohomology class $\alpha + \iota^{\ast}\alpha$. Because $\wt{\gamma}$ is the pre-image of a curve of $M$, we have that $\iota \wt{\gamma} = \wt{\gamma}$. Therefore  $\alpha + \iota^{\ast}\alpha$ is an
  $\iota^{\ast}$-invariant element of $H^1(\wt{M}; \ZZ)$ that assigns $1$ to $\wt{\gamma}$. Such a cohomology
  class cannot be a pullback of a class on $M$ since the pullback of a cohomology class on $M$ would assign an
  even value to $\wt{\gamma}$.

What the above example does show is that for any $\alpha\in H^1(\wt{M}; \ZZ)$ that is  $\iota^{\ast}$-invariant, the class $2\alpha$ definitely is a pullback of class in $H^1(M; \ZZ)$.

\subsection{Inverting the Poincar\'e duality map for embedded surfaces}
\label{sec:invert-poincare}

For either an orientable or non-orientable 3-manifold $M$, given $\alpha\in H^1(M; \ZZ)$, we can construct a dual map $f_\alpha$.  The preimage of a regular value $q\in S^1$ $f_{\alpha}^{-1}(q)$, will be an embedded surface. %In the orientable setting, the homology class of this embedded surface is well-defined, independent both of the choice of representative $1$-form in its cohomology class and the choice of regular value. While the homology class is also well defined in the non-orientable setting, the homology class is trivial when the embedded surface is non-orientable.
We will invert this construction: given an embedded surface $S$, we want a
closed $1$-form $\alpha$ such that the surface $S$ comes from $\alpha$ in the manner described
above.

When $M$ is orientable, Poincar\'e duality determines a closed 1-form corresponding to any embedded surface.  In this section, we create an ad hoc version of Poincar\'e duality for non-orientable surfaces in Theorem \ref{thm:Poincare-duality}.
%associating embedded surfaces to $1$-forms.
However, we need a version of the orientability condition for embedded non-orientable surfaces that we call \emph{relative orientability}.

\p{Relative orientability}
  Let $M$ be a $3$-manifold, and $S$ an embedded surface in $M$. The surface $S$ is said to be {\it relatively
  oriented with respect to $M$} if there is a nowhere vanishing normal vector field on $S$. Two
  such normal vector fields are said to induce the same orientation if locally they induce the
  same orientation after picking a local frame for $S$. A surface $S$ is \emph{relatively oriented}
  if both $S$ and the choice of positive normal vector field are specified.

Note that relative orientability is a strictly weaker notion than orientability. If $S$ and $M$ are
orientable, then $S$ is relatively orientable with respect to $M$. But even if $M$ is
non-orientable, a non-orientable embedded surface $S$ may be relatively orientable with respect to $M$. For instance, let $S$ be
the fiber of a non-orientable mapping torus $M$.  The pre-image under the bundle map of a non-vanishing vector field on $S^1$ is a non-orientable vector field on $M$.
%It is not the case that every embedded surface in a non-orientable $3$-manifold is relatively orientable.

\p{A surface that is not relatively orientable in a 3-manifold}
  Let $S$ be the standard torus $\RR^2/\ZZ^2$, and let $\varphi$ map $(x,y)$ to $(-x, y)$. Then $\varphi$ is an
  orientation-reversing homeomorphism.  Therefore the mapping torus $M_\varphi$ is non-orientable. Consider a vertical line $\gamma$ in $S$ preserved by $\varphi$, i.e. the line
  $x = 0$. The image of $\gamma$ in $S$ under the suspension flow in $M$ is a subsurface of $M$,
  which we'll call $S'$. The normal direction to $S'$ when restricted to $S$ is $\frac{\partial}{\partial x}$. Because the suspension flow reverses the direction of $\gamma$, the
  the normal vector field cannot be continuously extended to all of $M$.
  This means that the surface $S'$ is not relatively orientable in $M$ (despite being orientable itself.)

However, if both $M$ and an embedded surface are non-orientable, the surface will be relatively orientable.
\begin{prop}
  \label{prop:relative-orientability}
  Let $M$ be a non-orientable $3$-manifold, and let $S$ be an embedded connected non-orientable surface in $M$.
  Then $S$ is relatively orientable with respect to $M$.
\end{prop}
\begin{proof}
  Let $\wt{M}$ be the orientation double cover of $M$, and $\wt{S}$ be the pre-image of $S$ under the double cover. The
  restriction of the orientation reversing deck transformation $\iota:\wt{M}\rightarrow\wt{M}$ to $\wt{S}$ is an orientation reversing homeomorphism of $\wt{S}$.
  Let $(v_1, v_2)$ be positively oriented local frame for the tangent space to $\wt{S}$. Let $n$ be an outward pointing normal vector to $\wt{S}$ so the local frame $(v_1, v_2, n)$ is positively oriented. Since $\iota$ reverses the orientation of both $\wt{S}$ and $\wt{M}$, $(\iota(v_1), \iota(v_2))$ and $(\iota(v_1), \iota(v_2), \iota(n))$ are both negatively oriented. Then $\iota(n)$ is outward pointing.
  %, since the quoti(nent $S$ is non-orientable.
  %That means $S$ leaves the outward pointing normal
  %direction from $\wt{S}$ invariant, and that descends to an outward pointing normal direction on $S$. This
  %shows that $S$ is relatively orientable with respect to $M$.
  Therefore the outward pointing normal direction on $\wt{S}$ descends to an outward pointing normal direction on $S$, and $S$ is relatively orientable in $M$.
\end{proof}

We care about relatively orientable surfaces because for these surfaces can be mapped to cohomology classes.
\begin{thm}[Poincar\'e duality for non-orientable $3$-manifolds]
  \label{thm:Poincare-duality}
  Let $M$ be a non-orientable $3$-manifold, and let $S$ be a relatively oriented embedded
  surface. Then there exists a cohomology class $[\alpha]$ in $H^1(M; \ZZ)$ and a regular value $q\in S^1$ such that for some
  representative $\alpha$, $S=f_{\alpha}^{-1}(q)$. Furthermore, $\alpha$ assigns positive values to the positively oriented normal vector
  field on $S$.
\end{thm}

The idea of the proof of this theorem is fairly straightforward. Starting with the embedded surface
$S$ in $M$, we look at the pre-image $\wt{S}$ in the orientation double cover $\wt{M}$. We show
that the Poincar\'e dual to $\wt{S}$ is $\iota^{\ast}$-invariant.
\begin{lem}
  \label{lem:PD1}
  Let $S$ be a relatively oriented embedded surface in $M$, and $\wt{S}$ its pre-image in
  $\wt{M}$. Then the Poincar\'e dual to $[\wt{S}]$ is $\iota^{\ast}$-invariant.
\end{lem}
\begin{proof}
  If $S$ is relatively oriented with respect to $M$, then the relative orientation lifts to a relative orientation of $\wt{S}$ with respect to $\wt{M}$. Since $\wt{S}$ and $\wt{M}$ are orientable, this defines an orientation on $\wt{S}$,
  and thus the homology class $[\wt{S}]$ is well defined.

  The deck transformation $\iota$ reverses the orientation on $\wt{S}$. Indeed, let $(v_1, v_2, v_3)$ be a local frame for some point in $\wt{S}$ such that
  $v_3$ is the outward pointing normal vector field. Since the outward pointing normal vector
  field descends to the quotient by the orientation reversing map $\iota$.  Therefore $\iota(v_3)$
  must also be outward pointing. Since $\iota$ reverses the
  orientation on $\wt{M}$ but preserves the direction of $\iota(v_3)$, $\iota$ must reversing the orientation on the pair
  $(v_1, v_2)$.\becca[inline]{Is orientation really the best term in reference to the pair?} In particular, that means $\iota$ reverses the orientation on $\wt{S}$.

  This means $[\wt{S}]$ is in the $-1$-eigenspace of the $\iota_{\ast}$ action on
  $H_2(\wt{M}; \RR)$. Let $\wt{\alpha}$ be the the Poincar\'e dual to $[\wt{S}]$. The 1-form $\wt{\alpha}$ is $\iota^{\ast}$-invariant. This follows from
  the following chain of equalities which hold for all closed $2$-forms $\omega$.  We use the fact that $\iota^2=id$ in the first and third equalities.
  \begin{align*}
    \int_{\iota_{\ast}\wt{S}} \omega &= \int_{\wt{S}} \iota^{\ast}\omega &&\text{(By a change of variables)} \\
                                     &= \int_{\wt{M}} \wt{\alpha} \wedge \iota^{\ast} \omega &&\text{(Poincar\'e duality)} \\
                                     &=\int_{\wt{M}} \iota^{\ast} \left( \iota^{\ast}\wt{\alpha} \wedge \omega \right) \\
    &= \int_{\wt{M}} - \left( \iota^{\ast} \wt{\alpha} \wedge \omega \right) &&\text{($\iota$ is orientation reversing)}
  \end{align*}
  On the other hand, the following equalities follow from the fact that
  $\iota_{\ast}[\wt{S}] = -[\wt{S}]$.
  \begin{align*}
    \int_{\iota_{\ast}\wt{S}} \omega &= - \int_{\wt{S}} \omega \\
                              &= - \int_{\wt{M}} \wt{\alpha} \wedge \omega
  \end{align*}
  Because $$\int_{\wt{M}}\wt{\alpha}\wedge\omega=\int_{\wt{M}}\iota^\ast\wt{\alpha}\wedge\omega$$ for all $\omega$, it follows that $\wt{\alpha}$ is
  $\iota^{\ast}$-invariant.
\end{proof}

We now have an $\iota^{\ast}$-invariant $1$-form $\wt{\alpha}$.  Then we construct the map $f_{\widetilde{\alpha}}:\wt{M}\rightarrow S^1$ such that for a regular value $p\in S^1$, the surface $\wt{S}=f_{\wt{\alpha}}^{-1}(p)$. The next claim we want to make is that the map $f_{\wt{\alpha}}: \wt{M} \to S^1$ factors through the quotient $M$.
\begin{lem}
  \label{lem:PD2}
Let $p:\wt{M}\rightarrow M$ be the orientation double cover. For all points $y \in \wt{M}$, $f_{\wt{\alpha}}(y) = f_{\wt{\alpha}}(\iota (y))$.
\end{lem}
\begin{proof}
  Recall that $f_{\wt{\alpha}}(y)$ is given by the following integral formula.
  \begin{align*}
    f_{\wt{\alpha}}(y) = \int_{x_0}^y \wt{\alpha} \mod \ZZ,
  \end{align*}
  where $x_0$ is a basepoint in $\wt{M}$. Since $f_{\wt{\alpha}}(y)$ is equal to
  $f_{\wt{\alpha}}(\iota(y))$ for all $y$, we have the following:
  \begin{align*}
    \left(  \int_{x_0}^y \wt{\alpha} - \int_{x_0}^{\iota(y)} \wt{\alpha} \right) \in \ZZ.
  \end{align*}
  By a change of variables, and using the $\iota^{\ast}$-invariance of $\wt{\alpha}$, the left hand
  side of the above condition can be transformed, giving us the following condition.
  \begin{align*}
    \label{cond:integer}
    \left( \int_{x_0}^{\iota(x_0)} \wt{\alpha} \right) \in \ZZ
  \end{align*}
  %In other words, we want the integral of $\wt{\alpha}$ along any curve $\gamma$ from $x_0$ to $\iota(x_0)$ to be an integer. Equivalently,
  Let $\gamma$ be a simple one-sided curve based at $p(x_0)$. The preimage $p^{-1}(\gamma)$ in $\widetilde{M}$ is a simple closed curve based at $x_0$, call it $\delta$.  It will suffice to show that the integral of
  $\wt{\alpha}$ along $\delta$ is an even integer.

  The parity of $\displaystyle \int_{\delta} \wt{\alpha}$ is precisely the parity
  of the intersection number of $\delta$ and $\wt{S}$.  But the intersection of $\delta$ and $\wt{S}$ is even because it is twice the intersection of $p(\delta)=\gamma$ and $p(\widetilde{S})=S$.   %Furthermore, both $\delta$ and $\wt{S}$ are lifts of a curve and surface from $M$. Which means the number of intersections they have in $\wt{M}$ is twice the number of intersections have in $M$. But the latter number must be an integer, and thus the former number must be an even integer, showing that condition \eqref{cond:integer} holds.
  In particular,
  this shows that the map $f_{\wt{\alpha}}$ factors through, proving the lemma.
\end{proof}
We now have everything we need to finish proving Theorem \ref{thm:Poincare-duality}.
\begin{proof}[Proof of Theorem \ref{thm:Poincare-duality}]
  Starting with a relatively oriented surface $S$ in $M$, we look at its pre-image $\wt{S}$ in
  $\wt{M}$ under the orientation double cover. The relative orientation of the preimage gives us the homology class $[\wt{S}]$, and
  we get a $1$-form $\wt{\alpha}$, which is Poincar\'e dual to the homology class of $\wt{S}$.
  More specifically, we have a map $f_{\wt{\alpha}}$ and a regular value $q\in S^1$ such that $\wt{S}=f^{-1}_{\wt{\alpha}}(q)$.
  By Lemma \ref{lem:PD1}, $\wt{\alpha}$ is $\iota^{\ast}$-invariant, and by Lemma \ref{lem:PD2}, the map $f_{\wt{\alpha}}$ factors through $M$ to a map $f_{\alpha}:M\to S^1$.  The map $f_\alpha$ has the property that $f_{\alpha}^{-1}(q) =
  S$. By pulling back $d\theta$ on $S^1$ under $f_\alpha$, we obtain the desired 1-form $\alpha$ in $H^1(M; \ZZ)$.
\end{proof}

\subsection{Oriented sums of surfaces}
\label{sec:orient-sums-surf}

%We now have a way of going from an embedded surface to an element of $H^1(M; \ZZ)$. To make this mapping even
%more useful, we'll describe a way of adding two surfaces via the operation of taking
The next step in studying embedded non-orientable surfaces will be to describing
\emph{oriented sums}.  The oriented sum of two surfaces embedded in a manifold $M$ indeed is additive in both the Euler characteristic and $H^1(M;\RR)$. This
operation is well-known in the case of orientable $3$-manifolds (along with orientable embedded
surfaces), but we will sketch out the relevant details.  We then extend the construction relatively orientable embedded surfaces. %The same construction works for relatively orientable surfaces; one just needs to verify consistency.

\p{Oriented sum for oriented manifolds}
Let $S$ and $S'$ be oriented embedded surfaces in an oriented manifold $M$. Assume that $S$ and $S'$ intersect
trasversally. Thus, $S \cap S'$ is a disjoint union of copies of $S^1$. For each component $S\cap S'$, take a tubular neighborhood that has cross section as in Figure \ref{fig:cross-section}.

%\autoref{fig:cross-section}.
\begin{figure}
  \centering
  \incfig[0.2]{cross-section}
  \caption{Cross section of intersection of $S$ and $S'$.}
  \label{fig:cross-section}
\end{figure}

We then perform a surgery on the leaves of $S$ and $S'$ so that the outward pointing normal vector fields match as in Figure \ref{fig:surgery}.% We have two possible
%choices: we could join the left $S$ leaf to either the top or the bottom $S'$ leaf. Since both $S$ and $S'$ are oriented submanifolds of $M$, there is an outward pointing normal vector field on $S$ and $S'$. Suppose the outward normal vector field on $S$ points upwards and the outward normal vector on $S'$ points to the right. In that case, we'd glue the left $S$ leaf to the bottom $S'$ leaf to maintain a consistent outward normal vector field. See \autoref{fig:surgery} to see how the choice affects orientability.
\begin{figure}
  \centering
  \incfig[0.3]{surgery}
  \caption{On the left, the normal vectors on $S$ and $S'$ are consistent. On the right, they aren't.}
  \label{fig:surgery}
\end{figure}

By performing this surgery at all the intersections, we get a new submanifold $S''$ (which may have
multiple components). This new submanifold $S''$ is the oriented sum of $S$ and $S'$. The operation
of taking oriented sums is additive on Euler characteristic, as well as the homology classes (and thus
the cohomology classes of their Poincar\'e duals).
\begin{align*}
  \chi(S'') &= \chi(S) + \chi(S') \\
  [S''] &= [S] + [S'] \\
\end{align*}

\p{Oriented sum for non-orientable manifolds}
Let $M$ be a non-orientable manifold and let $S$ and $S'$ be embedded surfaces in $M$ that are relatively oriented.
%Observe that in order to canonically choose the right leaves to join, all we need is a relative orientation for both $S$ and $S'$. %That suggests that the same construction ought to work.
%Like in the case of an orientable ambient manifold, at every transversal intersection, we perform surgery based on the outwards pointing normal vector field.
%We need to verify that this construction is consistent with the covering map: i.e. taking
%the oriented sum of $S$ and $S'$ is the same as taking the oriented sum of $\wt{S}$ and $\wt{S'}$
%and then taking the quotient by $\iota$.

%Let $\gamma$ be a component of $S\cap S'$ in $M$.  %, and let $\wt{\gamma}_1$ and $\wt{\gamma}_2$ be the distinct path lifts of $\gamma$ in $\wt{M}$ under the orientation double cover $p$.
We will define the oriented sum on $S$ and $S'$ as follows.  Let $p:\wt{M}\rightarrow M$ be the orientation double cover and let $\iota$ be the orientation reversing deck transformation of $\wt{M}$.  As above, let $\wt{S}=p^{-1}(S)$ and $\wt{S}'=p^{-1}(S')$, which are embedded oriented surfaces in $\wt{M}$.  The oriented sum of $S$ and $S'$ is the image under $p$ of the oriented sum of $\wt{S}$ and $\wt{S}'$ (as defined above for oriented surfaces in oriented manifolds).  We need to justify that this operation is well-defined.

As in the proof of Lemma \ref{lem:PD1}, $\iota$ preserves the relative orientation, and thus leaves the outward
normal vector fields on $\wt{S}$ and $\wt{S}'$ invariant. Therefore a leaf $\ell$ of $\wt{S}$ is surgered with a leaf of $\ell'$ of $\wt{S}'$ if and only if $\iota(\ell)$ and $\iota(\ell')$ are surgered.  Therefore surgery factors through $p$ and the oriented double sum is well-defined.  %The oriented sum of $S$ and %The oriented sum of embeddein $M$ that is consistent with the oriented sum on the orientation double cover.%We need to show that when we surger the leaves of $\wt{S}$ to a leaf of $\wt{S'}$ along $\wt{\gamma}_1$, .

\p{Example} Let $\gamma$ be a component of $S\cap S'$ and $\wt{\gamma}_1$ and $\wt{\gamma}_2$ be the path lifts of $\gamma$.  Consider
\autoref{fig:consistency}, which shows the outward point normal vectors to $\wt{S}$ and $\wt{S'}$,
which determine which leaves are glued together along $\wt{\gamma}_1$ and $\wt{\gamma}_2$.
\begin{figure}
  \centering
  \incfig[0.4]{consistency}
  \caption{Neighborhoods of $\wt{\gamma}_1$ and $\wt{\gamma}_2$, with the outward pointing normal vector field.}
  \label{fig:consistency}
\end{figure}

The normal vector field tells us that the left $\wt{S}$ leaf gets glued to the bottom $\wt{S'}$
leaf near $\wt{\gamma}_1$ and $\wt{\gamma}_2$. Since $\iota(\wt{\gamma}_1)=\wt{\gamma_2})$, the outward pointing normal vector fields point the same (relative) directions.  %Consider now the deck transformation $\iota$. %Note that $\iota$ is an orientation reversing self map for $\wt{M}$, $\wt{S}$ and $\wt{S'}$. We've


\p{Additivity} By the consistency of the oriented sum in $M$ and $\wt{M}$, it easily follows that the oriented sum
is additive in Euler characteristic, as well as in terms of Poincar\'e dual, since the Poincar\'e
dual was also defined by passing to the orientation double cover.

\section{Mapping classes with small stretch factors}
\label{sec:mapping-classes-with}

In this section, we provide a strategy to compute pseudo-Anosov homeomorphisms with small stretch factors.
%The key tools from \autoref{sec:thur-norm-non-orientable} are Theorems \ref{thm:strong-duality} and \ref{thm:classifying-fibrations}, and the operation of oriented sums for non-orientable surfaces.

\subsection{Mapping class groups of non-orientable surfaces}
\label{sec:backgr-mapp-class}
Let $\no$ be a non-orientable surface and let $\wt{\no}$ and the covering map $p:\wt{\no}\rightarrow \no$ be its orientation double covering space.
Every homeomorphism $\varphi: \no \to \no$, has a unique orientation-preserving lift $\wt{\varphi}: \wt{\no} \to \wt{\no}$.


A consequence is that lifting homeomorphisms induces a monomorphism between homeomorphisms of $\no$ and orientation-preserving homeomorphisms of $\wt{\no}$.
Every homotopy of $\no$ lifts to a homotopy of $\wt{\no}$.
%Further, if $f,g:\no\to\no$ are homeomorphisms such that their orientation preserving lifts are homotopic, then $f$ and $g$ are homotopic.
%\becca[inline]{Do we need this (the previous) sentence?  It does not follow naturally from the sentence before it and actually requires proof/citation.  But it is only required if we want the inclusion to be injective.}
%\sayantan[inline]{We don't really need this to be injective, but the fact is probably folklore: we could prove it if we needed to, but it would be more elementary covering space stuff, and wouldn't add anything useful.}
%\becca[inline]{That's what I thought (we could also just cite it, there's a paper that proves it somewhere out there).  But I think we should just remove that sentence.}
Therefore there is an inclusion from the mapping class group of $\no$ to the (orientation-preserving) mapping class group of $\wt{\no}$.
This inclusion also respects the Nielsen-Thurston classification of mapping classes, both qualitatively, and quantitatively, as the following proposition shows.
\begin{prop}
  \label{prop:2}
  Let $\varphi:\no\rightarrow\no$ be a homeomorphism and let $\wt{\varphi}:\wt{\no}\rightarrow\wt{\no}$ be the orientation-preserving lift of $\varphi$.  Then:
  \begin{enumerate}[(i)]
  \item $\varphi$ is periodic if and only if $\wt{\varphi}$ is periodic,
  \item $\varphi$ is reducible if and only if $\wt{\varphi}$ is reducible, and
  \item $\varphi$ is pseudo-Anosov if and only if $\wt{\varphi}$ is pseudo-Anosov.  Moreover if $\varphi$ has stretch factor $\lambda$, then $\wt{\varphi}$ also has stretch factor $\lambda$.
  \end{enumerate}
\end{prop}
\begin{proof}
 % It's easy to see that if $\varphi$ is periodic, so it $\wt{\varphi}$, and the other way round.
  %If $\varphi$ is reducible, that means it leaves some multicurve $\gamma$ in $\no$ invariant, which means $\wt{\varphi}$ leaves the pre-image of $\gamma$ invariant as well.
  %Conversely, if $\wt{\varphi}$ leaves some multicurve $\wt\gamma$ invariant, so does $\iota \circ \wt{\varphi}$, since they commute.\becca[inline]{Need more here.  I'll think about it (See Aramayona--Leininger--Souto ``Injections on mapping class groups": there exist (branched) covering spaces where pA lift to reducible.  This is not such a case, but why?)}
  %That means the union of $\wt\gamma$ and $\iota(\wt\gamma)$, where $\iota$ is the orientation reversing deck transformation, is also a multi-curve and thus descends to a multi-curve on $\no$ that is left invariant by $\varphi$.\becca[inline]{The image of the multicurve may be non-simple or trivial}
  %If $\varphi$ is neither periodic nor reducible, it must be pseudo-Anosov.  By exclusion, $\wt{\varphi}$ must also be pseudo-Anosov.
  The fact that the map from $\Mod(\no)$ to $\Mod(\wt{\no})$ is type-preserving follows from Aramayona--Leininger--Souto \cite[Lemma 10]{aramayona2009injections} (while the statement of the Lemma is for orientable surfaces, the argument, which we will skip, is identical for non-orientable surfaces).

  Suppose now that $\varphi:\no\rightarrow\no$ is a psuedo-Anosov homeomorphism with stretch factor $\lambda$ and stable and unstable foliations $\mathcal{F}_s$ and $\mathcal{F}_u$ respectively.
  Let $\wt{\mathcal{F}_s}$ and $\wt{\mathcal{F}_u}$ denote the lifts of the stable and unstable foliations to the orientation double cover.
  We need to show that the following identities hold for all simple closed curves $\gamma$ in $\wt{\no}$ (see \cite[Expos\'e 5]{FLP} for the definition of intersection number with measured foliations; the fact that these identities suffice follows from \cite[Lemma 9.15]{FLP}):
%  \sayantan[inline]{Provide citations for stretch factor claim and also intersection number with foliations definition.}
  \begin{align}
      \label{eq:unstable-foliation}
      i(\gamma, \wt{\varphi}(\wt{\mathcal{F}_u})) &= \lambda \cdot i(\gamma, \wt{\mathcal{F}_u}) \\
      \label{eq:stable-foliation}
      i(\gamma, \wt{\varphi}(\wt{\mathcal{F}_s})) &= \frac{1}{\lambda} \cdot i(\gamma, \wt{\mathcal{F}_s}).
  \end{align}

  To see that these identities hold, we partition $\gamma$ into short arcs $\{\gamma_i\}$ such that the restriction of the covering map $p$ to a neighbourhood of each arc is a homeomorphism.
  %The local homeomorphism lets us compute the intersection number for each arc $\gamma_i$ by instead computing the intersection number on the surface $\no$:
  Then we have:
  \begin{align}
  \label{eq:push1}
    i(\gamma_i, \wt{\mathcal{F}_u}) &= i(p(\gamma_i), \mathcal{F}_u) \\
  \label{eq:push2}
    i(\gamma_i, \wt{\varphi}(\wt{\mathcal{F}_u})) &= i(p(\gamma_i), \varphi(\mathcal{F}_u)).
  \end{align}
  Since we know that $\mathcal{F}_u$ is the unstable foliation for $\varphi$ with stretch factor $\lambda$, we can compute the ratio of the right hand side of \eqref{eq:push1} and \eqref{eq:push2}:
  \begin{align}
      \label{eq:ratio}
      i(p(\gamma_i), \varphi(\mathcal{F}_u)) = \lambda \cdot i(p(\gamma_i), \mathcal{F}_u).
  \end{align}
  Combining \eqref{eq:push1}, \eqref{eq:push2}, and \eqref{eq:ratio}, and summing over all $\gamma_i$ gives us that \eqref{eq:unstable-foliation} holds. A similar argument also verifiproves that \eqref{eq:stable-foliation} holds.
\end{proof}



\subsection{Constructing pseudo-Anosov maps on nearby surfaces using oriented sums}
\label{sec:constr-psuedo-anos}
%\becca[inline]{I rewrote the section.  The old version is in comments.}
The goal of this section is to prove that the stretch factor of any pseudo-Anosov homeomorphism provides an asymptotic upper bound for the minimum stretch factor.  We do this in Proposition \ref{prop:asymptotic}.


\begin{prop}\label{prop:asymptotic}
Let $\no_g$ be a non-orientable surface of genus $g$ and let $\varphi:\no_g\rightarrow \no_g$ be a pseudo-Anosov homeomorphism with stretch factor $\lambda$.  Let $N_\varphi$ be the mapping torus of $\no_g$ by $\varphi$.  Let $\no_{g'}$ be an incompressible surface embedded in $N_\varphi$ that is transverse to the supension flow associated to $\varphi$.  Then for all $k\in\ZZ^+$, there is a pseudo-Anosov homeomorphism of the oriented sum $\no_{g}+k\no_{g'}$ with stretch factor at most $\lambda$.
\end{prop}

Our strategy for proving Proposition \ref{prop:asymptotic} is to find fiber bundles of $N_{\varphi}$ over $S^1$ that have fiber $\no_g+k\no_{g'}$.  We then apply a special case of Thurston's hyperbolization theorem, which says that the mapping torus of an orientable surface $S$ by a homeomorphism $\varphi$ is hyperbolic if and only if $\varphi$ is pseudo-Anosov \cite[Theorem 0.1]{thurston_hyp}.  In particular, Thurston's theorem implies that if $M=M_\varphi$ fibers over $S^1$ in two ways, either both mondromies are pseudo-Anosov or neither monodromy is pseudo-Anosov.  Finally, we adapt theorems of Fried and Matsumoto (Theorem \ref{thm:fm}) and Agol--Leininger--Margalit (Theorem \ref{thm:alm}) to work for mapping tori with non-orientable fibers.


\medskip
We will repeatedly use the following two facts for orientable surfaces and 3-manifolds:
\begin{enumerate}
 % \label{thm:norm-minimizing}
 \item An orientable surface $S$ minimizes the Thurston norm in its homology class if and only if $S$ is incompressible.
\item  If an orientable 3-manifold $M$ fibers over $S^1$, then the fiber is incompressible.
\end{enumerate}


%As a consequence, if we can decompose the mapping torus of a pseudo-Anosov map as the mapping torus of some other homeomorphism, then the other homeomorphism must also be pseudo-Anosov.


%Let $\no$ be a non-orientable surface, and $\varphi$ be a pseudo-Anosov homeomorphism.
%Let $N_\varphi$ be the associated mapping torus and $f:N\rightarrow S^1$ the associated fibration.
%Suppose we can construct another relatively oriented surface $\no'$ inside $N$ such that $\no'$ is transverse to the suspension flow direction associated to $f$. Let $\alpha$ be the Poincar\'e dual of $\no$ and $\alpha'$ be the Poincar\'e dual of $\no'$.  By Theorem \ref{thm:classifying-fibrations} $\alpha$ lies in a cone $\mathcal{C}_\varphi \subset H^1(N_\varphi; \ZZ)$ associated to $\varphi$ with other $1$-forms also coming from fibrations.
%Furthermore, $\alpha'$ lies in the closure of $\mathcal{C}_\varphi$.
%All positive integer linear combinations of $\alpha$ and $\alpha'$ are lattice elements of $\mathcal{C}_\varphi$.
%Each linear combination of $\alpha$ and $\alpha'$ is Poincar\'e dual to an oriented sum of $\no$ and $\no'$ in $N$.  Under reasonably mild conditions on $\no$ and $\no'$, we can show that the fiber of $f$ is homeomorphic to the oriented sum $\no + \no'$ of $\no$ and $\no'$.

\begin{prop}
  \label{thm:oriented-sum}
  Let $\no'$ be an incompressible surface embedded in $N$ that is transverse to the suspension flow associated to $\varphi$.
  Let $\alpha$ be the Poincar\'e dual of $\no$ and $\alpha'$ the Poincar\'e dual of $\no'$.
  If the oriented sum of $\no$ and $\no'$ is connected, then $\no + \no'$ is homeomorphic to the fiber of the fibration given by $\alpha + \alpha'$.
\end{prop}
\begin{proof}
  Let $p:\wt{N}\rightarrow N$ be the orientation double cover of $N$.
  The surface $\no$ is incompressible because it is a fiber of $f$; therefore $p^{-1}(\no)$ is also incompressible.  Then the Thurston norm of $\no$ of $\alpha$ is $2\chi_-(\no)$.  Likewise, the Thurston norm of $\alpha'$ is $2\chi_-(\no')$.

By \autoref{thm:classifying-fibrations} (ii), $\alpha'$ lies in the same cone in $H^1(N;\ZZ)$ as $\alpha$.  The Thurston norm $x$ on $H^1(N;\ZZ)$ is linear function on that cone.
 Since the Thurston norm is also linear on oriented sums of $\no$ and $\no'$, we have:
  \begin{align*}
    x(\alpha + \alpha') &= x(\alpha) + x(\alpha') \\
                        &= 2\chi_-(\no) + 2\chi_-(\no') \\
                        &= 2\chi_-(\no + \no').
  \end{align*}
  %The last equality follows from the linearity of the oriented sum.
  Because $2\chi_-(\no+\no')$ achieves the Thurston norm of $\alpha+\alpha'$, the preimage $p^{-1}(\no+\no')$ achieves the Thurston norm of the pullback of $\alpha+\alpha'$ under $p$.  Therefore $p^{-1}(\no+\no')$ is incompressible.  Then $\no+\no'$ is also incompressible.


  By Theorem \ref{thm:classifying-fibrations} (i), we have that $\alpha + \alpha'$ corresponds to some other fibration $f'':N\rightarrow S^1$.
  By Theorem \ref{thm:strong-duality}, the fiber of $f''$ must be homeomorphic to $\no + \no'$.  %Since the Poincar\'e dual to $\no + \no'$ is $\alpha + \alpha'$, and $\no + \no'$ is incompressible.
\end{proof}

In the proof of Proposition \ref{prop:asymptotic}, we will use Proposition \ref{thm:oriented-sum} along with a theorem of Thurston to obtain a pseudo-Anosov homemorphism $\varphi_k$ of the surface of genus of genus $g+kg'$.  We the use Theorems \ref{thm:fm} and \ref{thm:alm} to obtain a upper bound on the stretch factor of $\varphi_k$.

%The next step is to show that the stretch factor of the pseudo-Anosov on the new surfaces is less than the stretch factor of the original surface.  We use the following two theorems that hold for orientable 3-manifolds.

\begin{thm}[Fried \cite{fried1982flow,fried1983transitive}, Matsumoto\cite{matsumoto1987topological}]
  \label{thm:fm}
  Let $M$ be an orientable hyperbolic $3$-manifold and let $\mathcal{K}$ be the union of cones in $H^1(M; \RR)$ whose lattice points correspond to fibrations over $S^1$.
  There exists a strictly convex function $h: \mathcal{K} \to \RR$ satisfying the following properties:
  \begin{enumerate}[(i)]
  \item For all $c > 0$ and $u \in \mathcal{K}$, $h(cu) =  \frac{1}{c}h(u)$,
  \item For every primitive lattice point $u \in \mathcal{K}$, $h(u) = \log(\lambda)$, where $\lambda$ is the
    stretch factor of the pseudo-Anosov map associated to this lattice point, and
  \item $h(u)$ goes to $\infty$ as $u$ approaches $\partial \mathcal{K}$.
  \end{enumerate}
\end{thm}

\begin{thm}[Agol-Leininger-Margalit]
  \label{thm:alm}
  Let $\mathcal{K}$ be a fibered cone for a mapping torus $M$ and let $\overline{\mathcal{K}}$ be its closure
  in $H^1(M;\RR)$. If $u \in \mathcal{K}$ and $v \in \overline{\mathcal{K}}$, then $h(u+v) < h(u)$.
\end{thm}

\begin{proof}[Proof of Proposition \ref{prop:asymptotic}]
The oriented sum $\mathcal{S}=\no_g+k\no_{g'}$ constructed in Proposition \ref{thm:oriented-sum} is a surface of genus $g+kg'$, and is $\mathcal{S}$ is homeomorphic to a fiber of $N_\varphi$ given by $\alpha+k\alpha'$.  Let $\varphi_{k}:\mathcal{S}\rightarrow\mathcal{S}$ be the monodromy of $N_\varphi$ over $\mathcal{S}$.  By Thurston's theorem, $\varphi_k$ is pseudo-Anosov.  We claim that $\varphi_k$ has stretch factor at most $\lambda$.


Let $p:\wt{N}\rightarrow N_\varphi$ be the orientation double cover of $N_\varphi$. Let $h\mid_{N}$ be the restriction of $h$ to the pullback  $p^\ast(H^1(N_\varphi; \RR))$ in $H^1(\wt{N}; \RR)$.
The restriction $h\mid_N$ satisfies all the properties of Theorems \ref{thm:fm} and \ref{thm:alm}.

 Let $\wt{\varphi}$ be the orientation preserving lift of $\varphi$ to $p^{-1}(\no)$.  Since $\wt{\alpha}$ is the pullback of $\alpha$, the $\wt{\varphi}$ is the pseudo-Anosov homeomorphism associated to $\wt{\alpha}$.  By Proposition \ref{prop:2}, the stretch factor of $\wt{\varphi}$ is $\lambda$.

Let $\mathcal{K}$ be the cone in $H^1(N_\varphi;\RR)$ containing $\alpha$.  Since $\no_{g'}$ is transverse to the suspension flow in the direction of $\varphi$, we have that $\alpha'$ is in the closure of $\mathcal{K}$ in $H^1(N;\RR)$.  Let $\wt{\alpha}$ be the pullback of $\alpha$ under $p$ and let $\wt{\alpha}'$ be the pullback of $\alpha'$ under $p$.  Then $h\mid_N(\wt{\alpha}+\wt{\alpha}')<h\mid_N(\wt{\alpha})$.  By Theorem \ref{thm:fm}, $h(\wt{\alpha})$ is equal to the stretch factor of the pseudo-Anosov homeomorphism associated to $\wt{\alpha}$.  Therefore we have $h\mid_N(\wt{\alpha}+\wt{\alpha}')<\log(\lambda)$. It follows that the stretch factor of $\varphi_k$ is less than $\lambda$.
\end{proof}

%
%In this section, we answer the following question:
%\begin{question}
%  If $\no_g$ has a pseudo-Anosov map with stretch factor $\lambda$, does there exist a pseudo-Anosov map on $\no_{g'}$ with stretch factor at most $\lambda$, for all $g' > g$?
%\end{question}
%\becca[inline]{This might just be a note to self: I don't understand what the difference between $\lambda$ and $\lambda'$ being controlled means.}
%\sayantan[inline]{\sout{This refers to getting an upper bound on $|\ln \lambda - \ln \lambda^{\prime}|$ as a function of $g$ and $g'$. I've changed the question to the more precise version.} You were right in pointing out that the theorem at the end of this section doesn't answer the earlier question; rather it answers the following question, which is weaker, but good enough for our purposes.}
%\becca[inline]{Thanks Sayantan.  Now that I understand, I think I don't like that we ask it as a question without providing the answer right away.  It's probably clearer just to say exactly what we are going to do in the section.}
%To do so, we first prove Theorem \ref{thm:oriented-sum}, which allows us to construct different surfaces and pseudo-Anosov maps on them which have the same mapping torus.
%
%We will repeatedly use the following two facts:
%\begin{enumerate}
% % \label{thm:norm-minimizing}
% \item A surface $S$ minimizes the Thurston norm in its homology class if and only if it is incompressible.
%\item  If $M$ fibers over $S^1$, then the fiber is incompressible.
%\end{enumerate}
%\becca[inline]{Is the point that these facts are only known for orientable surfaces and we have to finagle a little to make them work for non-orientable in Theorem 3.2?}
%\sayantan[inline]{Yes, that's right.}
%
%We also have the following theorem of Thurston relating hyperbolicity of mapping tori and pseudo-Anosov maps.
%\begin{thm}[Thurston's Hyperbolization Theorem]
 % If $M$ is the mapping torus of a surface $S$ and a homeomorphism $\varphi$, then $M$ is hyperbolic if and only if $\varphi$ is pseudo-Anosov.
%\end{thm}
%
%%As a consequence, if we can decompose the mapping torus of a pseudo-Anosov map as the mapping torus of some other homeomorphism, then the other homeomorphism must also be pseudo-Anosov.
%In particular, if a manifold is a mapping torus of two different homeomorphisms $\varphi$ and $\varphi'$, then $\varphi$ is pseudo-Anosov if and only if $\varphi'$ is pseudo-Anosov.
%
%Next, we characterize the possible fibers of a mapping torus.
%
%%Let $\no$ be a non-orientable surface, and $\varphi$ be a pseudo-Anosov homeomorphism.
%%Let $N_\varphi$ be the associated mapping torus and $f:N\rightarrow S^1$ the associated fibration.
%%Suppose we can construct another relatively oriented surface $\no'$ inside $N$ such that $\no'$ is transverse to the suspension flow direction associated to $f$. Let $\alpha$ be the Poincar\'e dual of $\no$ and $\alpha'$ be the Poincar\'e dual of $\no'$.  By Theorem \ref{thm:classifying-fibrations} $\alpha$ lies in a cone $\mathcal{C}_\varphi \subset H^1(N_\varphi; \ZZ)$ associated to $\varphi$ with other $1$-forms also coming from fibrations.
%%Furthermore, $\alpha'$ lies in the closure of $\mathcal{C}_\varphi$.
%%All positive integer linear combinations of $\alpha$ and $\alpha'$ are lattice elements of $\mathcal{C}_\varphi$.
%%Each linear combination of $\alpha$ and $\alpha'$ is Poincar\'e dual to an oriented sum of $\no$ and $\no'$ in $N$.  Under reasonably mild conditions on $\no$ and $\no'$, we can show that the fiber of $f$ is homeomorphic to the oriented sum $\no + \no'$ of $\no$ and $\no'$.
%
%\begin{thm}
 % \label{thm:oriented-sum}
 % Let $\no$ be a non-orientable surface and $\varphi$ a homeomorphism of $\no$.
 % Let $N_\varphi$ be the associated mapping torus and $f:N_\varphi\rightarrow S^1$ the fibration.
 % Let $\no'$ be an incompressible surface embedded in $N$ that is transverse to the suspension flow direction associated to $f$.
 % Let $\alpha$ be the Poincar\'e dual of $\no$ and $\alpha'$ the Poincar\'e dual of $\no'$.
 % If the oriented sum of $\no$ and $\no'$ is connected, then $\no + \no'$ is homeomorphic to the fiber of the fibration given by $\alpha + \alpha'$.
%\end{thm}
%\begin{proof}
 % Let $p:\wt{N}\rightarrow N$ be the orientation double cover of $N$.
 % The surface $\no$ is incompressible because it is a fiber of $f$; therefore its pre-image under $p$ is also incompressible.  Therefore the Thurston norm of $\no$ of $\alpha$ is $2\chi_-(\no)$.  Likewise, the Thurston norm of $\alpha'$ is $2\chi_-(\no')$.
%
% Both $\alpha$ and $\alpha'$ lie in a cone over a fibered face in $H^1*N;\ZZ)$.  Therefore the Thurston norm $x$ on $H^1(N;\ZZ)$ is linear function on that cone.
% Since the Thurston norm is linear on oriented sums of $\no$ and $\no'$, we have:
 % \begin{align*}
%    x(\alpha + \alpha') &= x(\alpha) + x(\alpha') \\
   %                     &= 2\chi_-(\no) + 2\chi_-(\no') \\
  %                      &= 2\chi_-(\no + \no').
 % \end{align*}
  %%The last equality follows from the linearity of the oriented sum.
 % Because $2\chi_(\no+\no')$ achieves the Thurston norm of $\alpha+\alpha'$, the preimage $p^{-1}(\no+\no')$ achieves the Thurston norm of the pullback of $\alpha+\alpha'$ under $p$.  Therefore $p^{-1}(\no+\no')$ is incompressible.  Then $\no+\no'$ is also incompressible.
 % \becca[inline]{The previous version of what was above was incomplete.  Perhaps this version is too complete (I think it's correct though).}

 % By Theorem \ref{thm:classifying-fibrations}, we have that $\alpha + \alpha'$ corresponds to some other fibration $f'':N\rightarrow S^1$.
 % By Theorem \ref{thm:strong-duality}, the fiber of $f''$ must be homeomorphic to $\no + \no'$.  %Since the Poincar\'e dual to $\no + \no'$ is $\alpha + \alpha'$, and $\no + \no'$ is incompressible.
%\end{proof}

%Theorem \ref{thm:oriented-sum} is the first step in answering the question posed at the beginning of this subsection.
%If $\no'$ in Theorem \ref{thm:oriented-sum} has Euler characteristic $-1$, then then the oriented sum $\no+k\no'$ has genus equal to the genus of $\no$ plus $k$.
%\becca[inline]{If $\no'$ has Euler characteristic -1, can't we realize all genera greater than 2?}
%The next step is to show that the stretch factor of the pseudo-Anosov on the new surfaces is lesser than the stretch factor of the original surface.

%That is done using the following two theorems, due to Fried--Matsumoto \cite{fried1982flow}, \cite{fried1983transitive}, and \cite{matsumoto1987topological} and Agol--Leininger--Margalit  \cite{agol6983pseudo} in the orientable case.

%\begin{thm}[Fried-Matsumoto]
 % \label{thm:fm}
 % Let $M$ be a hyperbolic $3$-manifold and let $\mathcal{K}$ be the union of cones in $H^1(M; \RR)$ whose lattice points correspond to fibrations over $S^1$.
 % There exists a strictly convex function $h: \mathcal{K} \to \RR$ satisfying the following properties.
 % \begin{enumerate}[(i)]
 % \item For all $c > 0$ and $u \in \mathcal{K}$, $h(cu) =  \frac{1}{c}h(u)$.
 % \item For every primitive lattice point $u \in \mathcal{K}$, $h(u) = \log(\lambda)$, where $\lambda$ is the
 %   stretch factor of the pseudo-Anosov map associated to this lattice point.
  %\item $h(u)$ goes to $\infty$ as $u$ approaches $\partial \mathcal{K}$.
 % \end{enumerate}
%\end{thm}
%
%\begin{thm}[Agol-Leininger-Margalit]
 % \label{thm:alm}
 % Let $\mathcal{K}$ be a fibered cone for a mapping torus $M$ and let $\overline{\mathcal{K}}$ be its closure in $H^1(M;\RR)$. If $u \in \mathcal{K}$ and $v \in \overline{\mathcal{K}}$, then $h(u+v) < h(u)$.
%\end{thm}
%
%\begin{proof}[Proof of Theorems \ref{thm:fm} and \ref{thm:alm} for non-orientable $M$]
%Consider the pullback of $H^1(M; \RR)$ to $H^1(\wt{M}; \RR)$, where $\wt{M}$ is the orientation double cover, and restrict the function $h$ to the image of the pullback.
%It is easy to verify that the restriction satisfies all the listed properties as well.
%\end{proof}
%
%
%This answers the question we posed at the beginning of this subsection, and we use the results of this section in Section \ref{sec:application} to construct pseudo-Anosov maps with small stretch factors.

\section{Minimal stretch factors for non-orientable surfaces with marked points}
\label{sec:application}

Recall that associated to every pseudo-Anosov homeomorphism there is a number, the \textit{dilatation} or \textit{stretch factor}, the amount that the stable and unstable foliations of the pseudo-Anosov change by. Given a surface $S$, it is natural to ask what we can say about the set of all possible stretch factors, i.e.
\begin{align*}
    \{\log(\lambda(f)) | f \in \text{Mod}(S) \text{ is pseudo-Anosov}\}
\end{align*}

We call this set the \textit{spectrum} of $S$. We define the spectrum in terms of the logarithm of the stretch factors, as this is equivalent to the \textit{topological entropy} of a pseudo-Anosov homeomorphism. Topological entropy is a natural measure of complexity that one can assign to any topological map. A first step at understanding this set of entropies associated to a surface is considering the following quantity
\begin{align*}
    l_{g,n} =\min\{\log(\lambda(f)) | f \in \text{Mod}(\mathcal{S}_{g,n}) \text{ is pseudo-Anosov}\}.
\end{align*}

The study of this minimal stretch factor $l_{g,n}$ was initiated by Penner in his work \cite{penner1991bounds}. In this work Penner studied the asymptotic behavior of minimal stretch factors of non punctured orientable surfaces, i.e. the behavior of $l_{g,0}$. He showed that there exists positive constants $A_1$ and $A_2$ such that for any $g \geq 2$
\begin{align*}
    \frac{A_1}{g} \leq l_{g,0} \leq \frac{A_2}{g}.
\end{align*}
Hence showing that asymptotically $l_{g,0}$ behaves like $\frac{1}{g}$ for $g \geq 2$. The work by Yazdi in \cite{yazdi2018pseudo} that we aim to generalize using our results on the Thurston norm for non-orientable manifolds is a generalization of these first steps by Thurston. Yazdi asks what the asymptotic behavior of $l_{g,n}$ is when we look at rays in the $(g,n)$ plane with $n$ being constant. To this extent he proves the following result.
\begin{thm}[Yazdi]
\label{thm:yazdi1}
For any fixed $n \in \mathbb{N}$, there are positive constants $B_1 = B_1(n)$ and $B_2 = B_2(n)$ such that for any $g \geq 2$
\begin{align*}
    \frac{B_1}{g} \leq l_{g,n} \leq \frac{B_2}{g}.
\end{align*}
\end{thm}

Yazdi's result is one of many recent results in studying the aymptotics of $l_{g,n}$ for different subsets of the $(g,n)$ plane. See the introductions of \cite{yazdi2018pseudo} and \cite{tsai2009asymptotic} for more examples of results of this form. Yazdi proves an additional result along these lines for a large subset of the $(g,n)$ plane, one containing balls of arbitrary large radii.

\begin{thm}[Yazdi]
    \label{thm:yazdi2}
    There exists positive constants $A$, $B$ and $C$ such that for any $n \geq 1$ and $g \geq Cn\log^2(n)$ we have
    \begin{align*}
        \frac{B}{g} \leq l_{g,n} \leq \frac{A}{g}
    \end{align*}

\end{thm}

Our goal is to show these results also hold for non-orientable surfaces, albeit with possibly different constants. That is, if we let $\mathcal{N}_{g,n}$ be the genus $g$ non-orientable surface with $n$ punctures and let
$$l'_{g,n} = \min\{\log(\lambda(f)) \, \vert \, f \in \text{Mod}(\mathcal{N}_{g,n})\ \text{ is pseudo-Anosov}\}$$
Then we prove the following results
\begin{thm}
\label{thm:stretch}
~\begin{enumerate}
    \item For any fixed $n \in \mathbb{N}$, there are positive constants $B'_1 = B'_1(n)$ and $B'_2 = B'_2(n)$ such that for any $g \geq 2$ $$\frac{B'_1}{g} \leq l'_{g,n} \leq \frac{B'_2}{g}$$.
    \item There exists positive constants $A'$, $B'$ and $C'$ such that for any $n \geq 1$ and $g \geq C'n\log^2(n)$ $$\frac{B'}{g} \leq l'_{g,n} \leq \frac{A'}{g}$$
\end{enumerate}
\end{thm}

Recall that Proposition \ref{prop:2} states a pseudo-Anosov on a non-orientable surface lifts to a pseudo-Anosov with the same stretch factor on the double orientation cover. Thus, overlooking a few details we will describe at the end of this section, Yazdi's lower bound ``lifts" to a lower bound for non-orientable surfaces. The work in Yazdi's paper is done in constructing a family of pseudo-Anosovs, one for each surface $S_{g,n}$, that have 'small' stretch factors. Our goal, and the primary part of the proof of the first part of Theorem \ref{thm:stretch}, will be to replicate Yazdi's construction on non-orientable surfaces, using our extension of the Thurston norm.

\subsection{The Yazdi Construction}

Yazdi provides a 5 step construction in order to produce a pseudo-Anosov map for every punctured orientable surface that has stretch factor bounded above by the desired quantity. We will reproduce each step here, giving our version of Yazdi's construction for non-orientable surfaces.

\begin{center}
\textbf{Step 1: Constructing the Surfaces}
\end{center}

The first step in the construction is defining a family of surfaces that exhibit a sort of rotational symmetry. Using this symmetry, if one shows that a power of some homeomorphism is pseudo-Ansov, then so was the original homeomorphism. Yazdi cites this insight as being due to Penner in his construction in \cite{penner1991bounds}.

Note that we will try to follow Yazdi's notation as close as we can, in order to make it clear to the reader how our construction replicates his.

We will define a family of surfaces $P_{n,k}$ in the way originally done by Yazdi. Let $T$ be an orientable surface of genus 5 with 3 boundary components $c,d$ and $e$. Orient $T$ and give $c,d$ and $e$ the induced orientations from this orientation. Now add two cross-caps to $T$ but keep the boundaries of $T$ oriented. Let $p$ (respectively $q$) be a puncture (respectively a marked point) on the boundary component $e$ of $T$, with oriented arcs $r$ and $s$ connecting them on $\partial T$. See Figure \ref{fig:buildingblock} for a picture of $T$.

\begin{figure}[]
    \centering
    \resizebox{.25\totalheight}{!}{\incfig{YazdiTypeSurface}}
    \caption{The surface $T$, the building block for our Yazdi construction}
    \label{fig:buildingblock}
\end{figure}

Let $T_{i,j}$ be copies of the surface $T$, where $i,j \in \mathbb{Z}$. We will use similar notations to refer to the boundary components of $T_{i,j}$. Define an infinite surface $S_\infty$ as the quotient
$$S_\infty \coloneqq \left( \bigcup T_{i,j} \right)/\sim,$$
where $i,j \in \mathbb{Z}$. The equivalence relation $\sim$ is defined as
\begin{align*}
    c_{i,j} \sim d_{i+1,j} \hspace{1em}, \hspace{1em} r_{i,j} \sim s_{i,j+1}
\end{align*}
and the gluing maps for the boundary components are by orientation-reversing homeomorphisms.

We have two natural shift maps $\overline{\rho_1},\overline{\rho_2}: T_\infty \xrightarrow[]{} T_\infty$ that act by
\begin{gather*}
    \overline{\rho_1} \text{ sends } T_{i,j} \text{ to } T_{i+1,j}, \\
    \overline{\rho_2} \text{ sends } T_{i,j} \text{ to } T_{i,j+1}.
\end{gather*}

Note that these maps commute. Define the surface $P_{n,k}$ as the quotient of the surface $T_\infty$ by the covering action of the group generated by $(\overline{\rho_1})^n$ and $(\overline{\rho_2})^k$. Therefore, $\overline{\rho_1}$ and $\overline{\rho_2}$ induce maps on the surface $P_{n,k}$, which we denote by $\rho_1$ and $\rho_2$.

A question that naturally arises is why did we choose the surface $T$ for our building block? It comes down to two main problems:
\begin{enumerate}
    \item The combinatorics of the curves make the associated matrix we get from the Penner construction satisfy the conditions of Lemma Y
    \item Having a curve $\gamma$ such that it and its image under our map form the boundary of an embedded $\mathcal{N}_3$ in our surface, which we will need in later steps (\textcolor{red}{Perhaps this was outlined above}).
\end{enumerate}

\begin{lem}
Define the sequence
\begin{align*}
    g_{n,k} &= (14k - 2)n + 2, k \geq 3, n \geq 1
\end{align*}
    The genus of $P_{n,k}$ is $g_{n,k}$.
\end{lem}
\begin{proof}
    Consider the subsurface $U \subset P_{n,k}$ defined as $$U = \left( \bigcup_{i =0}^{k-1} T_{0,i} \right)/\sim.$$ Then $U$ is a compact, non-orientable surface of genus $12k$ with $2k$ boundary components, and forms a fundamental domain for the covering action of $\overline{\rho_1}$ on $T_\infty$. We have $$\chi(U) = 2 - 12k - 2k = 2 - 14k.$$ Thus $$\chi(P_{n,k}) = n \cdot \chi(U) = -n(14k - 2),$$ since $P_{n,k}$ is formed by gluing $n$ copies of $U$ together along circle boundary components. Therefore $\chi(P_{n,k}) = 2 - g_{n,k} = -n(14k - 2) \implies g_{n,k} = n(14k - 2) + 2$.
\end{proof}

\begin{center}
\textbf{Step 2: Constructing the Maps}
\end{center}


Following Yazdi, we will now construct maps $f_{n,k}: P_{n,k} \xrightarrow[]{} P_{n,k}$ that are defined as a composition of specific Dehn twists followed by a finite order mapping class. The key insight is that a power of this map will be a composition of Dehn twists that satisfy the criteria to be a Penner construction and thus pseudo-Anosov. This is how we take advantage of the rotational symmetry given to us by how the $P_{n,k}$ are constructed.

\begin{figure}[h]
    \centering
    \resizebox{.35\totalheight}{!}{\incfig{CurvesOnSurface}}
    \caption{The curves that lie on $T_{0,0}$}
    \label{fig:curves}
\end{figure}

\caleb[inline]{I have the graphics fully drawn I just need to mess around with the labeling on the curves to make them readable}

Recall that for non-orientable surfaces, we don't initially have a well-defined notion of a positive or negative Dehn twist. In order to do a Penner construction, we need to ensure that the curves we are working with are marked inconsistently. Then do Dehn twists around the curves according to these markings. Note that our labeling of the curves already gives us an inconsistent marking though. For any alpha curve $\alpha_i$, we let the marking $\phi_{\alpha_i}$ be orientation preserving and for beta curves $\beta_j$ let $\phi_{\beta_j}$ be orientation reversing. Since alpha curves only intersect beta curves (and vice versa), we have an inconsistent marking at each point of intersection.

Let $\mathcal{B}$ be the union of all $\beta$ curves except $\beta_1$ in $T_{0,0} \cup T_{0,1} \cup T_{1,0}$ (see figures below). Let $\rho_1(\mathcal{B})$ be the image of $\mathcal{B}$ under $\rho_1$. Define $\phi_b$ as the composition of Dehn twists along all the curves in the set $\overline{\mathcal{B}} \coloneqq \mathcal{B} \cup \rho_1(\mathcal{B}) \cup \dots \cup \rho_1^{n-1}(\mathcal{B})$. Since the curves in $\overline{\mathcal{B}}$ are disjoint, Dehn twists along them commute. Therefore, it is not necessary to specify the order in which we compose these Dehn twists in $\phi_b$. Let $\mathcal{R}$ be the union of all $\alpha$ curves except $\alpha_1$ in $T_{0,0}$. Define $\mathcal{R}$ and $\phi_r$ in the exact same way.

Let $\alpha_1,\beta_1 \subset T_{0,0}$ be the curves in Figure Z. Let $\phi$ be the composition of Dehn twists along all the curves $\alpha_1, \rho_1(\alpha_1), \dots, \rho_1^{n-1}(\alpha_1)$ followed by Dehn twists along all the curves $\beta_1,\rho_1(\beta_1),\dots,\rho_1^{n-1}(\beta_1)$. Define
\begin{align*}
    f_{n,k} &\coloneqq \rho_2 \circ \phi \circ \phi_b \circ \phi_r
\end{align*}
We are using the same notation as Yazdi, so the composition is from right to left. It follows from the Penner construction that $(f_{n,k})^k$ is pseudo-Anosov. Hence $f_{n,k}$ itself is pseudo-Anosov and invariant train tracks $\tau^1_{n,k}$ for $f_{n,k}$ can be obtained from Penner's construction that we described above.

\begin{center}
\textbf{Step 3: The Mapping Torus}
\end{center}

We have now constructed an infinite family of non-orientable surfaces and pseudo-Anosov maps, but there's an alarming issue with it. Looking back at Lemma Z, the genera of this family of surfaces does not encapsulate every positive integer. We are going to use our extension of the Thurston norm to "fill in the gaps", by constructing fibers for fibrations of the mapping tori of the pseudo-Anosov maps we defined above.

Let $M_{n,k}$ be the mapping torus of $f_{n,k}$ .Likewise, let $\widetilde{M_{n,k}}$ denote the mapping tori of $\widetilde{f_{n,k}}$, where $\widetilde{f_{n,k}}$ is the orientation preserving lift of $f_{n,k}$ to the orientation double cover of $P_{n,k}$. Note that it follows that $\widetilde{M_{n,k}}$ is the orientation double cover of $M_{n,k}$.

Let $\mathcal{K}_{n,k}$ denote the fibered face of $H_2(\widetilde{M_{n,k}},\mathbb{R})$ corresponding to the map $\widetilde{f_{n,k}}$. We will show that $M_{n,k}$ contains a closed non-orientable surface of genus 3 that lifts to a closed orientable surface of genus 2 in $\widetilde{M_{n,k}}$ that is contained in the closure of $\mathcal{K}_{n,k}$.

\textcolor{red}{Will come back to this section later, first I need to go and do the oriented sum}

\begin{lem}
\label{lem:genus3}
There is a non-trivial homology classes $0 \neq [\widetilde{F_{n,k}}] \in H_2(\widetilde{M_{n,k}};\mathbb{Z})$ represented by orientable surfaces of genus two that is a lift of a non-orientable surfaces of genus three $F_{n,k}$ in $M_{n,k}$. Moreover, $\widetilde{F_{n,k}}$ is Thurston norm-minimizing and lie in the closure $\overline{\mathcal{K}_{n,k}}$.
\end{lem}
\begin{proof}

\end{proof}

\begin{lem}
Let $\iota: \widetilde{M_{n,k}} \xrightarrow[]{} \widetilde{M_{n,k}}$ denote the deck transformation that generates the deck group of the orientation double cover. Likewise, let $\iota_*: H_2(\widetilde{M_{n,k}};\mathbb{R}) \xrightarrow[]{} H_2(\widetilde{M_{n,k}};\mathbb{R})$ denote its action on second homology. Then $\iota_*([\wt{F_{n,k}}]) = -[\wt{F_{n,k}}]$.
\end{lem}
\begin{proof}

By the way we have defined $\wt{F_{n,k}}$ as a lift of an embedded subsurface in $M_{n,k}$, we know that $\iota$ sends $\wt{F_{n,k}}$ to itself. We want to see that $\iota$ is also orientation reversing when restricted to $\wt{F_{n,k}}$.

To begin, we first need to see what our surface $\wt{F_{n,k}}$ looks like embedded in $\wt{M_{n,k}}$. Recall the way that $F_{n,k}$ is defined as an embedded $\mathcal{N}_1$ with two boundary components, $\gamma$ and $f^k(\gamma) = \hat{\gamma}$ in one of the fibers of $M_{n,k}$ union the tubes formed by following $\gamma$ $k$ times around the suspension flow in $M_{n,k}$. Let's observe what happens to our embedded genus 1 with 2 boundary components in a single fiber after it is lifted to the orientation double cover. The orientation double cover of a genus 1 non-orientable surface is $S^2$, and we can see here that our embedded surface will lift to a sphere with four boundary components, one can see this by imagining two copies of our embedded subsurface being glued along their single cross-cap. For ease of notation, let's denote the two lifts of $\gamma$ and $\hat{\gamma}$ as $\gamma_0,\gamma_1$ and $\hat{\gamma}_0,\hat{\gamma}_1$ respectively. These curves form the boundary of the sphere with four boundary components that is sitting in our single fiber in $\wt{M_{n,k}}$.

Recall that $\wt{M_{n,k}}$ is not only the double orientation cover of $M_{n,k}$, but is also the mapping torus of $\wt{f_{n,k}}$. Looking at these maps, if we let $p$ denote the covering map for $\wt{P_{n,k}} \xrightarrow[]{} P_{n,k}$, then we know that $p \circ \wt{f_{n,k}} = f_{n,k} \circ p$. This tells us that $\wt{f_{n,k}}$ sends $\gamma_0$ to $\hat{\gamma_0}$ and $\gamma_1$ to $\hat{\gamma}_1$. Thus we can see that the tube traced out by following the suspension flow of $\gamma$ to $\hat{\gamma}$ gets lifted to tubes following the suspension flow of $\gamma_0$ to $\hat{\gamma_0)}$ and $\gamma_1$ to $\hat{\gamma_1}$. These tubes glued to our sphere with four boundary components give us our genus 2 surface in the cover.

We will now show that $\iota$ restricted to $\wt{F_{n,k}}$ is orientation reversing by showing that it is orientation reversing on the individual components, i.e. the sphere with boundary and the two tubes. First note that the boundary components of the of the sphere with boundary are also curves that lie in one of the fibers of $\wt{M_{n,k}}$. Suppose that we give an orientation to our fiber which induces orientations on our curves. Since $\iota$ is orientation reversing on the fiber, it must reverse the orientation of our curves, and thus reverses the orientations of the boundaries of our sphere when we restrict $\iota$. Thus $\iota$ must be orientation reversing on the whole of the sphere with boundary components. \textcolor{red}{I know you made a slightly different argument for tubes Sayantan, but we can't we just use the same exact arguement for the tubes since the tubes are bounded by these curves?}

Now that we know that $\iota$ is orientation reversing on $\wt{F_{n,k}}$, we know that $\iota_*: H_2(\wt{F_{n,k}}) \xrightarrow{} \wt{F^i_{n,k}}$ acts by sending the fundamental class $[\wt{F_{n,k}}]$ to its negative. We also know that $\wt{F_{n,k}}$ is viewed as a representative for an element of $H_2(\wt{M_{n,k}})$ by the image of $[\wt{F_{n,k}}] \in H_2(\wt{F_{n,k}})$ under the map on second homology induced by the inclusion $i: \wt{F_{n,k}} \xrightarrow[]{} \wt{M_{n,k}}$. Since $\iota$ can be restricted to $\wt{F_{n,k}}$, it is a map of the pair $(\wt{M_{n,k}},\wt{F_{n,k}})$ and thus by the naturality of the long exact sequence of a pair, $\iota_*$ and $i_*$ commute. This tells us that $\iota_*: H^2(\wt{M_{n,k}}) \xrightarrow[]{} H^2(\wt{M_{n,k}})$ acts by $\iota_*([\wt{F_{n,k}}]) = -[\wt{F_{n,k}}]$, giving us our desired result.
\end{proof}

\begin{center}
\textbf{Step 4: Bounding the Stretch Factor}
\end{center}

In \cite{yazdi2018pseudo}, Yazdi shows that the family of pseudo-Anosov maps that we have constructed all have the log of their stretch factor bounded above by a similar factor. In order to to do this, recall in Section \ref{sec:background} we saw that pseudo-Anosov maps give rise to matrices whose Perron-Frobenius eigenvalue is our stretch factor. So a way to find an upper bound of the stretch factor of the maps we have constructed is to bound the spectral radius of the associated matrices. The following lemma by Yazdi does just this for a specific class of matrices that our examples are based off of.

\begin{lem}[Yazdi]
\label{lem:spectral}
Let $A$ be a non-negative integral matrix, $\Gamma$ be the adjacency graph of $A$, and $V(\Gamma)$ the set of vertices of $\Gamma$. For each $v \in V(\Gamma)$, define $v^+$ to be the set of vertices $u$ such that there is an oriented edge from $v$ to $u$. Let $D$ and $k$ be fixed natural numbers. Assume the following conditions hold for $\Gamma$: \begin{enumerate}
    \item For each $v \in V(\Gamma)$ we have $\deg_{\text{out}}(v) \leq D$.
    \item There is a partition $V(\Gamma) = V_1 \cup \dots \cup V_k$ such that for each $v \in V_i$ we have $v^+ \subset V_{i+1}$, for any $1 \leq i \leq k$ except possibly when $i = 1$ or 3 (indices are mod $k$).
    \item For each $v \in V_1$, we have $v^+ \subset V_2 \cup V_3$.
    \item For each $v \in V_3$ we have $v^+ \subset V_3 \cup V_4$, and for $u \in V^+ \cap V_3$ we have $u^+ \subset V_4$.
    \item For all $3 < j \leq k$ and each $v \in V_j$, the set $v^+$ consists of a single element.
\end{enumerate}
\end{lem}

With this result in hand, we can now show that in the same way as Yazdi, the stretch factors for our main family of examples are all bounded above in the way we hope.

\begin{lem}
There exists a universal positive constants $C'$ and such that for every $n \geq 1$ and $k \geq 3$:
$$\log(\lambda_{n,k}) \leq C'\frac{n}{g_{n,k}}$$
\end{lem}

\begin{proof}
We have purposefully constructed our examples so our curves are in the same ``general form" as Yazdi's were and thus they will still satisfy the criteria of Lemma 1. Though we still want to explicitly show that this is the case.

First for consistency of notation, let
\begin{align*}
    \mathcal{A} \coloneqq \mathcal{B} \cup \mathcal{R} &\cup \{\alpha_1,\beta_1\}, \overline{\mathcal{A}} \coloneqq \mathcal{A} \cup \rho_1(\mathcal{A}) \cup \dots \cup \rho_1^{n-1}(\mathcal{A}) \\
    &\hat{\mathcal{A}} \coloneqq \overline{\mathcal{A}} \cup \rho_2(\overline{\mathcal{A}}) \cup \dots \cup \rho_2^{k-1}(\overline{\mathcal{A}}).
\end{align*}
Thus $\hat{\mathcal{A}}$ is all the curves on our surface we are Dehn twisting around to get $f_{n,k}$.

Recall from above that we stated we need to find the eigenvalue of the matrix that represents the action of $f_{n,k}$ on the subspace of the cone of transverse measures that is spanned by the measures assigning $1$ to single curves in $\hat{\mathcal{A}}$ and 0 to everything else. Let $A$ be said matrix and $\Gamma$ the adjacency graph of $A$. In order to bound the spectral radius of $A$, we need to show that $\Gamma$ satisfies the criteria of Lemma 1. To do this we first need to partition the vertices of $\gamma$, which is equivalent to a partition of the curves in $\hat{\mathcal{A}}$: $$\mathcal{A} = \bigcup_{i=1}^k \rho_2^{i-2}(\overline{\mathcal{A}}).$$ Then define $V_i$ for $1 \leq i \leq k$ as the vertices of $\Gamma$ corresponding to elements in $\rho_2^{i-2}(\overline{\mathcal{A}})$.

We can now check the conditions of Lemma 1, based on the combinatorics of the curves on our surface:
\begin{enumerate}
    \item
    \item As above, we now have a partition of our vertices where $V_i \coloneqq \rho_2^{i-2}(\overline{\mathcal{A}})$. So suppose that $v \in V_i$, $i \neq 1,3$, is a vertex that correpsonds to a curve $c \in \hat{\mathcal{A}}$. By the partitioning $c$ must be a curve in $\rho_2^{i-2}(\overline{\mathcal{A}})$, for $i \neq 1,3$. Note in order for all vertices of this form to have $v^+ \subset V_{i+1}$, we need to see that $f_{n,k}$
    \item
    \item
    \item All the curves corresponding to an element of $V_j$, $3 < j \leq k$ are disjoint from all the curves in $\overline{A}$. Thus $f_{n,k}$ just acts by rotation.
\end{enumerate}

Setting $\lambda = \lambda_{n,k}$, Lemma \ref{lem:spectral} implies that
\begin{gather*}
    \lambda^{k-1} = \rho(A)^{k-1} = \rho(A^{k-1}) \leq 4D^4 \implies (k-1)\cdot \log(\lambda) \leq \log(4D^4) \\
    \implies \frac{k}{2}\log(\lambda) \leq (k-1)\log(\lambda) \leq \log(4D^4)
\end{gather*}

On the other hand, we know $g_{n,k} = (14k - 2)n + 2 \leq 14kn$. Therefore
\begin{align*}
    \log(\lambda) \leq 2\log(4D^4)\cdot\frac{1}{k} \leq 2\log(4D^4)\cdot \frac{14n}{g_{n,k}} = C'\frac{n}{g_{n,k}}
\end{align*}
where $C' \coloneqq 28\log(4D^4)$.
\end{proof}

\begin{center}
\textbf{Step 5: Filling in the Gaps}
\end{center}

Recall that $\wt{P_{n,k}}$ is the double orientation cover of $P_{n,k}$ and also the fiber of $\wt{M_{n,k}}$. Like Yazdi, we are going to be considering the homology classes $[\wt{P^r_{n,k}}] \coloneqq [\wt{P_{n,k}}] + r[\wt{F_{n,k}}]$. Representatives for these homology classes can be found by taking the oriented sum.

\begin{lem}
The surfaces $\wt{P^r_{n,k}}$ are Thurston norm-minimizing, with genera equal to $\wt{g^r_{n,k}} \coloneqq g_{n,k} + r - 1$. As $r$ varies between $0$ and $14n$, the genera of $\wt{P^r_{n,k}}$ cover the range between $\wt{g_{n,k}}$ and $\wt{g_{n,k+1}}$. Moreover, $\wt{P^r_{n,k}}$ are fibers of fibrations of $\wt{M_{n,k}}$ with pseudo-Anosov monodromy that fixes $4n$ of the singularities of the invariant foliation and descend to fibrations of $M_{n,k}$.
\end{lem}

\begin{proof}
    %\textcolor{red}{For now I'm going to write this section as if the oriented sum section above didn't exist, because I'm not sure how I want to structure that section yet.}
    All statements of this lemma except for the last follow in the same way as Lemma 3.5 in \cite{yazdi2018pseudo}. We know that
    $$\chi(\wt{P^r_{n,k}}) = \chi(\wt{P_{n,k}}) + r\cdot\chi(\wt{F_{n,k}}) = (-2(g_{n,k} - 1) + 2)-2r = -2(g_{n,k} + r - 1) + 2.$$ This proves the identity for the genus of $\wt{P^r_{n,k}}.$ To see that $\wt{P^r_{n,k}}$ is Thurston norm-minimizing, note that $[\wt{P^r_{n,k}}] \subset \mathcal{K}_{n,k}$ due to $\wt{P_{n,k}} \subset \mathcal{K}_{n,k}$ and $\wt{F_{n,k}} \subset \overline{\mathcal{K}_{n,k}}$. Note that from Theorem 3 in \cite{thurston1986norm}, one can deduce that the fiber of a fibration of a 3-manifold is Thurston norm-minimizing in its second homology class. This along with Lemma \ref{lem:genus3} and the linearity of the Thurston norm on a fibered face tells us
    \begin{align*}
        x([\wt{P^r_{n,k}}]) = x([\wt{P_{n,k}}]) + rx([\wt{F_{n,k}}]) = \chi_-(\wt{P_{n,k}}) + 2r = 2(g_{n,k} + r - 1) - 2
    \end{align*}
    so $\wt{P^r_{n,k}}$ is also norm minimizing.

    Just as Yazdi says, the homology class $[\wt{P^r_{n,k}}] = [P_{n,k}] + r[F_{n,k}]$ is clearly integral. It is also primitive since there is a curve in $\wt{M_{n,k}}$ that intersects $P_{n,k}$ transversely and exactly once, while avoiding $F_{n,k}$. (?). Since $[P^r_{n,k}]$ is integral, primitive and lies in the fibered face $C_{n,k}$, by Theorem \ref{thm:Thur1} it is the fiber of a fibration of $\wt{M_{n,k}}$. Since the monodromy $\wt{f_{n,k}}$ is pseudo-Anosov, all monodromies that correspond to this face are pseudo-Anosov by Theorem \ref{thm:ThurHyp}, so in particular $\wt{f^r_{n,k}}$ is pseudo-Anosov.

    As we see in Yazdi, the singularities of the stable foliation of $f_{n,k}$ that are fixed are the $2n$ intersection points of the axis of $\rho_1$ with $P_{n,k}$. Thus $\wt{f_{n,k}}$ has $4n$ singularities of its stable foliation. Furthermore we have already seen that the surface $F_{n,k}$ can be isotoped to be transverse to the suspension flow and disjoint from the orbit of the $2n$ singularities of $f_{n,k}$, thus $\wt{F_{n,k}}$ can be isotoped to be disjoint from the orbit of these $4n$ singularities of $\wt{f_{n,k}}$. Hence the monodromy $\wt{f^r_{n,k}}$ still fixes the corresponding $4n$ singularities on $P^r_{n,k}$.

    This all suffices for the orientable case, but for the non-orientable case we need to also show that our fibrations of $\wt{M_{n,k}}$ defined by our maps $\wt{f^r_{n,k}}$ and surfaces $P^r_{n,k}$ descend to fibrations on $M_{n,k}$. Recall that Lemma X gave us the exact criteria for a fibration of $\wt{M_{n,k}}$ to descend to a fibration of $M_{n,k}$. It is precisely when:
    \begin{enumerate}
        \item The corresponding 1-form $\wt{\alpha}$ is the pullback of a 1-form on $M_{n,k}$, and
        \item $\int_{x_0}^{\iota(x_0)} \alpha \in \mathbb{Z}$ for some chosen basepoint $x_0 \in \wt{M_{n,k}}$.
    \end{enumerate}
    For our situation, this corresponding 1-form is exactly the Poincar\'e dual of $[\wt{P^r_{n,k}}]$, which we will denote $\wt{\alpha}$. Since $[\wt{P^r_{n,k}}] = [\wt{P_{n,k}}] + r\cdot[\wt{F_{n,k}}]$ and both of the latter homology classes come from lifting surfaces in $M$, we know that $\wt{\alpha}$ satisfies both of these conditions by Lemmas X and Y.
\end{proof}

We can now prove our version of Yazdi's Lemma 3.6:

\begin{lem}
\label{lem:bound}
There exists a constant $C > 0$ such that for every $n \geq 1$, $k \geq 3$, and $0 \leq r \leq 6n$ we have $$\log(\lambda^r_{n,k}) \leq C\frac{n}{\wt{g^r_{n,k}}}$$
\end{lem}
\begin{proof}
    Let $\mathcal{K} = \wt{\mathcal{K}_{n,k}}$ be our fibered faces and $h: \mathcal{K} \xrightarrow[]{} \mathbb{R}$ the function described in Theorem \ref{thm:alm}. Note that we have
    $$\wt{g^r_{n,k}} = \wt{g_{n,k}} + r \leq \wt{g_{n,k}} + 14n < 2\wt{g_{n,k}} < 2g_{n,k}$$
    Thus
    $$h([\wt{P^r_{n,k}}]) < h([\wt{P_{n,k}}]) \leq C'\frac{n}{g_{n,k}} \leq 2C'\frac{n}{\wt{g^r_{n,k}}} $$
\end{proof}

So now we have that our surfaces $\wt{P^r_{n,k}}$ are fibers of fibrations of $\wt{M_{n,k}}$ that descend to fibrations of $M_{n,k}$ and their monodromies are pseudo-Anosov with their stretch factors bounded. The pseudo-Anosov monodromies $\wt{f^r_{n,k}}$ descend to pseudo-Anosov monodromies on $M_{n,k}$ with the same stretch factor. Thus our upper bound of $2C'\frac{n}{\wt{g^{r}_{n,k}}}$ still holds, but we do need to make a slight modification. This bound is in terms of $\wt{g^r_{n,k}}$, the genus on the fiber in the double orientation cover, but the genus of our fiber downstairs will be one greater, thus $2C'\frac{n}{\wt{g^r_{n,k}}} = 2C'\frac{n}{g^r_{n,k} - 1} \leq 2C'\frac{n}{\frac{1}{2}g^r_{n,k}} = 4C'\frac{n}{g^r_{n,k}}$.

We can now think of $f^r_{n,k}$ as a map on a non-orientable surface of genus $g^r_{n,k}$. Note from above we know that $g^r_{n,k}$ covers all natural numbers between $g_{n,k}$ and $g_{n,k+1}$, thus this set of genera for all $r$ covers all natural numbers larger than $g_{n,3} = 40n + 2$. Recall that all of these surfaces will have $2n$ singularities, so we can either puncture $n$ or $n + 1$ to account for all possible number of punctures.

We can now give a proof of the first half of Theorem \ref{thm:stretch}:

\begin{manualtheorem}{\ref{thm:stretch}.1}
For any fixed $n \in \mathbb{N}$, there are positive constants $B_1' = B_1'(n)$ and $B_2' = B_2'(n)$ such that for any $g \geq 2$
$$\frac{B_1'}{g} \leq l'_{g,n} \leq \frac{B_2'}{g}$$
\end{manualtheorem}
\begin{proof}
    ~
    We begin by proving the upper bound. By Lemma \ref{lem:bound} and above, we have that there exists a number $C' > 0$ such that for $g \geq 40n + 2$, $l'_{g,n} \leq 4C'\frac{n}{g}$. So we take $B'_2(n)$ to be:
    $$B'_2(n) = \max\{4C'n, l'_{1,n}, 2l'_{2,n}, \dots, (40n + 1)l'_{40n+1,n}\}$$
    For the lower bound, first recall that the lift of any pseudo-Anosov on a non-orientable surface $N_{g,n}$ to the double cover $S_{g-1,2n}$ is still pseudo-Anosov with the same stretch factor. Thus any lower bound on $l_{g-1,2n}$ is a lower bound for $l'_{g,n}$. In \cite{penner1991bounds}, Penner proved that
    $$l_{g,n} \geq \frac{\log(2)}{12g - 12 + 4n}$$
    And thus $$l'_{g,n} \geq l_{g-1,2n} \geq \frac{\log(2)}{12g - 24 + 8n} > \frac{\log(2)}{12ng}$$
    And so like Yazdi, we can set $B_2'(n) = \frac{\log(2)}{12n}$
\end{proof}


\bibliographystyle{plain}
\bibliography{references}

\end{document}