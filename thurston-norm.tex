\section{Thurston norm for non-orientable $3$-manifolds}
\label{sec:thur-norm-non-orientable}

\subsection{Background on non-orientable surfaces and their mapping tori}
\label{sec:backgr-non-orient}

A genus $g$ non-orientable surface is the connect-sum of $g$ copies of
$\mathbb{RP}^2$, analogous to how a genus $g$ surface is the connect-sum of $g$ copies of a torus
$S^1 \times S^1$.  A common way of visualizing non-orientable surfaces is to think of them as orientable
surfaces with \emph{crosscaps} attached. We attach a crosscap to a surface $S$ by first deleting a small open disc $D\subset S$, and
identifying the boundary of that disc (on the surface) via the antipodal map. In pictures, this is often denoted by
an X inscribed in a circle, see \autoref{fig:buildingblock} for an example of a surface with two crosscaps
attached.  Let $\no_{g,n}$ be a non-orientable surface obtained by attaching $g$ crosscaps to $S^2$ and marking $n$ points in $S^2$.  The integer $g$ is referred to as the genus of $\no_{g,n}$.  The compact non-orientable surfaces are classified by the triple $(g,n,b)$ where $g$ is the genus, $n$ is the number of marked points and $b$ is the boundary.

\p{The orientation double cover} The orientation double cover of $\no_{g,n}$ is an orientable surface $\os_{g-1, 2n}$ of genus $g-1$ and $2n$ marked points and a covering map $p$ defined as follows. The surface $\os_{g-1,2n}$ has an orientation reversing deck transformation $\iota: \os_{g-1, 2n} \to \os_{g-1,
  2n}$ of order 2. The covering map from $p:\os_{g-1,2n}\to \no_{g,n}$ is the quotient of $\os_{g-1,2n}$. If the genus and the
number of marked points is clear from the context, we will drop the subscripts and just use
$\no$ and $\os$ for the non-orientable surface and its orientation double cover respectively.

Every homeomorphism $\varphi: \no \to \no$, has a unique orientation preserving lift, that is a homeomorphism $\wt{\varphi}: \os \to \os$ with $p\wt{\varphi}=\varphi p$. %This is an easy exercise in covering space theory, but we'll give a proof here for completeness.
%\begin{prop}
 % For any homeomorphism $\varphi: \no \to \no$, there exists a unique orientation preserving lift
 % $\wt{\varphi}: \os \to \os$. If the non-orientable surface has marked points fixed by $\varphi$,
  %then the orientation preserving lift $\wt{\varphi}$ may not fix the marked points, but the lift
  %of $\varphi^2$ will fix the marked points.
%\end{prop}
%\begin{proof}
 % One can always lift a homeomorphism $\varphi: \no \to \no$ if $\varphi$ preserves the subgroup of
  %$\pi_1(\no)$ corresponding to the cover $\os$. This subgroup can be concretely described as the
  %subgroup generated by the two sided curves in $\no$, i.e. the curves whose tubular neighbourhoods
  %are cylinders, and not M\"obius strips. Such a subgroup is clearly preserved by any homeomorphism
  %$\varphi$, which means we always have a lift. There will be two choices for such a lift, since
  %$\os$ is a two-sheeted cover. These two lifts $\wt{\varphi}_1$ and $\wt{\varphi}_2$ are related
  %by the following identity.
  %\begin{align*}
%\wt{\varphi}_1 = \iota \circ \wt{\varphi}_2
 % \end{align*}
  %Since $\iota$ is orientation reversing, only one of $\wt{\varphi}_1$ or $\wt{\varphi}_2$ is
  %orientation preserving, which gives us a unique choice.
%
 % If $\no$ has marked points that $\varphi$ fixes, then the lift $\wt{\varphi}$ may or may not swap
 % the pre-images of the marked points. But the square of the lift will definitely fix the
 % pre-images as well, which proves the second part of the proposition.
%\end{proof}

A consequence is that lifting homeomorphisms induces a monomorphism between orientation preserving homeomorphisms of $\no$ and (orientation preserving) homeomorphisms of $\os$.  Every homotopy of $\no$ lifts to a homotopy of $\os$.  Moreover, if $f,g:\no\to\no$ are homeomorphisms such that their orientation preserving lifts $\widetilde{f},\widetilde{g}$ of $\os$ are homotopic, then $f$ and $g$ are homotopic.  Therefore there is an inclusion from the mapping class group of $\no$ to the (orientation preserving) mapping class group of $\os$.  This inclusion also respects the Nielsen-Thurston classification of mapping classes, both qualitatively, and
quantitatively, as the following proposition shows.
\begin{prop}
  \label{prop:2}
  If $\varphi$ is a self-homeomorphism of $\no$ and $\wt{\varphi}$ is its orientation preserving lift on $\os$, then:
  \begin{enumerate}[(i)]
  \item $\varphi$ is periodic if and only if $\wt{\varphi}$ is periodic,
  \item $\varphi$ is reducible if and only if $\wt{\varphi}$,
  \item $\varphi$ is pseudo-Anosov if and only if $\wt{\varphi}$ is pseudo-Anosov.  Moreover if $\varphi$ has stretch factor $\lambda$, then $\wt{\varphi}$ also has stretch factor $\lambda$.
  \end{enumerate}
\end{prop}
\begin{proof}
  It's easy to see that if $\varphi$ is periodic, so it $\wt{\varphi}$, and the other way round. If $\varphi$
  is reducible, that means it leaves some multicurve $\gamma$ in $\no$ invariant, which means $\wt{\varphi}$ leaves
  the preimage of $\gamma$ invariant as well. Conversely, if $\wt{\varphi}$ leaves some multicurve
  $\wt\gamma$ invariant, so does $\iota \circ \wt{\varphi}$, since they commute. That means the union of $\wt\gamma$
  and $\iota(\wt\gamma)$ is also a multi-curve and thus descends to a multi-curve on $\no$ that is left invariant
  by $\varphi$. Since any mapping class of $\no$ that is neither periodic nor reducible must be pseudo-Anosov on $\no$ must lift to a pseudo-Anosov on $\no$
  and vice versa.

  Suppose now that $\varphi$ is a psuedo-Anosov on $\no$ with stretch factor $\lambda$ and expanding and contracting
  foliations $\mu_e$ and $\mu_c$ respectively. Since $\varphi$ is a pseudo-Anosov map, the following
  identity involving the intersection form $i$ holds for all closed curves $\gamma$ in $\no$.
  \begin{align}
    \label{eq:1}
    i(\varphi^{-1}\gamma, \mu_e) &= i(\gamma, \varphi(\mu_e)) \\
                               &= \lambda \cdot i(\gamma, \mu_e)
  \end{align}
  A similar identity holds for $\mu_c$.
  \begin{align}
    \label{eq:2}
    i(\varphi^{-1}\gamma, \mu_c) &= i(\gamma, \varphi(\mu_c)) \\
                               &= \frac{1}{\lambda} \cdot i(\gamma, \mu_c)
  \end{align}
  Note now that the foliations can be lifted to the double cover: call their lifts $\wt{\mu}_e$ and
  $\wt{\mu}_c$. For any closed curve $\wt{\gamma}$ of $\os$, consider its intersection number with the
  foliations. Observe that computing the intersection number is a local calculation. Start by picking an open
  cover $U$ on $\no$ such that all the open sets in $U$ are homeomorphic to the connected components of their
  pre-image in $\os$. By picking a partition of unity subordinate to this cover, one can compute the intersection
  number by restricting computation on each open set in the cover. This calculation lifts to the orientation double
  cover, giving us the following identity.
  \begin{align}
    \label{eq:3}
    i(\wt{\gamma}, \wt{\mu}_e) = i(\gamma, \mu_e)
  \end{align}
  Combining identities \eqref{eq:1} and \eqref{eq:3}, we get the following identity for intersection numbers
  on $\os$.
  \begin{align*}
    i(\wt{\varphi}^{-1} (\wt{\gamma}), \wt{\mu}_e) = \lambda \cdot i(\wt{\gamma}, \wt{\mu}_e)
  \end{align*}
  We get a similar expression for $\wt{\mu}_c$, which proves that $\wt{\varphi}$ has the same stretch factor
  as $\varphi$, thus proving the proposition.
\end{proof}
% Part (ii) of Proposition \ref{prop:2} is going to be useful in an application of our main result,
% where we'll be computing asymptotics for the minimal stretch factor of a pseudo-Anosov map on
% $\no_{g,n}$.

%Finally, the last thing we need to know about mapping classes on non-orientable surfaces is how to construct examples of pseudo-Anosov maps.
In the case of orientable surfaces, the Penner construction is used to construct pseudo-Anosov maps, as well compute their stretch factors. It turns out the Penner construction also works in the non-orientable setting, with some minor modifications. This construction is presented in detail in Section 2 of \cite{Strenner_2017}, but we give an outline of the key ideas below.

\p{The Penner construction} The Penner construction in the orientable setting starts with a pair of filling multicurves $A = \{a_1,\dots,a_n\}$
and $B = \{b_1,\dots,b_m\}$.  A Penner construction is a composition of positive Dehn twists around curves in $A$ and negative Dehn twists about curves in $B$ that uses a Dehn twist about each curve in $A\cup B$ at least once.  Penner proves that this construction is pseudo-Anosov \cite{penner1988construction}. The problem with making this work for
non-orientable surfaces is that when defining Dehn twists about curves on a non-orientable surface, there is not a well-defined notion of a left or right Dehn twist. For non-orientable surfaces we will use a set of filling two-sided curves that are \textit{marked inconsistently}.

Each two-sided curve $c$ on a non-orientable surface $N$ has a neighborhood homeomorphic to an
annulus $A$ by a homeomorphism $\phi: A \xrightarrow{} N$, called a \textit{marking}. In this
context, we can define the Dehn twist $T_{c,\phi}(x)$ around $(c,\phi)$ in the following manner.
\begin{align*}
  T_{c,\phi}(x) =
  \begin{cases}
    \phi \circ T \circ \phi^{-1}(x) & \text{for } x \in \phi(A) \\
    x & \text{for } x \in N - \phi(A)
  \end{cases}
\end{align*}
Here $T$ is the standard Dehn twist on $A$, i.e. $T(\theta,t) = (\theta + 2\pi t,t)$. If we fix an
orientation of $A$, then we can pushforward this orientation to $S$. We say two marked curves
$(c,\phi_c)$ and $(d,\phi_d)$ that intersect at a point $p$ are marked inconsistently if the
pushforward of the orientation of $A$ by $\phi_c$ and $\phi_d$ disagree in a neighborhood of $p$.
If all our curves are marked inconsistently and are filling, then once again a composition of Dehn
twists around them that use all the curves at least once will be pseudo-Anosov.

\p{Train tracks} The Penner construction not only promises that our map is pseudo-Anosov, but it also gives a way to
compute the stretch factor of our map (see \cite{penner1988construction}).  The proof of the fact
that the composition is pseudo-Anosov, and the computation of its stretch factor works the same is
in the orientable setting.  Let $\varphi$ be a pseudo-Anosov homeomorphism of $\no$.  A {\it train track} is an embedded graph in $\no$ such that for every vertex of valence three or greater, all adjacent edges have the same tangent vector.  An {\it invariant train track for $\varphi$} is a train track track $\tau$ such that $\varphi(\tau)$ is homotopic to $\tau$.  Let $\mathcal{C}$ be a collection of curves in $\no$. %Consider now the collection of transverse measures on our train track $\tau$.
For every curve $\gamma \in\mathcal{C}$, there is an associated transverse measure
$\mu_\gamma$ for $\tau$ that assigns $1$ to all edges lying in $\gamma$ and 0 to everything else. Let $V_\tau$
be the cone of transverse measures on $\tau$, and $H$ the subspace of $V_\tau$ spanned by the
transverse measure associated to curves in $\mathcal{C}$.
%\begin{align*}
 % H = \mathrm{span}(\{\mu_\gamma \mid \gamma \text{ is a connected curve in } \mathcal{C}\}).
%\end{align*}
The measures $\mu_\gamma$ are linearly independent and form the \textit{standard basis} for $H$. The subspace $H$ is invariant under the action of $\varphi$ on $V_\tau$, thus $\varphi$ has a linear action on $H$. If we let $A$
be the matrix representing this action in the standard basis, then the stretch factor of $\varphi$,
$\lambda(\varphi)$, is the Perron-Frobenius eigenvalue of $\varphi$.

Another tool that is used to study mapping classes is the associated \emph{mapping torus}, a $3$-manifold, constructed using a surface $\no$ and a mapping class $[\varphi]$.
Given a surface $S$ and a
homeomorphism $\varphi: S \to S$, one can construct a $3$-manifold $M_\varphi$ via the
\emph{mapping torus} construction.
\begin{align*}
  M_\varphi \coloneqq \frac{S \times [0,1]}{(x,1) \sim (\varphi(x), 0)}
\end{align*}
%Inverting this construction is also a problem of interest: given a $3$-manifold $M$, is it the mapping torus of some surface and self-homeomorphism $(S,\varphi)$? In how many ways can one express a $3$-manifold as a mapping torus?
Mapping tori are surface bundles over $S^1$, or \emph{fibrations over $S^1$}, denoted
$S\rightarrow M\rightarrow S^1$. A fibration defines a {flow} on $M$, called the
\emph{suspension flow}, where for any $x_0\in S$ and $t_0\in S^1$ the pair $(x_0,t_0)$ is sent
to $(x_0,t_0+t)$.  The \emph{fiber} of a fibration is the pre-image in $M$ of any point in
$S^1$. The mapping torus for any non-orientable surface and any mapping class will always be a
non-orientable $3$-manifold, but that is not the only way to get a non-orientable mapping
torus. The mapping torus of an orientable surface, and a non-orientable mapping class is also
a non-orientable $3$-manifold. To distinguish between the two kinds of non-orientable mapping
tori, we consider the fiber in the double cover. Suppose $M$ is a non-orientable mapping
torus, $f: M \to S^1$ is the fibration, and $\wt{M}$ is the orientation double cover of $M$.
Consider the fiber of $f$, i.e. $f^{-1}(\theta)$ for some $\theta \in S^1$. In the first case,
it will be a non-orientable surface, and in the second case, it will be an orientable surface.
We can then take the pre-image of the fiber under the covering map $p: \wt{M} \to M$. In the
first case, the pre-image of the fiber will be a single connected component, consisting of the
orientation double cover of the fiber, and in the second case, it will be two connected components, consisting of disjoint copies of the fiber. {\color{red} Turn this into a lemma?}

In what follows, we will restrict our attention to mapping tori of surfaces and pseudo-Anosov
mapping classes. The goal of this section is to transport some of the background machinery, namely the theory of Thurston norm to non-orientable surfaces, and in \autoref{sec:fibered-faces} use those results to construct pseudo-Anosov mapping classes with
small stretch factors on non-orientable surfaces. We recall the context and definition of the
Thurston norm for orientable $3$-manifolds.

\paragraph{Thurston norm on orientable $3$-manifolds}
Given a orientable closed $3$-manifold $M$, the Thurston norm is a semi-norm on its second
homology group $H_2(M; \RR)$: to define the norm, we need to make some preliminary remarks and
define a function.  Let $S$ be a connected surface. Define the complexity of $S$ to be
$\chi_-(S) = \max\{-\chi(S),0\}$. If a surface $S$ has multiple components $S_1,\ldots,S_m$
then $\chi_-(S)=\displaystyle\sum_{i=1}^m\chi_-(S_i)$.  Furthermore, for an \emph{orientable}
3-manifold $M$, it is known that elements in $H_2(M ; \ZZ)$ can be represented by embedded
surfaces inside of $M$. This lets us define a norm function $x$ for every homology class $a$
in $H_2(M; \ZZ)$:
\begin{align*}
  x(a) = \min\{\chi_-(S) \mid [S] = a \text{ and $S$ is compact, properly embedded and oriented}\}.
\end{align*}

We then linearly extend $x$ to the rational points.  There is then a is a unique continuous extention of $x$ to $\RR$, that is: a $\RR$-valued function on $H_2(M; \RR)$ called the {\it Thurston norm}. The unit ball for the Thurston norm is a convex polyhedron, and thus it makes
sense to talk about the \emph{faces} of the unit ball.

{\color{red} Mention incompressible surfaces here.}

\subsection{The problem with homology in non-orientable $3$-manifolds}
\label{sec:probl-with-homol}

A first attempt at defining the Thurston norm for a compact non-orientable $3$-manifold might
go as follows.  Let $S$ be an embedded surface in a non-orientable $3$-manifold $M$.  Define
the complexity function $\chi_-$, much like in the case of orientable $3$-manifolds, and then
define the norm of a homology class $a \in H^2(M; \ZZ)$ by minimizing $\chi_-(S')$ over all
$S'$ representing $a$. This is quite unsatisfying: the construction assigns
zero norm to all embedded non-orientable surfaces, since their fundamental classes are
trivial, and thus map to $0$ in $H_2(M; \ZZ)$. But we would like the incompressible surfaces
in non-orientable $3$-manifolds to have a positive norm. There are plenty of incompressible
surfaces even in non-orientable $3$-manifolds, namely fibers of fibrations over $S^1$, and a
useful definition of Thurston norm should find them.

It turns out that the fundamental problem with non-orientable $3$-manifolds is that homology is a
very coarse invariant: too coarse to detect embedded non-orientable surfaces. Our workaround will
be to deal with the first cohomology $H^1(M)$ rather than the second homology $H_2(M)$. By Poincar\'e duality they are the same in orientable 3-manifolds, but the same is not true for non-orientable $3$-manifolds.

To see why Poincar\'e duality for non-orientable 3-manifolds fails, consider an orientable $3$-manifold $M$. We can explicitly work out the map from
$H^1(M; \ZZ)$ to $H_2(M; \ZZ)$ given by Poincar\'e duality.
To do so, we set up a correspondence between elements of $H^1(M; \ZZ)$ and homotopy classes of maps from
$M$ to $S^1$ as follows. Given a cohomology class $[\alpha]$ in $H^1(M; \ZZ)$, choose a representative $1$-form $\alpha$,
and a basepoint $y_0$ in $M$. The associated map $f_{\alpha}$ is given by the following formula
\begin{align}\label{form:map}
  f_{\alpha}(y) \coloneqq  \int_{y_0}^y \alpha \mod \ZZ
\end{align}
Changing the basepoint or the representative $1$-form gives a different map to $S^1$ that is homotopic to the
original map (see Section 5.1 of \cite{calegari2007foliations} for the details). One can recover the $1$-form
$\alpha$ from the map $f_{\alpha}$ by pulling back the canonical volume form $d\theta$ on $S^1$ along $f_{\alpha}$.

Let $q\in S^1$ be a regular value and let $S = f_{\alpha}^{-1}(q)$ be a surface. To construct a homology class, we choose an orientation on $S$ by declaring
that the outwards pointing normal vectors on $S$ are assigned a positive value by the form
$\alpha$. Then $S$ inherits an orientation from the orientation on $M$, and we have defined a
fundamental class $[S]$. We claim that $[S]$ is the Poincar\'e dual to $\alpha$.
\begin{lem}
  Let $q$ and $q'$ be two regular values of the function $f_{\alpha}$ and let $S=f_\alpha^{-1}(q)$ and $S'=f_\alpha^{-1}(q')$. Then for any closed $2$-form $\omega$ on $M$,
  the following identity holds:
  \begin{align*}
    \int_{S} \omega = \int_{S'} \omega.
  \end{align*}
  Furthermore, the following identity also holds:
  \begin{align*}
    \int_S \omega = \int_M \alpha \wedge \omega.
  \end{align*}
  In particular, the homology class of $S$ is Poincar\'e dual to $\alpha$.
\end{lem}
\begin{proof}
  The first part of the lemma follows from the fact that $S$ and $S'$ are homologous,
  i.e. $f^{-1}_{\alpha}([q, q'])$ is a singular $3$-chain that has $S$ and $S'$ as boundaries. From Stokes'
  theorem, we get the following:
  \begin{align*}
    \int_{S - S'} \omega &= \int_{f_{\alpha}^{-1}([q, q'])} d\omega \\
                         &= 0.
  \end{align*}

  To prove the second claim, observe that we can break up the second integral as a product integral:
  \begin{align*}
    \int_M \alpha \wedge \omega &= \int_{S^1} \left(   \int_{f_{\alpha}^{-1}(\theta)} \omega \right) d\theta.
  \end{align*}
  The above equation is true because $\alpha$ is the pullback of $d\theta$ along the map $f_{\alpha}$. Observe
  that the inner integral only makes sense when $\theta$ is a regular value, but by Sard's theorem, almost
  every $\theta \in [0,1]$ is a regular value, so the right hand side is well-defined. By the first claim, the inner integral is a constant function, as we vary over the $\theta$ which are regular values of $f_{\alpha}$.
  Then the integral of $d\theta$ over $S^1$ is $1$, giving us the identity we want:
  \begin{align*}
    \int_M \alpha \wedge \omega = \int_S \omega.
  \end{align*}
\end{proof}
What we have here is an explicit formula for the Poincar\'e duality map. For orientable $3$-manifolds, this
is an isomorphism, and more specifically the following theorem is true.
\begin{thm}[Poincar\'e duality for orientable $3$-manifolds]
  Let $M$ be an orientable $3$-manifold, and let $S$ be an oriented embedded surface. Then there exists a $1$-form
  $\alpha$ and a regular value $q\in M$ such that $S = f_{\alpha}^{-1}(q)$.
\end{thm}

Note that the maps from the $1$-form to a homology class of an embedded surface still makes sense for a non-orientable $3$-manifold $M$. However in that case the map from $H^1(M; \ZZ)$ to $H_2(M; \ZZ)$ has a nontrivial kernel.

\p{Failure of Poincar\'e duality for non-orientable 3-manifolds}
  Let $\no$ be a non-orientable surface, and let $\varphi$ be any homeomorphism of $\no$. Let $M$ be the mapping torus
  of $(\no, \varphi)$. We then have a map $f: M \to S^1$ given by mapping to the base of the mapping torus.
  Pulling back the form $d\theta$ under $f$, we get a closed but not exact $1$-form $\alpha$ on $M$. Observe
  that $f_{\alpha} = f$, because of how we constructed $\alpha$. Furthermore $f_{\alpha}^{-1}(0)$ is $\no$ inside
  $M$. Thus, the ``Poincar\'e duality map'' for $M$ maps a non-trivial element $\alpha \in H^1(M; \ZZ)$ to the
  zero element $[\no] \in H_2(M; \ZZ)$. In particular, we end up losing information
  going from $H^1(M)$ to $H_2(M)$.


The above example also suggests an alternative strategy of defining the Thurston norm for non-orientable
$3$-manifolds: rather than working with $H_2(M; \RR)$, we can instead work with $H^1(M; \RR)$. We will also be interested in getting a partial
inverse for this map: given a non-orientable surface $\no$ inside a non-orientable
$3$-manifold $M$, we would like to understand if $M$ can be realized as the mapping torus of some homeomorphism of
$\no$.

\subsection{Thurston norm for non-orientable $3$-manifolds}
\label{sec:thurston-norm-non}

For this section, we'll use $M$ to denote a non-orientable $3$-manifold, and $\wt{M}$ to denote its
orientation double cover. We will denote by $\iota$ the orientation reversing deck transformation
of $\wt{M}$, and the covering map $\wt{M} \to M$ by $p$. If $M=M_\varphi$ is the mapping torus of the
non-orientable surface $\no$ and a pseudo-Anosov map $\varphi: \no \to \no$, then $\wt{M}$ is the
mapping torus of $(\os, \wt{\varphi})$, where $\os$ is the orientable double cover of $\no$, and
$\wt{\varphi}$ is the orientation preserving lift of $\varphi$.

Since we have already concluded that the first cohomology is the right space on which to define the
Thurston norm, we need to relate $H^1(M; \RR)$ and $H^1(\wt{M}; \RR)$.  In particular, we will pullback $H^1(M;\RR)$ to $H^1(\wt{M};\RR)$ under $p$.

\begin{lem}
  \label{lem:injective}
  The pullback $p^{\ast}:H^1(M;\RR)\rightarrow H^1(\wt{M};\RR)$ maps $H^1(M; \RR)$ bijectively to the $\iota^{\ast}$-invariant subspace of
  $H^1(\wt{M}; \RR)$.
\end{lem}
\begin{proof}
  For any $1$-form $\alpha$ on $M$, $p^{\ast}(\alpha)$ will be
  $\iota^{\ast}$-invariant. %This means that the image of $p^{\ast}$ lands inside the $\iota^{\ast}$-invariant subspace.
  To check that $p^\ast$ is injective, consider a $1$-form $\alpha$ on $M$ such that $p^{\ast}\alpha$
  is exact. We thus have a smooth function $g:\wt{M}\to\RR$ such that the following relation holds:
    \begin{align*}
        dg = p^{\ast} \alpha.
    \end{align*}
    But since $p^{\ast}\alpha$ is $\iota^{\ast}$-invariant, we must have $dg = \iota^{\ast} dg$,
    Because $\iota^\ast$ commutes with the exterior derivative, we have $dg = d(\iota^{\ast}g)$. That means $g$
    and $\iota^{\ast}g$ differ by a constant, but that constant must be $0$ since $\iota$ is finite
    order. Thus $g$ is $p^*$-equivariant and therefore descends to a function on $M$. Therefore $\alpha$ is exact, which
    proves injectivity of $p^{\ast}$.

    To show surjectivity, let $[\alpha]$ be an element in
    $H^1(\wt{M}; \RR)$ that is $\iota^{\ast}$-invariant and let $\alpha$ be a representative.  Define $\beta\in H^1(M;\RR)$ in local coordinates such that $\beta$ takes on the value of $\alpha-\frac{df}{2}$.  Then $\beta$ is well-defined because $\alpha-\frac{df}{2}$ is $\iota^\ast$ invariant.  Indeed, for some smooth function $g$:
    \begin{align*}
        \alpha - \iota^{\ast}(\alpha) = dg.
    \end{align*}
    Applying $\iota^\ast$ to both sides of the equality, we have $\iota^{\ast}dg = -dg$. Then $\alpha$ is the pullback of $\beta$ under %Then we have that the $1$-form $\alpha - \frac{dg}{2}$ is $\iota^{\ast}$-invariant, and thus in the image of
    $p^{\ast}$. This proves surjectivity, and the lemma.
\end{proof}

Lemma \ref{lem:injective} tells us that $H^1(M; \RR)$ is a subspace of $H^1(\wt{M}; \RR)$, so we define the Thurston norm on $H^1(M; \RR)$ by restricting the Thurston norm on the orientable 3-manifold $\widetilde{M}$ to the subspace $p^*(H^1(M;\RR))$ of $H^1(\widetilde{M};\RR)$.
%\begin{cor}
%The pullback $p^\ast:H^(M;\ZZ)\rightarrow H^1(\wt{M};\ZZ)$ maps bijectively to a finite-index subgroup of $\iota^\ast$-invariant subspace of $H^1(\wt{M};\ZZ)$.
%\end{cor}

\p{Thurston norm for non-orientable 3-manifolds}
Let $M$ be a non-orientable 3-manifold and $\wt{M}$ its orientation double cover.  Let $\wt{x}$ be the Thurston norm on $H^1(\wt{M};\RR)$ defined in Section \ref{sec:thurst-fiber-face}.
  Let $\alpha\in H^1(M;\RR).$
  The {\it Thurston norm on $H^1(M; \RR)$}, is the norm $x: H^1(M;\RR)\rightarrow \RR$ defined:
  \begin{align*}
    x(\alpha) \coloneqq \wt{x}(p^{\ast}\alpha).
  \end{align*}

We now extend properties of the Thurston norm for orientable manifolds to the Thurston norm on $H^1(M;\RR)$.
\begin{thm}
  The unit ball with respect to the dual Thurston norm on $\left( H^1(M; \RR) \right)^{\ast}$ is a polyhedron in $(H^1(M,\RR))^\ast$
  whose vertices are lattice points $\{\pm \beta_1, \ldots \pm \beta_k\}$. The unit ball $B_1$ with respect to
  Thurston norm is a polyhedron given by the following inequalities.
  \begin{align*}
    B_1 = \left\{ a\in H^1(M,\RR) \mid \left| \beta_i(a) \right| \leq 1 \text{ for $1\leq i \leq k$} \right\}
  \end{align*}
\end{thm}

\begin{proof}
  The proof is identical to the original proof of Thurston
  \cite[Theorem 2]{thurston1986norm}. Because the norm of an element of
  $H^1(M; \ZZ)$ is the Thurston norm of the corresponding element in $H^1(\wt{M}; \ZZ)$, the norm of any element in
  $H^1(M; \ZZ)$ is also an integer.  The linear algebra follows identically.
\end{proof}

Observe that the way we defined the Thurston norm for non-orientable $3$-manifolds is lacking in two
ways. First of all, in the orientable case, the Thurston norm is a norm on the second homology, and thus also
embedded surfaces. We have already seen how working with second homology does not quite work, which is why we defined the analogue of the Thurston norm for non-orientable surfaces on the
first cohomology. %We would still like to talk about the norm of an embedded surface though, even if the homology class of that surface may be trivial.  This is something we'll see in Section
In Section \ref{sec:invert-poincare}, we will develop a version of Poincar\'e duality for non-orientable 3-manifolds to better understand embedded non-orientable surfaces.

The second shortcoming of the definition of Thurston norms for non-orientable manifolds is that Lemma \ref{lem:injective} gives a bijection between $H^1(M;\RR)$ and a subspace of $H^1(\wt{M};\RR)$.  But the Thurston norm describes the relationship between fibrations an orientable manifold $\wt{M}$ over $S^1$ and {\it lattice points} $H^1(\wt{M}; \ZZ)$. % that when working with fibrations over $S^1$, the elements of $H^1(M; \ZZ)$ are the elements of interest, rather than the elements of $H^1(M; \RR)$. %Lemma \ref{lem:injective} tells us that elements of $H^1(M; \RR)$ are precisely the $\iota^{\ast}$-invariant elements of $H^1(\wt{M}; \RR)$.
However there are $\iota^{\ast}$-invariant
elements on $H^1(\wt{M}; \ZZ)$ that are not pullbacks of elements of $H^1(M; \ZZ)$.  So there is not a bijection between $H^1(M; \ZZ)$ and fibrations of $M$ over $S^1$.

\p{Failure of surjectivity}
  Let $\no$ be a non-orientable surface, $\os$ its orientation double cover. Let $\gamma$ be a one-sided curve
  on $\no$, i.e. a curve whose preimage $\wt{\gamma}$ in $\os$ has a single component. Let the $3$-manifolds $M$ be the mapping torus of $\no$ with some pseudo-Anosov $\varphi$ and $\wt{M}$ the mapping torus of $\os$ with the orientation preserving lift of $\varphi$.  We can then
  consider $\gamma$ and $\wt{\gamma}$ as curves in the $3$-manifolds $M$ and $\wt{M}$.

  Extend $\wt{\gamma}$ to a basis $\mathcal{B}$ of $H_1(\wt{M}; \ZZ)$. We can construct an element
  of $H^1(\wt{M}; \ZZ)$ by simply assigning integer values to the elements of $\mathcal{B}$. Define $\alpha\in H^1(\wt{M};\ZZ)$
  that assigns $0.5$ to $\wt{\gamma}$ and an integer value to every element of $\mathcal{B}$. Consider the
  cohomology class $\alpha + \iota^{\ast}\alpha$. Because $\wt{\gamma}$ is the pre-image of a curve of $M$, we have that $\iota \wt{\gamma} = \wt{\gamma}$. Therefore  $\alpha + \iota^{\ast}\alpha$ is an
  $\iota^{\ast}$-invariant element of $H^1(\wt{M}; \ZZ)$ that assigns $1$ to $\wt{\gamma}$. Such a cohomology
  class cannot be a pullback of a class on $M$ since the pullback of a cohomology class on $M$ would assign an
  even value to $\wt{\gamma}$.

What the above example does show is that for any $\alpha\in H^1(\wt{M}; \ZZ)$ that is  $\iota^{\ast}$-invariant, the class $2\alpha$ definitely is a pullback of class in $H^1(M; \ZZ)$.

\subsection{Inverting the Poincar\'e duality map for embedded surfaces}
\label{sec:invert-poincare}

For either an orientable or non-orientable 3-manifold $M$, given $\alpha\in H^1(M; \ZZ)$, we can construct a dual map $f_\alpha$.  The preimage of a regular value $q\in S^1$ $f_{\alpha}^{-1}(q)$, will be an embedded surface. %In the orientable setting, the homology class of this embedded surface is well-defined, independent both of the choice of representative $1$-form in its cohomology class and the choice of regular value. While the homology class is also well defined in the non-orientable setting, the homology class is trivial when the embedded surface is non-orientable.
We will invert this construction: given an embedded surface $S$, we want a
closed $1$-form $\alpha$ such that the surface $S$ comes from $\alpha$ in the manner described
above.

When $M$ is orientable, Poincar\'e duality determines a closed 1-form corresponding to any embedded surface.  In this section, we create an ad hoc version of Poincar\'e duality for non-orientable surfaces in Theorem \ref{thm:Poincare-duality}.
%associating embedded surfaces to $1$-forms.
However, we need a version of the orientability condition for embedded non-orientable surfaces that we call \emph{relative orientability}.

\p{Relative orientability}
  Let $M$ be a $3$-manifold, and $S$ an embedded surface in $M$. The surface $S$ is said to be {\it relatively
  oriented with respect to $M$} if there is a nowhere vanishing normal vector field on $S$. Two
  such normal vector fields are said to induce the same orientation if locally they induce the
  same orientation after picking a local frame for $S$. A surface $S$ is \emph{relatively oriented}
  if both $S$ and the choice of positive normal vector field are specified.

Note that relative orientability is a strictly weaker notion than orientability. If $S$ and $M$ are
orientable, then $S$ is relatively orientable with respect to $M$. But even if $M$ is
non-orientable, a non-orientable embedded surface $S$ may be relatively orientable with respect to $M$. For instance, let $S$ be
the fiber of a non-orientable mapping torus $M$.  The pre-image under the bundle map of a non-vanishing vector field on $S^1$ is a non-orientable vector field on $M$.
%It is not the case that every embedded surface in a non-orientable $3$-manifold is relatively orientable.

\p{A surface that is not relatively orientable in a 3-manifold}
  Let $S$ be the standard torus $\RR^2/\ZZ^2$, and let $\varphi$ map $(x,y)$ to $(-x, y)$. Then $\varphi$ is an
  orientation-reversing homeomorphism.  Therefore the mapping torus $M_\varphi$ is non-orientable. Consider a vertical line $\gamma$ in $S$ preserved by $\varphi$, i.e. the line
  $x = 0$. The image of $\gamma$ in $S$ under the suspension flow in $M$ is a subsurface of $M$,
  which we'll call $S'$. The normal direction to $S'$ when restricted to $S$ is $\frac{\partial}{\partial x}$. Because the suspension flow reverses the direction of $\gamma$, the
  the normal vector field cannot be continuously extended to all of $M$.
  This means that the surface $S'$ is not relatively orientable in $M$ (despite being orientable itself.)

However, if both $M$ and an embedded surface are non-orientable, the surface will be relatively orientable.
\begin{prop}
  \label{prop:relative-orientability}
  Let $M$ be a non-orientable $3$-manifold, and let $S$ be an embedded connected non-orientable surface in $M$.
  Then $S$ is relatively orientable with respect to $M$.
\end{prop}
\begin{proof}
  Let $\wt{M}$ be the orientation double cover of $M$, and $\wt{S}$ be the pre-image of $S$ under the double cover. The
  restriction of the orientation reversing deck transformation $\iota:\wt{M}\rightarrow\wt{M}$ to $\wt{S}$ is an orientation reversing homeomorphism of $\wt{S}$.
  Let $(v_1, v_2)$ be positively oriented local frame for the tangent space to $\wt{S}$. Let $n$ be an outward pointing normal vector to $\wt{S}$ so the local frame $(v_1, v_2, n)$ is positively oriented. Since $\iota$ reverses the orientation of both $\wt{S}$ and $\wt{M}$, $(\iota(v_1), \iota(v_2))$ and $(\iota(v_1), \iota(v_2), \iota(n))$ are both negatively oriented. Then $\iota(n)$ is outward pointing.
  %, since the quoti(nent $S$ is non-orientable.
  %That means $S$ leaves the outward pointing normal
  %direction from $\wt{S}$ invariant, and that descends to an outward pointing normal direction on $S$. This
  %shows that $S$ is relatively orientable with respect to $M$.
  Therefore the outward pointing normal direction on $\wt{S}$ descends to an outward pointing normal direction on $S$, and $S$ is relatively orientable in $M$.
\end{proof}

We care about relatively orientable surfaces because for these surfaces can be mapped to cohomology classes.
\begin{thm}[Poincar\'e duality for non-orientable $3$-manifolds]
  \label{thm:Poincare-duality}
  Let $M$ be a non-orientable $3$-manifold, and let $S$ be a relatively oriented embedded
  surface. Then there exists a cohomology class $[\alpha]$ in $H^1(M; \ZZ)$ and a regular value $q\in S^1$ such that for some
  representative $\alpha$, $S=f_{\alpha}^{-1}(q)$. Furthermore, $\alpha$ assigns positive values to the positively oriented normal vector
  field on $S$.
\end{thm}

The idea of the proof of this theorem is fairly straightforward. Starting with the embedded surface
$S$ in $M$, we look at the pre-image $\wt{S}$ in the orientation double cover $\wt{M}$. We show
that the Poincar\'e dual to $\wt{S}$ is $\iota^{\ast}$-invariant.
\begin{lem}
  \label{lem:PD1}
  Let $S$ be a relatively oriented embedded surface in $M$, and $\wt{S}$ its pre-image in
  $\wt{M}$. Then the Poincar\'e dual to $[\wt{S}]$ is $\iota^{\ast}$-invariant.
\end{lem}
\begin{proof}
  If $S$ is relatively oriented with respect to $M$, then the relative orientation lifts to a relative orientation of $\wt{S}$ with respect to $\wt{M}$. Since $\wt{S}$ and $\wt{M}$ are orientable, this defines an orientation on $\wt{S}$,
  and thus the homology class $[\wt{S}]$ is well defined.

  The deck transformation $\iota$ reverses the orientation on $\wt{S}$. Indeed, let $(v_1, v_2, v_3)$ be a local frame for some point in $\wt{S}$ such that
  $v_3$ is the outward pointing normal vector field. Since the outward pointing normal vector
  field descends to the quotient by the orientation reversing map $\iota$.  Therefore $\iota(v_3)$
  must also be outward pointing. Since $\iota$ reverses the
  orientation on $\wt{M}$ but preserves the direction of $\iota(v_3)$, $\iota$ must reversing the orientation on the pair
  $(v_1, v_2)$.\becca[inline]{Is orientation really the best term in reference to the pair?} In particular, that means $\iota$ reverses the orientation on $\wt{S}$.

  This means $[\wt{S}]$ is in the $-1$-eigenspace of the $\iota_{\ast}$ action on
  $H_2(\wt{M}; \RR)$. Let $\wt{\alpha}$ be the the Poincar\'e dual to $[\wt{S}]$. The 1-form $\wt{\alpha}$ is $\iota^{\ast}$-invariant. This follows from
  the following chain of equalities which hold for all closed $2$-forms $\omega$.  We use the fact that $\iota^2=id$ in the first and third equalities.
  \begin{align*}
    \int_{\iota_{\ast}\wt{S}} \omega &= \int_{\wt{S}} \iota^{\ast}\omega &&\text{(By a change of variables)} \\
                                     &= \int_{\wt{M}} \wt{\alpha} \wedge \iota^{\ast} \omega &&\text{(Poincar\'e duality)} \\
                                     &=\int_{\wt{M}} \iota^{\ast} \left( \iota^{\ast}\wt{\alpha} \wedge \omega \right) \\
    &= \int_{\wt{M}} - \left( \iota^{\ast} \wt{\alpha} \wedge \omega \right) &&\text{($\iota$ is orientation reversing)}
  \end{align*}
  On the other hand, the following equalities follow from the fact that
  $\iota_{\ast}[\wt{S}] = -[\wt{S}]$.
  \begin{align*}
    \int_{\iota_{\ast}\wt{S}} \omega &= - \int_{\wt{S}} \omega \\
                              &= - \int_{\wt{M}} \wt{\alpha} \wedge \omega
  \end{align*}
  Because $$\int_{\wt{M}}\wt{\alpha}\wedge\omega=\int_{\wt{M}}\iota^\ast\wt{\alpha}\wedge\omega$$ for all $\omega$, it follows that $\wt{\alpha}$ is
  $\iota^{\ast}$-invariant.
\end{proof}

We now have an $\iota^{\ast}$-invariant $1$-form $\wt{\alpha}$.  Then we construct the map $f_{\widetilde{\alpha}}:\wt{M}\rightarrow S^1$ such that for a regular value $p\in S^1$, the surface $\wt{S}=f_{\wt{\alpha}}^{-1}(p)$. The next claim we want to make is that the map $f_{\wt{\alpha}}: \wt{M} \to S^1$ factors through the quotient $M$.
\begin{lem}
  \label{lem:PD2}
Let $p:\wt{M}\rightarrow M$ be the orientation double cover. For all points $y \in \wt{M}$, $f_{\wt{\alpha}}(y) = f_{\wt{\alpha}}(\iota (y))$.
\end{lem}
\begin{proof}
  Recall that $f_{\wt{\alpha}}(y)$ is given by the following integral formula.
  \begin{align*}
    f_{\wt{\alpha}}(y) = \int_{x_0}^y \wt{\alpha} \mod \ZZ,
  \end{align*}
  where $x_0$ is a basepoint in $\wt{M}$. Since $f_{\wt{\alpha}}(y)$ is equal to
  $f_{\wt{\alpha}}(\iota(y))$ for all $y$, we have the following:
  \begin{align*}
    \left(  \int_{x_0}^y \wt{\alpha} - \int_{x_0}^{\iota(y)} \wt{\alpha} \right) \in \ZZ.
  \end{align*}
  By a change of variables, and using the $\iota^{\ast}$-invariance of $\wt{\alpha}$, the left hand
  side of the above condition can be transformed, giving us the following condition.
  \begin{align*}
    \label{cond:integer}
    \left( \int_{x_0}^{\iota(x_0)} \wt{\alpha} \right) \in \ZZ
  \end{align*}
  %In other words, we want the integral of $\wt{\alpha}$ along any curve $\gamma$ from $x_0$ to $\iota(x_0)$ to be an integer. Equivalently,
  Let $\gamma$ be a simple one-sided curve based at $p(x_0)$. The preimage $p^{-1}(\gamma)$ in $\widetilde{M}$ is a simple closed curve based at $x_0$, call it $\delta$.  It will suffice to show that the integral of
  $\wt{\alpha}$ along $\delta$ is an even integer.

  The parity of $\displaystyle \int_{\delta} \wt{\alpha}$ is precisely the parity
  of the intersection number of $\delta$ and $\wt{S}$.  But the intersection of $\delta$ and $\wt{S}$ is even because it is twice the intersection of $p(\delta)=\gamma$ and $p(\widetilde{S})=S$.   %Furthermore, both $\delta$ and $\wt{S}$ are lifts of a curve and surface from $M$. Which means the number of intersections they have in $\wt{M}$ is twice the number of intersections have in $M$. But the latter number must be an integer, and thus the former number must be an even integer, showing that condition \eqref{cond:integer} holds.
  In particular,
  this shows that the map $f_{\wt{\alpha}}$ factors through, proving the lemma.
\end{proof}
We now have everything we need to finish proving Theorem \ref{thm:Poincare-duality}.
\begin{proof}[Proof of Theorem \ref{thm:Poincare-duality}]
  Starting with a relatively oriented surface $S$ in $M$, we look at its pre-image $\wt{S}$ in
  $\wt{M}$ under the orientation double cover. The relative orientation of the preimage gives us the homology class $[\wt{S}]$, and
  we get a $1$-form $\wt{\alpha}$, which is Poincar\'e dual to the homology class of $\wt{S}$.
  More specifically, we have a map $f_{\wt{\alpha}}$ and a regular value $q\in S^1$ such that $\wt{S}=f^{-1}_{\wt{\alpha}}(q)$.
  By Lemma \ref{lem:PD1}, $\wt{\alpha}$ is $\iota^{\ast}$-invariant, and by Lemma \ref{lem:PD2}, the map $f_{\wt{\alpha}}$ factors through $M$ to a map $f_{\alpha}:M\to S^1$.  The map $f_\alpha$ has the property that $f_{\alpha}^{-1}(q) =
  S$. By pulling back $d\theta$ on $S^1$ under $f_\alpha$, we obtain the desired 1-form $\alpha$ in $H^1(M; \ZZ)$.
\end{proof}

\subsection{Oriented sums of surfaces}
\label{sec:orient-sums-surf}

%We now have a way of going from an embedded surface to an element of $H^1(M; \ZZ)$. To make this mapping even
%more useful, we'll describe a way of adding two surfaces via the operation of taking
The next step in studying embedded non-orientable surfaces will be to describing
\emph{oriented sums}.  The oriented sum of two surfaces embedded in a manifold $M$ indeed is additive in both the Euler characteristic and $H^1(M;\RR)$. This
operation is well-known in the case of orientable $3$-manifolds (along with orientable embedded
surfaces), but we will sketch out the relevant details.  We then extend the construction relatively orientable embedded surfaces. %The same construction works for relatively orientable surfaces; one just needs to verify consistency.

\p{Oriented sum for oriented manifolds}
Let $S$ and $S'$ be oriented embedded surfaces in an oriented manifold $M$. Assume that $S$ and $S'$ intersect
trasversally. Thus, $S \cap S'$ is a disjoint union of copies of $S^1$. For each component $S\cap S'$, take a tubular neighborhood that has cross section as in Figure \ref{fig:cross-section}.

%\autoref{fig:cross-section}.
\begin{figure}
  \centering
  \incfig[0.2]{cross-section}
  \caption{Cross section of intersection of $S$ and $S'$.}
  \label{fig:cross-section}
\end{figure}

We then perform a surgery on the leaves of $S$ and $S'$ so that the outward pointing normal vector fields match as in Figure \ref{fig:surgery}.% We have two possible
%choices: we could join the left $S$ leaf to either the top or the bottom $S'$ leaf. Since both $S$ and $S'$ are oriented submanifolds of $M$, there is an outward pointing normal vector field on $S$ and $S'$. Suppose the outward normal vector field on $S$ points upwards and the outward normal vector on $S'$ points to the right. In that case, we'd glue the left $S$ leaf to the bottom $S'$ leaf to maintain a consistent outward normal vector field. See \autoref{fig:surgery} to see how the choice affects orientability.
\begin{figure}
  \centering
  \incfig[0.3]{surgery}
  \caption{On the left, the normal vectors on $S$ and $S'$ are consistent. On the right, they aren't.}
  \label{fig:surgery}
\end{figure}

By performing this surgery at all the intersections, we get a new submanifold $S''$ (which may have
multiple components). This new submanifold $S''$ is the oriented sum of $S$ and $S'$. The operation
of taking oriented sums is additive on Euler characteristic, as well as the homology classes (and thus
the cohomology classes of their Poincar\'e duals).
\begin{align*}
  \chi(S'') &= \chi(S) + \chi(S') \\
  [S''] &= [S] + [S'] \\
\end{align*}

\p{Oriented sum for non-orientable manifolds}
Let $M$ be a non-orientable manifold and let $S$ and $S'$ be embedded surfaces in $M$ that are relatively oriented.
%Observe that in order to canonically choose the right leaves to join, all we need is a relative orientation for both $S$ and $S'$. %That suggests that the same construction ought to work.
%Like in the case of an orientable ambient manifold, at every transversal intersection, we perform surgery based on the outwards pointing normal vector field.
%We need to verify that this construction is consistent with the covering map: i.e. taking
%the oriented sum of $S$ and $S'$ is the same as taking the oriented sum of $\wt{S}$ and $\wt{S'}$
%and then taking the quotient by $\iota$.

%Let $\gamma$ be a component of $S\cap S'$ in $M$.  %, and let $\wt{\gamma}_1$ and $\wt{\gamma}_2$ be the distinct path lifts of $\gamma$ in $\wt{M}$ under the orientation double cover $p$.
We will define the oriented sum on $S$ and $S'$ as follows.  Let $p:\wt{M}\rightarrow M$ be the orientation double cover and let $\iota$ be the orientation reversing deck transformation of $\wt{M}$.  As above, let $\wt{S}=p^{-1}(S)$ and $\wt{S}'=p^{-1}(S')$, which are embedded oriented surfaces in $\wt{M}$.  The oriented sum of $S$ and $S'$ is the image under $p$ of the oriented sum of $\wt{S}$ and $\wt{S}'$ (as defined above for oriented surfaces in oriented manifolds).  We need to justify that this operation is well-defined.

As in the proof of Lemma \ref{lem:PD1}, $\iota$ preserves the relative orientation, and thus leaves the outward
normal vector fields on $\wt{S}$ and $\wt{S}'$ invariant. Therefore a leaf $\ell$ of $\wt{S}$ is surgered with a leaf of $\ell'$ of $\wt{S}'$ if and only if $\iota(\ell)$ and $\iota(\ell')$ are surgered.  Therefore surgery factors through $p$ and the oriented double sum is well-defined.  %The oriented sum of $S$ and %The oriented sum of embeddein $M$ that is consistent with the oriented sum on the orientation double cover.%We need to show that when we surger the leaves of $\wt{S}$ to a leaf of $\wt{S'}$ along $\wt{\gamma}_1$, .

\p{Example} Let $\gamma$ be a component of $S\cap S'$ and $\wt{\gamma}_1$ and $\wt{\gamma}_2$ be the path lifts of $\gamma$.  Consider
\autoref{fig:consistency}, which shows the outward point normal vectors to $\wt{S}$ and $\wt{S'}$,
which determine which leaves are glued together along $\wt{\gamma}_1$ and $\wt{\gamma}_2$.
\begin{figure}
  \centering
  \incfig[0.4]{consistency}
  \caption{Neighborhoods of $\wt{\gamma}_1$ and $\wt{\gamma}_2$, with the outward pointing normal vector field.}
  \label{fig:consistency}
\end{figure}

The normal vector field tells us that the left $\wt{S}$ leaf gets glued to the bottom $\wt{S'}$
leaf near $\wt{\gamma}_1$ and $\wt{\gamma}_2$. Since $\iota(\wt{\gamma}_1)=\wt{\gamma_2})$, the outward pointing normal vector fields point the same (relative) directions.  %Consider now the deck transformation $\iota$. %Note that $\iota$ is an orientation reversing self map for $\wt{M}$, $\wt{S}$ and $\wt{S'}$. We've


\p{Additivity} By the consistency of the oriented sum in $M$ and $\wt{M}$, it easily follows that the oriented sum
is additive in Euler characteristic, as well as in terms of Poincar\'e dual, since the Poincar\'e
dual was also defined by passing to the orientation double cover.
