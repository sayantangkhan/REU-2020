\section{Background}
\label{sec:background}

\subsection{Mapping classes of non-orientable surfaces}
\label{sec:mapping-classes-non}

A genus $g$ non-orientable surface is the connect-sum of $g$ copies of
$\mathbb{RP}^2$, analogous to how a genus $g$ surface is the connect-sum of $g$ copies of a torus
$S^1 \times S^1$.  A common way of visualizing non-orientable surfaces is to think of them as orientable
surfaces with \emph{crosscaps} attached. We attach a crosscap to a surface $S$ by first deleting a small open disc $D\subset S$, and
identifying the boundary of that disc (on the surface) via the antipodal map. In pictures, this is often denoted by
an X inscribed in a circle, see \autoref{fig:buildingblock} for an example of a surface with two crosscaps
attached.  Let $\no_{g,n}$ be a non-orientable surface obtained by attaching $g$ crosscaps to $S^2$ and marking $n$ points in $S^2$.  The integer $g$ is referred to as the genus of $\no_{g,n}$.  The compact non-orientable surfaces are classified by the triple $(g,n,b)$ where $g$ is the genus, $n$ is the number of marked points and $b$ is the boundary.  

\p{The orientation double cover} The orientation double cover of $\no_{g,n}$ is an orientable surface $\os_{g-1, 2n}$ of genus $g-1$ and $2n$ marked points and a covering map $p$ defined as follows. The surface $\os_{g-1,2n}$ has an orientation reversing deck transformation $\iota: \os_{g-1, 2n} \to \os_{g-1,
  2n}$ of order 2. The covering map from $p:\os_{g-1,2n}\to \no_{g,n}$ is the quotient of $\os_{g-1,2n}$. If the genus and the
number of marked points is clear from the context, we will drop the subscripts and just use
$\no$ and $\os$ for the non-orientable surface and its orientation double cover respectively.

Every homeomorphism $\varphi: \no \to \no$, has a unique orientation preserving lift, that is a homeomorphism $\wt{\varphi}: \os \to \os$ with $p\wt{\varphi}=\varphi p$. %This is an easy exercise in covering space theory, but we'll give a proof here for completeness.
%\begin{prop}
 % For any homeomorphism $\varphi: \no \to \no$, there exists a unique orientation preserving lift
 % $\wt{\varphi}: \os \to \os$. If the non-orientable surface has marked points fixed by $\varphi$,
  %then the orientation preserving lift $\wt{\varphi}$ may not fix the marked points, but the lift
  %of $\varphi^2$ will fix the marked points.
%\end{prop}
%\begin{proof}
 % One can always lift a homeomorphism $\varphi: \no \to \no$ if $\varphi$ preserves the subgroup of
  %$\pi_1(\no)$ corresponding to the cover $\os$. This subgroup can be concretely described as the
  %subgroup generated by the two sided curves in $\no$, i.e. the curves whose tubular neighbourhoods
  %are cylinders, and not M\"obius strips. Such a subgroup is clearly preserved by any homeomorphism
  %$\varphi$, which means we always have a lift. There will be two choices for such a lift, since
  %$\os$ is a two-sheeted cover. These two lifts $\wt{\varphi}_1$ and $\wt{\varphi}_2$ are related
  %by the following identity.
  %\begin{align*}
%\wt{\varphi}_1 = \iota \circ \wt{\varphi}_2
 % \end{align*}
  %Since $\iota$ is orientation reversing, only one of $\wt{\varphi}_1$ or $\wt{\varphi}_2$ is
  %orientation preserving, which gives us a unique choice.
%
 % If $\no$ has marked points that $\varphi$ fixes, then the lift $\wt{\varphi}$ may or may not swap
 % the pre-images of the marked points. But the square of the lift will definitely fix the
 % pre-images as well, which proves the second part of the proposition.
%\end{proof}

A consequence is that lifting homeomorphisms induces a monomorphism between orientation preserving homeomorphisms of $\no$ and (orientation preserving) homeomorphisms of $\os$.  Every homotopy of $\no$ lifts to a homotopy of $\os$.  Moreover, if $f,g:\no\to\no$ are homeomomorphisms such that their orientation preserving lifts $\widetilde{f},\widetilde{g}$ of $\os$ are homotopic, then $f$ and $g$ are homotopic.  Therefore there is an inclusion from the mapping class group of $\no$ to the (orientation preserving) mapping class group of $\os$.  This inclusion also respects the Nielsen-Thurston classification of mapping classes, both qualitatively, and
quantitatively, as the following proposition shows.
\begin{prop}
  \label{prop:2}
  If $\varphi$ is a self-homeomorphism of $\no$ and $\wt{\varphi}$ is its orientation preserving lift on $\os$, then:
  \begin{enumerate}[(i)]
  \item $\varphi$ is periodic if and only if $\wt{\varphi}$ is periodic,
  \item $\varphi$ is reducible if and only if $\wt{\varphi}$, 
  \item $\varphi$ is pseudo-Anosov if and only if $\wt{\varphi}$ is pseudo-Anosov.  Moreover if $\varphi$ has stretch factor $\lambda$, then $\wt{\varphi}$ also has stretch factor $\lambda$.
  \end{enumerate}
\end{prop}
\begin{proof}
  It's easy to see that if $\varphi$ is periodic, so it $\wt{\varphi}$, and the other way round. If $\varphi$
  is reducible, that means it leaves some multicurve $\gamma$ in $\no$ invariant, which means $\wt{\varphi}$ leaves
  the preimage of $\gamma$ invariant as well. Conversely, if $\wt{\varphi}$ leaves some multicurve
  $\wt\gamma$ invariant, so does $\iota \circ \wt{\varphi}$, since they commute. That means the union of $\wt\gamma$
  and $\iota(\wt\gamma)$ is also a multi-curve and thus descends to a multi-curve on $\no$ that is left invariant
  by $\varphi$. Since any mapping class of $\no$ that is neither periodic nor reducible must be pseudo-Anosov on $\no$ must lift to a pseudo-Anosov on $\no$
  and vice versa.

  Suppose now that $\varphi$ is a psuedo-Anosov on $\no$ with stretch factor $\lambda$ and expanding and contracting
  foliations $\mu_e$ and $\mu_c$ respectively. Since $\varphi$ is a pseudo-Anosov map, the following
  identity involving the intersection form $i$ holds for all closed curves $\gamma$ in $\no$.
  \begin{align}
    \label{eq:1}
    i(\varphi^{-1}\gamma, \mu_e) &= i(\gamma, \varphi(\mu_e)) \\
                               &= \lambda \cdot i(\gamma, \mu_e)
  \end{align}
  A similar identity holds for $\mu_c$.
  \begin{align}
    \label{eq:2}
    i(\varphi^{-1}\gamma, \mu_c) &= i(\gamma, \varphi(\mu_c)) \\
                               &= \frac{1}{\lambda} \cdot i(\gamma, \mu_c)
  \end{align}
  Note now that the foliations can be lifted to the double cover: call their lifts $\wt{\mu}_e$ and
  $\wt{\mu}_c$. For any closed curve $\wt{\gamma}$ of $\os$, consider its intersection number with the
  foliations. Observe that computing the intersection number is a local calculation. Start by picking an open
  cover $U$ on $\no$ such that all the open sets in $U$ are homeomorphic to the connected components of their
  pre-image in $\os$. By picking a partition of unity subordinate to this cover, one can compute the intersection
  number by restricting computation on each open set in the cover. This calculation lifts to the orientation double
  cover, giving us the following identity.
  \begin{align}
    \label{eq:3}
    i(\wt{\gamma}, \wt{\mu}_e) = i(\gamma, \mu_e)
  \end{align}
  Combining identities \eqref{eq:1} and \eqref{eq:3}, we get the following identity for intersection numbers
  on $\os$.
  \begin{align*}
    i(\wt{\varphi}^{-1} (\wt{\gamma}), \wt{\mu}_e) = \lambda \cdot i(\wt{\gamma}, \wt{\mu}_e)
  \end{align*}
  We get a similar expression for $\wt{\mu}_c$, which proves that $\wt{\varphi}$ has the same stretch factor
  as $\varphi$, thus proving the proposition.
\end{proof}
% Part (ii) of Proposition \ref{prop:2} is going to be useful in an application of our main result,
% where we'll be computing asymptotics for the minimal stretch factor of a pseudo-Anosov map on
% $\no_{g,n}$.

%Finally, the last thing we need to know about mapping classes on non-orientable surfaces is how to construct examples of pseudo-Anosov maps. 
In the case of orientable surfaces, the Penner construction is used to construct pseudo-Anosov maps, as well compute their stretch factors. It turns out the Penner construction also works in the non-orientable setting, with some minor modifications. This construction is presented in detail in Section 2 of \cite{Strenner_2017}, but we give an outline of the key ideas below.

\p{The Penner construction} The Penner construction in the orientable setting starts with a pair of filling multicurves $A = \{a_1,\dots,a_n\}$
and $B = \{b_1,\dots,b_m\}$.  A Penner construction is a composition of positive Dehn twists around curves in $A$ and negative Dehn twists about curves in $B$ that uses a Dehn twist about each curve in $A\cup B$ at least once.  Penner proves that this construction is pseudo-Anosov \cite{penner1988construction}. The problem with making this work for
non-orientable surfaces is that when defining Dehn twists about curves on a non-orientable surface, there is not a well-defined notion of a left or right Dehn twist. For non-orientable surfaces we will use a set of filling two-sided curves that are \textit{marked inconsistently}.

Each two-sided curve $c$ on a non-orientable surface $N$ has a neighborhood homeomorphic to an
annulus $A$ by a homeomorphism $\phi: A \xrightarrow{} N$, called a \textit{marking}. In this
context, we can define the Dehn twist $T_{c,\phi}(x)$ around $(c,\phi)$ in the following manner.
\begin{align*}
  T_{c,\phi}(x) =
  \begin{cases}
    \phi \circ T \circ \phi^{-1}(x) & \text{for } x \in \phi(A) \\
    x & \text{for } x \in N - \phi(A)
  \end{cases}
\end{align*}
Here $T$ is the standard Dehn twist on $A$, i.e. $T(\theta,t) = (\theta + 2\pi t,t)$. If we fix an
orientation of $A$, then we can pushforward this orientation to $S$. We say two marked curves
$(c,\phi_c)$ and $(d,\phi_d)$ that intersect at a point $p$ are marked inconsistently if the
pushforward of the orientation of $A$ by $\phi_c$ and $\phi_d$ disagree in a neighborhood of $p$.
If all our curves are marked inconsistently and are filling, then once again a composition of Dehn
twists around them that use all the curves at least once will be pseudo-Anosov.

\p{Train tracks} The Penner construction not only promises that our map is pseudo-Anosov, but it also gives a way to
compute the stretch factor of our map (see \cite{penner1988construction}).  The proof of the fact
that the composition is pseudo-Anosov, and the computation of its stretch factor works the same is
in the orientable setting.  Let $\varphi$ be a pseudo-Anosov homeomorphism of $\no$.  A {\it train track} is an embedded graph in $\no$ such that for every vertex of valence three or greater, all adjacent edges have the same tangent vector.  An {\it invariant train track for $\varphi$} is a train track track $\tau$ such that $\varphi(\tau)$ is homotopic to $\tau$.  Let $\mathcal{C}$ be a collection of curves in $\no$. %Consider now the collection of transverse measures on our train track $\tau$. 
For every curve $\gamma \in\mathcal{C}$, there is an associated transverse measure
$\mu_\gamma$ for $\tau$ that assigns $1$ to all edges lying in $\gamma$ and 0 to everything else. Let $V_\tau$
be the cone of transverse measures on $\tau$, and $H$ the subspace of $V_\tau$ spanned by the
transverse measure associated to curves in $\mathcal{C}$.
%\begin{align*}
 % H = \mathrm{span}(\{\mu_\gamma \mid \gamma \text{ is a connected curve in } \mathcal{C}\}).
%\end{align*}
The measures $\mu_\gamma$ are linearly independent and form the \textit{standard basis} for $H$. The subspace $H$ is invariant under the action of $\varphi$ on $V_\tau$, thus $\varphi$ has a linear action on $H$. If we let $A$
be the matrix representing this action in the standard basis, then the stretch factor of $\varphi$,
$\lambda(\varphi)$, is the Perron-Frobenius eigenvalue of $\varphi$.

\subsection{Thurston's theory of fibered faces}
\label{sec:thurst-fiber-face}

\becca[inline]{Does this first paragraph belong here or after the definition of Thurston norm?} Given a surface $S$ and a homeomorphism $\varphi: S \to S$, one can construct a $3$-manifold $M_\varphi$ via
the \emph{mapping torus} construction.
\begin{align*}
  M_\varphi \coloneqq \frac{S \times [0,1]}{(x,1) \sim (\varphi(x), 0)}
\end{align*}
%Inverting this construction is also a problem of interest: given a $3$-manifold $M$, is it the mapping torus of some surface and self-homeomorphism $(S,\varphi)$? In how many ways can one express a $3$-manifold as a mapping torus? 
Mapping tori are exactly the surface bundles over $S^1$, or \emph{fibrations over $S^1$}, denoted $S\rightarrow M\rightarrow S^1$. A fibration defines a {flow} on $M$,
called the \emph{suspension flow}, where for any $x_0\in S$ and $t_0\in S^1$ the pair $(x_0,t_0)$ is sent to $(x_0,t_0+t)$. %It turns out that the Thurston norm is an extremely useful tool when studying fibrations over $S^1$ (and other related objects).

Given a orientable closed $3$-manifold $M$, the Thurston norm is a semi-norm on its second homology
group $H_2(M; \RR)$: to define the norm, we need to make some preliminary remarks and define a
function.  Let $S$ be a connected surfaces  Define the complexity of $S$ to be $\chi_-(S) = \max\{-\chi(S),0\}$. If a
surface $S$ has multiple components $S_1,\cdots,S_m$ then $\chi_-(S)=\displaystyle\sum_{i=1}^m\chi_-(S_i)$.  Furthermore, for an \emph{orientable} 3-manifold $M$, it is known that
elements in $H_2(M ; \ZZ)$ can be represented by embedded surfaces inside of $M$. This lets us
define a norm function $x$ for every homology class $a$ in $H_2(M; \ZZ)$:
\begin{align*}
  x(a) = \min\{\chi_-(S) \mid [S] = a \text{ and $S$ is compact, properly embedded and oriented}\}.
\end{align*}

We then linearly extend $x$ to the rational points.  There is then a is a unique continuous extention of $x$ to $\RR$, that is: a $\RR$-valued function on $H_2(M; \RR)$ called the {\it Thurston norm}. The unit ball for the Thurston norm is a convex polyhedron, and thus it makes
sense to talk about the \emph{faces} of the unit ball.

The way this ties back up with the question of realizing the $3$-manifold $M$ as a fibration over
$S^1$ is via the following remarkable theorem of Thurston \cite{thurston1986norm}.  We use the restatement from \cite{yazdi2018pseudo}.

\begin{thm}[Thurston]
  \label{thm:Thur1}
  Let $\mathcal{F}$ be the set of homology classes in $H_2(M; \RR)$ that are representable by fibers of
  fibrations of $M$ over the circle.
\begin{enumerate}[(i)]
\item Elements of $\mathcal{F}$ are in one-to-one correspondence with (non-zero) lattice points
  inside some union of cones over open faces of the unit ball in the Thurston norm.
\item If a surface $F$ is transverse to the suspension flow associated to some fibration of
  $M \xrightarrow[]{} S^1$ then $[F]$ lies in the closure of the corresponding cone in $H_2(M;\RR)$.
\end{enumerate}
\end{thm}
%Note that one needs to clarify here what the homology class $[F]$ actually is, since given an embedded surface $F$, one has two canonical choices of $[F]$ depending upon the orientation of $F$. 
The class $[F]$ referred to in the theorem has orientation such that the positive flow direction is pointing outwards relative to the surface.

The open faces whose cones contain the fibers of fibrations are what are referred to as
\emph{fibered faces}.  %It turns out one can recover even more information about the homeomorphism $\phi$ of which $M$ is the mapping torus. The following result of Thurston tells us precisely when the map $f$ is psuedo-Anosov.

\begin{thm}[Thurston's Hyperbolization Theorem]
    \label{thm:ThurHyp}
  If $M_\varphi$ is the mapping torus of $(S, \varphi)$, then $M_\varphi$ is hyperbolic if and only if $\varphi$ is pseudo-Anosov.
\end{thm}

In particular, a consequence of the above theorem is that if $M_\varphi$ is the mapping torus of $(S', \varphi')$ then $\varphi'$ is pseudo-Anosov if and only if $\varphi$ is pseudo-Anosov. From Theorem \ref{thm:Thur1}, we know that all such pairs $(S, \varphi)$ correspond to lattice points in some union of
cones in $H_2(M;\RR)$. One strategy to study possible stretch factors of pseudo-Anosov mapping classes is to study how the stretch factors change within a cone in $H_2(M_\varphi,\RR)$. To this end, we have the following two
theorems, due to Fried-Matsumoto (see \cite{fried1982flow}, \cite{fried1983transitive}, and
\cite{matsumoto1987topological}) and Agol-Leininger-Margalit (see \cite{agol6983pseudo}).

\begin{thm}[Fried-Matsumoto]
  \label{thm:fm}
  Let $M$ be a hyperbolic $3$-manifold and let $\mathcal{K}$ be the union of cones in
  $H_2(M; \RR)$ whose lattice points correspond to fibrations over $S^1$.  There exists a strictly
  convex function $h: \mathcal{K} \to \RR$ satisfying the following properties.
  \begin{enumerate}[(i)]
  \item For all $c > 0$ and $u \in \mathcal{K}$, $h(cu) =  \frac{1}{c}h(u)$.
  \item For every primitive lattice point $u \in \mathcal{K}$, $h(u) = \log(\lambda)$, where $\lambda$ is the
    stretch factor of the pseudo-Anosov map associated to this lattice point.
  \item $h(u)$ goes to $\infty$ as $u$ approaches $\partial \mathcal{K}$.
  \end{enumerate}
\end{thm}

\begin{thm}[Agol-Leininger-Margalit]
  \label{thm:alm}
  Let $\mathcal{K}$ be a fibered cone for a mapping torus $M$ and let $\overline{\mathcal{K}}$ be its closure
  in $H_2(M;\RR)$. If $u \in \mathcal{K}$ and $v \in \overline{\mathcal{K}}$, then $h(u+v) < h(u)$.
\end{thm}
Theorems \ref{thm:fm} and \ref{thm:alm} make a quantitative link between the lattice points in the cones over
the fibered faces and the stretch factor of the associated pseudo-Anosov maps, demonstrating the power of
Thurston's fibered theory of fibered faces.

\p{Incompressible surfaces} %It is also useful to know if an embedded surface $S$ minimizes the Thurston norm in its homology class. It turns out that incompressible surfaces 
Let $S$ be a surface with positive genus embedded in a $3$-manifold $M$.  The surface $S$ is said to be {\it incompressible} if there
  exists no embedded disc $D$ in $M$ such that $D \cap S = \partial D$, and $D$ intersects $S$ transversally.  The following theorem is due to Thurston.
\begin{thm}
  A surface $S$ minimizes the Thurston norm in its homology class if and only if it is incompressible.
\end{thm}

\p{Examples} The following two families of examples of incompressible surfaces will be fairly important for us.
  Let $M$ be a hyperbolic $3$-manifold that fibers over $S^1$. Then an embedded surface $S$ is incompressible
  if either of the two following conditions hold:
  \begin{enumerate}[(i)]
  \item $S$ is the fiber of a fibration.
  \item $S$ has genus $2$. In this case, $S$ is incompressible.  Indeed, if it were not norm minimizing in its homology
    class, there would be an embedded surface of genus 0 or 1. But then $M$ cannot be hyperbolic.
  \end{enumerate}

The first example encompasses a large class of incompressible surfaces, as demonstrated by the following theorem.
\begin{thm}[Theorem 4 from \cite{thurston1986norm}]
  \label{thm:ThurIsotope}
  Any incompressible surface $S$ in the homology class of a fiber of a fibration is isotopic to the fiber.
\end{thm}

%\becca[inline]{good paragraph, probably belongs in the intro} One of the primary goals of this paper to extend this definition of Thurston norm to non-orientable manifolds and be able to state the non-orientable version of the theorems above. Most of the work in doing so is concentrated in determining the right analog of Thurston norm for non-orientable surfaces, and then making Theorem \ref{thm:Thur1} work with that definition, which is what we will do in Section \ref{sec:fibered-face-theory}. Once we have the versions of the theorems for non-orientable surfaces, we'll generalize a trick due to McMullen that lets one construct pseudo-Anosov maps with small stretch factor to non-orientable surfaces. Finally we will prove bounds on the asymptotics of minimal stretch factors for non-orientable surfaces in Section \ref{sec:application} by adapting the methods in \cite{yazdi2018pseudo} for non-orientable surfaces.
