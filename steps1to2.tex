We will first imitate the first two steps of Yazdi's construction, constructing two infinite families of non-orientable surfaces.

\textcolor{red}{Should we instead just say we do Yazdi's construction but just using a non-orientable building block?}

\textbf{Step 1:} We will define two families of surfaces $P_{n,k}$ and $Q_{n,k}$, first defining $P_{n,k}$ in the following way originally done by Yazdi. Let $T$ be a non-orientable surface of genus 12 with 3 boundary components $c,d$ and $e$. We give $c, d$ and $e$ a consistent choice of orientation (\textcolor{red}{How do we define what this means. Possible idea, we first imagine $P_{n,k}$ as a genus 5 orientable surface, orient the boundaries accordingly, and then add on the cross-caps}). Let $p$ (respectively $q$) be a puncture (respectively a marked point) on the boundary component $e$ of $T$. Let $r$ and $s$ be the two oriented arcs connecting $p$ and $q$ in $\partial T$. See Figure X for a picture of $T$. Let $T_{i,j}$ be copies of the surface $T$, where $i,j \in \mathbb{Z}$. We will use similar notations to refer to the boundary components of $T_{i,j}$. Define the infinite surface $S_\overline$ as the quotient $$S_\infty \coloneqq \left( \bigcup T_{i,j} \right)/\sim,$$ where $i,j \in \mathbb{Z}$. The equivalence relation $\sim$ is defined as $$c_{i,j} \sim d_{i+1,j} \hspace{1em}, \hspace{1em} r_{i,j} \sim s_{i,j+1}$$ where $i,j \in \mathbb{Z}$, and the gluing maps for the boundary components are by orientation-reversing homeomorphisms.

There are two natural maps $\overline{\rho_1},\overline{\rho_2}: S_\infty \xrightarrow[]{} S_\infty$ that act by shifts as follows $$\overline{\rho_1} \text{ sends } T_{i,j} \text{ to } T_{i+1,j},$$ $$\overline{\rho_2} \text{ sends } T_{i,j} \text{ to } T_{i,j+1}.$$ Note that these maps commute. Define the surface $P_{n,k}$ as the quotient of the surface $S_\infty$ by the covering action of the group generated by $(\overline{\rho_1})^n$ and $(\overline{\rho_2})^k$. Therefore, $\overline{\rho_1}$ and $\overline{\rho_2}$ induce maps on the surface $P_{n,k}$, which we denote by $\rho_1$ and $\rho_2$. 

We can define an analogous family of surface $Q_{n,k}$ by starting with a different building block, $R$. Let $R$ be a non-orientable surface of genus 13 with 3 boundary components $c,d$ and $e$. We again give $c, d$ and $e$ a consistent choice of orientation as above, and let $p$,$q$,$r$ and $s$ be as above. See Figure Y for a picture of $R$. We then let $R_{i,j}$ be copies of $R$, where $i,j \in \mathbb{Z}$ and then define a surface $N_\infty$ defined exactly like $S_\infty$ above. We then analogously have maps $\overline{\rho_1},\overline{\rho_2}: N_\infty \xrightarrow[]{} N_\infty$ acting as shifts, and use them to define the family of surfaces $Q_{n,k}$.

\begin{lem}
Define the sequences
\begin{align}
    g_{n,k} &= (14k - 2)n + 2, k \geq 3, n \geq 1 \\
    g'_{n,k} &= (15k - 2)n + 2, k \geq 3, n \geq 1.
\end{align}
    The genus of $P_{n,k}$ and $Q_{n,k}$ are $g_{n,k}$ and $g'_{n,k}$ respectively. 
\end{lem}
\begin{proof}
    We will first prove the lemma for $P_{n,k}$. Consider the subsurface $U \subset P_{n,k}$ defined as $$U = \left( \bigcup_{i =0}^{k-1} T_{0,i} \right)/\sim.$$ Then $U$ is a compact, non-orientable surface of genus $12k$ with $2k$ boundary components, and forms a fundamental domain for the covering action of $\overline{\rho_1}$ on $S_\infty$. We have $$\chi(U) = 2 - 12k - 2k = 2 - 14k.$$ Thus $$\chi(P_{n,k}) = n \cdot \chi(U) = -n(14k - 2),$$ since $P_{n,k}$ is formed by gluing $n$ copies of $U$ together along circle boundary components which have zero Euler characteristic. Therefore $$\chi(P_{n,k}) = 2 - g_{n,k} = -n(14k - 2) \implies g_{n,k} = n(14k - 2) + 2.$$ Likewise, the same argument holds for $Q_{n,k}$, but replacing 12 with 13 gives $g'_{n,k} = n(15k - 2) + 2$.
\end{proof}

\textbf{Step 2:} Following Yazdi, we will now define maps $f_{n,k}: P_{n,k} \xrightarrow[]{} P_{n,k}$ and $h_{n,k}: Q_{n,k} \xrightarrow[]{} Q_{n,k}$ that are defined as a composition of suitable Dehn twists, followed by a finite order mapping class. Let us specify the curves along which we do the Dehn twists.

\textcolor{red}{Should we define two separate maps like I did above, or just say everything works for both cases to limit how much notation we have to use?}

The following statements will hold for both $P_{n,k}$ and $Q_{n,k}$. Let $\mathcal{B}$ be the union of all $\beta$ curves except $\beta_1$ in $T_{0,0} \cup T_{0,1} \cup T_{1,0}$ (respectively $R_{0,0} \cup R_{0,1} \cup R_{1,0}$) (see figures below). Let $\rho_1(\mathcal{B})$ be the image of $\mathcal{B}$ under $\rho_1$. Define $\phi_b$ as the composition of positive Dehn twists along all the curves in the set $\overline{\mathcal{B}} \coloneqq \mathcal{B} \cup \rho_1(\mathcal{B}) \cup \dots \cup \rho_1^{n-1}(\mathcal{B})$. Since the curves in $\overline{\mathcal{B}}$ are disjoint, Dehn twists along them commute. Therefore, it is no necessary to specify the order in which we compose these Dehn twists in $\phi_b$. Let $\mathcal{R}$ be the union of all $\alpha$ curves except $\alpha_1$ in $T_{0,0}$. Define $\mathcal{R}$ and $\phi_r$ similarly but use negative Dehn twists this time.

Let $\alpha_1,\beta_1 \subset T_{0,0}$ be the curves in Figure Z. Let $\phi$ be the composition of negative dehn twists along all the curves $\alpha_1, \rho_1(\alpha_1), \dots, \rho_1^{n-1}(\alpha_1)$ followed by positive Dehn twists along all the curves $\beta_1,\rho_1(\beta_1),\dots,\rho_1^{n-1}(\beta_1)$. Define 
\begin{align*}
    f_{n,k} &\coloneqq \rho_2 \circ \phi \circ \phi_b \circ \phi_r \\
    h_{n,k} & \coloneqq \rho_2 \circ \phi \circ \phi_b \circ \phi_r
\end{align*}
using the curves in $P_{n,k}$ and $Q_{n,k}$ respectively. We are using notation so the composition is from right to left. It follows from Penner's construction of pseudo-Anosov maps that $(f_{n,k})^k$ and $(h_{n,k})^k$ are pseudo-Anosov. Hence $f_{n,k}$ and $h_{n,k}$ are themselves pseudo-Anosov and invariant train tracks $\tau^1_{n,k}$ and $\tau^2_{n,k}$ for $f_[n,k]$ and $h_{n,k}$ respectively can be obtained from Penner's construction by smoothing the intersection points.

\textcolor{red}{Do we need to give a stronger argument in order to use Penner for non-orientable? Something about two-sided filling curves lift to filling curves in the double cover and if a lift is pseudo-Anosov than the original map is pseudo-Anosov?}
